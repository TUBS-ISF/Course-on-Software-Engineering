%\subsection{What You Should Know}
%
%\begin{frame}{\myframetitle{}}
%	\begin{fancycolumns}
%%		\begin{note}{Fundamentals of Software Engineering}
%%			\begin{itemize}
%%				\item development processes
%%				\item object-oriented programming
%%				\item design patterns
%%				\item UML class diagrams
%%				\item modularity
%%			\end{itemize}
%%			\ifuniversity{magdeburg}{$\Rightarrow$ \emph{Software Engineering}}
%%		\end{note}
%	\nextcolumn
%%		\begin{note}{Fundamentals of Theoretical Computer Science}
%%			\begin{itemize}
%%				\item set theory
%%				\item propositional logic
%%				\item complexity theory
%%			\end{itemize}
%%			\ifuniversity{magdeburg}{
%%				$\Rightarrow$ \emph{Logik}\\
%%				$\Rightarrow$ \emph{Grundlagen der Theoretischen Informatik I}
%%			}
%%		\end{note}
%%		\begin{note}{Exercise}
%%			solid programming skills in Java
%%
%%			\ifuniversity{magdeburg}{
%%				$\Rightarrow$ \emph{Einführung in die Informatik}\\
%%				$\Rightarrow$ \emph{Algorithmen und Datenstrukturen}
%%			}
%%		\end{note}
%	\end{fancycolumns}
%\end{frame}

\subsection{Structure of This Course}

\begin{frame}{\myframetitle{}}
	\lectureseriesoverview[1]
\end{frame}

\begin{frame}{\myframetitle{}}
	\small
	\begin{fancycolumns}[columns=3,t,widths={26,34,38},animation=none]
		\begin{definition}{Part A: How to develop \emph{correct} software?}
			A common problem of software is its \emph{insufficient quality}. In the first part we will investigate the basics in software engineering to improve the quality of software. First, we start with the right \emph{choice of a programming language} and the available tools to support the development. Second, we give an overview on software quality and the two orthogonal \emph{testing techniques}, namely white-box and black-box testing. %Third, we illustrate the role of \emph{changes to software} and their impact on software quality. Finally, we discuss how to \emph{keep track of versions} and how to detect problems with software faster.
			Finally, we discuss how to \emph{operate and maintain software}.
		\end{definition}
	\nextcolumn
		\begin{definition}{Part B: How to develop software in \emph{schedule} and \emph{budget}?}
			Developing high quality software is typically conflicting with the available budget and the time available for development. Good software engineering cannot concentrate only on quality, but also needs to focus on how software projects meet their schedule and budget. Here it is not enough to consider a single project in isolation, but instead maintain well functioning teams in the long run. First, we start with a basic introduction into \emph{project management}, which is keeping track of the progress of a software project and adjusting the development where required. Second, we give an overview on different \emph{process models} such as scrum, which split the development into dedicated phases.
		\end{definition}
	\nextcolumn
		\begin{definition}{Part C: How to develop \emph{needed} software?}
			Even if software is of high quality and its development has met its time and budget constraints, it does not necessarily mean that the software is useful at all. Software is supposed to solve a problem of its users and it is far from trivial to develop software that is actually needed. The project cartoon illustrates common problems, which we distinguish into whether the problem to be solved is understood (\emph{requirements} and \emph{system modeling} aka. analysis) and whether the envisioned solution fits its purpose (\emph{software architecture} and \emph{software design} aka. design). Finally, software engineering is not only about developing software from scratch but also about \emph{reusing existing software} where appropriate, with the potential to drastically reduce costs and time-to-market.
		\end{definition}
	\end{fancycolumns}
\end{frame}

%\subsection{What You Will Learn}

%\subsection{What You Might Need}
%
%\begin{frame}{\myframetitle{}}
%%	\myframeicon{\mytitlesource{\fospl, \featureide}}
%%	\begin{fancycolumns}
%%		\begin{exampletight}{Recommended Literature for Lecture \& Exercise}
%%			\centering
%%			\parbox{0.49\linewidth}{
%%				\centering
%%				\pic[width=\linewidth]{cover-fospl}
%%				\emph{theory-focused}
%%			}
%%			\parbox{0.475\linewidth}{
%%				\centering
%%				\pic[width=\linewidth]{cover-featureide}
%%				\emph{practice-oriented}
%%			}
%%		\end{exampletight}
%%	\nextcolumn
%%		\begin{exampletight}{Recommended Tool Support for the Exercise}
%%			\centering
%%			\picDark[width=\linewidth]{featureide-feature-model-editor}\\[.5ex]
%%			\pic[width=0.25\linewidth]{featureide-logo}
%%		\end{exampletight}
%%	\end{fancycolumns}
%\end{frame}

%\subsection{Credit for the Slides}
%
%\begin{frame}{\myframetitle{}}
%	\ifuniversity{anonymous}{\mynote{}{\centering\huge Anonymous Authors}}
%	\unlessuniversity{anonymous}{
%		\myframeicon{\href{https://github.com/SoftVarE-Group/Course-on-Software-Product-Lines}{\pic[scale=.75]{cc-by-sa}}}
%	}
%	\begin{fancycolumns}[columns=3,animation=none]
%	\nextcolumn
%		\unlessuniversity{anonymous}{
%			\begin{note}{Thomas Thüm}
%				\centering
%				\href{https://www.uni-ulm.de/en/in/sp/team/thuem/}{\adjincludegraphics[height=.45\textheight,trim={.125\width} 0 {.125\width} 0,clip]{thomas-thuem}}
%
%				\small Professor at Paderborn University
%
%				software engineering
%
%				FeatureIDE team leader
%			\end{note}
%		}
%	\nextcolumn
%	\end{fancycolumns}
%\end{frame}
