% TODO replace links each year
\newcommand{\StudIPLectureLink}{https://studip.tu-braunschweig.de/dispatch.php/course/details?sem_id=8fa526030e67f6d8e691f934c2e256cd&again=yes}
\newcommand{\StudIPExerciseLink}{https://studip.tu-braunschweig.de/dispatch.php/course/details?sem_id=4b34a635107cde7ef107e91137838beb&again=yes}
\newcommand{\StudIPLecture}{\href{\StudIPLectureLink}{Stud.IP (Lecture)}}
\newcommand{\StudIPExercise}{\href{\StudIPExerciseLink}{Stud.IP (Exercise)}}

\subsection{About This Course} 
\begin{frame}{\insertsubsection}
	\begin{fancycolumns}
		\begin{definition}{Software Engineering 1}
			\begin{itemize}
				\item Abbreviation: SE1
				\item Credits: 5 ECTS %(45h+75h+30h=150h)
				\item Semester hours: 2+1
				\item Courses of studies:
				\begin{itemize}
					\item B.\,Sc.: Informatik, Wirtschaftsinformatik, Informationssystemtechnik, \ldots
					\item M.\,Sc.: Digitale Kommunikation und Medientechnologien, \ldots
					%\item \ldots
				\end{itemize}
			\end{itemize}
		\end{definition}
		\nextcolumn
		\begin{definition}{Course Organization}
			\begin{itemize}
				\item 2 hours: typically one lecture per week, see schedule in \StudIPLecture
				\item 1 hour: one exercise (2 hours) every two weeks, see schedule in \StudIPExercise
				\item February: written exam
				\item March: written retake exam
				\item No exams in summer term
			\end{itemize}
		\end{definition}
	\end{fancycolumns}
	\uncover<3->{\begin{note}{Disclaimer}
		Passing this course (study achievement and exam) is required to participate in the SEP \deutsch{Softwareentwicklungspraktikum} in summer term!
	\end{note}}
\end{frame}

\xkcdframe{2385}

\subsection{The Lectures}
\begin{frame}{\insertsubsection}
	\begin{fancycolumns}[widths={60}]
		\begin{definition}{The Lectures}
			\begin{itemize}
				\item weekly, Tuesday 11:30--13:00
%				\item except for holidays \deutsch{Pfingstmontag}
				\item lecture follows the Sandwich model
				\item three lecture parts
				\item interactive tasks in between (some require smartphone, tablet, notebook)
				\item lecture recordings planned (but without any guarantee)
				\item slides \href{https://github.com/TUBS-ISF/Course-on-Software-Engineering}{published under CC-BY-SA-4.0 license}
			\end{itemize}
		\end{definition}
	\nextcolumn
		\begin{note}{Adam Osborne} % TODO add picture and source
			\mycite{The most valuable thing you can make is a mistake -- you can't learn anything from being perfect.}
		\end{note}
	\end{fancycolumns}
\end{frame}

\subsection{The Exercises}
\begin{frame}{\insertsubsection}
	\begin{fancycolumns}[widths={43}]
		\begin{definition}{The Exercises}
			\begin{itemize}
				\item goal: deeper understanding of lecture topics, preparation for the exam
				\item bi-weekly, concrete dates in \StudIPExercise
				\item starting in \emph{third} week (special exercise in \emph{second} week)
				\item at most \emph{12--15 participants} in each exercise
				\item registration to time slots opens October 22, 2025 at 9:30am via \StudIPExercise
				\item switching between or attending multiple exercises is \emph{not allowed}
				\item holidays: alternative dates discussed in affected exercises
			\end{itemize}
		\end{definition}
		\nextcolumn
		\begin{definition}{Voting System \deutsch{Votierungssystem}}
			\begin{itemize}
				\item 6 sheets with 5-8 tasks each
				\item participants prepare solutions for the tasks \emph{before} the exercise
				%\item in the 15 minutes before the exercise, participants can vote \deutsch{votieren} tasks they prepared in a dedicated list \deutsch{Votierungsliste}
				\item participants submit solutions the day before the exercise to vote \deutsch{votieren} for presenting the respective subtask
				\item in the 15 minutes before the exercise, participants can double check their votes \deutsch{Votierungen} in a dedicated list \deutsch{Votierungsliste}
				\item during the exercise one or multiple prepared participants are selected to \emph{give a presentation} of their solution at the whiteboard or beamer
				\item at least 50\,\% votes \deutsch{Votierungspunkte} and at least 3 presentations \deutsch{Vortragspunkte} required to pass the exercise \deutsch{Studienleistung}
				%\item at least 75\,\% votes required to get the grade improvement \deutsch{Notenbonus für eine Notenstufe}
				%\item students may prepare solutions in groups but every member needs to be able to explain the solution if voted
			\end{itemize}
		\end{definition}
	\end{fancycolumns}
\end{frame}

\begin{frame}{\insertsubsection}
	\begin{fancycolumns}[widths={40}]
		\begin{note}{Exception Handling}
			\begin{itemize}
				\item tasks without prepared participants will not be discussed
				\item unprepared/cheating participants risk to loose one or all votes for the current sheet (incl. presentation points)
				\item forms of plagiarism are reported and may lead to an exmatriculation
			\end{itemize}
		\end{note}
		\nextcolumn
		\begin{note}{Rules for Generative AI}
			\begin{itemize}
				\item there are only six exercises as practice before the exam
				\item using generative AI as a shortcut (a) will reduce your options to practice and (b) will waste time of your colleagues
				\item tools as ChatGPT are by default prohibited in this course (unless explicitly allowed for a subtask)
				\item publishing the exercise sheets (or parts thereof) is not allowed (includes uploads to LLMs)
			\end{itemize}
		\end{note}
	\end{fancycolumns}
\end{frame}

\begin{frame}{\insertsubsection}
	\begin{fancycolumns}
		\begin{definition}{Special Exercises in Second Week}
			The following rules apply only to the second week!
			\begin{itemize}
				\item time slot of both alternating weeks merged
				\item special exercise on git and control flow graphs
				\item git is required for submission of solutions
				\item control flow graphs are used in lecture and exercise on testing
				\item come unprepared (i.e., no voting)
			\end{itemize}
		\end{definition}
		\nextcolumn
		\begin{note}{How to Find Answers}
			\begin{enumerate}
				\item Ask questions during the lecture (e.g., during interactive parts)
				\item Check information in StudIP
				\begin{itemize}
					\item Check already answered questions
					\item Ask your own questions and answer questions of fellow students
				\end{itemize}
				\item Ask questions in your exercise
				%\item Meet Thomas in his consultation hour:\\Wednesday 2pm in IZ 348\\(preregistration useful)
				\item Slowest option: Contact us via  \href{mailto:christopher.rau@tu-braunschweig.de?cc=thomas.thuem@tu-braunschweig.de&subject=[SE1]}{e-mail}
			\end{enumerate}
		\end{note}
	\end{fancycolumns}
\end{frame}

\subsection{Lego Scrum}
\begin{frame}{\insertsubsection}
	\begin{fancycolumns}[animation=none]
		\centering\pic[width=\linewidth]{legoscrum/legoscrum2023-01-13}
		\nextcolumn
		\centering\pic[width=\linewidth]{legoscrum/legoscrum2023-01-14}
	\end{fancycolumns}
	\centering Want to learn how software is developed in practice?
\end{frame}
\begin{frame}{\insertsubsection}
	\begin{fancycolumns}[columns=3,animation=none]
		\centering\vspace{20mm}
		
		\pic[width=\linewidth]{legoscrum/legoscrum2024-02-14results3}
		
		Build.
		\nextcolumn
		\centering\vspace{10mm}
		
		\pic[width=\linewidth]{legoscrum/legoscrum2023-01-14results1}
		
		(Y)our.
		\nextcolumn
		\centering\pic[width=\linewidth]{legoscrum/legoscrum2023-01-13results1}
		
		City.
	\end{fancycolumns}
\end{frame}

\subsection{Literature for This Course}
\begin{frame}{\insertsubsection}
	\begin{fancycolumns}[animation=none]
		\centering\pic[height=50mm]{books/sommerville-softwarenegineering}
		\nextcolumn
		\begin{definition}{\mysource{\sommerville}}
			\begin{itemize}
				\item \sommervillelink{Ian Sommerville. Software Engineering, 10. Edition, Pearson, 2018.}
				\begin{itemize}
					\item German, English, and earlier versions
					\item \href{https://software-engineering-book.com/videos/}{Videos by Ian Sommerville and others available online}
				\end{itemize}
				\item More literature announced in each lecture
			\end{itemize}
		\end{definition}
	\end{fancycolumns}
\end{frame}
