\subsection{Development of Similar Products: Elevators}
\begin{frame}{\insertsubsection}
	\uncover<5->{\begin{note}{}
		\centering All these elevators are different but also share similarities.
	\end{note}}

	~

	\begin{fancycolumns}[columns=4,T]
		\centering\pic[width=\linewidth]{elevators/elevator1-out}
		
		two buttons
	\nextcolumn
		\centering\pic[width=\linewidth]{elevators/elevator2-out2}
		
		one button
	\nextcolumn
		\centering\pic[width=\linewidth]{elevators/elevator3-out}
		
		keyhole
	\nextcolumn
		\centering\pic[width=\linewidth]{elevators/elevator2-out1}
		
		floor display
	\end{fancycolumns}
\end{frame}

\begin{frame}{\insertsubsection}
	\uncover<5->{\begin{note}{}
			\centering Shall we implement each product from scratch or can we reuse parts thereof?
	\end{note}}

	~

	\begin{fancycolumns}[columns=4,widths={28,21,28,21},T]
		\centering\pic[width=\linewidth]{elevators/elevator1-in2}
		
		no button to close door
	\nextcolumn
		\centering\pic[width=\linewidth]{elevators/elevator3-in1}
		
		two keyholes
	\nextcolumn
		\centering\pic[width=\linewidth]{elevators/elevator4-in}
		
		keycard
	\nextcolumn
		\centering\pic[width=\linewidth]{elevators/elevator2-in2}
		
		double tap for~undo
	\end{fancycolumns}
\end{frame}

\subsection{Reuse with Components}
\begin{frame}{\insertsubsection}
	\begin{fancycolumns}[animation=none]
		\pic[page=25,width=\linewidth,trim={.5\width} {.1\height} {.05\width} {.2\height},clip]{printedslides/2021wt/architecture}
	\nextcolumn
		\begin{definition}{Reuse with Client-Server Architectures}
			Different client components can communicate with the same server component
		\end{definition}
		\uncover<2->{\begin{note}{Discussion}
			\begin{itemize}
				\item Reuse of server component for multiple clients
				\item Lower effort than developing multiple servers
				\item Unit testing: only one server needs to be tested
				\item Integration testing: any change to the server may break an existing client
				\item Changing a client may require changes to the server or its interface (API); changing the server may require changes to all clients
				\item Challenge: how to design stable APIs?
			\end{itemize}
		\end{note}}
	\end{fancycolumns}
\end{frame}

\begin{frame}{\insertsubsection}
	\begin{fancycolumns}[animation=none]
		\pic[page=24,width=\linewidth,trim={.5\width} {.1\height} {.05\width} {.2\height},clip]{printedslides/2021wt/architecture}
		\nextcolumn
		\begin{definition}{Reuse with Layered Architectures}
			Layer can be replaced by compatible one with same interface (e.g., IP v4 by IP v6)
		\end{definition}
		\uncover<2->{\begin{note}{Discussion}
			\begin{itemize}
				\item Replacing some layers and reusing others
				\item Ideally only new layers need to be implemented
				\item In practice: adaptations to existing layers may be necessary
				\item Again: interplay needs to be checked with integration testing
				\item Challenge: how to design interoperable layers?
			\end{itemize}
		\end{note}}
	\end{fancycolumns}
\end{frame}

% TODO add log4j exploit here illustrating how often it was reused and how often it was vulnerable

\subsection{Reuse with Runtime Parameters}
\begin{frame}{\insertsubsection}
	\uncover<3->{\begin{definition}{}
		\centering All functionality implemented in one product. Parameters used to activate varying behavior.
	\end{definition}}
	\begin{fancycolumns}
		\pic[width=\linewidth]{reuse/runtime-parameters-win10-cmd-dir2}
	\nextcolumn
		\pic[width=\linewidth]{reuse/runtime-parameters-win10-cmd-dir}
	\end{fancycolumns}
\end{frame}

\begin{frame}{\insertsubsection}
	\begin{fancycolumns}[animation=none]
		\picDark[width=\linewidth,trim={.35\width} {.12\height} {.1\width} {.45\height},clip]{reuse/runtime-parameters} % \mysource{\featureide}
	\nextcolumn
		\begin{definition}{Reuse with Runtime Parameters}
			Complete product is reused and extended with new behavior (e.g., features on demand in cars)
		\end{definition}
		\uncover<2->{\begin{note}{Discussion}
				\begin{itemize}
					\item Configuration: reuse with new combination of existing parameters
					\item Extension: implementation of new parameters and parameter values
					\item Challenges:
						\begin{itemize}
							\item larger binaries compared to excluded components
							\item safety and security risks by unused functionality
							\item side effects: any change to a parameter may break other parameters
						\end{itemize}
				\end{itemize}
		\end{note}}
	\end{fancycolumns}
\end{frame}

\subsection{Ad-Hoc Reuse with Clone-and-Own}
\begin{frame}{\insertsubsection}
	\centering\trunkbranch{}{}
	\begin{fancycolumns}[T,widths={30}]
		\begin{definition}{Clone-and-Own}
			Complete product is copied (cloned) and modified as needed (owned)
			
			aka. cloning in the large
		\end{definition}
	\nextcolumn
		\begin{note}{Discussion}
				\begin{itemize}
					\item Copying often done by branching or forking in version control
					\item Fast and flexible implementation strategy for adaptation
					\item Challenges:
					\begin{itemize}
						\item features and adaptations lost in the implementation
						\item consistent changes of clones hardly possible in practice
						\item clones are an obstacle during evolution and maintenance
						\item no actual reuse, often considered an anti pattern
					\end{itemize}
				\end{itemize}
		\end{note}
	\end{fancycolumns}
\end{frame}
