%\lstset{style=java,style=featurehouse,style=jml,tabsize=2,escapechar=|,style=scriptsize,frame=none}%style=slides,
\newcommand{\featuremodule}[1]{\hfill#1~~~}

% TODO use resize box to fill code to available width \resizebox{60mm}{!}{}
\lstset{style=java,tabsize=2,escapechar=|,frame=none,style=small,backgroundcolor=,commentstyle=\color{blue}\selectfont}%style=featurehouse,style=jml,style=slides,
\newcommand{\centeredlisting}[1]{
	\begin{center}
		\begin{tikzpicture}[every node/.style={rounded corners,draw,thick,fill=white},listing/.style={inner xsep=4pt,inner ysep=3pt}]
			\node[listing] at (0,0) {\makebox{\usebox{#1}}};
		\end{tikzpicture}
	\end{center}
}

%%%%%%%%%% source code with bugs %%%%%%%%%%

\newsavebox{\edgelexicalerror}
\begin{lrbox}{\edgelexicalerror}
	\begin{lstlisting}
class Node {
}

class Edge {
	Node node, node|\tpkt{a1}´\tpkt{a2}|;
}|\errorunderline{a}|
	\end{lstlisting}
\end{lrbox}

\newsavebox{\edgelexicalerrorfix}
\begin{lrbox}{\edgelexicalerrorfix}
	\begin{lstlisting}
class Node {
}

class Edge {
	Node |\tpkt{a1}|first, second|\tpkt{a2}|;
}|\infounderline{a}|
	\end{lstlisting}
\end{lrbox}

\newsavebox{\edgesyntaxerror}
\begin{lrbox}{\edgesyntaxerror}
	\begin{lstlisting}
class Node {
}

class Edge {
	Node first, second;

	Edge(Node first, Node second|\tpkt{a1}|)|\tpkt{a2}|
		this.first = first;
		this.second = second;
	}
}|\errorunderline{a}|
	\end{lstlisting}
\end{lrbox}

\newsavebox{\edgesyntaxerrorfix}
\begin{lrbox}{\edgesyntaxerrorfix}
	\begin{lstlisting}
class Node {
}

class Edge {
	Node first, second;

	Edge(Node first, Node second|\tpkt{a1}|) {|\tpkt{a2}|
		this.first = first;
		this.second = second;
	}
}|\infounderline{a}|
	\end{lstlisting}
\end{lrbox}

\newsavebox{\edgetypeerror}
\begin{lrbox}{\edgetypeerror}
	\begin{lstlisting}
class Edge {
	Node first, second;
	Edge(Node first, Node second) {
		this.first = first;
		this.second = second;
	}
	public static void main(String[] args) {
		Node a = new Node();
		Node b = new Node();
		Edge e = |\tpkt{a1}|new Edge()|\tpkt{a2}|;
		System.out.println(e);
	}
}|\errorunderline{a}|
\end{lstlisting}
\end{lrbox}

\newsavebox{\edgetypeerrorfix}
\begin{lrbox}{\edgetypeerrorfix}
\begin{lstlisting}
class Edge {
	Node first, second;
	Edge(Node first, Node second) {
		this.first = first;
		this.second = second;
	}
	public static void main(String[] args) {
		Node a = new Node();
		Node b = new Node();
		Edge e = new Edge(|\tpkt{a1}|a,b|\tpkt{a2}|);
		System.out.println(e);
	}
}|\infounderline{a}|
\end{lstlisting}
\end{lrbox}

\newsavebox{\edgeruntimeerror}
\begin{lrbox}{\edgeruntimeerror}
	\begin{lstlisting}
class Edge {
	Node first, second;
	Edge(Node first, Node second) {
		this.first = first;
		this.second = second;
	}
	boolean equals(Edge e) {
		return first.equals(|\tpkt{a1}|e.first|\tpkt{a2}|) && second.equals(e.first);
	}
	public static void main(String[] args) {
		Edge e = new Edge(new Node(), new Node());
		System.out.println(e.equals(null));
	}
}|\errorunderline{a}|
	\end{lstlisting}
\end{lrbox}

% warning: any update of the code below may create inconsistencies with screenshot stored in designbycontract/javadoc.png
\newsavebox{\javadoc}
\begin{lrbox}{\javadoc}
	\begin{lstlisting}
class Edge {
	Node first, second;
	Edge(Node first, Node second) {
		this.first = first;
		this.second = second;
	}
	/** Comparison of edges depending on connected nodes.
	 * @param e edge that is not null
	 * @return true if both edges connect same nodes
	 */
	boolean equals(Edge e) {
		return first.equals(e.first) && second.equals(e.first);
	}
	public static void main(String[] args) {
		Edge e = new Edge(new Node(), new Node());
		System.out.println(e.equals(null));
	}
}
	\end{lstlisting}
\end{lrbox}

\newsavebox{\defensiveprogramming}
\begin{lrbox}{\defensiveprogramming}
	\begin{lstlisting}
class Edge {
	Node first, second;
	Edge(Node first, Node second) {
		this.first = first;
		this.second = second;
	}
	boolean equals(Edge e) {
		|\tpkt{a1}|if (e == null)|\tpkt{a2}|
			|\tpkt{b1}|throw new RuntimeException(|\tpkt{b2}|
				|\tpkt{c1}|"non-null parameter expected");|\tpkt{c2}|
		return first.equals(e.first) && second.equals(e.first);
	}
	public static void main(String[] args) {
		Edge e = new Edge(new Node(), new Node());
		System.out.println(e.equals(null));
	}
}|\infounderline{a,b,c}|
	\end{lstlisting}
\end{lrbox}

\newsavebox{\runtimeassertions}
\begin{lrbox}{\runtimeassertions}
	\begin{lstlisting}
class Edge {
	Node first, second;
	Edge(Node first, Node second) {
		this.first = first;
		this.second = second;
	}
	boolean equals(Edge e) {
		|\tpkt{a1}|assert (e == null) : "non-null parameter expected";|\tpkt{a2}|
		return first.equals(e.first) && second.equals(e.first);
	}
	public static void main(String[] args) {
		Edge e = new Edge(new Node(), new Node());
		System.out.println(e.equals(null));
	}
}|\infounderline{a}|
	\end{lstlisting}
\end{lrbox}

\newsavebox{\unittests}
\begin{lrbox}{\unittests}
	\begin{lstlisting}
class Edge {
	Node first, second;
	Edge(Node first, Node second) {
		this.first = first;
		this.second = second;
	}
	boolean equals(Edge e) {
		return first.equals(e.first) && second.equals(e.first);
	}
}
|\tpkt{d1}|import org.junit.Test;|\tpkt{d2}|
|\tpkt{e1}|public class TEdge {|\tpkt{e2}|
	|\tpkt{c1}|@Test|\tpkt{c2}|
	|\tpkt{f1}|public|\tpkt{f2}| void testEqualsNull() {
		Edge e = new Edge(new Node(), new Node());
		assert !e.equals(|\tpkt{b1}|null|\tpkt{b2}|);
	}
}|\infounderline{c,d,e,f}|
	\end{lstlisting}
\end{lrbox}

\newsavebox{\jml}
\begin{lrbox}{\jml}
	\begin{lstlisting}
class Edge {
	//@ invariant first != null && second != null;
	Node first, second;
	Edge(Node first, Node second) {
		this.first = first;
		this.second = second;
	}
	/*@ requires e != null;
		@ ensures \result <==> first.equals(e.first) &&
		@   second.equals(e.second); @*/
	boolean equals(Edge e) {
		return first.equals(e.first) && second.equals(e.first);
	}
}
	\end{lstlisting}
\end{lrbox}

\newsavebox{\jmlisqrt}
\begin{lrbox}{\jmlisqrt}
	\begin{lstlisting}
/*@ public normal_behavior
	@ requires y >= 0;
	@ assignable \nothing;
	@ ensures 0 <= \result
	@   && \result * \result <= y
	@   && ((0 <= (\result + 1) * (\result + 1))
	@   ==> y < (\result + 1) * (\result + 1));
	@*/
public static int isqrt(int y) {
	return (int) Math.sqrt(y);
}
	\end{lstlisting}
\end{lrbox}

\newsavebox{\edgeruntimeerrorfix}
\begin{lrbox}{\edgeruntimeerrorfix}
	\begin{lstlisting}
class Edge {
	Node first, second;
	Edge(Node first, Node second) {
		this.first = first;
		this.second = second;
	}
	boolean equals(Edge e) {
		return first.equals(e.first) && second.equals(e.first);
	}
	public static void main(String[] args) {
		Edge e = new Edge(new Node(), new Node());
		System.out.println(e.equals(|\tpkt{a1}|e|\tpkt{a2}|));
	}
}|\infounderline{a}|
	\end{lstlisting}
\end{lrbox}

\newsavebox{\edgelogicalerror}
\begin{lrbox}{\edgelogicalerror}
	\begin{lstlisting}
class Edge {
	Node first, second;
	Edge(Node first, Node second) {
		this.first = first;
		this.second = second;
	}
	boolean equals(Edge e) {
		return first.equals(e.first) && second.equals(e.first);
	}
	public static void main(String[] args) {
		Edge e = new Edge(new Node(), new Node());
		System.out.println(e.equals(e));
	}
}
	\end{lstlisting}
\end{lrbox}

\newsavebox{\edgelogicalerrorfix}
\begin{lrbox}{\edgelogicalerrorfix}
	\begin{lstlisting}
class Edge {
	Node first, second;
	Edge(Node first, Node second) {
		this.first = first;
		this.second = second;
	}
	boolean equals(Edge e) {
		return first.equals(e.first) && second.equals(e.|\tpkt{a1}|second|\tpkt{a2}|);
	}
	public static void main(String[] args) {
		Edge e = new Edge(new Node(), new Node());
		System.out.println(e.equals(e));
	}
}|\infounderline{a}|
	\end{lstlisting}
\end{lrbox}

\newsavebox{\codeone}
\begin{lrbox}{\codeone}
\begin{lstlisting}
class Edge {
  Node first, second;
	Edge(Node first, Node second) {
		this.first = first;
		this.second = second;
	}
  boolean |\tpkt{b1}|equals(Edge e)|\tpkt{b2}| {
    return first.equals(e.first) && second.equals(e.first);
  }
	public static void main(String[] args) {
		Node a = new Node();
		Node b = new Node();
		Edge e = new Edge(a, b);
		System.out.println(e.|\tpkt{a1}|equals()|\tpkt{a2}|);
	}
}
\end{lstlisting}
\end{lrbox}

\newsavebox{\bugone}
\begin{lrbox}{\bugone}
\begin{lstlisting}
class Edge {
  Node first, second;
	Edge(Node first, Node second) {
		this.first = first;
		this.second = second;
	}
  boolean |\tpkt{b1}|equals(Edge e)|\tpkt{b2}| {
    return first.equals(e.first) && second.equals(e.first);
  }
	public static void main(String[] args) {
		Node a = new Node();
		Node b = new Node();
		Edge e = new Edge(a, b);
		System.out.println(e.|\tpkt{a1}|equals()|\tpkt{a2}|);
	}
}|\errorunderline{a}\warningunderline{b}|
\end{lstlisting}
\end{lrbox}

\newsavebox{\codetwo}
\begin{lrbox}{\codetwo}
\begin{lstlisting}
class Edge {
  Node first, second;
	Edge(Node first, Node second) {
		this.first = first;
		this.second = second;
	}
  boolean equals(Edge e) {
    return first.equals(|\tpkt{a1}|e.first|\tpkt{a2}|) && second.equals(e.first);
  }
	public static void main(String[] args) {
		Node a = new Node();
		Node b = new Node();
		Edge e = new Edge(a, b);
		System.out.println(e.equals(|\tpkt{b1}|null|\tpkt{b2}|));
	}
}|\infounderline{b}|
\end{lstlisting}
\end{lrbox}

\newsavebox{\bugtwo}
\begin{lrbox}{\bugtwo}
\begin{lstlisting}
class Edge {
  Node first, second;
	Edge(Node first, Node second) {
		this.first = first;
		this.second = second;
	}
  boolean equals(Edge e) {
    return first.equals(|\tpkt{a1}|e.first|\tpkt{a2}|) && second.equals(e.first);
  }
	public static void main(String[] args) {
		Node a = new Node();
		Node b = new Node();
		Edge e = new Edge(a, b);
		System.out.println(e.equals(|\tpkt{b1}|null|\tpkt{b2}|));
	}
}|\errorunderline{a}\warningunderline{b}|
\end{lstlisting}
\end{lrbox}

\newsavebox{\bugtwoagain}
\begin{lrbox}{\bugtwoagain}
\begin{lstlisting}
class Edge {
	Node first, second;
	Edge(Node first, Node second) {
		this.first = first; this.second = second;
	}
	boolean equals(Edge e) {
		return first.equals(|\tpkt{a1}|e.first|\tpkt{a2}|) && second.equals(e.first);
	}
}
|\tpkt{d1}|import org.junit.Test;|\tpkt{d2}|
|\tpkt{e1}|public class TEdge {|\tpkt{e2}|
	|\tpkt{c1}|@Test|\tpkt{c2}|
	|\tpkt{f1}|public|\tpkt{f2}| void test1() {
		Node a = new Node();
		Node b = new Node();
		Edge e = new Edge(a, b);
		System.out.println(e.equals(|\tpkt{b1}|null|\tpkt{b2}|));
	}
}|\errorunderline{a}\warningunderline{b}\infounderline{c,d,e,f}|
\end{lstlisting}
\end{lrbox}

\newsavebox{\codethree}
\begin{lrbox}{\codethree}
\begin{lstlisting}
class Edge {
	Node first, second;
	Edge(Node first, Node second) {
		this.first = first; this.second = second;
	}
	boolean equals(Edge e) {
		return first.equals(|\tpkt{a1}|e.first|\tpkt{a2}|) && second.equals(e.first);
	}
}
|\tpkt{d1}|import org.junit.Test;|\tpkt{d2}|
|\tpkt{e1}|public class TEdge {|\tpkt{e2}|
	|\tpkt{c1}|@Test|\tpkt{c2}|
	|\tpkt{f1}|public|\tpkt{f2}| void test1() {
		Node a = new Node();
		Node b = new Node();
		Edge e = new Edge(a, b);
		System.out.println(e.equals(|\tpkt{b1}|e|\tpkt{b2}|));
	}
}|\infounderline{b}|
\end{lstlisting}
\end{lrbox}

\newsavebox{\bugthree}
\begin{lrbox}{\bugthree}
\begin{lstlisting}
class Node {}
class Edge {
	Node first, second;
	Edge(Node first, Node second) {
		this.first = first; this.second = second;
	}
	boolean equals(Edge e) {
		return first.equals(e.first) && second.equals(e.|\tpkt{a1}|first|\tpkt{a2}|);
	}
}
|\tpkt{d1}|import org.junit.Test;|\tpkt{d2}|
|\tpkt{e1}|public class TEdge {|\tpkt{e2}|
	|\tpkt{c1}|@Test|\tpkt{c2}|
	|\tpkt{f1}|public|\tpkt{f2}| void test1() {
		Node a = new Node();
		Node b = new Node();
		|\tpkt{b1}|System.out.println(new Edge(a, b).equals(new Edge(a, b)));|\tpkt{b2}|
	}
}|\errorunderline{b}\warningunderline{a}|
\end{lstlisting}
\end{lrbox}

\newsavebox{\codefour}
\begin{lrbox}{\codefour}
\begin{lstlisting}
class Node {}
class Edge {
	Node first, second;
	Edge(Node first, Node second) {
		this.first = first; this.second = second;
	}
	boolean equals(Edge e) {
	    return first.equals(e.first) && second.equals(e.|\tpkt{a1}|first|\tpkt{a2}|);
	}
}
|\tpkt{d1}|import org.junit.*;|\tpkt{d2}|
|\tpkt{e1}|public class TEdge {|\tpkt{e2}|
	|\tpkt{c1}|@Test|\tpkt{c2}|
	|\tpkt{f1}|public|\tpkt{f2}| void test1() {
		Node a = new Node();
		Node b = new Node();
		|\tpkt{b1}|Assert.assertTrue|\tpkt{b2}|(new Edge(a, b).equals(new Edge(a, b)));
	}
}|\infounderline{b,d}|
\end{lstlisting}
\end{lrbox}

\newsavebox{\bugfour}
\begin{lrbox}{\bugfour}
	\begin{lstlisting}
class Node {}
class Edge {
	Node first, second;
	Edge(Node first, Node second) {
		this.first = first; this.second = second;
	}
	boolean equals(Edge e) {
		return first.equals(e.first) && second.equals(e.|\tpkt{a1}|first|\tpkt{a2}|);
	}
}
|\tpkt{d1}|import org.junit.*;|\tpkt{d2}|
|\tpkt{e1}|public class TEdge {|\tpkt{e2}|
	|\tpkt{c1}|@Test|\tpkt{c2}|
	|\tpkt{f1}|public|\tpkt{f2}| void test1() {
		Node a = new Node();
		Node b = new Node();
		|\tpkt{b1}|Assert.assertTrue(new Edge(a, b).equals(new Edge(a, b)));|\tpkt{b2}|
	}
}|\warningunderline{a}\errorunderline{b}|
	\end{lstlisting}
\end{lrbox}

\newsavebox{\codefive}
\begin{lrbox}{\codefive}
	\begin{lstlisting}
class Node {}
class Edge {
	Node first, second;
	Edge(Node first, Node second) {
		this.first = first; this.second = second;
	}
	boolean equals(Edge e) {
		return first.equals(e.first) && second.equals(e.|\tpkt{a1}|second|\tpkt{a2}|);
	}
}
|\tpkt{d1}|import org.junit.*;|\tpkt{d2}|
|\tpkt{e1}|public class TEdge {|\tpkt{e2}|
	|\tpkt{c1}|@Test|\tpkt{c2}|
	|\tpkt{f1}|public|\tpkt{f2}| void test1() {
		Node a = new Node();
		Node b = new Node();
		|\tpkt{b1}|Assert.assertTrue|\tpkt{b2}|(new Edge(a, b).equals(new Edge(a, b)));
	}
}|\infounderline{a}|
	\end{lstlisting}
\end{lrbox}

\newsavebox{\dbc}
\begin{lrbox}{\dbc}
\begin{lstlisting}
class Node {}
class Edge {
  //@ invariant first != null && second != null;
  Node first, second;
	Edge(Node first, Node second) {
		this.first = first;
		this.second = second;
	}
  /*@ requires e != null;
    @ ensures \result <==> first.equals(e.first) &&
    @   second.equals(e.second); @*/
  boolean equals(Edge e) {
    return first.equals(e.first) && second.equals(e.first);
  }
	public static void main(String[] args) {
		Node a = new Node();
		Node b = new Node();
		System.out.println(new Edge(a, b).equals(null));
	}
}
\end{lstlisting}
\end{lrbox}

\newsavebox{\callee}
\begin{lrbox}{\callee}
\begin{lstlisting}

boolean equals(Edge e) {
	return first.equals(e.first)
	  && second.equals(e.first);
}
|~|
\end{lstlisting}
\end{lrbox}

\newsavebox{\caller}
\begin{lrbox}{\caller}
\begin{lstlisting}
static void main(String[] args) {
	Node a = new Node();
	Node b = new Node();
	System.out.println(
	  new Edge(a, b).equals(null));
}
\end{lstlisting}
\end{lrbox}

\newsavebox{\blamepre}
\begin{lrbox}{\blamepre}
\begin{lstlisting}
/*@ requires e != null; @*/
\end{lstlisting}
\end{lrbox}

\newsavebox{\blamepost}
\begin{lrbox}{\blamepost}
\begin{lstlisting}
/*@ ensures \result <==>
  @ first.equals(e.first) &&
  @ second.equals(e.second); @*/
\end{lstlisting}
\end{lrbox}

%%%%%%%%%% source code for fop %%%%%%%%%%

\newsavebox{\fopbase}
\begin{lrbox}{\fopbase}
	\begin{minipage}{72mm}
\begin{lstlisting}
class Edge {|\featuremodule{Base}|
  Node first, second;
	Edge(Node first, Node second) {
		this.first = first;
		this.second = second;
	}
  boolean equals(Edge e) {
    return first.equals(e.first) &&
      second.equals(e.second);
  }
[...] }
\end{lstlisting}
	\end{minipage}
\end{lrbox}

\newsavebox{\fopweight}
\begin{lrbox}{\fopweight}
	\begin{minipage}{72mm}
\begin{lstlisting}
class Edge {|\featuremodule{Weighted}|
  Integer weight = 0;
	void setWeight(Integer weight) {
		this.weight = weight;
	}
  boolean equals(Edge e) {
    return original(e) && weight == e.weight;
  }
[...] }
\end{lstlisting}
	\end{minipage}
\end{lrbox}

%%%%%%%%%% source code for plain contracting %%%%%%%%%%

\newsavebox{\pcbase}
\begin{lrbox}{\pcbase}
	\begin{minipage}{72mm}
\begin{lstlisting}
class Edge {|\featuremodule{Base}|
  //@ invariant first != null && second != nulljml;
  Node first, second;
  /*@ requires e != null;|\tpkt{a2}\tpkt{b2}\tpkt{d2}|
    @ ensures \result ==> first.equals(e.first) &&
    @ |~| second.equals(e.second); @*/
  boolean equals(Edge e) {
    return first.equals(e.first) &&
      second.equals(e.second);
  }
[...] }
\end{lstlisting}
	\end{minipage}
\end{lrbox}

\newsavebox{\pcweight}
\begin{lrbox}{\pcweight}
	\begin{minipage}{72mm}
\begin{lstlisting}
class Edge {|\featuremodule{Weighted}|
  Integer weight = 0;
  boolean equals(Edge e) {
    return original(e) && weight == e.weight;
  }
[...] }
\end{lstlisting}
	\end{minipage}
\end{lrbox}

%%%%%%%%%% source code for behavioral subtyping %%%%%%%%%%

\newsavebox{\bhsuper}
\begin{lrbox}{\bhsuper}
	\begin{minipage}{55mm}
\begin{lstlisting}
class Edge {
  //@ invariant first != null && second != null;
  Node first, second;
  /*@ requires e != null;|\tpkt{a2}\tpkt{b2}\tpkt{d2}|
    @ ensures \result ==> first.equals(e.first) &&
    @ |~| second.equals(e.second); @*/
  boolean equals(Edge e) {
    return first.equals(e.first) &&
      second.equals(e.second);
  }
  [...]
}
\end{lstlisting}
	\end{minipage}
\end{lrbox}

\newsavebox{\bhsub}
\begin{lrbox}{\bhsub}
	\begin{minipage}{55mm}
\begin{lstlisting}
class WeightedEdge extends Edge {
  Integer weight = 0;
  /*@ also
    @ requires e != null && weight != null;|\tpkt{c2}\tpkt{e2}|
    @ ensures \result ==> weight == e.weight; @*/
  boolean equals(WeightedEdge e) {
    return super(e) && weight.equals(e.weight);
  }
  [...]
}
\end{lstlisting}
	\end{minipage}
\end{lrbox}

\newsavebox{\bhone}
\begin{lrbox}{\bhone}
	\begin{minipage}{40mm}
\begin{lstlisting}
|\tpkt{a1}|new Edge([...]).equals([...]);
\end{lstlisting}
	\end{minipage}
\end{lrbox}

\newsavebox{\bhtwo}
\begin{lrbox}{\bhtwo}
	\begin{minipage}{40mm}
\begin{lstlisting}
Edge e = new WeightedEdge([...]);
|\tpkt{d1}\tpkt{e1}|e.equals([...]);
\end{lstlisting}
	\end{minipage}
\end{lrbox}

\newsavebox{\bhthree}
\begin{lrbox}{\bhthree}
	\begin{minipage}{40mm}
\begin{lstlisting}
|\tpkt{b1}\tpkt{c1}|new WeightedEdge([...]).equals([...]);
\end{lstlisting}
	\end{minipage}
\end{lrbox}

%%%%%%%%%% source code for explicit contract refinement %%%%%%%%%%

\newsavebox{\ecrbase}
\begin{lrbox}{\ecrbase}
	\begin{minipage}{62mm}
\begin{lstlisting}
class Graph {|\featuremodule{Base}|
  //@ invariant edges != null;
  Collection<Edge> edges;
  /*@ requires e != null;
    @ ensures edges.contains(e); @*/
  void add(Edge e) {
    edges.add(e);
  }
[...] }
\end{lstlisting}
	\end{minipage}
\end{lrbox}

\newsavebox{\ecrmaxoc}
\begin{lrbox}{\ecrmaxoc}
	\begin{minipage}{62mm}
\begin{lstlisting}
class Graph {|\featuremodule{MaxEdges}|
  Integer maxEdges = 10;




  void add(Edge e) {
    if (countEdges() < maxEdges)
		  original(e);
  }
[...] }
\end{lstlisting}
	\end{minipage}
\end{lrbox}

\newsavebox{\ecrmaxco}
\begin{lrbox}{\ecrmaxco}
	\begin{minipage}{62mm}
\begin{lstlisting}
class Graph {|\featuremodule{MaxEdges}|
  Integer maxEdges = 10;
  /*@ requires e != null
	  @ |~| && maxEdges != null;
    @ ensures \old(edges.size()) < maxEdges
		@ |~| ==> edges.contains(e); @*/
  void add(Edge e) {
    if (countEdges() < maxEdges)
		  original(e);
  }
[...] }
\end{lstlisting}
	\end{minipage}
\end{lrbox}

\newsavebox{\ecrmax}
\begin{lrbox}{\ecrmax}
	\begin{minipage}{62mm}
\begin{lstlisting}
class Graph {|\featuremodule{MaxEdges}|
  Integer maxEdges = 10;
  /*@ requires original
	  @ |~| && maxEdges != null;
    @ ensures \old(edges.size()) < maxEdges
		@ |~| ==> original; @*/
  void add(Edge e) {
    if (countEdges() < maxEdges)
		  original(e);
  }
[...] }
\end{lstlisting}
	\end{minipage}
\end{lrbox}

%%%%%%%%%% variability encoding %%%%%%%%%%

\newsavebox{\variables}
\begin{lrbox}{\variables}
	\begin{minipage}{50mm}
\begin{lstlisting}
boolean directed = false;
boolean weighted = false;
boolean optimalConnection = false;
\end{lstlisting}
	\end{minipage}
\end{lrbox}

\newsavebox{\variablesd}
\begin{lrbox}{\variablesd}
	\begin{minipage}{50mm}
\begin{lstlisting}
boolean directed = true;
boolean weighted = false;
boolean optimalConnection = false;
\end{lstlisting}
	\end{minipage}
\end{lrbox}

\newsavebox{\variablesw}
\begin{lrbox}{\variablesw}
	\begin{minipage}{50mm}
\begin{lstlisting}
boolean directed = false;
boolean weighted = true;
boolean optimalConnection = false;
\end{lstlisting}
	\end{minipage}
\end{lrbox}

\newsavebox{\variableso}
\begin{lrbox}{\variableso}
	\begin{minipage}{50mm}
\begin{lstlisting}
boolean directed = true;
boolean weighted = true;
boolean optimalConnection = true;
\end{lstlisting}
	\end{minipage}
\end{lrbox}

%%%%%%%%%% miniatures %%%%%%%%%%

\newsavebox{\contracts}
\begin{lrbox}{\contracts}
	\begin{minipage}{\textwidth}
	\newcommand{\callercolor}{white}
	\newcommand{\calleecolor}{red}
	\vspace{15mm}\hspace{54mm}\begin{tikzpicture}[remember picture,overlay,xscale=3.5,yscale=3,every node/.style={rounded corners,draw,thick},listing/.style={preaction={fill=white},inner xsep=-1pt,inner ysep=-2pt},assumption/.style={shorten <= 2pt,shorten >= 2pt,->,line width=1pt,bend right=60}]
		\node[listing,name=callee,preaction={fill=\calleecolor!25!white,path fading=north,fading angle=15},label=above left:callee] at (-1,0) {\usebox{\callee}};
		\node[listing,name=caller,preaction={fill=\callercolor!25!white,path fading=north,fading angle=15},label=above right:caller] at (1,0) {\usebox{\caller}};
		\path[assumption] (caller) edge node[listing,label=above:precondition] {\usebox{\blamepre}} (callee);
		\path[assumption] (callee) edge node[listing,label=above:postcondition] {\usebox{\blamepost}} (caller);
	\end{tikzpicture}
	\end{minipage}
\end{lrbox}

%\newsavebox{\productline}
%\begin{lrbox}{\productline}
%	\begin{minipage}{\textwidth}
%	\vspace{15mm}\hspace{54mm}\begin{tikzpicture}[remember picture,overlay,xscale=3,yscale=1.6,node/.style={rounded corners,draw,thick,fill=white},gen/.style={->,thick,shorten <= 2pt,shorten >= -2pt},valid/.style={->,gray,dashed,thick,shorten <= 2pt,shorten >= 2pt},conf/.style={node,pos=.7}]
%		\node[node,name=codebase] at (-.7,1.5) {\includegraphics[height=.15\textwidth]{../../../pics/featureide/cd-gpl}};
%		\uncover<1->{\node[name=pgraph] at (-1.5,-1.5) {\pgraph};\draw[gen] (codebase) -- node[conf,name=c] {\config{}} (pgraph);}
%		\uncover<1->{\node[name=pdgraph] at (-.5,-1.5) {\pdgraph};\draw[gen] (codebase) -- node[conf,name=cd] {\config{-d}} (pdgraph);}
%		\uncover<1->{\node[name=pwgraph] at (.5,-1.5) {\pwgraph};\draw[gen] (codebase) -- node[conf,name=cw] {\config{-w}} (pwgraph);}
%		\uncover<1->{\node[name=pograph] at (1.5,-1.5) {~~~\pograph~~~};\draw[gen] (codebase) -- node[conf,name=co] {\config{-dwo}} (pograph);}
%		\uncover<1->{\node[node,name=fm] at (.7,1.5) {\fm};\draw[valid] (fm) -- (c);\draw[valid] (fm) -- (cd);\draw[valid] (fm) -- (cw);\draw[valid] (fm) -- (co);}
%	\end{tikzpicture}
%	\end{minipage}
%\end{lrbox}
%
%\newsavebox{\foc}
%\begin{lrbox}{\foc}
%	\begin{minipage}{\textwidth}
%		\vspace{10mm}\hspace{54mm}\begin{tikzpicture}[remember picture,overlay,xscale=6,yscale=1.5,every node/.style={rounded corners,draw,thick,preaction={fill=white}},mechanism/.style={anchor=east},used/.style={shorten <= 2pt,shorten >= 2pt,->,line width=1pt}]
%			\node at (0,0) {\includegraphics[width=\textwidth]{../../../pics/featureide/cd-gpl-contracts.pdf}};
%			\node[name=ecr,mechanism,preaction={fill=blue!25!white,path fading=north,fading angle=15}] at (1,-2) {Explicit contract refinement};
%			\uncover<1->{\node[name=pc,mechanism,preaction={fill=orange!25!white,path fading=north,fading angle=15}] at (1,0) {Plain contracting};
%			\node[name=bs,mechanism,preaction={fill=green!25!white,path fading=north,fading angle=15}] at (1,-1) {Behavioral subtyping};
%			\path[used] (pc) edge (.2,.7);
%			\path[used] (pc) edge (.7,.8);
%			\path[used] (bs) edge (.1,.1);
%			\path[used] (bs) edge (-.5,.9);
%			\path[used] (bs) edge (-.4,-.5);
%			\path[used] (ecr) edge (-.3,-1.2);
%			\path[used] (ecr) edge (-.35,-.3);}
%		\end{tikzpicture}
%	\end{minipage}
%\end{lrbox}
%
%\newsavebox{\proofcomp}
%\begin{lrbox}{\proofcomp}
%	\begin{minipage}{\textwidth}
%		\vspace{0mm}\hspace{54mm}\begin{tikzpicture}[remember picture,overlay,xscale=3,yscale=.8,node/.style={rounded corners,draw,thick,fill=white},gen/.style={->,thick,shorten <= 2pt,shorten >= -2pt},neg/.style={->,thick,shorten <= -2pt,shorten >= 2pt},do/.style={gen,shorten >= 2pt,dashed},label/.style={preaction={fill=white,opacity=.66}}]
%			\node[node,name=codebase] at (-.7,1.5) {\includegraphics[height=.15\textwidth]{../../../pics/featureide/cd-gpl}};
%			\node[label] at (-.7,1.5) {+ partial proofs};
%			\node[name=pgraph] at (-1.5,-1.5) {\pgraph};\draw[gen] (codebase) -- (pgraph);
%			\node[name=pdgraph] at (-.5,-1.5) {\pdgraph};\draw[gen] (codebase) -- (pdgraph);
%			\node[name=pwgraph] at (.5,-1.5) {\pwgraph};\draw[gen] (codebase) -- (pwgraph);
%			\node[name=pograph] at (1.5,-1.5) {\pograph};\draw[gen] (codebase) -- (pograph);
%			\node[label] at (-1.5,-1.5) {+ proofs};
%			\node[label] at (-.5,-1.5) {+ proofs};
%			\node[label] at (.5,-1.5) {+ proofs};
%			\node[label] at (1.5,-1.5) {+ proofs};
%			\node[node,name=pgraphx,align=center] at (-1.5,-4) {theorems\\+ proofs};\draw[neg] (pgraph) -- (pgraphx);
%			\node[node,name=pdgraphx,align=center] at (-.5,-4) {theorems\\+ proofs};\draw[neg] (pdgraph) -- (pdgraphx);
%			\node[node,name=pwgraphx,align=center] at (.5,-4) {theorems\\+ proofs};\draw[neg] (pwgraph) -- (pwgraphx);
%			\node[node,name=pographx,align=center] at (1.5,-4) {theorems\\+ proofs};\draw[neg] (pograph) -- (pographx);
%			\node[name=pgraphz] at (-1.25,-5) {\richtig};
%			\node[name=pdgraphz] at (-.25,-5) {\richtig};
%			\node[name=pwgraphz] at (.75,-5) {\falsch};
%			\node[name=pographz] at (1.75,-5) {\richtig};
%		\end{tikzpicture}
%	\end{minipage}
%\end{lrbox}
%
%\newsavebox{\varenc}
%\begin{lrbox}{\varenc}
%	\begin{minipage}{\textwidth}
%		\vspace{5mm}\hspace{54mm}\begin{tikzpicture}[remember picture,overlay,xscale=3,yscale=1.6,node/.style={rounded corners,draw,thick,fill=white},gen/.style={->,thick,shorten <= 2pt,shorten >= -2pt},conf/.style={node,pos=.5},listing/.style={node,inner xsep=-1pt,inner ysep=-2pt}]
%			\node[node,name=codebase] at (0,1.5) {\includegraphics[height=.15\textwidth]{../../../pics/featureide/cd-gpl}};
%			\node[name=simulator,label=below:metaproduct] at (-1.5,-1.5) {\pmeta};\draw[gen] (codebase) -- node[conf,name=fm] {\fm} (simulator);
%			\node[name=product,label=below:product] at (1.5,-1.5) {\pwgraph};\draw[gen] (codebase) -- node[conf,name=cw] {\config{-w}} (product);
%			\draw[->,green,dashed,thick,shorten <= 2pt,shorten >= 2pt] (simulator) -- node[listing,label=below:product simulation] {\usebox{\variablesw}} (product);
%		\end{tikzpicture}
%	\end{minipage}
%\end{lrbox}
%
