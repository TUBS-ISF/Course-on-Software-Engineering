\subsection{Architectural Patterns}
\begin{frame}{\insertsubsection\ \deutsch{Architekturmuster}}
	\slideArchitecturalPattern
\end{frame}

\subsection{Layered Architecture}
\begin{frame}{\insertsubsection\ \normalsize\deutsch{Schichtenarchitektur}}
	%\includegraphics[width=.3\textwidth]{schichten}
	\begin{fancycolumns}[animation=none]
		\begin{definition}{{Layered Architecture \mysource{\sommerville}}}
			\begin{itemize}
				\item \emph{Problem}: subsystems are hard to adapt and replace
				\item \emph{Idea}: decomposition into layers \deutsch{Schichten}
				\item layer provides services to layers above
				\item layer delegates subtasks to layers below
				\item strict layers: every layer can only access the next layer
				\item relaxed layers: every layer can access all layers below % TODO translation
				\item information hiding: layers hide implementation details behind interface
			\end{itemize}
		\end{definition}
		\nextcolumn
		\posthandout{\pic[page=24,width=\linewidth,trim=225 20 25 40,clip]{printedslides/2021wt/architecture}}
	\end{fancycolumns}
\end{frame}

\subsection{Client-Server Architecture}
\begin{frame}{\insertsubsection\ \normalsize\deutsch{2-Schichten-Architektur}}
	\begin{fancycolumns}[animation=none]
		\begin{definition}{Client-Server Architecture \mysource{\sommerville}}% TODO further literature needed?
			\begin{itemize}
				\item aka.\ 2-tier architecture
				\item \emph{Problem}: several clients need to access the same data
				\item \emph{Idea}: separation of application (client) and data management (server)
				\item clients initiate the communication with a server
				\item typical: multiple clients of the same kind
				\item optional: multiple clients of different kinds
			\end{itemize}
		\end{definition}
		\begin{example}{Example}
			a browser uses a URL to connect to a server in the world wide web and receives an HTML page
		\end{example}
		\nextcolumn
		\posthandout{\pic[page=25,width=\linewidth,trim=225 20 25 40,clip]{printedslides/2021wt/architecture}}
	\end{fancycolumns}
\end{frame}

\subsection{3-Tier Architecture}
\begin{frame}{\insertsubsection\ \normalsize\deutsch{3-Schichten-Architektur}}
	\begin{fancycolumns}[animation=none]
		\begin{definition}{3-Tier Architecture}
			\begin{itemize}
				\item \emph{Problem}: clients with same functionality but different presentation needed
				\item \emph{Idea}: separation of data presentation, application logic, and data management
				\item thin-client application: application logic on the server
				\item rich-client application: application logic in the client
			\end{itemize}
		\end{definition}% TODO add source
		\begin{example}{Rule of Thumb}
			If you can use the application offline, then it is most likely a rich-client application.
		\end{example}
		\nextcolumn
		\posthandout{\pic[page=26,width=\linewidth,trim=225 20 25 40,clip]{printedslides/2021wt/architecture}}
	\end{fancycolumns}
\end{frame}

\subsection{Peer-to-Peer Architecture}
\begin{frame}{\insertsubsection}
	\begin{fancycolumns}[animation=none]
		\begin{definition}{Peer-to-Peer Architecture}
			\begin{itemize}
				\item \emph{Problem}: high load on server and high risk of failure when transmitting all client data to the server
				\item \emph{Idea}: decentralized transmission of data
				\item peers connect to each other and transfer data directly
				\item peers take over client or server roles
				\item arbitrary, dynamic topology
			\end{itemize}
		\end{definition}% TODO add source
		\uncover<4->{
			\begin{example}{In Practice}
				often combined with a client-server architecture
			\end{example}
		}
		\vspace{5mm}
		\nextcolumn
		\only<2|handout:0>{\pic[width=\textwidth,trim=110 180 110 180,clip]{architecture/peer-to-peer-nolines}}%
		\only<3->{\pic[width=\textwidth,trim=110 180 110 180,clip]{architecture/peer-to-peer}}%
	\end{fancycolumns}
\end{frame}

\begin{frame}{\insertsubsection\ in Windows 10}
	\begin{fancycolumns}
		\centering\picDark[height=65mm]{architecture/peer-to-peer-windows10a}
		\nextcolumn
		\centering\picDark[height=65mm]{architecture/peer-to-peer-windows10b}
	\end{fancycolumns}
\end{frame}

% TODO move to dedicated lecture part in SE2
%\subsection{Model-View-Controller Architecture}
%\begin{frame}{\insertsubsection}
%	\begin{fancycolumns}[animation=none]
%		\begin{definition}{Model-View-Controller Architecture \mysource{\sommerville}}
%			\begin{itemize}
%				\item \emph{Context}: data is presented and manipulated over several views
%				\item \emph{Problem}: data inconsistent and new views hard to add
%				\item \emph{Idea}: separation into three components
%				\item model: stores the relevant data independent of their presentation
%				\item view: shows (a part of) the data independent of manipulations
%				\item controller: user interface for the manipulation of data
%			\end{itemize}
%		\end{definition}
%		\vspace{-1mm}
%		\uncover<2->{
%			\begin{example}{Example}
%				In a spreadsheet, data is presented in tables and diagrams. Changing values in a table leads to an update of affected diagrams and tables.
%			\end{example}
%		}
%		\nextcolumn
%		\uncover<3->{\posthandout{\pic[page=29,width=\linewidth,trim=225 20 20 40,clip]{printedslides/2021wt/architecture}}}
%	\end{fancycolumns}
%\end{frame}
% TODO selection update not well explained in the video: https://youtu.be/nfbsaGvQ7n4?t=2042
% not only to update the selection after a deletion, but also to find out which elements to delete when pressing the button

\subsection{Pipe-and-Filter Architecture}
\begin{frame}{\insertsubsection}
	\slidePipeAndFilter
\end{frame}

