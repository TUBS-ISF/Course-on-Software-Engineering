\subsection{Black-Box Testing}
\begin{frame}{\insertsubsection\ \mytitlesource{\ludewiglichter}}
	\begin{fancycolumns}
		\begin{note}{Motivation}
			\begin{itemize}
				\item source code not always available (e.g., outsourced components, obfuscated code)
				\item specific test cases derived from logical ones using arbitrary values
				\item specification not incorporated so far (only for expected results)
				\item invalid inputs not tested
				\item errors are not equally distributed
			\end{itemize}
		\end{note}
		\begin{definition}{Black-Box Testing \deutsch{Funktionstest}}
			\begin{itemize}
				\item test-case design based on specification
				\item source code and its inner structure is ignored (assumed as a black-box)
			\end{itemize}
		\end{definition}
		\nextcolumn
		\vspace{-5mm}
		\begin{definition}{1. Equivalence Class Testing}
			\begin{itemize}
				\item idea: classify inputs and outputs into equivalence classes
				\item assumption: equivalent test cases detect the same faults, one test case is sufficient
			\end{itemize}
		\end{definition}
		\begin{definition}{2. Boundary Testing}
			\begin{itemize}
				\item extension of equivalence class testing
				\item goal: use experience (e.g., off-by-one errors)
				\item for every equivalence class: consider smallest, typical, and largest value
			\end{itemize}
		\end{definition}
		\begin{example}{In Practice}
			\begin{itemize}
				\item often combinations of white-box and black-box testing
				\item more techniques with requirements or design
			\end{itemize}
		\end{example}
	\end{fancycolumns}
\end{frame}

\subsection{Equivalence Class Testing}
\begin{frame}{\insertsubsection}
	\begin{fancycolumns}[columns=3,widths={46,4,50},animation=none]
		\begin{exampletight}{JavaDoc Specification for Modulo Example}
			\pic[width=\linewidth]{testing/modulo-only-javadoc-small-nohint}
		\end{exampletight}
		\begin{example}{Equivalence Classes}
			\begin{itemize}
				\item input a: $a < 0$, $a \geq 0$
				\item input b: $b < 0$, $b \geq 0$
				\item output: $m = 0$, $m > 0$, exception
			\end{itemize}
		\end{example}
		\nextcolumn
		\nextcolumn
		\uncover<2->{
			\begin{example}{Test Cases}
				\begin{tabular}{c c c c}	% temp
					% TODO: fix M{c}{num-}
					%\begin{tabular}{M{c}{1-} M{c}{4-} M{c}{5-} M{c}{6-}}
					& TC1 & TC2 & TC3 \\
					\toprule
					$a < 0$ & X &  &  \\
					$a \geq 0$ &  & X & X \\
					\midrule
					$b < 0$ & X &  &  \\
					$b > 0$ &  & X &  \\
					$b = 0$ &  &  & X \\
					\midrule
					$m = 0$ & X &  &  \\
					$m > 0$ &  & X &  \\
					exception &  &  & X \\
					\midrule
					\uncover<3->{input a & -3 & 1 & 2} \\
					\uncover<3->{input b & -3 & 2 & 0} \\
					\uncover<3->{expected output & 0 & 1 & exception} \\
					\midrule
					\uncover<4->{result & 0\correct & 1\correct & timeout\wrong} \\
					\bottomrule
				\end{tabular}
			\end{example}
		}
	\end{fancycolumns}
\end{frame}

\subsection{Boundary Testing}
\begin{frame}{\insertsubsection}
	\begin{fancycolumns}[columns=3,widths={4,92}]
		\nextcolumn
		\begin{example}{Test Cases}
			\begin{tabular}{c c c c c c c c}	% temp
				% TODO: fix M{c}{num-}
				%\begin{tabular}{M{c}{1-} M{c}{1-} M{c}{1-} M{c}{1-} M{c}{2-} M{c}{3-} M{c}{4-} M{c}{5-}}
				& TC1 & TC2 & TC3 & TC4 & TC5 & TC6 & TC7 \\
				\toprule
				$a < 0$ & X &  &  & min & max &  &  \\
				$a \geq 0$ &  & X & X &  &  & min & max \\
				\midrule
				$b < 0$ & X &  &  & max &  & min &  \\
				$b > 0$ &  & X &  &  & max &  & min \\
				$b = 0$ &  &  & X &  &  &  &  \\
				\midrule
				$m = 0$ & X &  &  & X &  & X & X \\
				$m > 0$ &  & min &  &  & max &  &  \\
				exception &  &  & X &  &  &  &  \\
				\midrule
				\uncover<1->{input a & -3 & 1 & 2 & minInt & -1 & 0 & maxInt }\\
				\uncover<1->{input b & -3 & 2 & 0 & -1 & maxInt & minInt & 1 }\\
				\uncover<1->{expected output & 0 & 1 & exception & 0 & maxInt-1 & 0 & 0 }\\
				\midrule
				\uncover<2->{result & 0\correct & 1\correct & timeout\wrong & 0\correct & maxInt-1\correct & timeout\wrong & 1\wrong }\\
				\bottomrule
			\end{tabular}
		\end{example}
		\nextcolumn
	\end{fancycolumns}
\end{frame}
%// a smallest negative, b largest negative, smallest result
%assertEquals(0,ModuloExample.modulo(Integer.MIN_VALUE,-1));
%// a largest negative, b largest positive, largest result
%assertEquals(Integer.MAX_VALUE-1,ModuloExample.modulo(-1,Integer.MAX_VALUE));
%// a smallest positive, b smallest negative, smallest result
%//assertEquals(0,ModuloExample.modulo(0,Integer.MIN_VALUE)); // leads to endless loop for every a
%assertEquals(0,ModuloExample.modulo(0,Integer.MIN_VALUE+1));
%// a largest positive, b smallest positive, smallest result
%assertEquals(0,ModuloExample.modulo(Integer.MAX_VALUE,1));

\subsection{Detected Faults in Modulo Example}
\begin{frame}
	\begin{fancycolumns}[animation=none]
		\uncover<1->{
			\begin{exampletight}{Detected Faults in Modulo Example}
				\pic[width=\linewidth]{testing/modulo}
			\end{exampletight}
			\vspace{-1mm}
			\begin{example}{}
				\wrong~~~~TC3: infinite loop for $b = 0$,\\ missing exception compared to JavaDoc
			\end{example}
		}
		\vspace{-1mm}
		\uncover<2->{
			\begin{example}{}
				\wrong~~~~TC6: $b$ remains negative as\\ \texttt{-Integer.MIN\_VALUE == Integer.MIN\_VALUE}\\ and the loop condition is fulfilled for any integer
			\end{example}
		}
		\vspace{-1mm}
		\uncover<3->{
			\begin{example}{}
				\wrong~~~~TC7: indicates that $m > b$ in the loop condition should be fixed to $m \geq b$
			\end{example}
		}
		\nextcolumn
		\uncover<4->{
			\begin{exampletight}{Improved Modulo Example}
				\pic[width=\linewidth]{testing/modulo-fixed}
			\end{exampletight}
		}
		\vspace{-1mm}
		\uncover<5->{
			\begin{example}{Passes All Test Cases}
				\vspace{3mm}
				\correct~~~~~\correct~~~~~\correct~~~~~\correct~~~~~\correct~~~~~\correct~~~~~\correct~~~~~\correct~~~~~\correct~~~~~\correct~~~~~ 
			\end{example}
		}
	\end{fancycolumns}
\end{frame}

\subsection{Reasons for Positive Test Cases}
\begin{frame}{\insertsubsection}
	\begin{fancycolumns}
		\begin{note}{Reasons for Positive Test Cases}
			\begin{itemize}
				\item actual fault
				\uncover<2->{\item wrong test case (input and expected results do not match)}
				\uncover<3->{\item interaction with other programs/libraries
					\item fault in the compiler
					\item fault in the operating system / device drivers
					\item fault in the hardware or hardware defect
					\item not enough memory
					\item does not halt (cf.\ undecidability of the halting problem)
					\item bitflip due to cosmic ray
					\item \ldots}
			\end{itemize}
		\end{note}
		%		\mynote{Reasons for Negative Test Cases}{
			%			\begin{itemize}
				%				\setlength\itemsep{.5em}
				%				\item program correct
				%				\item see above
				%			\end{itemize}
			%		}
	\end{fancycolumns}
\end{frame}

\begin{frame}
	\begin{fancycolumns}[height=8.5cm]
		\pic[width=\linewidth,trim=0 425 0 75,clip]{people/edsger-dijkstra}
		\vspace{-7mm}
		
		\begin{note}{Edsger W. Dijkstra \mysource{\goodliffe}}
			\mycite{If debugging is the process of removing software bugs, then programming must be the process of putting them in.}
		\end{note}
		% 1930-2002, ACM Turing Award winner
		\nextcolumn
		\pic[width=\linewidth,trim=250 0 0 0,clip]{people/brian-kernighan}
		\vspace{-7mm}
		
		\begin{note}{Brian Kernighan (1978) \mysource{\elementsofprogrammingstyle}}
			\mycite{Everyone knows that debugging is twice as hard as writing a program in the first place. So if you're as clever as you can be when you write it, how will you ever debug it?}
		\end{note}
		% known for books on C and Go
	\end{fancycolumns}
\end{frame}
