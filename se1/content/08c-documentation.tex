\subsection{Natural Language Specification}
\begin{frame}{\insertsubsection}
	\begin{fancycolumns}[widths={55}]
		\begin{definition}{}
			\begin{itemize}
				\item used since 1950s
				\item since programmer is not necessarily the user
			\end{itemize}
		\end{definition}
		\begin{example}{Style Guidelines}
			\begin{itemize}
				\item one or two short sentences of natural language
				\item one message per sentence
				\item use active voice (e.g., \mycite{the system \ldots})
				\item consistent use of language (i.e., avoid synonyms)
				\item use shall for mandatory and should for desirable requirements
				\item use text highlighting
				\item avoid jargon, abbreviations, acronyms
				\item provide a rationale
			\end{itemize}
		\end{example}
		\nextcolumn
		\begin{note}{Pros}
			\begin{itemize}
				\item expressive
				\item intuitive
				\item universal
			\end{itemize}
		\end{note}
		\begin{note}{Cons}
			\begin{itemize}
				\item vague \deutsch{vage}
				\item ambiguous \deutsch{mehrdeutig}
				\item interpretation depends on reader's background
			\end{itemize}
		\end{note}
	\end{fancycolumns}
\end{frame}

\subsection{Structured Specifications}
\begin{frame}{\insertsubsection}
	\begin{fancycolumns}[widths={55}]
		\begin{definition}{}
			\begin{itemize}
				\item use of templates rather than free-form text
				\item distinguishes between function, description, input (origin), output (destination), pre- and postconditions, side effects, rationale, dependencies to other requirements, \ldots
				\item or any subset thereof
				\item user stories are typically structured specifications
			\end{itemize}
		\end{definition}
		\begin{example}{Example}
			\setlength\tabcolsep{1mm}
			\begin{tabularx}{\textwidth}{rX}
				Function & Exchange of IDs\\
				Description & Two devices send their IDs to each other.\\
				Precondition & Devices have been within a distance of 2m for at least 15 minutes.\\
				Postcondition & IDs are stored in the local database.\\
				Rationale & Contact tracing requires to temporarily identify the device of contacts.
			\end{tabularx}
		\end{example}
		\nextcolumn
		\begin{note}{Pros}
			\begin{itemize}
				\item {\color{gray} expressive}
				\item {\color{gray} intuitive}
				\item information easier to find
				\item less likely to forget certain information (e.g., rationale)
			\end{itemize}
		\end{note}
		\begin{note}{Cons}
			\begin{itemize}
				\item {\color{gray} vague}
				\item {\color{gray} ambiguous}
				\item {\color{gray} interpretation depends on reader's background}
				\item same structure can be too restrictive
				\item multiple structures can be confusing
			\end{itemize}
		\end{note}
	\end{fancycolumns}
\end{frame}

\subsection{Use Case Diagrams}
\begin{frame}[label=usecaseslide]{\insertsubsection\ \normalsize(since 1993)}
	\begin{fancycolumns}[animation=none]
		\begin{definition}{Use Case Diagram \deutsch{Anwendungsfalldiagramm}}
			Use case diagrams are a means to capture the requirements of systems, i.e., what systems are supposed to do. The key concepts specified in this diagram are actors, use cases, and subjects:%\\~
			\begin{itemize}
				\item Each \emph{subject} represents a system under consideration to which the use case applies. \deutsch{System}
				\item Each users and any other system that may interact with a subject is represented as an \emph{actor}. \deutsch{Akteur}
				\item A \emph{use case} is a specification of behavior in terms of verb and noun. \deutsch{Anwendungsfall}
			\end{itemize}
			\mysource{adapted from \umlspec}
		\end{definition}
		\nextcolumn
		%\posthandout{\pic[width=\linewidth,trim=0 50 25 40,clip]{blackboard/blackboard_use_case1_23_24}}
		%\posthandout{\pic[page=44,width=\linewidth,trim={.5\width} {.1\height} {.02\width} {.2\height},clip]{se09-requirements-print}}
		\posthandout{\pic[page=53,width=\linewidth,trim=225 20 15 40,clip]{printedslides/2021wt/requirements}}
	\end{fancycolumns}
\end{frame}

\begin{frame}{Example}
	\posthandout{\centering\pic[page=51,width=.9\linewidth,trim=25 20 25 0,clip]{printedslides/2021wt/requirements}}
\end{frame}

\subsection{Include and Extend Relationships}
\begin{frame}[label=includeandextendslide]{\insertsubsection}
	\begin{fancycolumns}[animation=none]
		\begin{definition}{Include Relationship}
			\textbf{Motivation}: make common parts of multiple use cases explicit%\\[1mm]
			
			\textbf{Relationship}: a \emph{base use case} may define an include relationship to an \emph{included use case}%\\[1mm]
			
			\textbf{Meaning}: included use case is always executed when the base use case is
		\end{definition}
		\begin{definition}{Extend Relationship}
			\textbf{Motivation}: make explicit that some use cases only happen under certain circumstances%\\[1mm]
			
			\textbf{Relationship}: an \emph{extend use case} may define an extend relationship to a \emph{base use case}%\\[1mm]
			
			\textbf{Meaning}: when the base use case is executed, the extend use case may or may not be executed
		\end{definition}
		\nextcolumn
		\posthandout{\pic[page=53,width=\linewidth,trim=225 20 15 50,clip]{printedslides/2021wt/requirements}}
		%\posthandout{\pic[page=46,width=\linewidth,trim={.5\width} {.2\height} {.02\width} {.2\height},clip]{se09-requirements-print}}
	\end{fancycolumns}
\end{frame}

% TODO add rules for use case diagrams

% TODO discuss advantages/disadvantages of use case diagrams

% TODO combination of textual and visual requirements

\pictureframe{
	\pic[width=\paperwidth]{emotions/walking-on-water}
}{
	\vspace{65mm}
	\begin{note}{Edward V. Berard (1993) \mysource{\href{https://en.wikiquote.org/wiki/Edward_V._Berard}{wikiquote.org}}}
		\mycite{Walking on water and developing software from a specification are easy if both are frozen.}
	\end{note}
}
