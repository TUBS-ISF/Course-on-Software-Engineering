% introduced: introduction
% reused: summary
\newcommand{\slideSEvsProgramming}{
	\begin{fancycolumns}[animation=none]
		\centering
		\pic[height=60mm]{misc/ulm-muenster}
	\nextcolumn
		\centering\pic[height=60mm]{misc/tarp-tent-cropped}
	\end{fancycolumns}
	\begin{note}{}
		\centering what is needed besides programming will be motivated and shown throughout this course
	\end{note}
}

% introduced: implementation
% reused: summary, se2-compilation
\newcommand{\slideProgrammingLanguagesToday}{
	\begin{fancycolumns}
		\begin{example}{Today\mysource{\jonesbestpractice\ + \handbuch}}
			\begin{itemize}
				\item 2002: C\# by Microsoft
				\item 2009: Go by Google
				\item 2010: Rust by Mozilla Research
				\item 2014: Swift by Apple
				\item thousands of programming languages
				\item very few programming languages used for more than 10 years
				\item languages used for more than 25 years: Ada, C, C++, COBOL, Java, Objective C, PL/I, SQL, Visual Basic, \ldots
			\end{itemize}
		\end{example}
		\nextcolumn
		\begin{note}{Many Languages\mysource{\jonesbestpractice}}
			\begin{itemize}
				\item good: fit for every use case
				\item bad: developer training for new and dead languages, expensive tool support
			\end{itemize}
		\end{note}
	\end{fancycolumns}
}

% introduced: implementation
% reused: se2-compilation
\newcommand{\slideTiobeDiagram}{
	\only<1,6->{\picDark[width=\linewidth]{history/tiobe}}%
	\only<2|handout:0>{\picDark[width=\linewidth]{history/tiobe-python}}%
	\only<3|handout:0>{\picDark[width=\linewidth]{history/tiobe-cpp}}%
	\only<4|handout:0>{\picDark[width=\linewidth]{history/tiobe-c}}%
	\only<5|handout:0>{\picDark[width=\linewidth]{history/tiobe-java}}%
}
\newcommand{\slideTiobeTable}{
	\centering\picDark[width=.9\linewidth]{history/tiobe-longterm}
}

% introduced: implementation
% reused: designpatterns
\newcommand{\slideWindowsCalculator}{
	\begin{frame}{Calc on Windows 10\ \mytitlesource{\href{https://youtu.be/LKDbQfzzGJo?t=1544}{youtube.de}}}
		\begin{fancycolumns}
			\centering\picDark[width=.66\linewidth]{failures/win10-calc-scientific}
			\nextcolumn
			\centering\picDark[width=.66\linewidth]{failures/win10-calc-standard}
		\end{fancycolumns}
	\end{frame}
}

% introduced: testing
% reused: summary
\newcommand{\slideMindmapQualityAssurance}[7]{
	\vspace{-12mm}\hfill
	\begin{tikzpicture}
		\path[small mindmap,
		every node/.style={concept,font=\scriptsize},
		emph/.style={font=\bfseries\scriptsize},
		concept color=foreground!20!background,
		level 1/.append style={level distance=25mm,sibling angle=360/6},
		level 2/.append style={level distance=20mm,sibling angle=360/6},
		level 3/.append style={level distance=20mm,sibling angle=360/8},
		]
		node {Quality Assurance \deutsch{Qualitätssicherung}}
		[counterclockwise from=210]
		child[#1] { node {constructive} 
			[clockwise from=225]
			child[concept color=blue!20!background,#4] { node {Coding Guidelines} }
		}
		child[#2] { node {analytical} 
			[counterclockwise from=240]
			child[concept color=green!20!background,#5] { node {analysis}
				[counterclockwise from=180]
				child { node {Compilation} }
				child { node {Code Reviews} }
			}
			child[concept color=red!20!background,#6] { node {execution}
				[counterclockwise from=315]
				child { node {White-Box Testing} }
				child { node {Black-Box Testing} }
			}
		}
		child[#3] { node {organizational} 
			[clockwise from=-45]
			child[concept color=orange!20!background,#7] { node {Software Project Management} }
		}
		;
	\end{tikzpicture}
}

% Modification of Mindmap above (\slideMindmapQualityAssurance) for SE2
\newcommand{\slideMindmapQualityAssuranceMod}[7]{
	\vspace{-12mm}\hfill
	\begin{tikzpicture}
		\path[small mindmap,
		every node/.style={concept,font=\scriptsize},
		emph/.style={font=\bfseries\scriptsize},
		concept color=foreground!20!background,
		level 1/.append style={level distance=25mm,sibling angle=360/6},
		level 2/.append style={level distance=20mm,sibling angle=360/6},
		level 3/.append style={level distance=20mm,sibling angle=360/8},
		]
		node {Quality Assurance \deutsch{Qualitätssicherung}}
		[counterclockwise from=210]
		child[#1] { node {constructive} 
			[clockwise from=225]
			child[concept color=blue!20!background,#4] { node {Coding Guidelines} }
		}
		child[#2] { node {analytical} 
			[counterclockwise from=240]
			child[concept color=green!20!background,#5] { node {analysis}
				[counterclockwise from=180]
				child { node {Compilation} }
				child { node {Code Reviews} }
				child { node {Static Analysis} } %added for SE2
			}
			child[concept color=red!20!background,#6] { node {execution}
				[counterclockwise from=315]
				child { node {White-Box Testing} }
				child { node {Black-Box Testing} }
			}
		}
		child[#3] { node {organizational} 
			[clockwise from=-45]
			child[concept color=orange!20!background,#7] { node {Software Project Management} }
		}
		;
	\end{tikzpicture}
}

% introduced: changes
% reused: summary
\newcommand{\slideEvolutionAndMaintenance}{
	\begin{fancycolumns}[t]
		\begin{note}{Evolution}
			\begin{itemize}
				\item new or removed functionality
				\item typically larger changes
				\item often foreseen changes
				\item results in upgrades, service packs, or cumulative updates
			\end{itemize}
		\end{note}
		\begin{example}{Minor Release}
			new minor version: 2.3.1 $\Rightarrow$ 2.4.0
		\end{example}
		\begin{example}{Major Release}
			new major version: 2.3.1 $\Rightarrow$ 3.0.0
		\end{example}
		\nextcolumn
		\begin{note}{Maintenance}
			\begin{itemize}
				\item mostly corrections
				\item typically smaller changes
				\item often unforeseen changes
				\item results in patches and hot fixes
			\end{itemize}
		\end{note}
		\begin{example}{Patch Release}
			new patch version: 2.3.1 $\Rightarrow$ 2.3.2
		\end{example}
		\begin{example}{Linux Kernel Releases\mysource{\href{https://en.wikipedia.org/wiki/Linux_kernel_version_history}{wikipedia.org}}}
			\begin{itemize}
				%\item 6.13.0 -- 6.13.11
				%\item 6.12.0 -- 6.12.23
				%\item 6.11.0 (Sep 24) -- 6.11.11 (Dec 24)
				%\item 6.10.0 (Jul 24) -- 6.10.14 (Oct 24)
				\item 6.9.0 (May 24) -- 6.9.10 (Jul 24)
				\item 6.8.0 (Mar 24) -- 6.8.12 (May 24)
				\item 6.7.0 (Jan 24) -- 6.7.12 (Apr 24)
				\item 6.6.0 (Oct 23) -- 6.6.87+ (Dec 26) LTS
			\end{itemize}
		\end{example}
	\end{fancycolumns}
}

% introduced: versioncontrol
% reused: ?
\newcommand{\slideBranchingAndMerging}{
	\begin{fancycolumns}[animation=none,b,widths={58}]
		\only<1|handout:0>{\trunkbranch{}{}}%
		\only<2|handout:0>{\trunkbranch{,draw=green}{}}%
		\only<3-|handout:1>{\trunkbranch{,draw=green}{,draw=orange}}%
		\only<1->{
			\begin{note}{}
				\begin{itemize}
					\item simultaneous, independent development
					\item option to merge in the future
					\item main development on branch \texttt{main} (formerly \texttt{master})
					\item parallel developments on branches for new features, bug fixes, multiple versions, \ldots
					\item avoiding name \texttt{master} for branches:\\\texttt{git config --global init.defaultBranch main}
				\end{itemize}
			\end{note}
		}
		\nextcolumn
		\uncover<4-|handout:1>{\picDark[width=\linewidth]{versioncontrol/master-main-github}}%
	\end{fancycolumns}
}

% introduced: management
% reused: summary
\newcommand{\forwardpass}[1]{#1}
\newcommand{\backwardpass}[1]{#1}
\newcommand{\buffer}[1]{#1}
\newcommand{\networknode}[7]{
	\begin{tikzpicture}[every node/.style={draw=foreground,fill=background,anchor=base,minimum height=5mm,text width=7.5mm,align=center},inner xsep=0mm,line width=.5pt,node distance=-.5pt]
		\node[text width=22.5mm] (task) {#1};
		\node (d) [above=of task] {#3};
		\node (es) [left=of d] {\forwardpass{#2}};
		\node (ef) [right=of d] {\forwardpass{#4}};
		\node (b) [below=of task] {\buffer{#6}};
		\node (ls) [left=of b] {\backwardpass{#5}};
		\node (lf) [right=of b] {\backwardpass{#7}};
	\end{tikzpicture}
}
\newcommand{\slideGanttAndNetwork}{
	\begin{fancycolumns}
		\begin{note}{Gantt Chart}
			\begin{itemize}
				\item very common technique
				\item many tools available
				\item great visualization of timing and progress
			\end{itemize}
		\end{note}
		\begin{exampletight}{}
			\centering\picDark[width=.85\linewidth]{management/gantt-chart}
		\end{exampletight}
		\nextcolumn
		\begin{note}{Network Diagram \deutsch{Netzplan}}
			\begin{itemize}
				\item clear visualization of dependencies
				\item explicitly includes buffer times\\(cf.\ metra potential method)
			\end{itemize}
		\end{note}
		\begin{exampletight}{}
			\centering
			\begin{tikzpicture}[xscale=3,yscale=-2.1,inner sep=0,edge/.style={->,>={Stealth[round]},semithick}]
				\node[fill=red!30!background,font=\tiny] (legend) at (1,0) {\networknode{\normalsize Task}{earliest start}{duration}{earliest finish}{latest start}{buffer}{latest finish}};
				\node (intro) at (1,1) {\networknode{Introduction}{0}{1}{1}{10}{10}{11}};
				\node (background) at (0,0) {\networknode{Background}{0}{3}{3}{0}{0}{3}};
				\node (concept) at (0,1) {\networknode{Concept}{3}{4}{7}{3}{0}{7}};
				\draw[edge] (background) to (concept);
			\end{tikzpicture}
		\end{exampletight}
	\end{fancycolumns}
}

% introduced: management
% reused: process
\newcommand{\definitionuserstory}[1]{
	\begin{definition}{#1User Story \mysource{\sommerville}}
		\begin{itemize}
			\item a scenario of use that might be experienced by a system user
			\item aka.\ \emph{story card} as user stories are sometimes written on physical cards
			\item user stories are typically prioritized by the customer
			\item subset of all user stories is chosen for the next release
		\end{itemize}
	\end{definition}
}

% introduced: process
% reused: 
\newcommand{\slideWaterfallModel}{
	\begin{fancycolumns}
		\begin{definition}{Waterfall Model}
			\begin{itemize}
				\item first process model, motivated by practice
				\item by \href{https://scholar.google.de/scholar?cluster=8624196209257442707}{Winston W. Royce 1970}
				\item each development phase ends by the approval of one or more documents (document-driven process model)
				\item phases do not overlap
				\item numerous variants with varying number of phases: 5--7
				\item here: simplified variant by Sommerville
			\end{itemize}
		\end{definition}
		\nextcolumn
		\diagramWaterfallModel
	\end{fancycolumns}
}
\newcommand{\diagramWaterfallModel}{
	\begin{tikzpicture}[yscale=-1.2,xscale=.75,phase/.style={draw=foreground,thick,rounded rectangle,fill=blue!10!background,align=center,text width=21mm},label/.style={auto,bend right,align=left},to/.style={->,>={Stealth[round]},thick}]
			\node[phase,visible on=<6->] (1) at (0,0) {Requirements Analysis};
			\node[phase,visible on=<7->] (2) at (1,1) {System and Software Design};
			\node[phase,visible on=<2->] (3) at (2,2) {Implementation and Unit Testing};
			\node[phase,visible on=<4->] (4) at (3,3) {Integration and System Testing};
			\node[phase,visible on=<3->] (5) at (4,4) {Operation and Maintenance};
			\draw[to,visible on=<8->,blue] (1) -| node[label] {System\\Specification} (2.30);
			\draw[to,visible on=<8->,blue] (2) -| node[label] {Design\\Specification} (3.30);
			\draw[to,visible on=<5->,blue] (3) -| node[label] {Program\\Documentation} (4.30);
			\draw[to,visible on=<5->,blue] (4) -| node[label] {Test\\Documentation} (5.30);
			\draw[to,visible on=<9|handout:0>,red] (5) -| (1.210);
			\draw[to,visible on=<9|handout:0>,red] (5) -| (2.210);
			\draw[to,visible on=<9|handout:0>,red] (5) -| (3.210);
			\draw[to,visible on=<9|handout:0>,red] (5) -| (4.210);
			\draw[to,visible on=<10->,orange] (2) -| (1.210);
			\draw[to,visible on=<10->,orange] (3) -| (2.210);
			\draw[to,visible on=<10->,orange] (4) -| (3.210);
			\draw[to,visible on=<10->,orange] (5) -| (4.210);
	\end{tikzpicture}
}

% introduced: process
% reused: maintenance
\newcommand{\diagramVModel}{
	\begin{tikzpicture}[yscale=-1.45,xscale=.45,phase/.style={draw=foreground,thick,rounded rectangle,fill=blue!10!background,align=center,text width=17mm},label/.style={midway,anchor=west},to/.style={->,>={Stealth[round]},thick}]
			\node[phase,visible on=<2->] (1) at (0,0) {Requirements Elicitation};
			\node[phase,visible on=<3->] (2) at (1,1) {System Modeling};
			\node[phase,visible on=<4->] (3) at (2,2) {Architecture Design};
			\node[phase,visible on=<5->] (4) at (3,3) {Software Design};
			\node[phase,visible on=<6->] (5) at (6,4) {Implemen-tation};
			\node[phase,visible on=<7->] (6) at (9,3) {Unit\\Testing};
			\node[phase,visible on=<8->] (7) at (10,2) {Integration Testing};
			\node[phase,visible on=<9->] (8) at (11,1) {System Testing};
			\node[phase,visible on=<10->] (9) at (12,0) {Acceptance Testing};
			\draw[to,visible on=<3->,blue] (1) to node[label] {Requirements Specification} (2);
			\draw[to,visible on=<4->,blue] (2) to node[label] {System Specification} (3);
			\draw[to,visible on=<5->,blue] (3) to node[label] {Architecture Spec.} (4);
			\draw[to,visible on=<6->,blue] (4) to node[label] {Design Spec.} (5.165);
			\draw[to,visible on=<7->,blue] (5.15) to node[label] {Component} (6);
			\draw[to,visible on=<8->,blue] (6) to node[label] {Components} (7);
			\draw[to,visible on=<9->,blue] (7) to node[label] {System} (8);
			\draw[to,visible on=<10->,blue] (8) to node[label] {System} (9);
			\draw[to,visible on=<11->,orange,dashed] (1) to node[auto] {test cases} (9);
			\draw[to,visible on=<12->,orange,dashed] (2) to (8);
			\draw[to,visible on=<12->,orange,dashed] (3) to (7);
			\draw[to,visible on=<12->,orange,dashed] (4) to (6);
	\end{tikzpicture}
}
\newcommand{\diagramForwardEngineering}{
	\begin{tikzpicture}[yscale=-1.45,xscale=.45,phase/.style={draw=foreground,thick,rounded rectangle,fill=blue!10!background,align=center,text width=17mm},label/.style={midway,anchor=west},to/.style={->,>={Stealth[round]},thick}]
			\node[phase,visible on={<1-2,6>}] (1) at (0,0) {Requirements Elicitation};
			\node[phase,visible on={<1-3>}] (2) at (1,1) {System Modeling};
			\node[phase,visible on={<1,3-4>}] (3) at (2,2) {Architecture Design};
			\node[phase,visible on={<1,4-5>}] (4) at (3,3) {Software Design};
			\node[phase,visible on={<1,5-6>}] (5) at (4,4) {Implemen-tation};
			\draw[to,visible on={<1-2>},blue] (1) to node[label] {Requirements Specification} (2);
			\draw[to,visible on={<1,3>},blue] (2) to node[label] {System Specification} (3);
			\draw[to,visible on={<1,4>},blue] (3) to node[label] {Architecture Specification} (4);
			\draw[to,visible on={<1,5>},blue] (4) to node[label] {Design Specification} (5);
			\draw[to,visible on={<6|handout:0>},blue] (1) to node[label] {Minor Change Request} (5);
	\end{tikzpicture}
}
\newcommand{\diagramReverseEngineering}{
	\begin{tikzpicture}[yscale=-1.45,xscale=.8,phase/.style={draw=foreground,thick,rounded rectangle,fill=blue!10!background,align=center,text width=17mm},label/.style={midway,anchor=west},to/.style={->,>={Stealth[round]},thick}]
			\node[phase,visible on={<1,2,6>}] (1) at (0,0) {Requirements Elicitation};
			\node[phase,visible on={<1,4,6>}] (2) at (1,1) {System Modeling};
			\node[phase,visible on={<1,5->}] (3) at (2,2) {Architecture Design};
			\node[phase,visible on={<1,3,6>}] (4) at (3,3) {Software Design};
			\node[phase,visible on={<1->}] (5) at (4,4) {Implemen-tation};
			\draw[to,visible on={<1,6>},orange] (2) to (1);
			\draw[to,visible on={<1,6>},orange] (3) to (2);
			\draw[to,visible on={<1,6>},orange] (4) to (3);
			\draw[to,visible on={<2|handout:0>},orange] (5) to (1);
			\draw[to,visible on={<4|handout:0>},orange] (5) to (2);
			\draw[to,visible on={<5|handout:0>},orange] (5) to (3);
			\draw[to,visible on={<1,3,6>},orange] (5) to (4);
	\end{tikzpicture}
}

% introduced: process
% reused: process
\newcommand{\diagramScrum}{
	\pic[width=\linewidth]{process/scrum}
}

% introduced: requirements
% reused: summary
\newcommand{\slideMindmapNonFunctionalRequirements}{
	{\newcommand{\requirements}{requirements}%
		\begin{tikzpicture}
			\path[%grow cyclic,
			small mindmap,
			%every node/.style=concept,
			concept color=green!20!background,
			%	text width=20mm,align=flush center,
			%	level 1/.append style={level distance=27mm,sibling angle=360/3},
			]
			node[concept] {non-functional \requirements}
			[counterclockwise from=195]
			child[concept color=blue!20!background] { node[concept] {product \requirements} 
				[counterclockwise from=75]
				child[visible on=<2->] { node[concept] {usability \requirements} }
				child[visible on=<3->] { node[concept] {efficiency \requirements} 
					[counterclockwise from=165]
					child { node[concept] {performance \requirements} }
					child { node[concept] {space \requirements} }
				}
				child[visible on=<4->] { node[concept] {dependability \requirements} }
				child[visible on=<5->] { node[concept] {security \requirements} }
			}
			child[concept color=red!20!background] { node[concept] {organizational \requirements} 
				[counterclockwise from=210]
				child[visible on=<6->] { node[concept] {environmental \requirements} }
				child[visible on=<7->] { node[concept] {operational \requirements} }
				child[visible on=<8->] { node[concept] {development \requirements} }
			}
			child[concept color=orange!20!background] { node[concept] {external \requirements} 
				[counterclockwise from=285]
				child[visible on=<9->] { node[concept] {regulatory \requirements} }
				child[visible on=<10->] { node[concept] {ethical \requirements} }
				child[visible on=<11->] { node[concept] {legislative \requirements}
					[counterclockwise from=-15]
					child { node[concept] {accounting \requirements} }
					child { node[concept] {safety/security \requirements} }
				}
			}
			;
	\end{tikzpicture}}
}

% introduced: modeling
% reused: architecture, design, summary
% 
% legend: gray (will not be taught), bold (taught in this lecture), normal font (will be taught)
% 02-requirements: use case diagrams
% 03-modeling: UML, activity diagrams, state machine diagrams
% 04-architecture: component diagrams
% 05-design: class diagrams, sequence diagrams
%
% parameters: use case, activity, state machine, component, class, sequence, structure, behavior, interactions
\tikzset{notTaughtUMLDiagrams/.style={text=gray}}
\newcommand{\slideMindmapUMLdiagrams}[9]{
	{\centering
		\begin{tikzpicture}
			\path[small mindmap,
			every node/.style={concept,font=\footnotesize},
			emph/.style={font=\bfseries\footnotesize},
			blue/.style={emph,text=blue},
			red/.style={emph,text=red},
			concept color=green!20!background,
			level 1/.append style={level distance=35mm,sibling angle=360/3},
			level 2/.append style={level distance=20mm,sibling angle=360/8},
			level 3/.append style={level distance=20mm,sibling angle=360/8},
			]
			node {UML Diagrams}
			[clockwise from=150]
			child[concept color=blue!20!background,#7] { node {Structure Diagrams} 
				[counterclockwise from=15]
				child { node[#4] {Component Diagram} }
				child { node[#5] {Class Diagram} }
				child[notTaughtUMLDiagrams] { node {Profile Diagram} }
				child[notTaughtUMLDiagrams] { node {Composite Structure Diagram} }
				child[notTaughtUMLDiagrams] { node {Deployment Diagram} }
				child[notTaughtUMLDiagrams] { node {Object Diagram} }
				child[notTaughtUMLDiagrams] { node {Package Diagram} }
			}
			child[concept color=red!20!background,#8] { node {Behavior Diagrams} 
				[clockwise from=135]
				child { node[#1] {Use Case Diagram} }
				child { node[#2] {Activity Diagram} }
				child { node[#3] {State Machine Diagram} }
				child[concept color=orange!20!background,#9] { node {Interaction Diagrams} 
					[clockwise from=30]
					child { node[#6] {Sequence Diagram} }
					child[notTaughtUMLDiagrams] { node {Communi-cation Diagram} }
					child[notTaughtUMLDiagrams] { node {Interaction Overview Diagram} }
					child[notTaughtUMLDiagrams] { node {Timing Diagram} }
				}
			}
			;
	\end{tikzpicture}}
}

% introduced: architecture
% reused: summary
\newcommand{\slideArchitecturalPattern}{
	\begin{fancycolumns}
		\centering
		\begin{definition}{Architectural Pattern\mysource{\sommerville}}
			\mycite{Architectural patterns capture the essence of an architecture that has been used in different software systems. [...] Architectural patterns are a means of reusing knowledge about generic system architectures.}
		\end{definition}
		\nextcolumn
		\begin{note}{Goals}
			\partofpage{35}{
				\pic[width=\textwidth]{architecture/database}
			}
			\partofpage{60}{
				\begin{itemize}
					\item preserve knowledge of software architects
					\item reuse of established architectures
					\item enable efficient communication
				\end{itemize}
			}
		\end{note}
	\end{fancycolumns}
}

% introduced: architecture
% reused: 
\newcommand{\slidePipeAndFilter}{
	\begin{fancycolumns}[animation=none]
		\begin{definition}{Pipe-and-Filter Architecture \mysource{\sommerville}}
			\begin{itemize}
				\item \emph{Problem}: data is processed in numerous processing steps, which are prone to change
				\item \emph{Idea}: modularization of each processing step into a component
				\item filter components process a stream of data continously
				\item pipes transfer data unchanged from filter output to filter input
			\end{itemize}
		\end{definition}
		\uncover<2->{
			\begin{example}{Pipe Operator in UNIX}
				\mycite{\texttt{ls -al | grep '2020' | grep -v 'Nov' | more}} searches files in a folder from the year 2020 except those from November and delivers the results in pages.
			\end{example}
		}
		\nextcolumn%
		\uncover<3->{\hfill\picDark[width=.8\textwidth]{architecture/pipe-and-filter}}%
	\end{fancycolumns}
}

% introduced: se1-reuse
% reused: se2-designpatterns
\newcommand{\slideGangOfFour}{
	\begin{frame}{\insertsubsection\ \mytitlesource{\gof}}
	\begin{fancycolumns}[animation=none]
		\centering\pic[width=.66\linewidth]{books/gof}
		\nextcolumn
		\centering\pic[height=27mm,trim=185 0 184 0,clip]{people/erich-gamma}
		\pic[height=27mm]{people/richard-helm}
		
		\pic[height=27mm,trim=0 0 5 0,clip]{people/ralph-johnson}
		\pic[height=27mm,trim=0 0 160 0,clip]{people/john-vlissides}
	\end{fancycolumns}
\end{frame}
}

% introduced: se1-reuse
% reused: se2-designpatterns
\newcommand{\slideDesignPatterns}{
	\begin{frame}{\insertsubsection\ \mytitlesource{\gofen}}
		\begin{fancycolumns}
			\begin{note}{Motivation}
				\mycite{Designing object-oriented software is hard, and designing \emph{reusable} object-oriented software is even harder. [...]  It takes a long time for novices to learn what good object-oriented design is all about. Experienced designers evidently know something inexperienced ones don't. What is it?}
			\end{note}
			\nextcolumn
			\begin{definition}{Design Patterns \deutsch{Entwurfsmuster}}
				\setlength\tabcolsep{1mm}
				\begin{tabularx}{\textwidth}{rX}				
					pattern name & for communication and high-level abstraction\\
					problem & when to apply the pattern\\
					solution & template on how to arrange classes and objects\\
					consequences & trade-offs of applying the pattern
				\end{tabularx}
			\end{definition}
			\begin{note}{Kinds of Patterns}
				\mycite{Creational patterns \deutsch{Erzeugungsmuster} concern the process of object creation. Structural patterns \deutsch{Strukturmuster} deal with the composition of classes or objects. Behavioral patterns \deutsch{Verhaltensmuster} characterize the ways in which classes or objects interact and distribute responsibility.}
			\end{note}
		\end{fancycolumns}
	\end{frame}
}

% introduced: se1-reuse
% reused: se2-designpatterns
%
% legend: gray (will not be taught), bold (taught in this lecture), normal font (will be taught)
% 
% parameters: object adapter, composite, decorator, singleton, abstract factory, observer, structural, creational, behavioral
\tikzset{notTaughtDesignPatterns/.style={text=gray}}
\newcommand{\slideMindmapDesignPatterns}[9]{
	{\centering
		\begin{tikzpicture}
			\path[small mindmap,
			every node/.style={concept,font=\scriptsize},
			emph/.style={font=\bfseries\scriptsize},
			blue/.style={emph,text=blue},
			red/.style={emph,text=red},
			concept color=green!20!background,
			level 1/.append style={level distance=35mm,sibling angle=65},
			level 2/.append style={level distance=20mm,sibling angle=360/12},
			level 3/.append style={level distance=20mm,sibling angle=360/8},
			]
			node {Design Patterns}
			[clockwise from=155]
			child[concept color=blue!20!background,#7] { node {Structural Patterns} 
				[clockwise from=215]
				child[#1] { node {Object Adapter} }
				child[notTaughtDesignPatterns] { node {Class Adapter} }
				child[notTaughtDesignPatterns] { node {Bridge} }
				child[#2] { node {Composite} }
				child[#3] { node {Decorator} }
				child[notTaughtDesignPatterns] { node {Facade} }
				child[notTaughtDesignPatterns] { node {Flyweight} }
				child[notTaughtDesignPatterns] { node {Proxy} }
			}
			child[concept color=red!20!background,#8] { node {Creational Patterns} 
				[clockwise from=150]
				child[#5] { node {Abstract Factory} }
				child[notTaughtDesignPatterns] { node {Builder} }
				child[notTaughtDesignPatterns] { node {Factory Method} }
				child[notTaughtDesignPatterns] { node {Prototype} }
				child[#4] { node {Singleton} }
			}
			child[concept color=orange!20!background,#9] { node {Behavioral Patterns} 
				[clockwise from=175]
				child[notTaughtDesignPatterns] { node {Chain of Responsibility} }
				child[notTaughtDesignPatterns] { node {Command} }
				child[notTaughtDesignPatterns] { node {Interpreter} }
				child[notTaughtDesignPatterns] { node {Iterator} }
				child[notTaughtDesignPatterns] { node {Mediator} }
				child[notTaughtDesignPatterns] { node {Memento} }
				child[#6] { node {Observer} }
				child[notTaughtDesignPatterns] { node {State} }
				child[notTaughtDesignPatterns] { node {Strategy} }
				child[notTaughtDesignPatterns] { node {Template Method} }
				child[#6] { node {Visitor} } % TODO check how to parameterize that for SE1
			}
			;
	\end{tikzpicture}}
}

% design patterns as uml class diagrams modeled with diagrams.net
\newcommand{\objectadapter}[1]{\picWhite[#1,page=1,trim=110 240 495 200,clip]{design/diagramsnet-designpatterns}}
\newcommand{\composite}[1]{\picWhite[#1,page=2,trim=200 300 405 115,clip]{design/diagramsnet-designpatterns}}
\newcommand{\compositeexample}[1]{\picWhite[#1,page=3,trim=200 100 400 100,clip]{design/diagramsnet-designpatterns}}
\newcommand{\decorator}[1]{\picWhite[#1,page=4,trim=185 205 355 135,clip]{design/diagramsnet-designpatterns}}
\newcommand{\decoratorexample}[1]{\picWhite[#1,page=5,trim=185 260 340 135,clip]{design/diagramsnet-designpatterns}}

\newcommand{\singleton}[1]{\picWhite[#1,page=6,trim=55 425 665 85,clip]{design/diagramsnet-designpatterns}}
\newcommand{\singletonexample}[1]{\picWhite[#1,page=7,trim=55 410 665 85,clip]{design/diagramsnet-designpatterns}}
\newcommand{\abstractfactory}[1]{\picWhite[#1,page=8,trim=200 355 280 65,clip]{design/diagramsnet-designpatterns}}

\newcommand{\observer}[1]{\picWhite[#1,page=9,trim=285 350 320 55,clip]{design/diagramsnet-designpatterns}}
\newcommand{\visitor}[1]{\picWhite[#1,page=10,trim=80 205 420 85,clip]{design/diagramsnet-designpatterns-new}}

% TODO integrate separate file for the calculator example
\newcommand{\profcalculator}[1]{\picWhite[#1,trim=190 185 130 105,clip]{design/diagramsnet-designpatterns}}
\newcommand{\profcalculatorcollaboration}[1]{\picWhite[#1,trim=70 0 20 90,clip]{design/diagramsnet-designpatterns-collaborations}}

% introduced: se2-02-evolution
% reused: se2-05-compilation
\newcommand{\slideSimplicityOverPerformance}{
	\begin{fancycolumns}
		\pic[width=\linewidth,trim=0 50 0 30,clip]{people/wes-dyer}
		\vspace{-7mm}
		
		\begin{note}{Wes Dyer \mysource{\href{https://twitter.com/CodeWisdom/status/801456038008520705}{twitter.com}}}
			\mycite{Make it correct, make it clear, make it concise, make it fast. In that order.}
		\end{note}
		% professor? lecturer?
		\nextcolumn
		\pic[width=\linewidth,trim=100 0 0 25,clip]{people/joshua-bloch}
		\vspace{-7mm}
		
		\begin{note}{Joshua J.\ Bloch (born 1961) \mysource{\href{https://twitter.com/codewisdom/status/876878006278598656}{twitter.com}}}
			\mycite{The cleaner and nicer the program, the faster it's going to run. And if it doesn't, it'll be easy to make it fast.}
		\end{note}
		% known for design of Java platform
	\end{fancycolumns}
}

\newcommand{\figcompilerarchfull}{
	\begin{tikzpicture}[every node/.style = {align = center, text depth = 0pt}]       
		\node[fill = background] (A) {Source Code\\as String};
		
		\node[below = 20pt of A, minimum width=2.25cm, draw, fill=background] (An1) {Scanner};
		\node[below = of An1, minimum width=2.25cm, draw, fill =background] (An2) {Parser};
		\node[below = of An2, minimum width=2.25cm, draw, fill = background] (An3) {Name and Type\\Analysis};        
		
		\begin{scope}[on background layer]
			\fill[fill = blue!25!background] ([xshift = -5pt, yshift = 5pt]An1.north west) rectangle node (H1) {} ([xshift = 5pt, yshift = -5pt]An3.south east);
		\end{scope}
		
		\node [left = 1.4cm of H1.center, color = blue, rotate=90, anchor = south, fill=background, inner sep = 0pt] {Analysis};
		
		\path[-latex]
		(A) edge ([yshift = 5pt]An1.north)
		(An1) edge node[align = right, left, font = \small] {Token\\Stream} (An2)           
		(An2) edge node[align = right, left, font = \small] {Syntax\\Tree} (An3);
		
		\node[right = 1.5cm of A, fill = background] (tgt) {Target Code\\as String};
		
		
		\node[below = 20pt of tgt, minimum width=2.25cm, draw, fill=background] (syn2) {Code Generator};
		\node[below = of syn2, minimum width=2.25cm, draw, fill=background] (syn1) {Translator};
		
		\node[below = of syn1, fill = background] (ast) {Abstract\\SyntaxTree};
		
		\path[]
		([yshift = -5pt]An3.south) edge ([yshift = -10pt] An3.south)
		([yshift = -10pt] An3.south) edge[-latex, to path={-| (\tikztotarget)}] (ast);  
		
		\path[-latex]
		(ast) edge ([yshift = -5pt] syn1.south)
		([xshift = -15pt]syn1.north) edge node[align = left, right, font = \small] (il) {Intermediate\\Language} ([xshift = -15pt]syn2.south)
		([yshift = 5pt]syn2.north) edge (tgt);
		;
		
		\begin{scope}[on background layer]
			\fill[fill = red!25!background] ([xshift = -5pt, yshift = 5pt]syn2.north west) rectangle node (H2) {} ([xshift = 5pt, yshift = -5pt]syn1.south east);
		\end{scope}
		
		\node [left = 1.4cm of H2.center, color = red, rotate=90, anchor = south, inner sep=0pt, fill=background] {Synthesis};
		
		
		\node[right = 20pt of il.east, fill = background] (op2) {Attribution \&\\Optimization};
		
		\draw (intersection cs:
		first line={(op2.center) -- ([yshift = 3cm] op2.center)},
		second line={(tgt.center) -- ([xshift = 3cm] tgt.center)}) node[fill = background] (op1) {Peep Hole\\Optimization}; 
		
		\draw (intersection cs:
		first line={(op2.center) -- ([yshift = 3cm] op2.center)},
		second line={(ast.center) -- ([xshift = 3cm] ast.center)}) node[fill = background] (op3) {Attribution \&\\Optimization}; 
		
		\path[]         
		([xshift = -18pt, yshift = 3pt]op1.west) edge[-latex] ([xshift = -3pt, yshift = 3pt]op1.west)   
		([xshift = -18pt, yshift = -3pt]op1.west) edge[latex-] ([xshift = -3pt, yshift = -3pt]op1.west) 
		
		([xshift = -18pt, yshift = 3pt]op2.west) edge[-latex] ([xshift = -3pt, yshift = 3pt]op2.west)   
		([xshift = -18pt, yshift = -3pt]op2.west) edge[latex-] ([xshift = -3pt, yshift = -3pt]op2.west) 
		
		([xshift = -18pt, yshift = 3pt]op3.west) edge[-latex] ([xshift = -3pt, yshift = 3pt]op3.west)   
		([xshift = -18pt, yshift = -3pt]op3.west) edge[latex-] ([xshift = -3pt, yshift = -3pt]op3.west)     
		;       
	\end{tikzpicture}       
}

\newcommand{\figcompilerarchnoop}{
	\begin{tikzpicture}[every node/.style = {align = center, text depth = 0pt}]       
		\node[fill = background] (A) {Source Code\\as String};
		
		\node[below = 20pt of A, minimum width=2.25cm, draw, fill=background] (An1) {Scanner};
		\node[below = of An1, minimum width=2.25cm, draw, fill = background] (An2) {Parser};
		\node[below = of An2, minimum width=2.25cm, draw, fill = background] (An3) {Name and Type\\Analysis};        
		
		\begin{scope}[on background layer]
			\fill[fill = blue!25!background] ([xshift = -5pt, yshift = 5pt]An1.north west) rectangle node (H1) {} ([xshift = 5pt, yshift = -5pt]An3.south east);
		\end{scope}
		
		\node [left = 1.4cm of H1.center, color = blue, rotate=90, anchor = south, fill=background, inner sep = 0pt] {Analysis};
		
		\path[-latex]
		(A) edge ([yshift = 5pt]An1.north)
		(An1) edge node[align = right, left, font = \small] {Token\\Stream} (An2)           
		(An2) edge node[align = right, left, font = \small] {Syntax\\Tree} (An3);
		
		\node[right = 1.5cm of A, fill = background] (tgt) {Target Code\\as String};
		
		
		\node[below = 20pt of tgt, minimum width=2.25cm, draw, fill=background] (syn2) {Code Generator};
		\node[below = of syn2, minimum width=2.25cm, draw, fill=background] (syn1) {Translator};
		
		\node[below = of syn1, fill = background] (ast) {Abstract\\SyntaxTree};
		
		\path[]
		([yshift = -5pt]An3.south) edge ([yshift = -10pt] An3.south)
		([yshift = -10pt] An3.south) edge[-latex, to path={-| (\tikztotarget)}] (ast);  
		
		\path[-latex]
		(ast) edge ([yshift = -5pt] syn1.south)
		([xshift = -15pt]syn1.north) edge node[align = left, right, font = \small] (il) {Intermediate\\Language} ([xshift = -15pt]syn2.south)
		([yshift = 5pt]syn2.north) edge (tgt);
		;
		
		\begin{scope}[on background layer]
			\fill[fill = red!25!background] ([xshift = -5pt, yshift = 5pt]syn2.north west) rectangle node (H2) {} ([xshift = 5pt, yshift = -5pt]syn1.south east);
		\end{scope}
		
		\node [left = 1.4cm of H2.center, color = red, rotate=90, anchor = south, inner sep=0pt, fill=background] {Synthesis};
	\end{tikzpicture}       
}

\newcommand{\figcompilerarchbase}{
	\begin{tikzpicture}[every node/.style = {align = center, text depth = 0pt}]       
		\node[fill = background] (A) {Source Code\\as String};
		
		\node[below = 20pt of A, minimum width=2.25cm] (An1) {\phantom{Scanner}};
		\node[below = of An1, minimum width=2.25cm] (An2) {\phantom{Parser}};
		\node[below = of An2, minimum width=2.25cm] (An3) {\phantom{Name and Type}\\\phantom{Analysis}};        
		
		\begin{scope}[on background layer]
			\fill[fill = blue!25!background] ([xshift = -5pt, yshift = 5pt]An1.north west) rectangle node (H1) {} ([xshift = 5pt, yshift = -5pt]An3.south east);
		\end{scope}
		
		\node [left = 1.4cm of H1.center, color = blue, rotate=90, anchor = south, fill=background, inner sep = 0pt] {Analysis};
		
		\path[-latex]
		(A) edge ([yshift = 5pt]An1.north);
		
		\node[right = 1.5cm of A, fill = background] (tgt) {Target Code\\as String};
		
		\node[below = 20pt of tgt, minimum width=2.25cm] (syn2) {\phantom{Code Generator}};
		\node[below = of syn2, minimum width=2.25cm] (syn1) {\phantom{Translator}};
		
		\node[below = of syn1, fill = background] (ast) {Abstract\\SyntaxTree};
		
		\path[]
		([yshift = -5pt]An3.south) edge ([yshift = -10pt] An3.south)
		([yshift = -10pt] An3.south) edge[-latex, to path={-| (\tikztotarget)}] (ast);  
		
		\path[-latex]
		(ast) edge ([yshift = -5pt] syn1.south)
		([yshift = 5pt]syn2.north) edge (tgt)
		;
		
		\begin{scope}[on background layer]
			\fill[fill = red!25!background] ([xshift = -5pt, yshift = 5pt]syn2.north west) rectangle node (H2) {} ([xshift = 5pt, yshift = -5pt]syn1.south east);
		\end{scope}
		
		\node [left = 1.4cm of H2.center, color = red, rotate=90, anchor = south, inner sep=0pt, fill=background] {Synthesis};
	\end{tikzpicture}    
}

\newcommand{\figCompiler}{
	\begin{tikzpicture}[every node/.style = {align = center, fill = background}, node distance = 1cm and 0.5cm]       
		\node[] (A) {Source Code};
		
		\node[right = of A, fill = black, text=white, minimum width=2cm] (Compiler) {Compiler};
		
		\node[right = of Compiler] (tgt) {Target Code};
		
		\path[-latex]
		(A) edge (Compiler)
		(Compiler) edge (tgt);
	\end{tikzpicture}       
}

\newcommand{\figInterpreter}{
	\begin{tikzpicture}[every node/.style = {align = center, fill = background}]     
		\node[] (A) {Source Code};
		
		\node[below = of A] (B) {Input Data};
		
		\path (A.center) -- node (H) {} (B.center);
		
		\node[right = of H, fill = black, text=white, minimum width=2cm] (Compiler) {Interpreter};
		
		\node[right = 0.5cm of Compiler] (tgt) {Output Data};
		
		\path[-latex]
		(A) edge[out = -90, in = 180] ([yshift = 2pt]Compiler.west)
		(B) edge[out = 90, in = 180] ([yshift = -2pt]Compiler.west)
		(Compiler) edge (tgt);
	\end{tikzpicture}       
}

\newcommand{\figCompilerWithAnalysisOut}{
	\begin{tikzpicture}[every node/.style = {align = center, fill = background}] 
		
		\node[fill = black, text=white, minimum width=2cm] (Compiler) {Compiler};
		
		\node[left = of Compiler] (H1) {};		 
		\node[right = of Compiler] (H2) {};
		
		\node[above = 10pt of H1.center] (A) {Source Code};
		
		\node[above = 10pt of H2.center] (X) {Target Code};
		\node[below = 10pt of H2.center] (Y) {Analysis Result};
		
		\path[-latex, looseness = 0.7]
		(A) edge[out = -90, in = 180] ([yshift = 2pt]Compiler.west)
		([yshift = 2pt]Compiler.east) edge[out = 0, in = -90] (X)
		([yshift = -2pt]Compiler.east) edge[out = 0, in = 90] (Y);
	\end{tikzpicture}       
}

\newcommand{\figInterpreterWithAnalysisOut}{
	\begin{tikzpicture}[every node/.style = {align = center, fill = background}]      
		
		\node[fill = black, text=white, minimum width=2cm] (Compiler) {Interpreter};
		
		\node[left = of Compiler] (H1) {};		 
		\node[right = of Compiler] (H2) {};
		
		\node[above = 10pt of H1.center] (A) {Source Code};
		\node[below = 10pt of H1.center] (B) {Input Data};
		
		\node[above = 10pt of H2.center] (X) {Output Data};
		\node[below = 10pt of H2.center] (Y) {Analysis Result};
		
		\path[-latex, looseness = 0.7]
		(A) edge[out = -90, in = 180] ([yshift = 2pt]Compiler.west)
		(B) edge[out = 90, in = 180] ([yshift = -2pt]Compiler.west)
		([yshift = 2pt]Compiler.east) edge[out = 0, in = -90] (X)
		([yshift = -2pt]Compiler.east) edge[out = 0, in = 90] (Y);
	\end{tikzpicture}       
}

\newcommand{\drawFileSymbol}[3][]{
	
	\fill[draw, fill=background,#1] (#2,#3) -- (#2+0.75,#3) -- (#2+1, #3-0.25) -- (#2+1, #3-1.5) -- (#2, #3-1.5) -- (#2,#3) (#2+0.75, #3) -- (#2+0.75, #3-0.25) -- (#2+1, #3-0.25) (#2+0.2, #3-0.6) -- (#2+0.8, #3-0.6) (#2+0.2, #3-0.8) -- (#2+0.8, #3-0.8) (#2+0.2, #3-1) -- (#2+0.8, #3-1);
}

\newcommand{\figJIT}{
	\begin{tikzpicture}[every node/.style = {text depth = 0pt}]
		
		\node[draw = blue, anchor = north west, fill = blue!30!background] (F) {
			\begin{tikzpicture}
				\node[minimum width=5cm, minimum height = 1cm, fill = gray!10!background, draw = gray!30!background] (HW) {};
				\node[below left = 0pt and 0pt of HW.north east, font = \small, align=right] {Real Machine /\\Hardware};
				
				\node[right = 5pt of HW.west, fill = background, draw, inner sep=4pt, rounded corners=4pt] (MC) {Machine Code};
				
				\node[above = 20pt of MC.north, align=center, fill = background, draw, inner sep=4pt] (Y) {JIT Translator};	
				
				\node[above = 14pt of Y.north, align=center, fill = background, draw, rounded corners = 4pt, inner sep=4pt] (X) {Java Byte\\Code};
				
				\path[-latex] 
				(X) edge (Y)
				(Y) edge ([yshift = 7pt]MC.north)
				;
			\end{tikzpicture}
		};
		
		\node[below left = 0pt and 0pt of F.north east, font = \small, align=right] {Java Virtual Machine};
		
		\node[above = 15pt of F.north, align=center, fill = background, draw, rounded corners = 4pt, inner sep=4pt, xshift = 1.5cm] (ID) {Input Data};
		
		
		\node[above = 13pt of F, xshift = -1.5cm] (D) {
			\begin{tikzpicture}
				\drawFileSymbol[scale = 0.5]{0.9}{0.45}
				\drawFileSymbol[scale = 0.5]{0.6}{0.3}
				\drawFileSymbol[scale = 0.5]{0.3}{0.15}
				\drawFileSymbol[scale = 0.5]{0}{0}
			\end{tikzpicture}
		};
		
		\node[yshift = -8pt, align=center, fill = background, draw, rounded corners = 4pt, inner sep=4pt] at (D) (E) {Class Files};
		
		
		\node[above = 15pt of D, align = center, draw, inner sep=4pt, fill = background] (C) {Analyzer \&\\Translator};
		\node[left = 2pt of C, rotate=90, anchor=south, fill = background] {\texttt{javac}};
		
		\node[above = 38pt of ID] (A) {
			\begin{tikzpicture}
				\drawFileSymbol[scale = 0.5]{0.9}{0.45}
				\drawFileSymbol[scale = 0.5]{0.6}{0.3}
				\drawFileSymbol[scale = 0.5]{0.3}{0.15}
				\drawFileSymbol[scale = 0.5]{0}{0}
			\end{tikzpicture}
		};
		\node[yshift = -10pt, align=center, fill = background, draw, rounded corners = 4pt, inner sep=4pt] at (A) (B) {Java Source\\Code Unit};
		
		\node[left = 15pt of F.west, align=center, fill = background, draw, rounded corners = 4pt, inner sep=4pt] (OD) {Output Data};
		
		\path[-latex]
		(B) edge (C)
		(C) edge (D)
		(E) edge ([xshift = -1.5cm]F.north)
		(ID) edge ([xshift = 1.5cm]F.north)
		(F) edge (OD)
		;
		
	\end{tikzpicture}
}

% introduced: se2-examples
% reused: se2-dynamic
\newcommand{\slideCrowdStrike}{
	\begin{fancycolumns}[b]
		\begin{definition}{\insertsubsection\mysource{\href{https://en.wikipedia.org/wiki/CrowdStrike}{wikipedia.org}, \href{https://www.bbc.com/news/articles/c23k4yyjxp3o}{bbc.com}}}
			\begin{itemize}
				\item CrowdStrike: American cybersecurity company
				\item Falcon: Windows kernel extension to prevent against threats
				\item about 10 patches released each days
				\item update on July 19, 2024 affected 8.5 million devices running Windows 10 or 11
				\item blue screen of death
				\item no restart possible without manual intervention
				\item largest IT outage
				\item about 5,000 (5 \%) flight cancelled
				\item Delta Airlines lost 500 million dollar
				\item overall financial damage of 10 billion dollar
				\item similar problem affected Linux devices already in April 2024
			\end{itemize}
		\end{definition}
		\nextcolumn
		\small\vspace{-20mm}
		\picDark[width=\linewidth]{failures/crowdstrike}
		\vspace{-4mm}
		\begin{note}{Opening in US Congress\mysource{\href{https://www.bbc.com/news/articles/c23k4yyjxp3o}{bbc.com}}}
			\mycite{A global IT outage that impacts \emph{every sector of the economy} is a catastrophe that we would expect to see in a movie [\ldots] we would expect to be carefully executed by a malicious and sophisticated nation-state actor [but] the largest IT outage in history was due to a \emph{mistake}.}
		\end{note}
		\begin{note}{Apology in US Congress\mysource{\href{https://www.bbc.com/news/articles/c23k4yyjxp3o}{bbc.com}}}
			\mycite{On behalf of everyone at CrowdStrike, I want to apologize. We are \emph{deeply sorry} and we are determined to prevent this from ever happening again. [\ldots] I want to underscore that this was \emph{not a cyber attack}.}
		\end{note}
	\end{fancycolumns}
}

% introduced: se1-testing
% reused: se2-dynamic
\newcommand{\slideSoftwareQuality}{
	\begin{fancycolumns}
		\begin{definition}{Quality \mysource{\ludewiglichter}}
			Quality is the entirety of properties and characteristics of a product or process that indicate adequacy with respect to given requirements.
		\end{definition}
		\begin{definition}{Quality Assurance \mysource{\ludewiglichter}}
			Quality assurance \deutsch{Qualitätssicherung} are all activities with the goal to improve the quality.
		\end{definition}
		\nextcolumn
		\begin{note}{Expectations on Quality \mysource{\sommerville}}
			\mycite{Because of their previous experiences with buggy, unreliable software, users sometimes have low expectations of software quality. They are not surprised when their software fails. When a new system is installed, users may tolerate failures because the benefits of use outweigh the costs of failure recovery. However, as a software product becomes more established, users expect it to become more reliable.\uncover<3->{ [...] If a software product or app is very cheap, users may be willing to tolerate a lower level of reliability.}\uncover<4->{ [...] Customers may be willing to accept the software, irrespective of problems, because the costs of not using the software are greater than the costs of working around the problems.}}
		\end{note}
	\end{fancycolumns}
}


% introduced: se1-testing
% reused: se2-dynamic
\newcommand{\slideSoftwareTesting}{
\begin{fancycolumns}[height=1cm,animation=none]
	\uncover<1->{
		\begin{definition}{Software Testing \mysource{\sommerville}}
			\mycite{Testing is intended to show that a program does what it is intended to do and to discover program defects before it is put into use.}
		\end{definition}
	}
	\uncover<2->{
		\begin{definition}{Validation Testing \mysource{\sommerville}}
			\mycite{Demonstrate to the developer and the customer that the software meets its requirements.}
		\end{definition}
	}
	\uncover<3->{
		\begin{definition}{Defect Testing \mysource{\sommerville}}
			\mycite{Find inputs or input sequences where the behavior of the software is incorrect, undesirable, or does not conform to its specification.}
		\end{definition}
	}
	\nextcolumn
	%\vspace{-12mm}
	\uncover<4->{
		\begin{note}{V\&V \mysource{\seeconomics}}
			\mycite{\emph{Validation}: Are we building the right product?\\\emph{Verification}: Are we building the product right?}
		\end{note}	
	} 
	% TODO better visualization of V&V (see Inas slides). move to V model?
	\vspace{-.7mm}
	\uncover<5->{
		\begin{note}{Stages of Testing \mysource{\sommerville}}
			\begin{itemize}
				\setlength\itemsep{.1em}
				\item[1.] \mycite{\emph{Development testing}, where the system is tested during development to discover
					bugs and defects}
				\item[2.] \mycite{\emph{Release testing}, where a separate testing team tests a complete version of the
					system before it is released to users}
				\item[3.] \mycite{\emph{User testing}, where users or potential users of a system test the system in their
					own environment}
			\end{itemize}
		\end{note}
	}
	\vspace{-.7mm}
	\uncover<6->{
		\begin{note}{}
			\mycite{In \emph{manual testing}, a tester runs the program with some test data and
				compares the results to their expectations. [...] In \emph{automated testing}, the tests are encoded in a program that is run each time the system under development is to be tested.} \mysource{\sommerville}
		\end{note}
	}
\end{fancycolumns}
}

% introduced: se1-testing
% reused: se2-dynamic
\newcommand{\explTestCases}{
	\uncover<1->{
		\begin{definition}{Systematic Test \mysource{\ludewiglichter}}
			A systematic test is a test, in which
			\begin{itemize}
				\item[1.] the setup is defined,
				\item[2.] the inputs are chosen systematically,
				\item[3.] the results are documented and evaluated by criteria being defined prior to the test. 
			\end{itemize}
		\end{definition}
	}
	\uncover<2->{
		\begin{definition}{Test Case \mysource{\ludewiglichter}}
			In a test, a number of test cases are executed, whereas each test case consists \emph{input values} for a single execution and \emph{expected outputs}. An \emph{exhaustive test} refers a test in which the test cases exercise all the possible inputs.
		\end{definition}
	}
}

% introduced: se1-testing
% reused: se2-dynamic
\newcommand{\figTestDesign}{
\\\vspace{-7mm}
\hfill
\begin{tikzpicture}
	\path[small mindmap,
	every node/.style={concept,font=\scriptsize},
	emph/.style={font=\bfseries\scriptsize},
	concept color=foreground!20!background]
	node {Test-Case Design}
	child[concept color=blue!20!background,visible on={<2->}] { node {experience-based} }
	child[concept color=green!20!background,visible on={<3->}] { node {structure-based} 
		child { node {white-box testing} }
	}
	child[concept color=orange!20!background,visible on={<4->}] { node {specification-based} 
		child { node {black-box testing} }
	}
	;
\end{tikzpicture}
}

% introduced: se1-development
% reused: se2-dynamic
\newcommand{\slideStagesTesting}{
\begin{fancycolumns}[animation=none]
	\begin{definition}{1. Unit Testing \deutsch{Komponententest}}
		\uncover<2->{\begin{itemize}
				\item each component is tested independently
				\item unit may stand for a component or smaller entities (package, class, method)
				\item tests created by the developers
				\item automation is common (e.g., JUnit)
		\end{itemize}}
	\end{definition}
	\begin{definition}{2. Integration Testing \deutsch{Integrationstest}}
		\uncover<3->{\begin{itemize}
				\item some components are integrated (e.g., into subsystems) and tested together
				\item detects inconsistencies in interfaces and communication between components
				\item top-down vs bottom-up integration
		\end{itemize}}
	\end{definition}
	\nextcolumn
	\begin{definition}{3. System Testing \deutsch{Systemtest}}
		\uncover<4->{\begin{itemize}
				\item all components are integrated to the complete system
				\item detects further inconsistencies and unanticipated interactions
				\item system is tested against system requirements
		\end{itemize}}
	\end{definition}
	\begin{definition}{4. Acceptance Testing \deutsch{Abnahmetest}}
		\uncover<5->{\begin{itemize}
				\item final stage in the testing process before accepted for operational use
				\item system is tested against user requirements and with real data
				\item performed by (potential) customer
		\end{itemize}}
	\end{definition}
\end{fancycolumns}
}