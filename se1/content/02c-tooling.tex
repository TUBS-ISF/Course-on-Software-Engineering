\subsection{Computer-Aided Software Engineering}
\begin{frame}{\insertsubsection}
	\begin{fancycolumns}[widths={55}]
		\begin{definition}{Terms \mysource{adapted from \ghezzi}}
			A \emph{tool} is an application that supports a particular activity. An \emph{environment} is a collection of related tools. Tools and environments aim at automating some of the activities that are involved in software engineering. The generic term for this field of study is \emph{computer-aided software engineering}.
		\end{definition}
	\end{fancycolumns}
\end{frame}

\widexkcdframe{378} % real programmers

\subsection{Overview on Development Tools}
\begin{frame}{\insertsubsection}
	\begin{fancycolumns}[widths={57}]
		\begin{note}{Variety of Tools \mysource{\ghezzi}}
			\begin{itemize}
				\item text(ual) editors: emacs, vim, ed, Word, \ldots
				\item graphical editors: UML editors, Powerpoint, \ldots
				\item assembler, compiler, interpreter
				\uncover<2->{
				\item configuration management tools: git, SVN, CVS, \ldots
				\item tracking tools (issue trackers): Github, Gitlab, \ldots
				\item tools for code navigation and refactoring
				\item tools for test specification, generation, execution, reporting
				\item tools for static and dynamic code analysis (e.g., debugger), reverse/reengineering, project management
				}
				\uncover<3->{
				\item integrated development environments (IDEs): Eclipse, Visual Studio, IntelliJ, Android Studio
				}
			\end{itemize}
		\end{note}
	\end{fancycolumns}
\end{frame}

\subsection{Demo on Tool Support in Eclipse}
\begin{frame}{\insertsubsection\ \mytitlesource{\href{https://youtu.be/Jxt77kTbFZ0?si=oGjCcdfji7NMmbM7&t=3318}{youtube.de}}}
	\centering\pic[width=.7\linewidth,trim=0 76 0 76,clip]{demo/livecoding2} % TODO update picture to recent version
\end{frame}

