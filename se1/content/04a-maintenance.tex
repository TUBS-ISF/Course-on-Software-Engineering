\subsection{Software Maintenance in Practice}

\subsubsection{New Year's Eve for Maintainers}
\begin{frame}{New Year's Eve: Party Time?!?}
	\begin{fancycolumns}
		\pic[width=\linewidth,trim=0 230 0 140,clip]{changes/london-fireworks}
		\vspace{-7mm}
		
		\begin{example}{}
			London Eye on January 1st, \sout{2008} 2017
		\end{example}
	\end{fancycolumns}
\end{frame}

\begin{frame}{\insertsubsubsection}
	\begin{fancycolumns}
		\begin{exampletight}{Meanwhile in the Internet}
			\picDark[width=\linewidth,trim=0 0 0 0,clip]{changes/cloudflare-offline}
		\end{exampletight}
		\nextcolumn
		\picDark[width=\linewidth,trim=0 0 0 0,clip]{changes/cloudflare-dns-control-panel}
		\vspace{-7mm}
		
		\begin{example}{Meanwhile in San Francisco \mysource{\href{https://blog.cloudflare.com/how-and-why-the-leap-second-affected-cloudflare-dns/}{cloudflare.com}}}
			\mycite{\emph{At midnight UTC} on New Year’s Day, deep inside Cloudflare’s custom RRDNS software, a number went negative when it should always have been, at worst, zero. [\ldots] %A little later this negative value caused RRDNS to panic. This panic was caught using the recover feature of the Go language. The net effect was that some DNS resolutions to some Cloudflare managed web properties failed.
				%The problem only affected customers who use CNAME DNS records with Cloudflare, and only affected a small number of machines across Cloudflare's 102 data centers.
				At peak approximately \emph{0.2\% of DNS queries} to Cloudflare were affected and less than \emph{1\% of all HTTP requests to Cloudflare encountered an error}.}
		\end{example}
	\end{fancycolumns}
\end{frame}

\subsubsection{Maintenance at Work}
\begin{frame}{\insertsubsubsection\ \mytitlesource{\href{https://blog.cloudflare.com/how-and-why-the-leap-second-affected-cloudflare-dns/}{cloudflare.com}}}
	\begin{fancycolumns}
		\begin{example}{Meanwhile in San Francisco}
			\mycite{This problem was quickly identified. The most affected machines were \emph{patched in 90 minutes} and the fix was \emph{rolled out worldwide by 06:45 UTC}. We are sorry that our customers were affected, but we thought it was worth writing up the root cause for others to understand.}
		\end{example}
		
		\begin{exampletight}{The Timeline}
			\picDark[width=\linewidth,trim=0 0 0 0,clip]{changes/cloudflare-timeline}
		\end{exampletight}
		\nextcolumn
		\begin{exampletight}{Monitoring}
			\picDark[width=\linewidth,trim=0 0 0 0,clip]{changes/cloudflare-timeline-diagram}
		\end{exampletight}
		
		\begin{note}{Fun Fact}
			San Francisco's standard time zone is UTC-8
		\end{note}
	\end{fancycolumns}
\end{frame}

%\subsection{Wisdom on Debugging}
%\begin{frame}{\insertsubsection}
%	\begin{fancycolumns}
%		\pic[width=\linewidth,trim=35 0 0 0,clip]{people/gordon-glegg}
%		\vspace{-7mm}
%		
%		\begin{note}{Gordon Glegg \mysource{\href{https://twitter.com/CodeWisdom/status/1140295217947512833}{twitter.com}}}
%			\mycite{Sometimes the problem is to discover what the problem is.}
%		\end{note}
%		\nextcolumn
%		\pic[width=\linewidth,trim=0 80 0 25,clip]{people/jessica-joy-kerr}
%		\vspace{-7mm}
%		
%		\begin{note}{Jessica Joy Kerr (2020) \mysource{\href{https://jessitron.com/2020/02/07/smaller-pieces-lower-pain/}{jessitron.com}}}
%			\mycite{In programming, it’s dangerous to work near your working memory threshold. You get more mistakes and more complicated code.}
%		\end{note}
%	\end{fancycolumns}
%\end{frame}

\subsubsection{The One-Character-Fix}
\begin{frame}{\insertsubsubsection\ \mytitlesource{\href{https://blog.cloudflare.com/how-and-why-the-leap-second-affected-cloudflare-dns/}{cloudflare.com}}}
	\begin{fancycolumns}
		\begin{exampletight}{The Patch}
			\picDark[width=\linewidth,trim=0 0 0 0,clip]{changes/cloudflare-fix}
		\end{exampletight}
		\nextcolumn
		\begin{example}{A Falsehood Programmers Believe About Time}
			\mycite{The root cause of the bug that affected our DNS service was the belief that \emph{time cannot go backwards}. In our case, some code assumed that the difference between two times would always be, at worst, zero.
				
				~
				
				RRDNS is written in Go and uses Go’s time.Now() function to get the time. Unfortunately, this function does not guarantee monotonicity.}
		\end{example}
	\end{fancycolumns}
\end{frame}

\subsubsection{Leap Seconds}
\begin{frame}{\insertsubsubsection\ \deutsch{Schaltsekunden}}
	\begin{fancycolumns}
		\centering\pic[width=.7\linewidth,trim=0 0 0 0,clip]{changes/leap-second}
		
		\begin{exampletight}{Leap Second 27 in 2016 \mysource{\href{https://www.spinellis.gr/blog/20170101/}{spinellis.gr}}}
			\picDark[width=\linewidth,trim=0 0 0 0,clip]{changes/leap-second-37}
		\end{exampletight}
		\nextcolumn
		\begin{exampletight}{27 Leap Seconds since 1972 \mysource{\href{https://en.wikipedia.org/wiki/Leap_second}{wikipedia.org}}}
			\hfill\picDark[width=.25\linewidth,trim=0 0 0 0,clip]{changes/leap-seconds1}
			\hfill\picDark[width=.25\linewidth,trim=0 0 0 0,clip]{changes/leap-seconds2}
			\hfill\picDark[width=.25\linewidth,trim=0 0 0 0,clip]{changes/leap-seconds3}
			\hfill{}
		\end{exampletight} % 190% screenshots with 20 lines
	\end{fancycolumns}
\end{frame}

\xkcdframe{2266} % leap second/day smearing

\subsection{Software Maintenance}
\begin{frame}{\insertsubsection\ \mytitlesource{\ludewiglichter}}
	\begin{fancycolumns}
		\begin{note}{Motivation}
			\begin{itemize}
				\item for software: no compensation of deterioration, repair, spare parts
				\item corrections (especially shortly after first delivery)
				\item modification and reconstruction
			\end{itemize}
		\end{note}
		\nextcolumn
		\begin{definition}{Operation and Maintenance Phase \mysource{\ieeesixten}}
			\mycite{The period of time in the software life cycle during which a software product is employed in its operational environment, monitored for satisfactory performance, and modified as necessary to correct problems or to respond to changing requirements.}
		\end{definition}
		\begin{definition}{Maintenance \mysource{\ieeesixten}}
			\mycite{The process of modifying a software system or component after delivery to correct faults, improve performance or other attributes, or adapt to a changed environment.}
		\end{definition}
	\end{fancycolumns}
\end{frame}
% TODO Operations

\subsection{Kinds of Maintenance}
\begin{frame}{\insertsubsection\ \mytitlesource{\ludewiglichter}}
	\begin{fancycolumns}
		\begin{definition}{Adaptive Maintenance \mysource{\lientzswanson}}
			\mycite{Software maintenance performed to make a computer program usable in a changed environment.} \hfill \deutsch{adaptive Wartung}
		\end{definition}
		\begin{example}{}
			desktop application for a new version of an operating system (e.g., from Windows 10 to 11)
		\end{example}
		\begin{definition}{Corrective Maintenance \mysource{\lientzswanson}}
			\mycite{Maintenance performed to correct faults in software.} \hfill \deutsch{korrektive Wartung}
		\end{definition}
		\begin{example}{}
			Windows calculator showing wrong formulas
		\end{example}
		\nextcolumn
		% TODO it is really annoying that this category has an overlap with preventive maintenance, as better maintainability typically also results in fewer problems
		\begin{definition}{Perfective Maintenance \mysource{\lientzswanson}}
			\mycite{Software maintenance performed to improve the performance, maintainability, or other attributes of a computer program.} \hfill \deutsch{perfektive Wartung}
		\end{definition}
		\begin{example}{}
			better handling of very large files in a text editor
		\end{example}
		\begin{definition}{Preventive Maintenance \mysource{\lientzswanson}}
			\mycite{Maintenance performed for the purpose of preventing problems before they occur.} \hfill \deutsch{präventive Wartung}
		\end{definition}
		\begin{example}{}
			2,000 year problem, leap seconds/years
		\end{example}
	\end{fancycolumns}
\end{frame}
% TODO add links to original literature

\subsection{Wisdom on Maintenance}
\begin{frame}{\insertsubsection}
	\begin{fancycolumns}
		\pic[width=\linewidth,trim=0 360 0 50,clip]{people/karolina-szczur}
		\vspace{-7mm}
		
		\begin{note}{Karolina Szczur \mysource{\href{https://twitter.com/codewisdom/status/1141686386103332864}{twitter.com}}}
			\mycite{Writing software as if we are the only person that ever has to comprehend it is one of the biggest mistakes and false assumptions that can be made.}
		\end{note}
		\nextcolumn
		\pic[width=\linewidth,trim=450 235 450 120,clip]{changes/psychopath}
		\vspace{-7mm}
		
		\begin{note}{John F. Woods \mysource{\href{https://groups.google.com/g/comp.lang.c++/c/rYCO5yn4lXw/m/oITtSkZOtoUJ}{google.com}}}
			\mycite{Always code as if the guy who ends up maintaining your code will be a violent psychopath who knows where you live.}
		\end{note}
	\end{fancycolumns}
\end{frame}

