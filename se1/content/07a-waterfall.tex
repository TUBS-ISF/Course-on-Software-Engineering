\subsection{Software Development Process}
\begin{frame}{\insertsubsection}
	\begin{fancycolumns}
		\begin{note}{Motivation}
			\begin{itemize}
				\item how to \emph{structure} the project?
				\item what are \emph{activities} and phases?\\analysis (requirements elicitation + system modeling), design (architectural + software design), implementation, test, deployment
				\item how to organize \emph{communication}?
				\item who has which \emph{responsibilities}?
				\item did we \emph{forget} anything?
				\item can we \emph{predict} the project result?
				\item how to \emph{manage and control} progress?
				\item how to share and elicit \emph{experience}?
				\item how to synchronize \emph{hardware} and software development?
			\end{itemize}
		\end{note}
		\nextcolumn
		\begin{definition}{Software Process \mysource{\sommerville}}
			\mycite{A \emph{software process} is a set of related activities that leads to the production of a software system. [...] \emph{Products} or deliverables are the outcomes of a process activity. [...] \emph{Roles} reflect the responsibilities of the people involved in the process. [...] Pre- and postconditions are \emph{conditions} that must hold before and after a process
				activity.}
		\end{definition}
		\uncover<3->{\begin{note}{Why Different Processes? \mysource{\sommerville}}
				\mycite{The process used in different companies depends on the type of software being developed, the requirements of the software customer, and the skills of the people writing the software.}
		\end{note}}
	\end{fancycolumns}
\end{frame}

\subsection{Recap}
\begin{frame}<12>[b]{\insertsubsection}
	\vspace{-20mm}\small\renewcommand{\projectcartoonwidth}{.18}
	\begin{fancycolumns}[widths={30},animation=none]
		\uncover<11->{\begin{definition}{Why Do Software Projects Fail?}
				\begin{itemize}
					\item many stakeholders with different background each
					\item miscommunication
					\item implicit or wrong assumptions
					\item time pressure
					\item high complexity
					\item \ldots
					\item numerous reasons specific to certain phases and roles
				\end{itemize}
		\end{definition}}
		\uncover<12->{\begin{note}{}\centering{}software engineering aims to\\reduce such problems\end{note}}
		\nextcolumn
		\centering\hprojectcartoon{01}{how the customer explained it} % se02-10 requirements
		\uncover<2->{\hprojectcartoon{02}{how the project leader understood it}} % se02-07 modeling
		\uncover<3->{\hprojectcartoon{03}{how the analyst designed it}} % se04-10 architecture anddesign
		\uncover<4->{\hprojectcartoon{04}{how the programmer implemented it}} % se02-10 implementation
		\uncover<5->{\hprojectcartoon{05}{what the beta testers received}} % se08-09 testing
		
		\uncover<6->{\hprojectcartoon{06}{how the business consultant described it}} % se02-03 process
		\uncover<7->{\hprojectcartoon{09}{how the customer was billed}} % se10 management and pricing
		\uncover<8->{\hprojectcartoon{10}{how it was supported}} % se12 maintenance
		\uncover<9->{\hprojectcartoon{12}{when it was delivered}} % se13 continous integration/delivery
		\uncover<10->{\hprojectcartoon{13}{what the customer really needed}} % se02-05
		%\hspace{-7mm}
		%\projectcartoon{07}{how the project was documented} % code documentation
		%\hprojectcartoon{18}{how patches were applied} % se11 evolution
		%\hprojectcartoon{17}{how it performed under load} % se14 compilation/quality assurance/performance
		%\hprojectcartoon{11}{what marketing advertised} % se15 reuse/product lines
		%\hprojectcartoon{16}{how open source version} % se16 open source/licensing
		%\projectcartoon{08}{what operations installed} % devops?
		%\projectcartoon{14}{what the digg effect can do to your site} % micro services?
		%\projectcartoon{15}{the disaster recover plan}
		%\hspace{-7mm}
	\end{fancycolumns}
\end{frame}

% TODO add slide(s) on requirements and design?

\subsection{Waterfall Model}
\begin{frame}{\insertsubsection{} \deutsch{Wasserfallmodell} \mytitlesource{\sommerville}}
	\slideWaterfallModel
\end{frame}

\begin{frame}{\insertsubsection\ \mytitlesource{\sommerville}}
	\begin{fancycolumns}[animation=none]
		\begin{definition}{1. Requirements Analysis}
			\mycite{The system’s services, constraints, and goals are established by consultation with system users. They are then defined in detail and serve as a \emph{system specification}.}
		\end{definition}
		\begin{definition}{2. System and Software Design}
			\mycite{The systems design process allocates the requirements to either hardware or software systems. It establishes an overall \emph{system architecture}. Software design involves identifying and describing the fundamental software system abstractions and their relationships.}
		\end{definition}
		\nextcolumn
		\vspace{-11mm}
		\begin{definition}{3. Implementation and Unit Testing}
			\mycite{During this stage, the software design is realized as a set of \emph{programs or program units}. Unit testing involves verifying that each unit meets its specification.}
		\end{definition}
		\begin{definition}{4. Integration and System Testing}
			\mycite{The individual program units or programs are integrated and tested as a complete system to ensure that the software requirements have been met. After testing, the \emph{software system is delivered} to the customer.}
		\end{definition}
		\begin{definition}{5. Operation and Maintenance}
			\mycite{Normally, this is the longest life-cycle phase. The system is installed and put into practical use. Maintenance involves \emph{correcting errors} that were not discovered in earlier stages of the life cycle [...].%, improving the implementation of system units, and enhancing the system’s services as new requirements are discovered.
			}
		\end{definition}
	\end{fancycolumns}
\end{frame}

\subsection{Waterfall Model -- Examples}
\begin{frame}{\insertsubsection\ \mytitlesource{\sommerville}}
	\begin{fancycolumns}[animation=none]
		\begin{example}{Example Domains}
			\begin{itemize}
				\item \emph{embedded systems} where software has to interface with hardware systems
				\item \emph{critical systems} with extensive safety and security analysis of specification and design
				\item \emph{large software systems} that are typically developed by several companies
			\end{itemize}
		\end{example}
	\nextcolumn
		\pic[width=\linewidth]{misc/lawn-mower-cropped}
	\end{fancycolumns}
\end{frame}

\subsection{Waterfall Model -- Discussion}
\begin{frame}{\insertsubsection\ \mytitlesource{\sommerville}}
	\begin{fancycolumns}
		\begin{note}{Advantages}
			\begin{itemize}
				\item easy to understand, manage, and control
				\item good for systems development (i.e., with high manufacturing costs for hardware)
				\item easier to use the same model as for hardware
				\item combination with formal system development feasible (e.g., B method)
			\end{itemize}
		\end{note}
		\nextcolumn
		\begin{note}{Disadvantages}
				\begin{itemize}
					\item for software development: stages should feed information to each other
					\item changes in previous stages are hard to achieve
					\item problems from previous stages left for later resolution
					\item freezing of requirements may lead to software not wanted by the user
					\item freezing of design may lead to bad structure and implementation tricks
					\item requires clear and stable requirements and good design upfront
				\end{itemize}
		\end{note}
	\end{fancycolumns}
\end{frame}

