\begin{frame}
	\begin{fancycolumns}[height=8.5cm]
		\pic[width=\linewidth,trim=0 20 0 0,clip]{people/louis-srygley}
		\vspace{-7mm}
		
		\begin{note}{Louis Srygley \mysource{\href{https://twitter.com/codewisdom/status/956225177439866882}{twitter.com}}}
			\mycite{Without requirements or design, programming is the art of adding bugs to an empty text file.}
		\end{note}
		\nextcolumn
		\pic[width=\linewidth,trim=0 75 0 75,clip]{people/alistair-cockburn}
		\vspace{-7mm}
		
		\begin{note}{Alistair Cockburn \mysource{\href{https://twitter.com/CompSciFact/status/934100633539567618}{twitter.com}}}
			\mycite{If it's your decision, it's design; if not, it's a requirement.}
		\end{note}
	\end{fancycolumns}
\end{frame}

\subsection{Recap: Process Models}
\begin{frame}[10]{\insertsubsection}
	\begin{fancycolumns}[widths={45}]
		\diagramWaterfallModel
		\nextcolumn
		\diagramVModel
	\end{fancycolumns}
\end{frame}

\xkcdframe{1741} % design decisions

\subsection{Recap: 14 Types of UML Diagrams}
\begin{frame}{\insertsubsection\ \mytitlesource{\umlspec}}
	\centering\only<1|handout:0>{\slideMindmapUMLdiagrams{blue}{blue}{blue}{blue}{}{}{}{}{}}%
	\only<2->{\slideMindmapUMLdiagrams{blue}{blue}{blue}{blue}{}{red}{}{}{}}%
\end{frame}

\subsection{Sequence Diagrams}
\begin{frame}{\insertsubsection{} \deutschertitel{Sequenzdiagramme}}
	\begin{fancycolumns}[animation=none]
		\begin{definition}{{Sequence Diagram \mysource{\umlrefman}}}
			\mycite{A sequence diagram displays an interaction as a two-dimensional chart. The \emph{vertical dimension} is the time axis; time proceeds down the page. The \emph{horizontal dimension} shows the roles that represent individual objects in the collaboration. 
				\uncover<2->{Each role is represented by a vertical column containing a head symbol and a vertical line -- a \emph{lifeline}. During the time an object exists, it is shown by a dashed line. During the time an execution specification of a procedure on the \emph{object is active}, the lifeline is drawn as a double line. [...]} 
				\uncover<3->{A \emph{message} is shown as an arrow from the lifeline of one object to that of another. The arrows are arranged in time sequence down the diagram. An \emph{asynchronous message} is shown with a stick arrowhead.}} \uncover<3->{\deutsch{Lebenslinie, Aktivierungsbalken, (a)synchrone Nachricht}}
		\end{definition}
		% TODO add nesting, split over several slides: Aktivierungsbalken (Beginn, Ende, Ausführung, Verschachtelung), Nachrichten (synchron, return, asynchron), Objekterzeugung und -löschung
		% \mycite{You can show the nesting of a focus of control (caused by recursion, a call to a self-operation, or by a callback from another object) by stacking another focus of control slightly to the right of its parent (and can do so to an arbitrary depth).} \umluserguide
		% deliberately ignored: specific objects underlined, structured control, 
		\nextcolumn
		\posthandout{\includegraphics[page=23,width=\linewidth,trim=225 20 20 40,clip]{printedslides/2021wt/design}}
		%\posthandout{\pic[width=\linewidth,trim=20 180 30 160,clip]{blackboard/blackboard_sequence_23_24}}
	\end{fancycolumns}
\end{frame}

\begin{frame}{Example of a Sequence Diagram}
	\posthandout{\pic[page=24,width=\linewidth,trim=20 20 20 40,clip]{printedslides/2021wt/design}}
\end{frame}

\subsection{Rules for Sequence Diagrams}
\begin{frame}{\insertsubsection}
	\begin{fancycolumns}
		\begin{note}{Rules for Sequence Diagrams}
			\begin{itemize}
				\item Activity of objects begins at the role or with a message
				\item Only active objects can send messages
				\item Every synchronous message has its own return (in this lecture, optional in UML)
				\item Every return has its own synchronous message
				\item Activities can be stacked to arbitrary, but finite depth
				\item A destroyed object is dead forever
				\item Ordering of messages for each single lifetime matters
				\item Distance of messages in the diagram does not imply timing constraints
			\end{itemize}
		\end{note}
	\end{fancycolumns}
\end{frame}

\subsection*{Recap: 14 Types of UML Diagrams}
\begin{frame}{\insertsubsection\ \mytitlesource{\umlspec}}
	\centering\slideMindmapUMLdiagrams{blue}{blue}{blue}{blue}{}{blue}{}{}{}
\end{frame}

