\subsection{History of Programming Languages}
\begin{frame}{\insertsubsection}
	\begin{fancycolumns}
		\begin{note}{Milestones\mysource{\jonesbestpractice}}
			\begin{itemize}
				\setlength\itemsep{.1em}
				\item controlling behavior of mechanical devices by wiring or with punchcards \deutsch{Lochkarten}
				\item machine languages used during World War II
				\uncover<2->{
					\item assembly languages: distinction between human-readable instructions (source code) and executable instructions (object code)
					\item birth of compilers and interpreters having a one-to-many mapping between source and object code (opposed to one-to-one mapping in assemblers)
				}
				\uncover<3->{
					\item structured programming pioneered by David Parnas and Edsger Dijkstra
					\item high-level programming languages: high number of executable for each human-readable instruction
					\item domain-specific languages, later general-purpose programming languages
				}
			\end{itemize}
		\end{note}
	\nextcolumn
		\uncover<4->{
		\begin{example}{Languages\mysource{\jonesbestpractice\ + \handbuch}}
			\begin{itemize}
				\setlength\itemsep{.1em}
				\item 1945: first high-level language Plankalkül by Konrad Zuse (compiler written in 1998)
				\item 1954: first professional high-level language FORTRAN (Formula Translator) by IBM
				\item 1963: Basic as general-purpose language
				\uncover<5->{
					\item 1959: functional language Lisp
					\item 1970: first object-oriented lang.\ Smalltalk-80
					\item 1970: declarative language SQL
				}
				\uncover<6->{
					\item 1971: Pascal by Niklaus Wirth for teaching
					\item 1974: very common procedural language C
					\item 1977: logical language Prolog
				}
				\uncover<7->{
					\item 1980: C++ as object-oriented extension of C
					\item 1990: object-oriented language Java
					\item 1990: functional language Haskell
				}
				\uncover<8->{
					\item 1991: multi-paradigm language Python % TODO for machine learning
					\item 1995: scripting language JavaScript % TODO for web applications
				}
			\end{itemize}
		\end{example}
	}
	\end{fancycolumns}
\end{frame}
% TODO revise and talk about paradigms and their languages

\subsection{Programming Languages Today}
\begin{frame}{\insertsubsection}
	\slideProgrammingLanguagesToday
\end{frame}

\subsection{Choice of Programming Languages}
\begin{frame}{\insertsubsection}
	\begin{fancycolumns}
		\begin{definition}{{Desired Properties\mysource{\ludewiglichter}}}
			\begin{itemize}
				\item modular implementation
				\item separation of inferfaces and implementations
				\item type system: strongly/weakly typed languages
				\item readable syntax (FORTRAN vs ALGOL60) % designed for fewer characters vs readability
				\item automatic pointer management (C vs Java)
				\item exception handling
			\end{itemize}
		\end{definition}
	\nextcolumn
		\begin{example}{Criteria in Practice}
			\begin{itemize}
				\item language required by the company or customer?
				\item existing infrastructure?
				\item domain-specific languages available?
				\item language known/liked by developers?
				\item available libraries?
				\item available tool support?
				\item language popularity?
				\item what may change in the future?
			\end{itemize}
		\end{example}
	\end{fancycolumns}
\end{frame}

\subsection{Popularity of Programming Languages}
\begin{frame}{\insertsubsection}
	\slideTiobeDiagram
\end{frame}
\begin{frame}{\insertsubsection}
	\slideTiobeTable
\end{frame}

\begin{frame}
	\begin{fancycolumns}[height=8.5cm]
		\pic[width=\linewidth,trim=0 20 0 20,clip]{people/patrick-mckenzie}
		\vspace{-7mm}
		
		\mynote{Patrick McKenzie \mysource{\href{https://twitter.com/CodeWisdom/status/1182702520696803329}{twitter.com}}}{\mycite{Every great developer you know got there by solving problems they were unqualified to solve until they actually did it.}}
		% computer scientist, entrepreneur, influencer
	\nextcolumn
		\pic[width=\linewidth,trim=0 55 0 10,clip]{people/bill-gates}
		\vspace{-7mm}
		
		\mynote{Bill Gates \mysource{\href{https://code.org/quotes}{code.org}}}{\mycite{Learning to write programs stretches your mind, and helps you think better, creates a way of thinking about things that I think is helpful in all domains.}}
		% richest person in 15 years between 1994 and 2014
	\end{fancycolumns}
\end{frame}

\subsection{Computer-Aided Software Engineering}
\begin{frame}{\insertsubsection}
	\begin{fancycolumns}[widths={55}]
		\begin{definition}{Terms \mysource{adapted from \ghezzi}}
			A \emph{tool} is an application that supports a particular activity. An \emph{environment} is a collection of related tools. Tools and environments aim at automating some of the activities that are involved in software engineering. The generic term for this field of study is \emph{computer-aided software engineering}.
		\end{definition}
	\end{fancycolumns}
\end{frame}

\widexkcdframe{378} % real programmers

\subsection{Overview on Development Tools}
\begin{frame}{\insertsubsection}
	\begin{fancycolumns}[widths={57}]
		\begin{note}{Variety of Tools \mysource{\ghezzi}}
			\begin{itemize}
				\item text(ual) editors: emacs, vim, ed, Word, \ldots
				\item graphical editors: UML editors, Powerpoint, \ldots
				\item assembler, compiler, interpreter
				\uncover<2->{
					\item configuration management tools: git, SVN, CVS, \ldots
					\item tracking tools (issue trackers): Github, Gitlab, \ldots
					\item tools for code navigation and refactoring
					\item tools for test specification, generation, execution, reporting
					\item tools for static and dynamic code analysis (e.g., debugger), reverse/reengineering, project management
				}
				\uncover<3->{
					\item integrated development environments (IDEs): Eclipse, Visual Studio, IntelliJ, Android Studio
				}
			\end{itemize}
		\end{note}
	\end{fancycolumns}
\end{frame}

%\subsection{Demo on Tool Support in Eclipse}
%\begin{frame}{\insertsubsection\ \mytitlesource{\href{https://youtu.be/Jxt77kTbFZ0?si=oGjCcdfji7NMmbM7&t=3318}{youtube.de}}}
%	\centering\pic[width=.7\linewidth,trim=0 76 0 76,clip]{demo/livecoding2} % TODO update picture to recent version
%\end{frame}
