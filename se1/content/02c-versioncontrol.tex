\subsection{Version Control}
\begin{frame}{\insertsubsection}
	\slideVersionControl
\end{frame}

\subsection{Git vs Mercurial}
\begin{frame}{\insertsubsection}
	\begin{fancycolumns}[animation=none]
		\picDark[width=\linewidth]{versioncontrol/bitbucket-mercurial1}
		\nextcolumn
		\picDark[width=\linewidth]{versioncontrol/bitbucket-mercurial2}
	\end{fancycolumns}
\end{frame}

\subsection{Centralized Version Control}
\begin{frame}{\insertsubsection}
	\vspace{-5mm}
	\begin{fancycolumns}
		\clientserver{computer}{computer}{computer}{Server}{Client X}{Client Y}{
			\draw[kante] (client1) to (server);
			\draw[kante] (server) to (client1);
			\draw[kante] (client2) to (server);
			\draw[kante] (server) to (client2);
		}
		\begin{definition}{Centralized Version Control Systems}
			\begin{itemize}
				\item client-server architecture
				\item server manages the main copy
				\item clients use server to synchronize files/folders
			\end{itemize}
		\end{definition}
		\nextcolumn
		\only<-2|handout:0>{\clientserver{computer}{computer}{computer}{Repository: Revision 1-10}{}{}{
		}}
		\only<3|handout:0>{\clientserver{computer}{computer}{computer}{Repository: Revision 1-10}{Working Copy: Revision 10}{}{
				\draw[kante] (server) to (client1);
		}}
		\only<4|handout:0>{\clientserver{computer}{computer}{computer}{Repository: Revision 1-10}{Working Copy: Revision 10}{Working Copy: Revision 5}{
				\draw[kante] (server) to (client2);
		}}
		\only<5|handout:0>{\clientserver{computer}{computer}{computer}{Repository: Revision 1-10}{Working Copy: 10 (changed)}{Working Copy: Revision 5}{
		}}
		\only<6>{\clientserver{computer}{computer}{computer}{Repository: Revision 1-11}{Working Copy: Revision 11}{Working Copy: Revision 5}{
				\draw[kante] (client1) to (server);
		}}
		\only<7|handout:0>{\clientserver{computer}{computer}{computer}{Repository: Revision 1-11}{Working Copy: Revision 11}{Working Copy: Revision 11}{
				\draw[kante] (server) to (client2);
		}}
		\begin{note}{Centralized Version Control Systems}
			\begin{itemize}
				\item repository on server
				\item working copy on each client
				\item new revision for every change on the server
				\item revisions used to undo changes, merge files
			\end{itemize}
		\end{note}
	\end{fancycolumns}
\end{frame}

\subsection{Distributed Version Control}
\begin{frame}{\insertsubsection}
	\slideDistributedVC
\end{frame}

\xkcdframe{1597} % git

\subsection{Clone, Fetch, Commit, and Push}
\slideCloneFetch

\slideCommitPush

\subsection{Commit Messages}
\xkcdframe{1296} % commit messages

\begin{frame}{How to Write Good Commit Messages?}
	\begin{fancycolumns}
		\begin{definition}{Structure of Commit Messages \mysource{\href{https://gist.github.com/robertpainsi/b632364184e70900af4ab688decf6f53}{github.com}}}
			Subject Line (required)
			\vspace{-1mm}
			\begin{itemize}
				\item short summary (72 chars or less)
				\item should complete the following sentence:\\
				\mycite{If applied, this commit will \ldots}
			\end{itemize}
			
			Message Body (optional)
			\vspace{-1mm}
			\begin{itemize}
				\item blank line followed by message body
				\item explain what has changed and why
			\end{itemize}
		\end{definition}
		\vspace{-1mm}
		\begin{note}{Integration with Issues \mysource{\href{https://docs.github.com/en/enterprise/2.16/user/github/managing-your-work-on-github/closing-issues-using-keywords}{github.com}, \href{https://docs.gitlab.com/ee/user/project/issues/managing_issues.html}{gitlab.com}}}
			\begin{itemize}
				\item write \textbf{\#42} to refer to issue -- Github/Gitlab will create links in both ways
				\item \textbf{close/fix/resolve/\ldots\ \#42} -- issue automatically closed when commit is pushed to default branch
			\end{itemize}
		\end{note}
		\nextcolumn
		\begin{definition}{Conventional Commits \mysource{\href{https://www.conventionalcommits.org/}{conventionalcommits.org}}}
			\begin{itemize}
				\item Machine readable subject line
				%\item <type>(<optional scope>): <subject line>
				\item \textbf{fix:} patches a bug (\textbf{patch})
				\item \textbf{feat:} introduces a new feature (\textbf{minor} change)
				\item \textbf{BREAKING CHANGE:} introduces a breaking API change (\textbf{major} change)
				\item \textbf{refactor:} applies a refactoring (patch)
				\item \ldots
			\end{itemize}
		\end{definition}
	\end{fancycolumns}
\end{frame}

\begin{frame}{Examples for Conventional Commit Messages}
	\centering\picDark[width=.8\linewidth,trim={.05\width} {0\height} {0\width} {.25\height},clip]{versioncontrol/profcalc-commits}
\end{frame}


\begin{frame}{Semantic Versioning}
	\begin{fancycolumns}
		
		
		\begin{note}{Motivation} 
			\begin{itemize}
				\item many different versioning schemes
				\item unclear which software versions are compatible with others
			\end{itemize}
		\end{note}
		
		\begin{definition}{Semantic Versioning \mysource{\href{https://semver.org/}{semver.org}}} 
			\begin{itemize}
				\item proposes a simple rule set for versions
				\item differentiation based on changes:
				\begin{itemize}
					\item breaking changes (major)
					\item added features (minor)
					\item fixes / refactorings (patch)
				\end{itemize}
				\item version scheme: major.minor.patch
				\item 0.x.y versions for initial development
				\item common initial version: 0.1.0
			\end{itemize}
		\end{definition}
		\vspace{-1mm}
		
		\nextcolumn
		\vspace{-5mm}
		\begin{note}{Versioning Rules \mysource{\href{https://semver.org/}{semver.org}}}
			A version is changed after a set of commits:
			\begin{table}[]
				\begin{tabular}{lrrr}
					Breaking changes & $\geq$ 1 & ?        & ?      \\
					Added features  & 0        & $\geq$ 1 & ?   \\
					Fixes / Refactorings            & 0        & 0        & $\geq$ 1   \\[1mm]
					& \vspace{-4mm}& & \\[1mm]
					\hline
					& \vspace{-4mm}& & \\[1mm]
					\textbf{Major}                & +1       & 0        & 0         \\
					\textbf{Minor}                & =        & +1       & 0     \\
					\textbf{Patch}                & =        & =        & +1    
				\end{tabular}
			\end{table}
			%			\begin{table}[]
				%				\begin{tabular}{ccc|ccc}
					%					\rotatebox{90}{Incompatible Changes} & \rotatebox{90}{Added functionality} & \rotatebox{90}{Bug fixes} & Major & Minor & Patch \\ \hline
					%					$\geq$ 1             & ?                   & ?         & +1    & 0     & 0     \\
					%					0                    & $\geq$ 1            & ?         & =     & +1    & 0     \\
					%					0                    & 0                   & $\geq$ 1  & =     & =     & +1   
					%				\end{tabular}
				%			\end{table}
		\end{note}
		
		\begin{example}{Examples}
			
			\begin{itemize}
				
				\item \textbf{major} 2.3.2 $\rightarrow$ 3.0.0
				
				\textit{Changed supported database}
				
				\item \textbf{minor} 1.2.0 $\rightarrow$ 1.3.0
				
				\textit{Added functionality to upload files}
				
				\item \textbf{patch} 1.12.0 $\rightarrow$ 1.12.1
				
				\textit{Fixed a bug when downloading large files}
				
			\end{itemize}
		\end{example}
	\end{fancycolumns}
\end{frame}


\subsection{Merge, Pull, and Ignore}
\slideMergePull

\slideIgnore

\subsection{Branching and Merging}
\begin{frame}{\insertsubsection}
	\slideBranchingAndMerging
\end{frame}

\begin{frame}[fragile]{Automatic Merge}
	\begin{columns}[onlytextwidth]
		\begin{column}{0.6\linewidth}
			\begin{columns}[T]
				\begin{column}{0.5\linewidth}
					Working Copy: Revision \only<1|handout:0>{10}\only<2-3|handout:0>{\emph{10$^*$}}\only<4>{\emph{11$^*$}}\\[2mm]
					
					\begin{onlyenv}<1|handout:0>
						\begin{lstlisting}[style=java,basicstyle=\fontfamily{pcr}\small\selectfont,numbers=none,escapechar=|]
class Foo {
	void bar() {
		print("Bar");
	}
}	
						\end{lstlisting}
					\end{onlyenv}
					\begin{onlyenv}<2-3|handout:0>
						\begin{lstlisting}[style=java,basicstyle=\fontfamily{pcr}\small\selectfont,numbers=none,escapechar=|]
class Foo {
	void bar() {
		print("Bar");
		print("\n");
	}
}	
						\end{lstlisting}
					\end{onlyenv}
					\begin{onlyenv}<4-|handout:1>
						\begin{lstlisting}[style=java,basicstyle=\fontfamily{pcr}\small\selectfont,numbers=none,escapechar=|]
class Foo {
	void bar() {
		print("Foo");
		print("Bar");
		print("\n");
	}
}	
						\end{lstlisting}
					\end{onlyenv}
				\end{column}
				\begin{column}{0.5\linewidth}					
					Repository: Revision \only<1-2|handout:0>{10}\only<3->{\emph{11}}
					
					\begin{onlyenv}<1-2|handout:0>
						\begin{lstlisting}[style=java,basicstyle=\fontfamily{pcr}\small\selectfont,numbers=none,escapechar=|]
class Foo {
	void bar() {
		print("Bar");
	}
}
						\end{lstlisting}
					\end{onlyenv}
					\begin{onlyenv}<3-|handout:1>
						\begin{lstlisting}[style=java,basicstyle=\fontfamily{pcr}\small\selectfont,numbers=none,escapechar=|]
class Foo {
	void bar() {
		print("Foo");
		print("Bar");
	}
}
						\end{lstlisting}
					\end{onlyenv}
				\end{column}
			\end{columns}
			
			\begin{example}{}
				\begin{itemize}
					\item<1-> Checkout of or update to revision 10
					\item<2-> Changing the working copy (\emph{10$^*$})
					\item<3-> In the meantime: New commit to repository (\emph{11})
					\item<4-> Update to head revision, automatically merged
				\end{itemize}			
			\end{example}
		\end{column}
		\begin{column}{0.5\linewidth}
		\end{column}
	\end{columns}
\end{frame}

\begin{frame}{Git Merge in Pictures}
	\begin{fancycolumns}[animation=none]
		\only<1|handout:0>{\pic[width=\linewidth]{versioncontrol/gitmerge1}}%
		\only<2|handout:1>{\pic[width=\linewidth]{versioncontrol/gitmerge2}}%
		\only<3|handout:0>{\pic[width=\linewidth]{versioncontrol/gitmerge3}}%
		\only<4|handout:0>{\pic[width=\linewidth]{versioncontrol/gitmerge4}}%
		\only<5|handout:0>{\pic[width=\linewidth]{versioncontrol/gitmerge5}}%
		\nextcolumn
		\only<-0>{\pic[width=\linewidth]{versioncontrol/gitmerge5}}%
	\end{fancycolumns}
\end{frame}

\begin{frame}[fragile]{Merge Conflicts}	
	\begin{columns}[T, onlytextwidth]
		\begin{column}{0.6\linewidth}
			\begin{columns}[T]
				\begin{column}{0.5\linewidth}
					Working Copy: Revision \only<1|handout:0>{10}\only<2-3|handout:0>{\emph{10$^*$}}\only<4>{\emph{11$^*$}}\\[2mm]
					
					\begin{onlyenv}<1|handout:0>
						\begin{lstlisting}[style=java,basicstyle=\fontfamily{pcr}\small\selectfont,numbers=none,escapechar=|]
class Foo {
	void bar() {
		print("Bar");
	}
}	
						\end{lstlisting}
					\end{onlyenv}
					\begin{onlyenv}<2-3|handout:0>
						\begin{lstlisting}[style=java,basicstyle=\fontfamily{pcr}\small\selectfont,numbers=none,escapechar=|]
class Foo {
	void bar() {
		print("Bar\n");
	}
}
						\end{lstlisting}
					\end{onlyenv}
					\begin{onlyenv}<4-|handout:1>
						\begin{lstlisting}[style=java,basicstyle=\fontfamily{pcr}\small\selectfont,numbers=none,escapechar=|]
class Foo {
	void bar() {		
		<<<<<<< .mine
		print("Bar\n");
		=======
		print("FooBar");
		>>>>>>> .r11
	}
}	
						\end{lstlisting}
					\end{onlyenv}
				\end{column}
				\begin{column}{0.5\linewidth}					
					Repository: Revision \only<1-2|handout:0>{10}\only<3->{\emph{11}}
					
					\begin{onlyenv}<1-2|handout:0>
						\begin{lstlisting}[style=java,basicstyle=\fontfamily{pcr}\small\selectfont,numbers=none,escapechar=|]
class Foo {
	void bar() {
		print("Bar");
	}
}
						\end{lstlisting}
					\end{onlyenv}
					\begin{onlyenv}<3-|handout:1>
						\begin{lstlisting}[style=java,basicstyle=\fontfamily{pcr}\small\selectfont,numbers=none,escapechar=|]
class Foo {
	void bar() {
		print("FooBar");
	}
}
						\end{lstlisting}
					\end{onlyenv}
				\end{column}
			\end{columns}
			
			\begin{example}{}
				\begin{itemize}
					\item<1-> Checkout of or update to revision 10
					\item<2-> Changing the working copy (\emph{10$^*$})
					\item<3-> In the meantime: New commit to repository (\emph{11})
					\item<4-> Update to head revision results in conflict
					\item<4-> Automatic merge fails: user has to provide a fix
				\end{itemize}			
			\end{example}
		\end{column}
		\begin{column}{0.5\linewidth}
		\end{column}
	\end{columns}
\end{frame}
