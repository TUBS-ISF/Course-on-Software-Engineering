% TODO explain the difference between requirements elicitation and system model + architecture design vs software design

\subsection{V-Model}
\begin{frame}{\insertsubsection}
	\begin{fancycolumns}
		\begin{definition}{V-Model \mysource{\ludewiglichter}}
			\begin{itemize}
				\item developed by the German Ministry of Defense \deutsch{Verteidigungsministerium} and required since 1992
				\item extension of the waterfall model: project-aligned activities such as quality assurance, configuration management, project management
				\item 1997 V-model 97: incremental development, inclusion of hardware, object-oriented development
				\item 2004 V-model XT (for extreme tailoring): adaptability, application beyond software
				% four project types of V-model XT
				\item integration of four testing stages
				%\item distinction between validation (right product) and verification (product correct)
				% fake? proposed 1979 by Barry Boehm
			\end{itemize}
		\end{definition}
		\nextcolumn
		\diagramVModel
		% TODO find reference for this kind of V model
	\end{fancycolumns}
\end{frame}

\subsection{Stages of Testing}
\begin{frame}{\insertsubsection\ \deutsch{Teststufen}\ \mytitlesource{\ludewiglichter, \sommerville}}
	\begin{fancycolumns}[animation=none]
		\begin{definition}{1. Unit Testing \deutsch{Komponententest}}
			\uncover<2->{\begin{itemize}
					\item each component is tested independently
					\item unit may stand for a component or smaller entities (package, class, method)
					\item tests created by the developers
					\item automation is common (e.g., JUnit)
			\end{itemize}}
		\end{definition}
		\begin{definition}{2. Integration Testing \deutsch{Integrationstest}}
			\uncover<3->{\begin{itemize}
					\item some components are integrated (e.g., into subsystems) and tested together
					\item detects inconsistencies in interfaces and communication between components
					\item top-down vs bottom-up integration
			\end{itemize}}
		\end{definition}
		\nextcolumn
		\begin{definition}{3. System Testing \deutsch{Systemtest}}
			\uncover<4->{\begin{itemize}
					\item all components are integrated to the complete system
					\item detects further inconsistencies and unanticipated interactions
					\item system is tested against system requirements
			\end{itemize}}
		\end{definition}
		\begin{definition}{4. Acceptance Testing \deutsch{Abnahmetest}}
			\uncover<5->{\begin{itemize}
					\item final stage in the testing process before accepted for operational use
					\item system is tested against user requirements and with real data
					\item performed by (potential) customer
			\end{itemize}}
		\end{definition}
	\end{fancycolumns}
\end{frame}
% TODO split on several slides and give examples for errors that should / should not be found in this phase (and why)

\subsection{V-Model -- Examples}
\begin{frame}{\insertsubsection\ \mytitlesource{\ludewiglichter}}
	\begin{fancycolumns}[animation=none]
		\begin{example}{Example Domains}
			\begin{itemize}
				\item since 1992 V-model required by German government (e.g., Bundeswehr), since 2004 V-model XT
				\item embedded, critical, and large software systems as for the waterfall model
			\end{itemize}
		\end{example}
	\nextcolumn
		\pic[width=\linewidth]{misc/lawn-mower-cropped}
		% TODO picture associated with German government?
	\end{fancycolumns}
\end{frame}

\subsection{V-Model -- Discussion}
\begin{frame}{\insertsubsection\ \mytitlesource{\ludewiglichter}}
	\begin{fancycolumns}
		\begin{note}{Advantages}
			\begin{itemize}
				\item quality assurance in several testing stages
				\item completeness helps to not miss activities
				%\item distinction between validation and verification
				\item V-model 97/XT are widely applicable (e.g., hardware, incremental)
			\end{itemize}
		\end{note}
		\nextcolumn
		\begin{note}{Disadvantages}
				\begin{itemize}
					\item complex and extensive process
					\item adaptations often required (cf.\ XT for extreme tailoring)
					\item overhead useful only for large software systems
					\item changes in requirements are problematic
				\end{itemize}
		\end{note}
	\end{fancycolumns}
\end{frame}

