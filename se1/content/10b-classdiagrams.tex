\begin{frame}{Recap: 14 Types of UML Diagrams\ \mytitlesource{\umlspec}}
	\centering\slideMindmapUMLdiagrams{blue}{blue}{blue}{blue}{red}{}{}{visible on={<-0>}}{}
\end{frame}

\subsection{Class Diagrams}
\begin{frame}{\insertsubsection\ \normalsize\deutsch{Klassendiagramme}}
	\begin{fancycolumns}[animation=none]
		\begin{definition}{{Class Diagram\mysource{\umluserguide}}}
			%\mycite{A \emph{class diagram} shows a set of classes, interfaces, and collaborations and their relationships.}
			\mycite{A \emph{class} is a description of a set of objects that share the same attributes, operations, relationships, and semantics. [...] 
				\uncover<2->{ An \emph{association} is a structural relationship that specifies that objects of one thing are connected to objects of another. [...]}
				\uncover<3->{When a class participates in an association, it has a specific role that it plays in that relationship; a \emph{role} is just the face the class at the far end of the association presents to the class at the near end of the association. [...]}
				\uncover<4->{When you state a \emph{multiplicity} at the far end of an association, you are specifying that, for each object of the class at the near end, how many objects at the [far] end may exist.}}
		\end{definition}
		\uncover<5->{\small
			\begin{example}{{Example Multiplicities (default=*)}\mysource{\umlrefman}}
				0..* (=*) or 1..* or 0..1 or 1..1 (=1) \ldots\ 2..5 \ldots
			\end{example}
			% default is actually unspecified, meaning that it take any value which is close to *
			% source: \umlrefman and https://coderanch.com/t/100225/engineering/defualt-multiplicity
		}
		\nextcolumn
		\posthandout{\picWhite[page=13,width=\linewidth,trim=225 20 20 40,clip]{printedslides/2021wt/design}}
		%\mynote{\umluserguide:}{\mycite{Class diagrams are the most common diagram found in modeling object-oriented systems.}}
		% deliberatively ignore: fully qualified names such as java::awt::Rectangle, active classes, packages, dependencies, association names
	\end{fancycolumns}
\end{frame}

\subsection{Attributes and Operations of Classes}
\begin{frame}{\insertsubsection}
	\begin{fancycolumns}[widths={51},animation=none]
		\begin{definition}{{Attributes and Operations \mysource{\umluserguide}}}
			\mycite{An \emph{attribute} is a named property of a class that describes a range of values that instances of the property may hold. [...]%
				\uncover<2->{ An \emph{operation} is the implementation of a service that can be requested from any object of the class to affect behavior. In other words, an operation is an abstraction of something you can do to an object that is shared by all objects of that class.}} 
			\uncover<3->{\emph{Static} attributes and operations exist only once for each class and are underlined (opposed to \emph{instance} ones).}
		\end{definition} % \deutsch{Attribute und Operationen)
		\uncover<4->{
			\begin{definition}{{Visibility Modifiers \mysource{\umluserguide}}}
				\begin{itemize}
					\item[\emph{--}] private is available only in this class
					\item[\emph{+}] public is available from each class
					\item[\emph{\#}] protected is available from each subclass
					\item[\emph{\raisebox{-0.9ex}{\~{}}}] package is available in classes of same package
				\end{itemize}
			\end{definition}
		}
		% TODO add primitive types? Int, Float, Boolean
		% deliberatively ignore: responsibilities, modeling of own primitive types
		\nextcolumn
		\posthandout{\picWhite[page=14,width=\linewidth,trim=225 20 20 40,clip]{printedslides/2021wt/design}}
	\end{fancycolumns}
\end{frame}
	
\subsection{Completeness of Attributes and Operations}
\begin{frame}{\insertsubsection}
	\begin{fancycolumns}
		\begin{note}{\umluserguide:}
			\mycite{When drawing a class, you don't have to show every attribute and every operation at once. In fact, in most cases, you can't (there are too many of them to put in one figure) and you probably shouldn't (only a subset of these attributes and operations are likely to be relevant to a specific view). For these reasons, you can elide a class, meaning that you can choose to show only some or none of a class's attributes and operations.}
		\end{note}
	\end{fancycolumns}
\end{frame}

\subsection{Aggregation and Composition of Classes}
\begin{frame}{\insertsubsection}
	\begin{fancycolumns}[animation=none]
		\begin{definition}{{Aggregation \mysource{adapted from \umluserguide}}}
			\emph{Aggregation} is a special association in which one class represents a larger thing (the whole), which consists of smaller things (the parts) (i.e., has-a relationship). In contrast, a plain association between two classes represents a structural relationship between peers, meaning that both classes are conceptually at the same level.
		\end{definition}
		\uncover<2->{
			\begin{definition}{{Composition \mysource{\umluserguide}}}
				\mycite{\emph{Composition} is a form of aggregation, with strong ownership and coincident lifetime as part of the whole. Parts with non-fixed multiplicity may be created after the composite itself, but once created they live and die with it. Such parts can also be explicitly removed before the death of the composite.}
			\end{definition}
		}
		\nextcolumn
		\posthandout{\picWhite[page=16,width=\linewidth,trim=225 20 20 40,clip]{printedslides/2021wt/design}}
	\end{fancycolumns}
\end{frame}

\widexkcdframe{2309}

\subsection{Inheritance Relationships}
\begin{frame}{\insertsubsection}
	\begin{fancycolumns}[animation=none]
		\begin{definition}{Generalization and Abstract Classes}
			\mycite{A \emph{generalization} is a relationship between a general kind of thing (called the superclass or parent) and a more specific kind of thing (called the subclass or child). Generalization is sometimes called an is-a-kind-of relationship: one thing is-a-kind-of a more general thing.} If a superclass cannot be instantiated it is an \emph{abstract class} (also denoted with \textit{italic name}). \deutsch{Generali-\\sierung, abstrakte Klasse}\mysource{\umluserguide}
		\end{definition}
		\uncover<2->{
			\begin{definition}{Interfaces and Realization}
				An \emph{interface} is a special class that only has static non-private attributes, abstract non-private operations, cannot be instantiated, and not be in a generalization relationship. Classes may \emph{realize} an interface. \deutsch{Schnittstelle, Realisierung}
			\end{definition}
		}
		\nextcolumn
		\posthandout{\picWhite[page=18,width=\linewidth,trim=225 20 20 40,clip]{printedslides/2021wt/design}}
	\end{fancycolumns}
\end{frame}

\subsection{Rules and Hints for Class Diagrams}
\begin{frame}{\insertsubsection}
	\begin{fancycolumns}
		\begin{note}{Rules for Class Diagrams}
			\begin{itemize}
				\item Class and attribute names are short nouns or noun phrases
				\item Operation names are verbs or short phrases starting with verbs
				\item Use camel case, whereas the first letter of attributes and operations is not capitalized
				\item A class has an arbitrary number of attributes/operations, any subset thereof shown in a diagram
				\item No cycles in inheritance relationships
				\item Abstract operations only live in abstract classes
				\item Interfaces cannot have private members (attributes/operations)
				\item Each class can only be a \textbf{part} of one composition
			\end{itemize}
		\end{note}
		\nextcolumn
		\begin{example}{{Hints on Inheritance \mysource{\umluserguide}}}
			\mycite{An object of the child class may be used for a variable or parameter typed by the parent, but not the reverse. In other words, generalization means that the child is substitutable for a declaration of the parent. A child inherits the properties of its parents, especially their attributes and operations. Often—but not always—the child has attributes and operations in addition to those found in its parents. An implementation of an operation in a child overrides an implementation of the same operation of the parent; this is known as polymorphism. To be the same, two operations must have the same signature (same name and parameters).}
		\end{example}
	\end{fancycolumns}
\end{frame}

\begin{frame}{Recap: 14 Types of UML Diagrams\ \mytitlesource{\umlspec}}
	\centering\slideMindmapUMLdiagrams{blue}{blue}{blue}{blue}{blue}{blue}{}{}{}%
	
	\uncover<2->{Six most important UML diagrams* discussed in this course
		
		*\tiny\href{https://dl.acm.org/doi/10.1145/1278201.1278205}{John Erickson and Keng Siau. 2007. Theoretical and practical complexity of modeling methods. Commun. ACM 50, 8 (August 2007), 46–51.}}
\end{frame}
