\subsection{Motivation for Modeling}
\begin{frame}{\insertsubsection}
	\begin{fancycolumns}
		\begin{note}{\umluserguide:}
			\mycite{A successful software organization is one that consistently deploys \emph{quality software} that meets the needs of its users. An organization that can develop such software in a \emph{timely and predictable} fashion, with an \emph{efficient and effective use of resources}, both human and material, is one that has a sustainable business.
				
				[...]
				
				Modeling is a central part of all the activities that lead up to the deployment of good software. We build models to \emph{communicate} the desired structure and behavior of our system. We build models to \emph{visualize and control} the system's architecture. We build models to better \emph{understand} the system we are building, often exposing opportunities for \emph{simplification and reuse}. And we build models to \emph{manage risk}.}
		\end{note}
		\nextcolumn
		\begin{note}{\umluserguide:}
			\mycite{We build models of complex systems because we cannot comprehend such a system in its entirety.}
		\end{note}
	\end{fancycolumns}
\end{frame}

\begin{frame}[b]{Recap: Software Engineering vs Programming}
	\slideSEvsProgramming
\end{frame}

\begin{frame}
	\begin{fancycolumns}
		\pic[width=\linewidth,trim=0 300 0 80,clip]{people/bjarne-stroustrup}
		\vspace{-7mm}
		
		\begin{note}{Bjarne Stroustrup (2000) \mysource{\cpp}}
			\mycite{The most important single aspect of software development is to be clear about what you are trying to build.}
		\end{note}
	\end{fancycolumns}
\end{frame}

\subsection{What is System Modeling?}
\begin{frame}{\insertsubsection}
	\begin{fancycolumns}
		\begin{definition}{System Modeling \mysource{\sommerville}}
			\mycite{\emph{System modeling} is the process of developing abstract models of a system, with each model presenting a different view or perspective of that system. [...] Models are used during the requirements engineering process to help derive the \emph{detailed requirements} for a system, during the design process to \emph{describe the system to engineers} implementing the system, and after implementation to \emph{document the system}’s structure and operation.}
		\end{definition}
	\end{fancycolumns}
\end{frame}

\subsection{What is a Model?}
\begin{frame}{\insertsubsection}
	\begin{fancycolumns}
		\begin{definition}{\umluserguide:}
			\mycite{A \emph{model} is a simplification of reality.}
		\end{definition}
		\begin{definition}{\sommerville:}
			\mycite{A \emph{model} is an abstract view of a system that deliberately ignores some system details.}
		\end{definition}
		\begin{note}{{Goals of Models \mysource{\umluserguide}}}
			\begin{itemize}
				\item visualize a system as it is (wanted)
				\item specify the structure or behavior of a system
				\item template to guide construction of a system
				\item document the decisions we have made
			\end{itemize}
		\end{note}
		\nextcolumn
		\begin{note}{\sommerville:}
			\mycite{It is important to understand that a system model is \emph{not a complete representation} of system. It purposely leaves out detail to make it \emph{easier to understand}. A model is an abstraction of the system being studied rather than an alternative representation of that system. A representation of a system should maintain all the information about the entity being represented. An abstraction \emph{deliberately simplifies a system} design and picks out the most salient characteristics.}
		\end{note}
	\end{fancycolumns}
\end{frame}

\subsection{What Language to Use for Modeling?}
\begin{frame}{\insertsubsection}
	\begin{fancycolumns}
		\myexample{Towards a Common Language}{
			\begin{itemize}
				\item Natural language? hard to abstract from details, already used in requirements
				\item Programming language? unfamiliar to people without programming skills in that language, too early to decide for the programming language
				\item Textual language? harder to understand
				\item Graphical language? makes use of our visual abilities, requires common understanding
				\item Problem: engineers need to be aware of all languages being used
				\item Solution: use a graphical language independent of company and domain
			\end{itemize}
		}
	\end{fancycolumns}
\end{frame}

% TODO new slides on History of Modeling Languages?

\xkcdframe{927} % 14+1 standards

\subsection{The Unified Modeling Language (UML)}
\begin{frame}{\insertsubsection}
	\begin{fancycolumns}
		\begin{definition}{UML \mysource{\umlrefman}}
			\mycite{The Unified Modeling Language (UML) is a general-purpose visual modeling language that is used to specify, visualize, construct, and document the artifacts of a software system.}
		\end{definition}
		\pause
		\begin{note}{\umluserguide:}
			\mycite{Modeling yields an understanding of a system. No one model is ever sufficient. Rather, you often need multiple models that are connected to one another [...].}
		\end{note}
	\end{fancycolumns}
\end{frame}

\subsection{Different Kinds of UML Diagrams}
\begin{frame}{\insertsubsection}
	\begin{fancycolumns}
		\begin{definition}{Structure Diagrams \deutsch{Strukturdiagramme}}
			\mycite{\emph{Structure diagrams} show the static structure of the objects in a system. That is, they depict those elements in a specification that are irrespective of time. The elements in a structure diagram represent the meaningful concepts of an application, and may include abstract, real-world and implementation concepts.}\mysource{\umlspec}
		\end{definition}	
		\nextcolumn
		\begin{definition}{Behavior Diagrams \deutsch{Verhaltensdiagramme}}
			\mycite{\emph{Behavior diagrams} show the dynamic behavior of the objects in a system, including their methods, collaborations, activities, and state histories. The dynamic behavior of a system can be described as a series of changes to the system over time.}\mysource{\umlspec}
		\end{definition}
	\end{fancycolumns}
\end{frame}
% TODO add paragraph on interaction diagrams?

\subsection{14 Types of UML Diagrams}
\begin{frame}{\insertsubsection\ \mytitlesource{\umlspec}}
	\only<-4>{\tikzset{notTaughtUMLDiagrams/.style={}}}%
	\only<-5|handout:0>{\slideMindmapUMLdiagrams{}{}{}{}{}{}{visible on={<2->}}{visible on=<3->}{visible on=<4->}}%
	\only<6|handout:0>{\slideMindmapUMLdiagrams{blue}{}{}{}{}{}{}{}{}}%
	\only<7->{\slideMindmapUMLdiagrams{blue}{red}{red}{}{}{}{}{}{}}%
	
	\uncover<5->{Six most important UML diagrams* discussed in this course
		
		*\tiny\href{https://dl.acm.org/doi/10.1145/1278201.1278205}{John Erickson and Keng Siau. 2007. Theoretical and practical complexity of modeling methods. Commun. ACM 50, 8 (August 2007), 46–51.}}
\end{frame}

% TODO talk about SysML? or at least mention in the slides?
