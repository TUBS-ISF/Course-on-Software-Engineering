\begin{frame}
	\begin{fancycolumns}[height=8.5cm]
		\pic[width=\linewidth,trim=0 425 0 75,clip]{people/edsger-dijkstra}
		\vspace{-7mm}
		
		\begin{note}{Edsger W. Dijkstra (1972) \mysource{\thehumbleprogrammer}}
			\mycite{Program testing can be a very effective way to show the presence of bugs, but it is hopelessly inadequate for showing their absence.}
		\end{note}
		% 1930-2002, ACM Turing Award winner
		\nextcolumn
		%\href{}{\includegraphics[width=\linewidth,trim=0 0 0 0,clip]{burt-rutan}}
		\vspace{56mm}
		
		\begin{note}{Burt Rutan \mysource{\href{https://www.routledge.com/Design-of-Biomedical-Devices-and-Systems-4th-edition/King-Fries-Johnson/p/book/9781138723061}{King~et~al.\ 2018}}}
			\mycite{Testing leads to failure, and failure leads to understanding.}
		\end{note}
		% unclear
	\end{fancycolumns}
\end{frame}

\subsection{Test Cases}
\begin{frame}{\insertsubsection\ \deutschertitel{Testfälle}}
	\begin{fancycolumns}[animation=none]
		\uncover<1->{
			\begin{definition}{Systematic Test \mysource{\ludewiglichter}}
				A systematic test is a test, in which
				\begin{itemize}
					\item[1.] the setup is defined,
					\item[2.] the inputs are chosen systematically,
					\item[3.] the results are documented and evaluated by criteria being defined prior to the test. 
				\end{itemize}
			\end{definition}
		}
		\uncover<2->{
			\begin{definition}{Test Case \mysource{\ludewiglichter}}
				In a test, a number of test cases are executed, whereas each test case consists \emph{input values} for a single execution and \emph{expected outputs}. An \emph{exhaustive test} refers a test in which the test cases exercise all the possible inputs.
			\end{definition}
		}
		\nextcolumn
		\uncover<3->{
			\begin{example}{Exhaustive Testing in Practice?}
				\texttt{boolean a, b, c;}\\			
				\texttt{int i, j;}\\~\\
				\texttt{bla(a,b,c)} \uncover<4->{has $2^3 = 8$ possible inputs}\\
				\texttt{blub(i,j)} \uncover<5->{has $(2^{32})^2 = 2^{64} \approx 10^{19}$ inputs}
				\uncover<6->{\begin{itemize}
						\item assuming $10^9$ test cases can be executed in 1~second (cf.\ CPU with more than 1 GHz)
						\item exhaustive test of \texttt{blub} takes $\approx 585$ years
						\item testing for a day would cover less than 0.0005 \% of the inputs
				\end{itemize}}
				\uncover<7->{How to test thousands of such methods several times a day?}
			\end{example}
		}
	\end{fancycolumns}
\end{frame}

\subsection{Test-Case Design}
\begin{frame}{\insertsubsection\ \deutschertitel{Testfallentwurf}}
	\begin{fancycolumns}
		\begin{note}{Goal \mysource{\ludewiglichter}}
			Detect a large number of failures with a low number of test cases. A test case (execution) is \emph{positive}, if it detects a failure, and \emph{successful} if it detects an unknown failure.
		\end{note}
		\begin{definition}{An ideal test case is \ldots \mysource{\ludewiglichter}}
			\begin{itemize}
				\item representative: represents a large number of feasible test cases
				\item failure sensitive: has a high probability to detect a failure
				\item non-redundant: does not check what other test cases already check
			\end{itemize}
		\end{definition}
		\nextcolumn
		\\\vspace{-7mm}
		\hfill
		\begin{tikzpicture}
			\path[small mindmap,
			every node/.style={concept,font=\scriptsize},
			emph/.style={font=\bfseries\scriptsize},
			concept color=foreground!20!background]
			node {Test-Case Design}
			child[concept color=blue!20!background,visible on={<2->}] { node {experience-based} }
			child[concept color=green!20!background,visible on={<3->}] { node {structure-based} 
				child { node {white-box testing} }
			}
			child[concept color=orange!20!background,visible on={<4->}] { node {specification-based} 
				child { node {black-box testing} }
			}
			;
		\end{tikzpicture}
	\end{fancycolumns}
\end{frame}

\subsection{Modulo in Different Programming Languages}
\begin{frame}{\insertsubsection}
	\begin{fancycolumns}[widths={55}]
		\begin{exampletight}{Overview by Torsten Curdt:}
			\centering\picDark[height=60mm]{testing/modulo-negative}
		\end{exampletight}
		\nextcolumn
		\begin{exampletight}{An Own Modulo Implementation}
			\centering\pic[height=60mm]{testing/modulo-with-javadoc-small-nohint} % TODO in 2023 update link
		\end{exampletight}
	\end{fancycolumns}
\end{frame}

\tikzset{emph/.style={thick,draw=blue,fill=blue!10!background},e1/.style={},e2/.style={},e3/.style={},e4/.style={},e5/.style={},e6/.style={},e7/.style={},e8/.style={}}
\renewcommand{\cfg}{ % TODO hack as command already exists, rename consistently
	\begin{tikzpicture}[yscale=-.5,xscale=.866,every node/.style={draw,semithick,circle,fill=background},to/.style={->,>={Stealth[round]},semithick,}]
		\node[e1,e2,e3,e4,e5,e6,e7,e8] (1) at (0,0) {1};
		\node[e1,e4,e5] (2) at (1,1) {2};
		\node[e1,e2,e4,e5,e6,e7,e8] (3) at (0,2) {3};
		\node[e1,e2,e4,e5,e6,e7,e8] (4) at (0,4) {4};
		\node[e1,e5,e7,e8] (5) at (1,5) {5};
		\node[e1,e2,e8] (6) at (0,6) {6};
		\draw[to,e1,e4,e5] (1) to (2);
		\draw[to,e2,e6,e7,e8] (1) to (3);
		\draw[to,e1,e4,e5] (2) to (3);
		\draw[to,e1,e2,e4,e5,e6,e7,e8] (3) to (4);
		\draw[to,e1,e5,e7,e8] (4) to[bend right=20] (5);
		\draw[to,e1,e2,e8] (4) to (6);
		\draw[to,e1,e5,e7,e8] (5) to[bend right=20] (4);
	\end{tikzpicture}
}

\subsection{White-Box Testing}
\begin{frame}{\insertsubsection\ \deutschertitel{Strukturtest}}
	\begin{fancycolumns}[animation=none]
		\begin{definition}{White-Box Testing \mysource{\ludewiglichter}}
			\begin{itemize}
				\setlength\itemsep{.1em}
				\item inner structure of test object is used
				\item idea: coverage of structural elements
				\item code translated into control flow graph
				\item specific test case (concrete inputs)\\derived from logical test case (conditions)\\derived from path in control flow graph
			\end{itemize}
		\end{definition}
		\uncover<2->{
			\begin{exampletight}{}
				\pic[width=\linewidth]{testing/modulo}
			\end{exampletight}
		}
		\nextcolumn
		\\\vspace{33mm}
		{\only<4->{\tikzset{e2/.append style={emph}}}\only<3->{\cfg}}
		%\only<3->{\tikzset{e1/.append style={emph}}\cfg}
	\end{fancycolumns}
\end{frame}
% control flow graphs as activity diagrams: \umluserguide, Chapter~20, Modeling an Operation

\subsection{Coverage Criteria}
\begin{frame}{\insertsubsection\ \deutschertitel{Überdeckungskriterien}}
	\begin{fancycolumns}[animation=none]
		\begin{definition}{Coverage Criteria \mysource{\ludewiglichter}}
			\begin{itemize}
				\item[1.] statement coverage \deutsch{Anweisungsüberdeck.}: all statements are executed for at least one test case
				\uncover<3->{\item[2.] branching coverage \deutsch{Zweigüberdeckung}: statement coverage and for each branching statement all branches have been exercised} % TODO not so easy to define as percentage
				\uncover<4->{\item[3.] term coverage \deutsch{Termüberdeckung}: branching coverage and terms ($n$) used in a branching statement are combined exhaustively ($2^n$)\hfill(simplified)}
				% TODO discuss path coverage?
			\end{itemize}
		\end{definition}
		\vspace{-1mm}
		\uncover<2->{\small
			\begin{example}{In Practice}
				100\% statement coverage not feasible in presence of dead code or some unreachable error handling
				
				\uncover<5->{100\% term coverage not feasible for certain dependencies between choices: \texttt{even() || odd()}}
			\end{example}
		}
		\nextcolumn
%				\\[5mm]
				\cfg%
	\end{fancycolumns}
\end{frame}

\subsection{Statement Coverage}
\begin{frame}
	\begin{fancycolumns}[t,animation=none]
		\begin{exampletight}{Statement Coverage for Modulo Example}
			\pic[width=\linewidth]{testing/modulo}
		\end{exampletight}
		\begin{example}{First Test Case}
			\setlength\tabcolsep{.5mm}
			\begin{tabularx}{\textwidth}{rl}				
				path: & \uncover<2->{1}\uncover<3->{, 2, 3, 4}\uncover<4->{, 5, 4}\uncover<5->{, 6}\\
				logical test case: & \uncover<3->{$b < 0$}\uncover<4->{ $\wedge$ $a > -b$}\uncover<5->{ $\wedge$ $a+b \leq -b$}\\
				specific test case: & \uncover<6->{$a = 5$, $b = -3$}\\
				expected result: & \uncover<7->{$m = 2$}
			\end{tabularx}
		\end{example}
		\nextcolumn
		\\[5mm]
		\only<1|handout:0>{\cfg}%
		\only<2|handout:0>{\tikzset{e3/.append style={emph}}\cfg}%
		\only<3|handout:0>{\tikzset{e4/.append style={emph}}\cfg}%
		\only<4|handout:0>{\tikzset{e5/.append style={emph}}\cfg}%
		\only<5->{\tikzset{e1/.append style={emph}}\cfg}%
		\uncover<8->{\correct}
	\end{fancycolumns}
\end{frame}

\subsection{Branching Coverage}
\begin{frame}[plain]
	\begin{fancycolumns}[t,animation=none]
		\begin{exampletight}{Branching Coverage for Modulo Example}
			\pic[width=\linewidth]{testing/modulo}
		\end{exampletight}
		\vspace{-1mm}
		\begin{example}{First Test Case}
			\setlength\tabcolsep{.5mm}
			\begin{tabularx}{\textwidth}{rl}				
				path: & 1, 2, 3, 4, 5, 4, 6\\
				logical test case: & $b < 0$ $\wedge$ $a > -b$ $\wedge$ $a+b \leq -b$\\
				specific test case: & $a = 5$, $b = -3$ and $m = 2$
			\end{tabularx}
		\end{example}
		\vspace{-1mm}
		\begin{example}{Second Test Case}
			\setlength\tabcolsep{.5mm}
			\begin{tabularx}{\textwidth}{rl}				
				path: & \uncover<2->{1}\uncover<3->{, 3, 4}\uncover<4->{, 6}\\
				logical test case: & \uncover<3->{$b \geq 0$}\uncover<4->{ $\wedge$ $0 \leq a \leq b$}\\
				specific test case: & \uncover<5->{$a = 0$, $b = 5$}\uncover<6->{ and $m = 0$}
			\end{tabularx}
		\end{example}
		\nextcolumn
		\\[5mm]
		{\tikzset{e1/.append style={emph}}\cfg}
		\only<1|handout:0>{\cfg}%
		\only<2|handout:0>{\tikzset{e3/.append style={emph}}\cfg}%
		\only<3|handout:0>{\tikzset{e6/.append style={emph}}\cfg}%
		\only<4->{\tikzset{e2/.append style={emph}}\cfg}%
		\uncover<7->{\correct}
		\\[15mm]
		\uncover<8->{\begin{note}{Ternary Operator in Statement 5}
				\begin{itemize}
					\item could also be treated as branching statement
					\item how to adapt the control flow graph then?
					\item are the two test cases still sufficient?
				\end{itemize}
		\end{note}}
	\end{fancycolumns}
\end{frame}

\subsection{Term Coverage}
\begin{frame}[plain]
	\begin{fancycolumns}[t,animation=none]
		\begin{exampletight}{Term Coverage for Modulo Example}
			\pic[width=\linewidth]{testing/modulo}
		\end{exampletight}
		\vspace{-1mm}
		\begin{example}{}
			\setlength\tabcolsep{.5mm}
			\begin{tabularx}{\textwidth}{rl}				
				logical test case: & $b < 0$ $\wedge$ $a > -b$ $\wedge$ $a+b \leq -b$\\
				specific test case: & $a = 5$, $b = -3$ and $m = 2$\uncover<14->{\correct}
			\end{tabularx}
		\end{example}
		\vspace{-1mm}
		\begin{example}{}
			\setlength\tabcolsep{.5mm}
			\begin{tabularx}{\textwidth}{rl}				
				logical test case: & $b \geq 0$ $\wedge$ $0 \leq a \leq b$\\
				specific test case: & $a = 0$, $b = 5$ and $m = 0$\uncover<14->{\correct}
			\end{tabularx}
		\end{example}
		\vspace{-1mm}
		\begin{example}{}
			\setlength\tabcolsep{.5mm}
			\begin{tabularx}{\textwidth}{rl}				
				path: & \uncover<8->{1}\uncover<9->{, 3, 4}\uncover<10->{, 5, 4}\uncover<11->{, 6}\\
				logical test case: & \uncover<9->{$b \geq 0$}\uncover<10->{ $\wedge$ $a < 0$}\uncover<11->{ $\wedge$ $0 \leq a+b \leq b$}\\
				specific test case: & \uncover<12->{$a = -2$, $b = 5$}\uncover<13->{ and $m = 3$}\uncover<14->{\correct}
			\end{tabularx}
		\end{example}
		\nextcolumn
		\\[5mm]
		{\tikzset{e1/.append style={emph}}\cfg}
		{\tikzset{e2/.append style={emph}}\cfg}
		\only<7|handout:0>{\cfg}%
		\only<8|handout:0>{\tikzset{e3/.append style={emph}}\cfg}%
		\only<9|handout:0>{\tikzset{e6/.append style={emph}}\cfg}%
		\only<10|handout:0>{\tikzset{e7/.append style={emph}}\cfg}%
		\only<11->{\tikzset{e8/.append style={emph}}\cfg}\\[5mm]
		\uncover<2->{$\neg(m < 0)$}\uncover<3->{ and $m > b$\correct}\\[5mm]
		\uncover<2->{$\neg(m < 0)$}\uncover<4->{ and $\neg(m > b)$\correct}\\[5mm]
		\uncover<5->{$m < 0$ and $m > b$}\uncover<6->{ impossible\only<15->{*}\alsocorrect}\\[5mm]
		\uncover<5->{$m < 0$ and $\neg(m > b)$}\uncover<13->{\correct}\\[3mm]
		\uncover<15->{\tiny*\,see third part of the lecture}
	\end{fancycolumns}
\end{frame}
