% UNIVERSITY-SPECIFIC ADJUSTMENTS
% usage: \def\university{ulm}

\ifdefined\university
\else
\def\university{}
\fi

\newcommand{\ifuniversity}[3][]{\ifthenelse{\equal{\university}{#2}}{#3}{#1}}
\newcommand{\unlessuniversity}[3][]{\ifuniversity[#3]{#2}{#1}}
\newcommand{\inputuniversity}[1]{\IfFileExists{#1-\university}{\input{#1-\university}}{}}

% TITLE
\title{Introduction to Software Engineering} % default title for the course
\ifuniversity{tubs}{\title[SE1]{Software Engineering 1}}

% LECTURERS AND TUTORS
\makeatletter
\let\author@old\author
\renewcommand{\author}[1]{
	\ifuniversity{tubs}{\author@old{Thomas Thüm, Christopher Rau}}
	\ifuniversity{recording}{\author@old{#1}}
	\ifuniversity{anonymous}{\author@old{Anonymous Authors}}
	\ifuniversity{}{\author@old{#1}}
}
\makeatother

% CONTENT OVERVIEW
\newcommand{\lectureseriesoverview}[1][]{ % shown at the start and end of each lecture
	\setlength{\leftmargini}{3.5ex}
	\begin{fancycolumns}[t,columns=3,animation=none]
		\mydefinition{\emphifequal{#1}{I}{Part A: Development of \mbox{\emph{Correct} Software}}}{
			\begin{enumerate}
				\item {\emphifequal{#1}{1}{Introduction}}
				\item {\emphifequal{#1}{2}{Implementation}}
				\item {\emphifequal{#1}{3}{Testing}}
				\item {\emphifequal{#1}{4}{Maintenance}}
%				\item {\emphifequal{#1}{4}{Software Changes}}
%				\item {\emphifequal{#1}{5}{Version Control}}
			\end{enumerate}
		}
		\nextcolumn
		\mydefinition{\emphifequal{#1}{II}{Part B: Development in \emph{Schedule} \& \emph{Budget}}}{
			\begin{enumerate}
				\setcounter{enumi}{4}
				\item {\emphifequal{#1}{5}{Project Management}}
				\item {\emphifequal{#1}{6}{Development Process}}
			\end{enumerate}
		}
		\nextcolumn
		\mydefinition{\emphifequal{#1}{III}{Part C: Development of \mbox{\emph{Needed} Software}}}{
			\begin{enumerate}
				\setcounter{enumi}{6}
				\item {\emphifequal{#1}{7}{Requirements}}
				\item {\emphifequal{#1}{8}{System Modeling}}
				\item {\emphifequal{#1}{9}{Software Architecture}}
				\item {\emphifequal{#1}{10}{Software Design}}
				\item {\emphifequal{#1}{11}{Software Reuse}}
				\item {\emphifequal{#1}{12}{Guest Lecture}}
			\end{enumerate}
		}
	\end{fancycolumns}
}
\newcommand{\skipPartA}{\addtocounter{lecture}{4}}
\newcommand{\skipPartB}{\addtocounter{lecture}{2}}

% COLOR SCHEME, LOGOS, PICTURES
\newcommand*{\lectureqr}[2][]{} % as default there is no qr code

\ifuniversity{recording}{
	\renewcommand{\deutsch}[1]{} % no German in English recordings
	\mode<beamer>{\renewcommand{\pic}[2][]{\includegraphics[#1]{#2}}} % avoid annoying tool tips during the recording
%	\setpicture[550]{jun22-clouds3} % default picture
}{
	\hypersetup{linkcolor=foreground, citecolor=red, filecolor=red, runcolor=red, urlcolor=red, colorlinks=true} % emphasizing links, but not when recording
	% TODO links do not work, find color for links in darkmode and lighmode
}

\ifuniversity{tubs}{
	\setpaths{
		{../pics/tubs/}
		{../pics/tubs/2024wt/}
	}
	
	\renewcommand*{\lectureqr}[2][width=.33\linewidth]{
		\begin{flushright}
			\picDark[#1]{qr#2}
		\end{flushright}
	}
	
	\usepackage{tubscolors}
	% redefine colors used in fancybeamer
	% Orange, OrangeLight (Yellow), OrangeDark (tubsRed)
	% Green, GreenLight, GreenDark
	% Blue, BlueLight, BlueDark
	% Violet, VioletLight, VioletDark
	% tubsGray (tubsGray60), tubsLightGray (tubsGray20)
	% tubsSecondaryLight, tubsSecondaryMedium, tubsSecondary, tubsSecondaryDark
	% x20, x40, x60, x80, x100

	\AtEndPreamble{
		\colorlet{red}{tubsRed}
		\colorlet{accent}{background}
		\colorlet{accenttwo}{red}
		\setbeamercolor{logobox}{bg=background}
		\setbeamercolor{myfooter}{fg=foreground,bg=accent}
		\setbeamercolor{subtitlebox}{fg=foreground,bg=accent}
		\fancylogos{tubs,isf}
		\ifdarkmode
			\colorlet{red}{red!90!white}
			\colorlet{green}{green!90!white}
			\colorlet{blue}{blue!90!white}
			\colorlet{orange}{orange!90!white}
			\colorlet{accenttwo}{accenttwo!25!darkgray}
			\setbeamercolor{mypagenumber}{fg=foreground,bg=accenttwo}
			\setbeamercolor{titlebox}{fg=foreground,bg=accenttwo}
			\fancylogos{tubs-black,isf}
		\fi

		% hack to force use of changed colors
		\UpdateBoxColor{definition}{orange}
		\UpdateBoxColor{example}{blue}
		\UpdateBoxColor{note}{red}
	}

	\setpicture[150]{campus/04-25-iz1} % default picture
}

\ifuniversity{anonymous}{
	\fancylogos{}
	\renewcommand{\pic}[2][]{\includegraphics[#1]{#2}}
	\renewcommand{\deutsch}[1]{}
	\renewcommand{\deutschertitel}[1]{}
}
