\xkcdframe{844}

\subsection{Activity Diagrams}
\begin{frame}{\insertsubsection}
	\begin{fancycolumns}[animation=none]
		\onlyhandout{\pic[page=3,width=\linewidth,trim=60 140 450 150,clip]{modeling/09-modeling_drawings}}
		\nextcolumn
		\begin{definition}{Activity Diagram \deutsch{Aktivitätsdiagramm}}
			An \emph{activity diagram} is a diagram visualizing activities and their order of execution. An activity diagram contains \emph{activities} (rounded box) that are connected by means of \emph{flows} (solid arrows). The execution begins at the \emph{initialization} (filled circle) and ends with the \emph{completion} node (bull's eye). \deutsch{Aktivität, Fluss, Startzustand, Endzustand}
		\end{definition}
		\pause%
		\begin{note}{Rules for Activity Diagrams}
			\begin{itemize}
				\item exactly one initialization node
				\item at least one activity
				\item every activity has one incoming and one outgoing flow
				\item every activity is reachable from initialization
				\item completion is reachable from every activity
			\end{itemize}
		\end{note}
		% What is the difference between activity and action? \umluserguide
	\end{fancycolumns}
\end{frame}

% TODO make sure that example contains activities (i.e., not only nouns)
\begin{frame}{Example of Sequential Activities}
	\onlyhandout{\pic[page=4,width=\linewidth,trim=60 80 80 120,clip]{modeling/09-modeling_drawings}}
\end{frame}

\subsection{Branching and Merging in Activity Diagrams}
\begin{frame}{\insertsubsection}
	\begin{fancycolumns}[animation=none]
		\onlyhandout{\pic[page=5,width=\linewidth,trim=60 120 450 150,clip]{modeling/09-modeling_drawings}}
		\nextcolumn
		\begin{definition}{{Branching and Merging \mysource{\umluserguide}}}
			\textbf{Motivation}: model control flow that depends on certain conditions (i.e., actions that may happen)
			
			\textbf{Branching}: A \emph{branch} has exactly one incoming and two or more outgoing flows. Each outgoing flow has a Boolean expression called \emph{guard}, which is evaluated on entering the branch. \deutsch{Verzweigung}
			
			\textbf{Merging}: A \emph{merge} has two or more incoming and exactly one outgoing flow. \deutsch{Zusammenführung}
		\end{definition}
		\pause
		\begin{note}{Further Rules for Activity Diagrams}
			\begin{itemize}
				\item guards on outgoing flows should not overlap (flow of control is unambiguous)
				\item guards should cover all possibilities (flow of control does not freeze)
				\item keyword \emph{else} possible for one guard \deutsch{sonst}
			\end{itemize}
		\end{note}
	\end{fancycolumns}
\end{frame}

\begin{frame}{Example of Conditional Activities}
	\onlyhandout{
		\pic[page=1,width=\linewidth,trim=60 80 80 150,clip]{modeling/09-modeling_drawings}
	}
\end{frame}

\subsection{Forking and Joining in Activity Diagrams}
\begin{frame}{\insertsubsection}
	\begin{fancycolumns}[animation=none]
		\onlyhandout{\pic[page=2,width=\linewidth,trim=60 140 450 120,clip]{modeling/09-modeling_drawings}}
		\nextcolumn
		\begin{definition}{{Forking and Joining \mysource{\umluserguide}}}
			\textbf{Motivation}: model concurrent control flows (i.e., activities that run in parallel)
			
			\textbf{Forking}: A \emph{fork} (thick horizontal or vertical line) has exactly one incoming and two or more outgoing flows. \deutsch{Gabelung}
			
			\textbf{Joining}: A \emph{join} (thick horizontal or vertical line) has two or more incoming and exactly one outgoing flow. \deutsch{Vereinigung}
		\end{definition}
		\pause
		\begin{note}{Further Rules for Activity Diagrams}
			\begin{itemize}
				\item branched paths must be merged eventually \deutsch{letztendlich}
				\item forked paths must be joined eventually
				\item only outgoing edges of branch nodes have guards
			\end{itemize}
		\end{note}
	\end{fancycolumns}
\end{frame}

	\begin{frame}{Example of Concurrent Activities}
		\onlyhandout{\pic[page=6,width=\linewidth,trim=60 100 10 120,clip]{modeling/09-modeling_drawings}}
	\end{frame}

\subsection{Swimlanes in Activity Diagrams}
\begin{frame}{\insertsubsection}
	\begin{fancycolumns}[animation=none]
		\onlyhandout{\pic[page=1,width=\linewidth,trim=60 100 150 150,clip]{modeling/09-modeling_drawings}}
		\nextcolumn
		\begin{definition}{{Swimlanes \mysource{\umluserguide}}}
			\textbf{Motivation}: group activities according to responsibilities
			
			\textbf{Swimlane}: An activity diagram may have no or at least two swimlanes. A \emph{swimlane} (rectangle) represents a high-level responsibility activities within an activity diagram. \deutsch{Verantwortlichkeitsbereiche}
		\end{definition}
		\pause
		\begin{note}{Further Rules for Activity Diagrams}
			\begin{itemize}
				\item each swimlane has a name unique within its diagram
				\item every activity belongs to exactly one swimlane
				\item only flows may cross swimlanes
			\end{itemize}
		\end{note}
	\end{fancycolumns}
\end{frame}

\begin{frame}{Recap: 14 Types of UML Diagrams\ \mytitlesource{\umlspec}}
	\slideMindmapUMLdiagrams{blue}{blue}{red}{}{}{}{}{}{}
\end{frame}

