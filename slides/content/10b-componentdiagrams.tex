\begin{frame}{Recap: 14 Types of UML Diagrams\ \mytitlesource{\umlspec}}
	\slideMindmapUMLdiagrams{blue}{blue}{blue}{red}{}{}{visible on={<2->}}{}{}
\end{frame}

\subsection{Component Diagrams}
\begin{frame}[label=componentslide]{\insertsubsection\ \deutsch{Komponentendiagramm}}
	\begin{fancycolumns}[animation=none]
		\begin{definition}{Component Diagram \mysource{adapted from \umluserguide}}
			A \emph{component} is a replaceable part of a system that conforms to and provides the realization of a set of interfaces. An \emph{interface} is a collection of operations that specify a service that is provided by or requested from a class or component. An interface that a component realizes is called a \emph{provided interface}, meaning an interface that the component provides as a service to other components. The interface that a component uses is called a \emph{required interface}, meaning an interface that the component conforms to when requesting services from other components. \deutsch{Komponente, angebotene/benötigte Schnittstelle}
		\end{definition}
		\nextcolumn
		\posthandout{\pic[page=16,width=\linewidth,trim=225 20 25 40,clip]{printedslides/2021wt/architecture}}
	\end{fancycolumns}
\end{frame}

\begin{frame}{Example of a Component Diagram}
	\posthandout{\pic[page=17,width=\linewidth,trim=0 20 0 40,clip]{printedslides/2021wt/architecture}}
\end{frame}

% TODO all current examples are missing situations in which a provided interface is to more than 1 required interfaces

\subsection{Hierarchical Component Diagrams}
\begin{frame}[label=nestedcomponentsslide]{\insertsubsection}
	\begin{fancycolumns}[animation=none]
		\begin{definition}{Nesting of Components\mysource{adapted from \umluserguide}}
			\textbf{Motivation}: decompose/structure large systems
			
			\textbf{Nesting} \deutsch{Verschachtelung}: A component may contain any number of \emph{subcomponents}. \deutsch{Teilkomponenten}
			
			\textbf{Ports and Delegates}: A \emph{port} is an explicit window into an encapsulated component. A \emph{delegate} connects provided or required interfaces with ports. 
		\end{definition}
		\nextcolumn
		\posthandout{\pic[page=18,width=\linewidth,trim=225 20 25 40,clip]{printedslides/2021wt/architecture}}
	\end{fancycolumns}
\end{frame}

\subsection{Rules for Component Diagrams}
\begin{frame}{\insertsubsection}
	\begin{fancycolumns}
		\begin{note}{Rules for Component Diagrams}
			\begin{itemize}
				\item component names are unique
				\item a component may have any number of required or provided interfaces
				\item every required interface is connected to provided interface
				\item every component is directly or indirectly connected to every other component
				\item subcomponents may be nested to any level
				\item when subcomponents communicate to a higher-level component, they need to communicate via ports
			\end{itemize}
		\end{note}
	\end{fancycolumns}
\end{frame}

\begin{frame}
	\begin{fancycolumns}[height=8.5cm]
		\pic[width=\linewidth,trim=175 0 0 0,clip]{people/gordon-bell}
		\vspace{-7mm}
		
		\begin{note}{Gordon Bell \mysource{\href{https://dl.acm.org/doi/10.1145/1968.381154}{Bentley 1984}}}
			\mycite{The cheapest, fastest, and most reliable components are those that aren’t there.}
		\end{note}
	\end{fancycolumns}
\end{frame}

\begin{frame}{Recap: 14 Types of UML Diagrams\ \mytitlesource{\umlspec}}
	\only<1|handout:0>{\slideMindmapUMLdiagrams{blue}{blue}{blue}{blue}{}{}{}{}{}}%
	\only<2->{\slideMindmapUMLdiagrams{blue}{blue}{blue}{blue}{red}{red}{}{}{}}%
\end{frame}
