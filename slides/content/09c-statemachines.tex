\subsection{Activity and State Machine Diagrams}
\begin{frame}{\insertsubsection}
	\begin{fancycolumns}
		\begin{note}{\umluserguide:}
			We can visualize the dynamics of execution in two ways: by emphasizing the flow of control from activity to activity (\emph{activity diagrams}) or by emphasizing the potential states and transitions among those states (\emph{state machine diagrams}).
		\end{note}
	\end{fancycolumns}
\end{frame}

\subsection{State Machine Diagrams}
\begin{frame}[label=statemachineslide]{\insertsubsection}
	\begin{fancycolumns}
		\begin{definition}{State Machine Diagram \deutsch{Zustandsdiagramm}}
			A \emph{state machine diagram} specifies the sequences of states the (a part of) the system goes through during its lifetime in response to events, together with its responses to those events. Every \emph{state} (rectangle with rounded corners) is characterized by a condition or situation. An \emph{event} is an occurrence of a stimulus that can trigger a state transition. A \emph{transition} (solid arrow) is a relationship between two states. \deutsch{Zustand, Ereignis, Zustandsübergang} \mysource{adapted from \umluserguide}
		\end{definition}
		\pause%
		\begin{note}{Rules for State Machine Diagrams}
			there is a single \emph{initial state} (filled circle) and any number of \emph{final states} (bull's eye) \deutsch{Start- und Zielzustand} --- see exception below
		\end{note}
		\nextcolumn
		\onlyhandout{\pic[page=8,width=\linewidth,trim=460 200 40 80,clip]{modeling/09-modeling_drawings}}
	\end{fancycolumns}
\end{frame}

	\begin{frame}[t]{Example of a State Machine Diagram}
		\onlyhandout{
			\begin{tikzpicture}
				\node[anchor=south,inner sep=0] at (0,2.1) {\pic[page=9,width=.86\linewidth,trim=0 0 120 120,clip]{modeling/09-modeling_drawings}};
				\fill[white] (-1,2) rectangle (5,3.5);
				\fill[white] (-3,2) rectangle (-6,3.5);
			\end{tikzpicture}
		}
	\end{frame}

% TODO introduce notation for actions?

\subsection{Hierarchical State Machine Diagrams}
\begin{frame}{\insertsubsection}
	\begin{fancycolumns}
		\begin{definition}{{Simple and Composite States \mysource{\umluserguide}}}
			\textbf{Motivation}: avoid duplicated transitions, improve overview in complex state machine diagrams
			
			\textbf{Simple State}: \mycite{A \emph{simple state} is a state that has no substructure.} \deutsch{einfacher Zustand}
			
			\textbf{Composite State}: \mycite{A state that has substates (i.e., nested states) is called a \emph{composite state}.} \deutsch{komplexer Zustand}
		\end{definition}
		\pause
		\begin{note}{Rules for State Machine Diagrams}
			\begin{itemize}
				\item every composite state has its own single \emph{initial state} \deutsch{Startzustand}
				\item substates may be nested to any level
			\end{itemize}
		\end{note}
		\nextcolumn
		\onlyhandout{
			\begin{tikzpicture}
				\node[anchor=south,inner sep=0] at (-5,2.1) {\pic[page=9,width=\linewidth,trim=0 0 150 120,clip]{modeling/09-modeling_drawings}};
				\fill[white] (-5.5,2) rectangle (0,3);
				\fill[white] (-8.5,2) rectangle (-6.5,3);
			\end{tikzpicture}
		}
	\end{fancycolumns}
\end{frame}

\begin{frame}{Recap: 14 Types of UML Diagrams\ \mytitlesource{\umlspec}}
	\slideMindmapUMLdiagrams{blue}{blue}{blue}{red}{}{}{}{}{}
\end{frame}

