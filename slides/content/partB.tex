% TODO L06 PROJECT MANAGEMENT

\ifuniversity{tubs}{\date{November 19, 2024}}

\author{Thomas Thüm}
\lecture{Project Management}{management}

\begin{frame}{\insertsubtitle}
	\alt<3->{%
		\projectcartoon{04}{how the programmer implemented it\\~\\\textbf{implementation}} % se02 implementation
		\projectcartoon{05}{what the beta testers received\\~\\\textbf{software\\testing}} % se03 testing
		\projectcartoon{18}{how patches were applied\\~\\\textbf{software\\changes}} % se04 changes
		\projectcartoon{12}{when it was delivered\\~\\~\\\textbf{version control}} % se05 version control
	}{%
		\hprojectcartoon{04}{how the programmer implemented it\\~\\\textbf{implementation}} % se02 implementation
		\hprojectcartoon{05}{what the beta testers received\\~\\\textbf{software\\testing}} % se03 testing
		\hprojectcartoon{18}{how patches were applied\\~\\\textbf{software\\changes}} % se04 changes
		\hprojectcartoon{12}{when it was delivered\\~\\~\\\textbf{version control}} % se05 version control
	}%
	\uncover<2->{\hprojectcartoon{09}{how the customer was billed\\~\\\textbf{project management}}} 
\end{frame}

\section{Introduction to Project Management}
\subsection{Software Development Project}
\begin{frame}{\insertsubsection\ \mytitlesource{\ludewiglichter}}
	\begin{fancycolumns}
		\begin{definition}{Software Development Project}
			\begin{itemize}
				\item aka.\ software engineering project
				\item temporary activity with start and end date
				\item has goals
				\begin{itemize}
					\item creation / modification of a software product
					\item creation / modification of components for future projects
					\item gain experience / knowledge
					\item capacity utilization \deutsch{Mitarbeiterauslastung}
					\item \ldots
				\end{itemize}
				\item is successful if goals are largely fulfilled
			\end{itemize}
		\end{definition}
	\end{fancycolumns}
\end{frame}

\subsection{Project Management}
\begin{frame}{\insertsubsection\ \mytitlesource{\sommerville}}
	\begin{fancycolumns}
		\begin{note}{Motivation}
			\mycite{Good management cannot guarantee project success. However,
				bad management usually results in project failure: The software may be delivered late,
				cost more than originally estimated, or fail to meet the expectations of customers.}
		\end{note}
		\pause
		\begin{definition}{Goals of Project Management}
			\begin{itemize}
				\item \mycite{deliver the software to the customer at the agreed \emph{time}
					\item keep overall \emph{costs} within budget
					\item deliver software that meets the customer’s \emph{expectations}
					\item maintain a coherent and well-functioning development \emph{team}}
			\end{itemize}
		\end{definition}
		\nextcolumn
		\pause
		\begin{example}{Project Management Depends on \ldots}
			\begin{tabularx}{\textwidth}{rX}				
				\emph{company size} & large companies have management hierarchies and reporting / budgeting / approval processes\\
				\emph{customers} & external customers (i.e., government agencies) usually have policies\\
				\emph{software size} & large systems require multiple development teams in different companies / locations\\
				\emph{software type} & safety-critical systems require all design decisions to be documented\\
				%\emph{org.\ culture} & some companies take higher risks\\
				\emph{dev.\ process} & project management heavily depends on process model
			\end{tabularx}
		\end{example}
	\end{fancycolumns}
\end{frame}

\xkcdframe{1319} % planning

\subsection{Activities in Project Management}
\begin{frame}{\insertsubsection\ \mytitlesource{\sommerville}}
	\small
	\begin{fancycolumns}
		\begin{definition}{Project Planning \deutsch{Projektplanung}}
			\mycite{Project managers are responsible for \emph{planning, estimating, and scheduling} project development and assigning people to tasks. They supervise the work to ensure that it is carried out to the required standards, and they \emph{monitor progress} to check that the development is on time and within budget.}
		\end{definition}
		\pause
		\begin{definition}{Risk Management \deutsch{Risikomanagement}}
			\mycite{Project managers have to \emph{assess the risks} that may affect a project, monitor these risks, and take action when problems arise.}
		\end{definition}
		\pause
		\begin{definition}{People Management \deutsch{Mitarbeiterführung}}
			\mycite{Project managers are responsible for \emph{managing a team} of people. They have to choose people for their team and establish ways of working that lead to effective team performance.}
		\end{definition}
		\nextcolumn
		\pause
		\begin{definition}{Reporting \deutsch{Berichterstattung}}
			\mycite{Project managers are usually responsible for \emph{reporting on the progress} of a project to customers and to the managers of the company developing the software. They have to be able to communicate at a range of levels, from detailed technical information to management summaries.}% They have to write concise, coherent documents that abstract critical information from detailed project reports. They must be able to present this information during progress reviews.
		\end{definition}
		\pause
		\begin{definition}{Proposal Writing \deutsch{Projektbeantragung}}
			\mycite{The first stage in a software project may involve writing a proposal to \emph{win a contract} to carry out an item of work. The proposal describes the objectives of the project and how it will be carried out. It usually includes \emph{cost and schedule estimates} and justifies why the project contract should be awarded to a particular organization or team. Proposal writing is a critical task as the survival of many software companies depends on having enough proposals accepted and contracts awarded.}
		\end{definition}
	\end{fancycolumns}
\end{frame}

\subsection{Risk Management}
\begin{frame}{\insertsubsection\ \mytitlesource{\sommerville}}
	\begin{fancycolumns}
		%\mynote{Motivation}{\mycite{Risk management is one of the most important jobs for a project manager. You can think of a risk as something that you’d prefer not to have happen.}}
		\begin{definition}{Risk}
			\setlength\tabcolsep{1mm}
			\begin{tabularx}{\textwidth}{rX}				
				\emph{Probability} & insignificant, low, moderate, high, very high\\
				\emph{Severity} & insignificant, tolerable, serious, catastrophic
			\end{tabularx}
		\end{definition}
		\pause
		\setlength\tabcolsep{1mm}
		\begin{note}{Classification of Risks}
			\begin{tabularx}{\textwidth}{rX}				
				\emph{Project Risks} & affect project schedule or resources: loss of an experienced system architect may result in longer development time\\
				\emph{Product Risks} & affect software quality: purchased component may not scale\\
				\emph{Business Risks} & affect organization / company: product of a competitor may reduce number of sales
			\end{tabularx}
		\end{note}
		\nextcolumn
		\pause
		\begin{definition}{Stages in Risks Management}
			\setlength\tabcolsep{1mm}
			\begin{tabularx}{\textwidth}{rX}				
				\emph{1. Risk Identification} & identify possible project, product, and business risks\\
				\emph{2. Risk Analysis} & assess likelihood and consequences\\
				\emph{3. Risk Planning} & plan how to address risks: avoidance or minimization of effects\\
				\emph{4. Risk Monitoring} & regularly assess risks and revise plans if needed
			\end{tabularx}
		\end{definition}
		\pause
		\begin{example}{Risks in Agile Development}
			reduced risks for requirements changes, increased risks for loss of staff due to fewer documentation
		\end{example}
	\end{fancycolumns}
\end{frame}

\subsection{People Management}
\begin{frame}{\insertsubsection\ \mytitlesource{\sommerville}}
	\begin{fancycolumns}
		\begin{note}{Motivation}
			\mycite{The people working in a software organization are its \emph{greatest assets}. It is expensive to recruit and retain good people, and it is up to software managers to ensure that the engineers working on a project are as \emph{productive} as possible. In successful companies and economies, this productivity is achieved when people are respected by the organization and are assigned responsibilities that reflect their skills and experience.}
		\end{note}
		\pause
		\begin{example}{In Practice}
			\mycite{Software engineers often have strong \emph{technical skills} but may lack the softer skills that enable them to \emph{motivate and lead a project development team}.}
		\end{example}
		\nextcolumn
		\pause
		\begin{definition}{Critical Factors}
			\setlength\tabcolsep{1mm}
			\begin{tabularx}{\textwidth}{rX}
				\emph{1. Consistency} & treat people comparably with similar rewards\\
				\emph{2. Respect} & let all people contribute and respect their differences in skills\\
				\emph{3. Inclusion} & consider views of least experienced peoples\\
				\emph{4. Honesty} & manager is honest about own skills and team performance
			\end{tabularx}
		\end{definition}
		\pause
		\begin{note}{Teamwork}
			\small\mycite{Most professional software is developed by project teams that range in size from two to several hundred people. However, as it is impossible for everyone in a large group to work together on a single problem, \emph{large teams are usually split} into a number of smaller groups. Each group is responsible for developing part of the overall system.}% The best size for a software engineering group is 4 to 6 members, and they should never have more than 12 members. When groups are small, communication problems are reduced. Everyone knows everyone else, and the whole group can get around a table for a meeting to discuss the project and the software that they are developing.
		\end{note}
	\end{fancycolumns}
\end{frame}

\lessonslearned{
	\item Software development projects
	\item Project management: goals, influences, activities
	\item Risk and people management
	\item Next: How to create realistic project plans?
}{
	\item \sommerville\mychapter{22} Project Management
	\item \ludewiglichter\mychapter{7.2} \deutsch{Software-Projekte}
}{
	\begin{enumerate}
		\item<+-> Form groups of 2-3 students
		\item<+-> Risk identification and analysis: Give an example for a risk of a messenger app and specify probability, severity, and classification
		\item<+-> Risk planning and monitoring: How to address the risk mentioned by one of your colleagues? What could change during the project?
		\item<+-> Repeat 2. and 3. with other examples
	\end{enumerate}
}

\section{Project Planning and Scheduling}
\subsection{Project Planning}
\begin{frame}{\insertsubsection}
	\begin{fancycolumns}
		\begin{note}{{Anonymous}} % TODO add citation?
			\mycite{A goal without a plan is just a wish.}
		\end{note}
	\end{fancycolumns}
\end{frame}

\begin{frame}{\insertsubsection\ \mytitlesource{\sommerville}}
	\begin{fancycolumns}
		\begin{note}{At the Proposal Stage}
			\begin{itemize}
				\item when bidding for a contract
				\item enough resources?
				\item price for the bidding?
				\item not all requirements known (i.e., system requirements) $\Rightarrow$ inevitable speculative
			\end{itemize}
		\end{note}
		\pause
		\begin{example}{Software Pricing}
			\begin{itemize}
				\item effort costs (software engineers / managers)
				\item hardware and software costs (incl.\ hardware maintenance and software support)
				\item travel and training costs
				\item price = estimated costs + profit + contingency (extra effort, 30--50\%)
			\end{itemize}
		\end{example}
		\nextcolumn
		\pause
		\begin{note}{On Project Startup}
			\begin{itemize}
				\item who will work on the project?
				\item how to split into increments?
				\item refine initial estimates
			\end{itemize}
		\end{note}
		\pause
		\begin{note}{Throughout the Project}
			\begin{itemize}
				\item update plan based on new insights
				\item learn about the software and team capabilities
				\item estimates get more accurate
			\end{itemize}
		\end{note}
	\end{fancycolumns}
\end{frame}

\xkcdframe{1425} % cost estimation

\subsection{Gantt Chart}
\begin{frame}{\insertsubsection\ \mytitlesource{\ludewiglichter, \sommerville}}
	\begin{fancycolumns}[widths={40,58}]
		\begin{definition}{Gantt Chart}
			\begin{itemize}
				\item named after Henry L. Gantt (1861--1919)
				\item bar chart with timeline on x axis and activities on the y axis
				\item optional: progress bars and marker for observation date
				\item optional: dependencies between tasks
				\item optional, not shown: highlight dependencies on the critical path
				\item \emph{critical path}: tasks whose delay also delays the project
			\end{itemize}
		\end{definition}
		\nextcolumn
		\begin{exampletight}{}
			\picDark[width=\linewidth]{management/gantt-chart}
		\end{exampletight}
		\pause[3]
		\begin{example}{}\centering
			What is wrong in the above picture?
		\end{example}
	\end{fancycolumns}
\end{frame}

\newcommand{\forwardpass}[1]{#1}
\newcommand{\backwardpass}[1]{#1}
\newcommand{\buffer}[1]{#1}
\newcommand{\networknode}[7]{
	\begin{tikzpicture}[every node/.style={draw=black,anchor=base,minimum height=5mm,text width=7.5mm,align=center},inner xsep=0mm,line width=.5pt,node distance=-.5pt]
		\node[text width=22.5mm] (task) {#1};
		\node (d) [above=of task] {#3};
		\node (es) [left=of d] {\forwardpass{#2}};
		\node (ef) [right=of d] {\forwardpass{#4}};
		\node (b) [below=of task] {\buffer{#6}};
		\node (ls) [left=of b] {\backwardpass{#5}};
		\node (lf) [right=of b] {\backwardpass{#7}};
	\end{tikzpicture}
}

\subsection{Network Diagram}
\begin{frame}{\insertsubsection\ \mytitlesource{\ludewiglichter, \sommerville}}
	\begin{fancycolumns}
		\begin{definition}{Network Diagram \deutsch{Netzplan}}
			\begin{itemize}
				\item aka.\ PERT charts
				\item directed, acyclic graph
				\item nodes represent tasks
				\item edges represent dependencies
			\end{itemize}
		\end{definition}
		\pause
		\begin{definition}{Metra Potential Method}
			Given project start date and \emph{duration} of each activity we can compute:
			\begin{itemize}
				\item \emph{earliest start} and \emph{earliest finish} time with \emph{forward pass}
				\item \emph{latest start} and \emph{latest finish} time with \emph{backwards pass}
				\item \emph{buffer} (time span between earliest and latest start/finish)
			\end{itemize}
		\end{definition}
		\nextcolumn
		\pause
		\begin{exampletight}{Example Network for a Bachelor's Thesis}
			\centering\vspace{1mm}
			\only<-3|handout:0>{\renewcommand{\forwardpass}[1]{}}
			\only<-4|handout:0>{\renewcommand{\backwardpass}[1]{}}
			\only<-5|handout:0>{\renewcommand{\buffer}[1]{}}
			\only<-6|handout:0>{\tikzset{emph/.style={}}}
			\only<7->{\tikzset{emph/.style={draw=red}}}
			\begin{tikzpicture}[xscale=3,yscale=-2.1,inner sep=0,edge/.style={->,>={Stealth[round]},semithick}]
				\node[fill=red!30!background,font=\tiny] (legend) at (1,0) {\networknode{\normalsize Task}{earliest start}{duration}{earliest finish}{latest start}{buffer}{latest finish}};
				\node (intro) at (1,1) {\networknode{Introduction}{0}{1}{1}{10}{10}{11}};
				\node (background) at (0,0) {\networknode{Background}{0}{3}{3}{0}{0}{3}};
				\node (concept) at (0,1) {\networknode{Concept}{3}{4}{7}{3}{0}{7}};
				\node (eval) at (0,2) {\networknode{Evaluation}{7}{4}{11}{7}{0}{11}};
				\node (summary) at (1,2) {\networknode{Summary}{11}{1}{12}{11}{0}{12}};
				\draw[edge,emph] (background) to (concept);
				\draw[edge,emph] (concept) to (eval);
				\draw[edge,emph] (eval) to (summary);
				\draw[edge] (intro) to (summary);
			\end{tikzpicture}
			\vspace{1mm}
		\end{exampletight}
		% TODO fix example: thesis over 13 weeks, no critical path if spread over 12 weeks (only summary is critical)
		% TODO include network-bachelor-thesis.pdf? separate version for dark mode?
	\end{fancycolumns}
\end{frame}

\subsection{Gantt Charts vs Network Diagrams}
\begin{frame}{\insertsubsection\ \mytitlesource{\sommerville}} % TODO check source
	\begin{fancycolumns}
		\begin{note}{Gantt Chart}
			\begin{itemize}
				\item very common technique
				\item many tools available
				\item great visualization of timing and progress
			\end{itemize}
		\end{note}
		\begin{exampletight}{}
			\centering\picDark[width=.85\linewidth]{management/gantt-chart}
		\end{exampletight}
		\nextcolumn
		\begin{note}{Network Diagram \deutsch{Netzplan}}
			\begin{itemize}
				\item clear visualization of dependencies
				\item explicitly includes buffer times\\(cf.\ metra potential method)
			\end{itemize}
		\end{note}
		\begin{exampletight}{}
			\centering
			\begin{tikzpicture}[xscale=3,yscale=-2.1,inner sep=0,edge/.style={->,>={Stealth[round]},semithick}]
				\node[fill=red!30!background,font=\tiny] (legend) at (1,0) {\networknode{\normalsize Task}{earliest start}{duration}{earliest finish}{latest start}{buffer}{latest finish}};
				\node (intro) at (1,1) {\networknode{Introduction}{0}{1}{1}{10}{10}{11}};
				\node (background) at (0,0) {\networknode{Background}{0}{3}{3}{0}{0}{3}};
				\node (concept) at (0,1) {\networknode{Concept}{3}{4}{7}{3}{0}{7}};
				\draw[edge] (background) to (concept);
			\end{tikzpicture}
		\end{exampletight}
	\end{fancycolumns}
\end{frame}


\lessonslearned{
	\item Project planning and software pricing
	\item Project scheduling with Gantt charts and network diagrams
	\item Next: How to estimate the project costs?
}{
	\item \sommerville\mychapter{23} Project Planning
	\item \ludewiglichter\mychapter{8.3.2} \deutsch{Projektphasen}
}{
	Fill out the network diagram in Stud.IP
	\lectureqr{06}
}

\section{Cost Estimation}
\subsection{Costs of the Corona-Warn-App}
\begin{frame}{\insertsubsection{} \mytitlesource{\href{https://dip.bundestag.de/vorgang/kosten-f\%C3\%BCr-die-corona-warn-app-im-vergleich-zu-\%C3\%A4hnlichen-anwendungen-in/294397?f.deskriptor=App\&rows=25\&pos=21}{bundestag.de}}}
	\centering\picDark[width=.6\linewidth]{management/cwa-welt}
\end{frame}

\subsection{User Stories}
\begin{frame}{\insertsubsection}
	\begin{fancycolumns}
		\begin{note}{Motivation for User Stories}
			\begin{itemize}
				\item split development\\ into manageable parts
				\item split cost estimation\\ into manageable parts
				\item unit of change used in change/release management (cf.~Lecture~5)
				\item keep track of progress / schedule
			\end{itemize}
		\end{note}
		\nextcolumn
		\definitionuserstory{}
	\end{fancycolumns}
\end{frame}

\subsection{Corona-Warn-App User Stories}
\begin{frame}{\insertsubsection}
	\centering\picDark[height=\textheightwithtitle]{management/cwa-issue65}
\end{frame}
\begin{frame}{\insertsubsection}
	\centering\picDark[height=\textheightwithtitle]{management/cwa-issue24}
\end{frame}
\begin{frame}{\insertsubsection}
	\centering\picDark[height=\textheightwithtitle]{management/cwa-issue235}
\end{frame}

\subsection{Planning Poker}
\begin{frame}{\insertsubsection}
	\begin{fancycolumns}
		\begin{definition}{Planning Poker \mysource{\href{https://dl.acm.org/doi/abs/10.5555/1036751}{Cohn 2005}}}
			\emph{Purpose} Estimate the relative effort required to complete user stories.
			\begin{itemize}
				\item introduce the user story
				\item each team member puts down a card with their estimate (face down)
				\item all team members reveal a card at the same time
				\item outliers justify their estimate
				\item repeat until consensus is reached
			\end{itemize}
			Cards use a \emph{modified Fibonacci sequence}: forces developers to take into account the uncertainty of the estimate
		\end{definition}
	\nextcolumn
		\pic[width=\linewidth]{management/planning-poker}
		\pause
		\begin{note}{Why not using \ldots}
			\begin{itemize}
				\item the numbers from 0 to 10?
				\item concrete time intervals (e.g., days)?
			\end{itemize}
		\end{note}
	\end{fancycolumns}
\end{frame}

\subsection{Burndown Chart}
\begin{frame}{\insertsubsection}
	\begin{fancycolumns}[widths={75}]
		\begin{exampletight}{}
			\picDark[width=\linewidth]{management/burndown-chart}
		\end{exampletight}
	\nextcolumn
		\begin{definition}{Burndown Chart}
			\begin{itemize}
				\item visualization of the development progress
				\item used by project management to track progress
			\end{itemize}
		\end{definition}
	\end{fancycolumns}
\end{frame}

\subsection{Recap: Software Pricing}
\begin{frame}<4>{\insertsubsection\ \mytitlesource{\sommerville}}
	\begin{fancycolumns}
		\begin{note}{At the Proposal Stage}
			\begin{itemize}
				\item when bidding for a contract
				\item enough resources?
				\item price for the bidding?
				\item not all requirements known (i.e., system requirements) $\Rightarrow$ inevitable speculative
			\end{itemize}
		\end{note}
		\pause
		\begin{example}{Software Pricing}
			\begin{itemize}
				\item effort costs (software engineers / managers)
				\item hardware and software costs (incl.\ hardware maintenance and software support)
				\item travel and training costs
				\item price = estimated costs + profit + contingency (extra effort, 30--50\%)
			\end{itemize}
		\end{example}
		\nextcolumn
		\pause
		\begin{note}{On Project Startup}
			\begin{itemize}
				\item who will work on the project?
				\item how to split into increments?
				\item refine initial estimates
			\end{itemize}
		\end{note}
		\pause
		\begin{note}{Throughout the Project}
			\begin{itemize}
				\item update plan based on new insights
				\item learn about the software and team capabilities
				\item estimates get more accurate
			\end{itemize}
		\end{note}
	\end{fancycolumns}
\end{frame}


\lessonslearned{
	\item Manageable units with \emph{user stories}
	\item Effort/cost estimation with \emph{planning poker}
	\item Tracking progress with \emph{burndown charts}
	\item Cost estimation is an iterative process
	\item Next: How to manage the process beyond implementation (e.g., testing)?
}{
	\item \sommerville\mychapter{23} Project Planning % TODO check citation
}{
	\begin{enumerate}
		\item<+-> Form groups of 3-4 students
		\item<+-> Choose a feature of an app of your choice
		\item<+-> In isolation: estimate effort in weeks to develop that feature
		\item<+-> In group: name your estimate and discuss outliers
		\item<+-> Repeat 3--4 until consensus is reached
	\end{enumerate}
}

%\faq{
%	\item
%}{
%	\item
%}{
%	\item
%}

\mode<beamer>{
	\addtocounter{framenumber}{-1}
	\begin{frame}{\inserttitle}
		\lectureseriesoverview[\insertlecturenumber]
	\end{frame}

	%\addtocounter{framenumber}{-1}
	%\againtitle % TODO does not work as we have redefined maketitle
}


% TODO L07 DEVELOPMENT PROCESS

\ifuniversity{tubs}{\date{November 26, 2024}}

\author{Thomas Thüm}
\lecture{Development Process}{process}

\begin{frame}{\insertsubtitle{} \deutsch{Vorgehensmodelle}}
	\alt<2->{%
		\hprojectcartoon{04}{how the programmer implemented it\\~\\\textbf{implementation}} % se02 implementation
		\hprojectcartoon{05}{what the beta testers received\\~\\\textbf{software\\testing}} % se03 testing
		\hprojectcartoon{18}{how patches were applied\\~\\\textbf{software\\changes}} % se04 changes
		\hprojectcartoon{12}{when it was delivered\\~\\~\\\textbf{version control}} % se05 version control
		\hprojectcartoon{09}{how the customer was billed\\~\\\textbf{project management}} % se06 project management
	}{%
		\hprojectcartoon{18}{how patches were applied\\~\\\textbf{software\\changes}} % se04 changes
		\hprojectcartoon{09}{how the customer was billed\\~\\\textbf{project management}} % se06 project management
		\hprojectcartoon{12}{when it was delivered\\~\\~\\\textbf{version control}} % se05 version control
		\hprojectcartoon{05}{what the beta testers received\\~\\\textbf{software\\testing}} % se03 testing
		\hprojectcartoon{04}{how the programmer implemented it\\~\\\textbf{implementation}} % se02 implementation
	}%
\end{frame}

\begin{frame}{The Process of Food Delivery}
	\begin{fancycolumns}
		\pic[width=\linewidth]{process/food-delivery}
	\nextcolumn
		\begin{example}{Phases of Food Delivery}
			eat, deliver, order, pay, choose, wait?
		\end{example}
	\end{fancycolumns}
\end{frame}

\section{Development Processes: The Waterfall Model}
\subsection{Software Development Process}
\begin{frame}{\insertsubsection}
	\begin{fancycolumns}
		\begin{note}{Motivation}
			\begin{itemize}
				\item how to \emph{structure} the project?
				\item what are \emph{activities} and phases?\\analysis (requirements elicitation + system modeling), design (architectural + software design), implementation, test, deployment
				\item how to organize \emph{communication}?
				\item who has which \emph{responsibilities}?
				\item did we \emph{forget} anything?
				\item can we \emph{predict} the project result?
				\item how to \emph{manage and control} progress?
				\item how to share and elicit \emph{experience}?
				\item how to synchronize \emph{hardware} and software development?
			\end{itemize}
		\end{note}
		\nextcolumn
		\begin{definition}{Software Process \mysource{\sommerville}}
			\mycite{A \emph{software process} is a set of related activities that leads to the production of a software system. [...] \emph{Products} or deliverables are the outcomes of a process activity. [...] \emph{Roles} reflect the responsibilities of the people involved in the process. [...] Pre- and postconditions are \emph{conditions} that must hold before and after a process
				activity.}
		\end{definition}
		\uncover<3->{\begin{note}{Why Different Processes? \mysource{\sommerville}}
				\mycite{The process used in different companies depends on the type of software being developed, the requirements of the software customer, and the skills of the people writing the software.}
		\end{note}}
	\end{fancycolumns}
\end{frame}

\subsection{Recap}
\begin{frame}<12>[b]{\insertsubsection}
	\vspace{-20mm}\small\renewcommand{\projectcartoonwidth}{.18}
	\begin{fancycolumns}[widths={30},animation=none]
		\uncover<11->{\begin{definition}{Why Do Software Projects Fail?}
				\begin{itemize}
					\item many stakeholders with different background each
					\item miscommunication
					\item implicit or wrong assumptions
					\item time pressure
					\item high complexity
					\item \ldots
					\item numerous reasons specific to certain phases and roles
				\end{itemize}
		\end{definition}}
		\uncover<12->{\begin{note}{}\centering{}software engineering aims to\\reduce such problems\end{note}}
		\nextcolumn
		\centering\hprojectcartoon{01}{how the customer explained it} % se02-10 requirements
		\uncover<2->{\hprojectcartoon{02}{how the project leader understood it}} % se02-07 modeling
		\uncover<3->{\hprojectcartoon{03}{how the analyst designed it}} % se04-10 architecture anddesign
		\uncover<4->{\hprojectcartoon{04}{how the programmer implemented it}} % se02-10 implementation
		\uncover<5->{\hprojectcartoon{05}{what the beta testers received}} % se08-09 testing
		
		\uncover<6->{\hprojectcartoon{06}{how the business consultant described it}} % se02-03 process
		\uncover<7->{\hprojectcartoon{09}{how the customer was billed}} % se10 management and pricing
		\uncover<8->{\hprojectcartoon{10}{how it was supported}} % se12 maintenance
		\uncover<9->{\hprojectcartoon{12}{when it was delivered}} % se13 continous integration/delivery
		\uncover<10->{\hprojectcartoon{13}{what the customer really needed}} % se02-05
		%\hspace{-7mm}
		%\projectcartoon{07}{how the project was documented} % code documentation
		%\hprojectcartoon{18}{how patches were applied} % se11 evolution
		%\hprojectcartoon{17}{how it performed under load} % se14 compilation/quality assurance/performance
		%\hprojectcartoon{11}{what marketing advertised} % se15 reuse/product lines
		%\hprojectcartoon{16}{how open source version} % se16 open source/licensing
		%\projectcartoon{08}{what operations installed} % devops?
		%\projectcartoon{14}{what the digg effect can do to your site} % micro services?
		%\projectcartoon{15}{the disaster recover plan}
		%\hspace{-7mm}
	\end{fancycolumns}
\end{frame}

% TODO add slide(s) on requirements and design?

\subsection{Waterfall Model}
\begin{frame}{\insertsubsection{} \deutsch{Wasserfallmodell} \mytitlesource{\sommerville}}
	\slideWaterfallModel
\end{frame}

\begin{frame}{\insertsubsection\ \mytitlesource{\sommerville}}
	\begin{fancycolumns}[animation=none]
		\begin{definition}{1. Requirements Analysis}
			\mycite{The system’s services, constraints, and goals are established by consultation with system users. They are then defined in detail and serve as a \emph{system specification}.}
		\end{definition}
		\begin{definition}{2. System and Software Design}
			\mycite{The systems design process allocates the requirements to either hardware or software systems. It establishes an overall \emph{system architecture}. Software design involves identifying and describing the fundamental software system abstractions and their relationships.}
		\end{definition}
		\nextcolumn
		\vspace{-11mm}
		\begin{definition}{3. Implementation and Unit Testing}
			\mycite{During this stage, the software design is realized as a set of \emph{programs or program units}. Unit testing involves verifying that each unit meets its specification.}
		\end{definition}
		\begin{definition}{4. Integration and System Testing}
			\mycite{The individual program units or programs are integrated and tested as a complete system to ensure that the software requirements have been met. After testing, the \emph{software system is delivered} to the customer.}
		\end{definition}
		\begin{definition}{5. Operation and Maintenance}
			\mycite{Normally, this is the longest life-cycle phase. The system is installed and put into practical use. Maintenance involves \emph{correcting errors} that were not discovered in earlier stages of the life cycle [...].%, improving the implementation of system units, and enhancing the system’s services as new requirements are discovered.
			}
		\end{definition}
	\end{fancycolumns}
\end{frame}

\subsection{Waterfall Model -- Examples}
\begin{frame}{\insertsubsection\ \mytitlesource{\sommerville}}
	\begin{fancycolumns}[animation=none]
		\begin{example}{Example Domains}
			\begin{itemize}
				\item \emph{embedded systems} where software has to interface with hardware systems
				\item \emph{critical systems} with extensive safety and security analysis of specification and design
				\item \emph{large software systems} that are typically developed by several companies
			\end{itemize}
		\end{example}
	\nextcolumn
		\pic[width=\linewidth]{misc/lawn-mower-cropped}
	\end{fancycolumns}
\end{frame}

\subsection{Waterfall Model -- Discussion}
\begin{frame}{\insertsubsection\ \mytitlesource{\sommerville}}
	\begin{fancycolumns}
		\begin{note}{Advantages}
			\begin{itemize}
				\item easy to understand, manage, and control
				\item good for systems development (i.e., with high manufacturing costs for hardware)
				\item easier to use the same model as for hardware
				\item combination with formal system development feasible (e.g., B method)
			\end{itemize}
		\end{note}
		\nextcolumn
		\begin{note}{Disadvantages}
				\begin{itemize}
					\item for software development: stages should feed information to each other
					\item changes in previous stages are hard to achieve
					\item problems from previous stages left for later resolution
					\item freezing of requirements may lead to software not wanted by the user
					\item freezing of design may lead to bad structure and implementation tricks
					\item requires clear and stable requirements and good design upfront
				\end{itemize}
		\end{note}
	\end{fancycolumns}
\end{frame}


\lessonslearned{
	\item Software development process: motivation and goal
	\item Waterfall model: phases, results, example domains, advantages and disadvantages
	\item Next: How is software developed for German government?
}{
	\item \sommerville\mychapter{2} Software Processes
}{
	\begin{enumerate} % TODO quiz?
		\item<+-> Form groups of 2-3 students
		\item<+-> ?
	\end{enumerate}
}

\section{V-Model for Extensive Testing}
% TODO explain the difference between requirements elicitation and system model + architecture design vs software design

\subsection{V-Model}
\begin{frame}{\insertsubsection}
	\begin{fancycolumns}
		\begin{definition}{V-Model \mysource{\ludewiglichter}}
			\begin{itemize}
				\item developed by the German Ministry of Defense \deutsch{Verteidigungsministerium} and required since 1992
				\item extension of the waterfall model: project-aligned activities such as quality assurance, configuration management, project management
				\item 1997 V-model 97: incremental development, inclusion of hardware, object-oriented development
				\item 2004 V-model XT (for extreme tailoring): adaptability, application beyond software
				% four project types of V-model XT
				\item integration of four testing stages
				%\item distinction between validation (right product) and verification (product correct)
				% fake? proposed 1979 by Barry Boehm
			\end{itemize}
		\end{definition}
		\nextcolumn
		\diagramVModel
		% TODO find reference for this kind of V model
	\end{fancycolumns}
\end{frame}

\subsection{Stages of Testing}
\begin{frame}{\insertsubsection\ \deutsch{Teststufen}\ \mytitlesource{\ludewiglichter, \sommerville}}
	\begin{fancycolumns}[animation=none]
		\begin{definition}{1. Unit Testing \deutsch{Komponententest}}
			\uncover<2->{\begin{itemize}
					\item each component is tested independently
					\item unit may stand for a component or smaller entities (package, class, method)
					\item tests created by the developers
					\item automation is common (e.g., JUnit)
			\end{itemize}}
		\end{definition}
		\begin{definition}{2. Integration Testing \deutsch{Integrationstest}}
			\uncover<3->{\begin{itemize}
					\item some components are integrated (e.g., into subsystems) and tested together
					\item detects inconsistencies in interfaces and communication between components
					\item top-down vs bottom-up integration
			\end{itemize}}
		\end{definition}
		\nextcolumn
		\begin{definition}{3. System Testing \deutsch{Systemtest}}
			\uncover<4->{\begin{itemize}
					\item all components are integrated to the complete system
					\item detects further inconsistencies and unanticipated interactions
					\item system is tested against system requirements
			\end{itemize}}
		\end{definition}
		\begin{definition}{4. Acceptance Testing \deutsch{Abnahmetest}}
			\uncover<5->{\begin{itemize}
					\item final stage in the testing process before accepted for operational use
					\item system is tested against user requirements and with real data
					\item performed by (potential) customer
			\end{itemize}}
		\end{definition}
	\end{fancycolumns}
\end{frame}
% TODO split on several slides and give examples for errors that should / should not be found in this phase (and why)

\subsection{V-Model -- Discussion}
\begin{frame}{\insertsubsection\ \mytitlesource{\ludewiglichter}}
	\begin{fancycolumns}
		\begin{note}{Advantages}
			\begin{itemize}
				\item quality assurance in several testing stages
				\item completeness helps to not miss activities
				%\item distinction between validation and verification
				\item V-model 97/XT are widely applicable (e.g., hardware, incremental)
			\end{itemize}
		\end{note}
		\nextcolumn
		\begin{note}{Disadvantages}
				\begin{itemize}
					\item complex and extensive process
					\item adaptations often required (cf.\ XT for extreme tailoring)
					\item overhead useful only for large software systems
					\item changes in requirements are problematic
				\end{itemize}
		\end{note}
	\end{fancycolumns}
\end{frame}

\subsection{V-Model -- Examples}
\begin{frame}{\insertsubsection\ \mytitlesource{\ludewiglichter}}
	\begin{fancycolumns}[animation=none]
		\begin{example}{Example Domains}
				\begin{itemize}
					\item since 1992 V-model required by German government (e.g., Bundeswehr), since 2004 V-model XT
					\item embedded, critical, and large software systems as for the waterfall model
				\end{itemize}
		\end{example}
		\nextcolumn
			% TODO picture associated with German government
	\end{fancycolumns}
\end{frame}


\lessonslearned{
	\item V-model
	\item Testing stages: unit testing, integration testing, system testing, acceptance testing
	\item Next: What is the most frequently used process model?
}{
	\item \ludewiglichter\mychapter{10.3} \deutsch{Das V-Modell}
	\item \sommerville\mychapter{2.2.3} Software Validation
	\item \sommerville\mychapter{8.1} Development Testing
}{
	\begin{enumerate}
		\item<+-> Form groups of 2-3 students
		\item<+-> 2 minutes of silence: Choose one of the four testing stages and think of an example of a fault that can be detected in that testing stage.
		\item<+-> 5--10 minutes in groups of 2--3 students: Discuss those faults and whether they could and should have been found in the other stages.
	\end{enumerate}
}

\section{Scrum for Agile Testing}
\subsection{Motivation for Agile Development}
\begin{frame}{\insertsubsection}
	\begin{fancycolumns}
		\begin{note}{Motivation \mysource{\sommerville}}
			\begin{itemize}
				\item businesses operate globally and in a rapidly changing environment
				\item software is part of almost all business operations
				\item new software has to be developed quickly
				\item often infeasible to derive a complete set of stable requirements
				\item plan-driven process models (e.g., waterfall) deliver software long after originally specified
			\end{itemize}
		\end{note}
		\nextcolumn
		\begin{definition}{Agile (Development) Methods \mysource{\sommerville}}
			Development of agile methods since late 1990s:
			\begin{itemize}
				\item[1.] specification, design, implementation are interleaved
				\item[2.] each increment is specified and evaluated by stakeholders (e.g., end-users)
				\item[3.] extensive tool support is used
			\end{itemize}
		\end{definition}
	\end{fancycolumns}
\end{frame}

\subsection{Manifesto of Agile Software Development}
\begin{frame}{\insertsubsection\ \mytitlesource{\href{https://agilemanifesto.org/}{agilemanifesto.org}}}
	\begin{fancycolumns}
		\pic[width=\linewidth,trim=0 250 0 0,clip]{people/agile-manifesto}
		\vspace{-7mm}
		
		\begin{note}{Ski Resort in Utah (February 2001)}
			\small 17 experts on software development:\\~\\Kent Beck, Mike Beedle, Arie van Bennekum, Alistair Cockburn, Ward Cunningham, Martin Fowler, James Grenning, Jim Highsmith, Andrew Hunt, Ron Jeffries, Jon Kern, Brian Marick, Robert C. Martin, Steve Mellor, Ken Schwaber, Jeff Sutherland, Dave Thomas
		\end{note}
		\nextcolumn
		\begin{definition}{Manifesto}
			\mycite{We are uncovering better ways of developing software by doing it and helping others do it. Through this work we have come to value:
				\begin{itemize}
					\item {\large individuals and interactions}\\\hfill over processes and tools
					\item {\large working software}\\\hfill over comprehensive documentation
					\item {\large customer collaboration}\\\hfill over contract negotiation
					\item {\large responding to change}\\\hfill over following a plan
				\end{itemize}
				That is, while there is value in the items on the right, we value the items on the left more.}
		\end{definition}
	\end{fancycolumns}
\end{frame}

\subsection{Principles behind the Agile Manifesto}
\begin{frame}[b]{\insertsubsection\ \mytitlesource{\href{https://agilemanifesto.org/}{agilemanifesto.org}}}
	\setlength\leftmargini{4mm}\vspace{-5mm}%
	\begin{fancycolumns}[widths={54},b,animation=none]
		\begin{definition}{Principles 1--6}
			\begin{itemize}
				\item[1.]<+-> \mycite{Our highest priority is to \emph{satisfy the customer} through early and continuous delivery of valuable software.
					\item[2.]<+-> \emph{Welcome changing requirements}, even late in development. Agile processes harness change for the customer's competitive advantage.
					\item[3.]<+-> \emph{Deliver working software frequently}, from a couple of weeks to a couple of months, with a preference to the shorter timescale.
					\item[4.]<+-> Business people and developers must \emph{work together daily} throughout the project.
					\item[5.]<+-> Build projects around \emph{motivated individuals}. Give them the environment and support they need, and trust them to get the job done.
					\item[6.]<+-> The most efficient and effective method of conveying information to and within a development team is \emph{face-to-face conversation}.}
			\end{itemize}
		\end{definition}
		\nextcolumn
		\begin{definition}{Principles 7--12}
			\begin{itemize}
				\item[7.]<+-> \mycite{\emph{Working software} is the primary measure of progress.
					\item[8.]<+-> Agile processes promote \emph{sustainable development}. The sponsors, developers, and users should be able to maintain a constant pace indefinitely.
					\item[9.]<+-> Continuous attention to \emph{technical excellence and good design} enhances agility.
					\item[10.]<+-> \emph{Simplicity}--the art of maximizing the amount of work not done--is essential.
					\item[11.]<+-> The best architectures, requirements, and designs emerge from \emph{self-organizing teams}.
					\item[12.]<+-> At regular intervals, the \emph{team reflects} on how to become more effective, then tunes and adjusts its behavior accordingly.}
			\end{itemize}
		\end{definition}
	\end{fancycolumns}
\end{frame}

\subsection{Scrum}
\begin{frame}{\insertsubsection}
	\begin{fancycolumns}[widths={45}]
		\definitionuserstory{Recap: }
		\begin{note}{}
				many agile methods rely on user stories: Scrum, Kanban, Extreme Programming (XP)
		\end{note}
		\nextcolumn
		\vspace{-10mm}
		\begin{definition}{Scrum \mysource{\sommerville}}
			\begin{itemize}
				% TODO add original authors and publication
				\item an agile method, most-widely used method
				\item no special development techniques (like pair programming, test-driven development)
				\item \emph{product backlog}: list of user stories, collected and priorized by the \emph{product owner}
				\item \emph{sprint backlog}: user stories selected by the scrum team for the next \emph{sprint}
			\end{itemize}
		\end{definition}
		\begin{exampletight}{}
			\diagramScrum
		\end{exampletight}
		% TODO better scrum visualization?
		%\myexampletight{}{\href{https://commons.wikimedia.org/wiki/File:Scrum_Framework.png}{\includegraphics[width=\linewidth]{scrum2}}}
	\end{fancycolumns}
\end{frame}

\subsection{Scrum Roles}
\begin{frame}{\insertsubsection}
	\begin{fancycolumns}
		\begin{definition}{Development Team \mysource{\sommerville}}
			\mycite{A self-organizing group of software developers, which should be \emph{no more than seven people}. They are responsible for developing the software and other essential project documents.}
		\end{definition}
		\begin{exampletight}{}
			\diagramScrum
		\end{exampletight}
		\nextcolumn
		\begin{definition}{Product Owner \mysource{\sommerville}}
			\mycite{An individual (or possibly a small group) whose job is to identify product features or requirements, prioritize these for development, and \emph{continuously review the product backlog} to ensure that the project continues to meet critical business needs. The product owner can be a customer but might also be a product manager in a software company or other stakeholder representative.}
		\end{definition}
		\uncover<3->{\begin{definition}{Scrum Master \mysource{\sommerville}}
				\mycite{The scrum master is responsible for ensuring that the scrum process is followed and \emph{guides the team} in the effective use of scrum. He or she is responsible for interfacing with the rest of the company and for ensuring that the scrum team is not diverted by outside interference.% The scrum developers are adamant that the scrum master should not be thought of as a project manager. Others, however, may not always find it easy to see the difference.
				}
		\end{definition}}
	\end{fancycolumns}
\end{frame}

\subsection{Scrum Terms}
\begin{frame}{\insertsubsection}
	\begin{fancycolumns}[widths={45}]
		\begin{definition}{(Potentially Shippable) Product Increment}
			\mycite{The software increment that is delivered from a sprint. The idea is that this should be potentially shippable, which means that it is in a \emph{finished state} and no further work, such as testing, is needed to incorporate it into the final product.% In practice, this is not always achievable.
			}\mysource{\sommerville}
		\end{definition}
		\uncover<2->{\begin{definition}{Product Backlog \mysource{\sommerville}}
				\mycite{This is a list of \emph{to-do items} that the scrum team must tackle. They may be feature definitions for the software, software requirements, user stories, or descriptions of supplementary tasks that are needed, such as architecture definition or user documentation.}
		\end{definition}}
		\nextcolumn
		\uncover<3->{\begin{definition}{Daily Scrum \mysource{\sommerville}}
				\mycite{A daily meeting (cf.\ stand-up meeting) of the scrum team that \emph{reviews progress and prioritizes work} to be done that day. Ideally, this should be a short face-to-face meeting that includes the whole team.}
		\end{definition}}
		\uncover<4->{\begin{definition}{Sprint \mysource{\sommerville}}
				\mycite{A development iteration. Sprints are usually \emph{2 to 4 weeks} long.}
		\end{definition}}
		\uncover<5->{\begin{definition}{Velocity \mysource{\sommerville}}
				\mycite{An estimate of how much product backlog effort a team can cover in a single sprint. Understanding a team’s velocity helps them estimate what can be covered in a sprint and provides a basis for \emph{measuring and improving performance}.}
		\end{definition}}
	\end{fancycolumns}
\end{frame}

\subsection{Recap: Planning Poker and Burndown Chart}
\begin{frame}{\insertsubsection}
	\begin{fancycolumns}[widths={35}]
		\pic[width=\linewidth]{management/planning-poker}
		\begin{note}{Planning Poker}
			\begin{itemize}
				\item used to estimate effort for new user stories arriving in the \emph{product backlog}
				\item estimation relative to the \emph{velocity}
			\end{itemize}
		\end{note}
	\nextcolumn
		\pic[width=\linewidth]{management/burndown-chart}
		\begin{note}{Burndown Chart}
			\begin{itemize}
				\item used for each \emph{sprint} (here three weeks)
				\item used by \emph{scrum master} to track progress
			\end{itemize}
		\end{note}
	\end{fancycolumns}
\end{frame}

% TODO necessary to add examples? would be consistent to other two parts

\subsection{Scrum -- Discussion}
\begin{frame}{\insertsubsection\ \mytitlesource{\sommerville}}
	\begin{fancycolumns}
		\begin{note}{Advantages}
			\begin{itemize}
				\item product is broken down into manageable and \emph{understandable chunks}
				\item \emph{unstable requirements} can be easily incorporated
				\item good \emph{team communication} and transparency
				\item \emph{customers can inspect increments} and understand how the product works
				\item establishes \emph{trust} between customers and developers
			\end{itemize}
		\end{note}
		\nextcolumn
		\begin{note}{Disadvantages}
			\begin{itemize}
				\item unclear how scale to \emph{larger teams}
				\item problematic when \emph{contract negotiation} is required (as customer pays for development time rather then set of requirements)
				\item \emph{documentation and testing} not explicitly covered (requires extra story cards)
				\item requires \emph{continuous customer input}
				\item \emph{tacit knowledge} not available during maintenance (of long-life systems)
				\item detailed documentation required for \emph{external regulation} and \emph{outsourcing}
			\end{itemize}
		\end{note}
	\end{fancycolumns}
\end{frame}


\lessonslearned{
	\item Motivation for agile development
	\item Agile Manifesto (four values, twelve principles)
	\item User stories
	\item Scrum: roles, terms, discussion
	\item Next: Are we developing the right product?
}{
	\item \sommerville\mychapter{3} Agile Software Development
}{
	Questions?
}

%\faq{
	%	\item
	%}{
	%	\item
	%}{
	%	\item
	%}

\mode<beamer>{
	\addtocounter{framenumber}{-1}
	\begin{frame}{\inserttitle}
		\lectureseriesoverview[\insertlecturenumber]
	\end{frame}

	%\addtocounter{framenumber}{-1}
	%\againtitle % TODO does not work as we have redefined maketitle
}

