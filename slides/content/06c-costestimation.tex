\subsection{Costs of the Corona-Warn-App}
\begin{frame}{\insertsubsection{} \mytitlesource{\href{https://dip.bundestag.de/vorgang/kosten-f\%C3\%BCr-die-corona-warn-app-im-vergleich-zu-\%C3\%A4hnlichen-anwendungen-in/294397?f.deskriptor=App\&rows=25\&pos=21}{bundestag.de}}}
	\centering\picDark[width=.6\linewidth]{management/cwa-welt}
\end{frame}

\subsection{User Stories}
\begin{frame}{\insertsubsection}
	\begin{fancycolumns}
		\begin{note}{Motivation for User Stories}
			\begin{itemize}
				\item split development\\ into manageable parts
				\item split cost estimation\\ into manageable parts
				\item unit of change used in change/release management (cf.~Lecture~5)
				\item keep track of progress / schedule
			\end{itemize}
		\end{note}
		\nextcolumn
		\begin{definition}{User Story \mysource{\sommerville}}
			\begin{itemize}
				\item a scenario of use that might be experienced by a system user
				\item aka.\ \emph{story card} as user stories are sometimes written on physical cards
				\item user stories are typically prioritized by the customer
				\item subset of all user stories is chosen for the next release
				%\item used in many agile methods
			\end{itemize}
		\end{definition}
	\end{fancycolumns}
\end{frame}

\subsection{Corona-Warn-App User Stories}
\begin{frame}{\insertsubsection}
	\centering\picDark[height=\textheightwithtitle]{management/cwa-issue65}
\end{frame}
\begin{frame}{\insertsubsection}
	\centering\picDark[height=\textheightwithtitle]{management/cwa-issue24}
\end{frame}
\begin{frame}{\insertsubsection}
	\centering\picDark[height=\textheightwithtitle]{management/cwa-issue235}
\end{frame}

\subsection{Planning Poker}
\begin{frame}{\insertsubsection}
	\begin{fancycolumns}
		\begin{definition}{Planning Poker \mysource{\href{https://dl.acm.org/doi/abs/10.5555/1036751}{Cohn 2005}}}
			\emph{Purpose} Estimate the relative effort required to complete user stories.
			\begin{itemize}
				\item introduce the user story
				\item each team member puts down a card with their estimate (face down)
				\item all team members reveal a card at the same time
				\item outliers justify their estimate
				\item repeat until consensus is reached
			\end{itemize}
			Cards use a \emph{modified Fibonacci sequence}: forces developers to take into account the uncertainty of the estimate
		\end{definition}
	\nextcolumn
		\pic[width=\linewidth]{management/planning-poker}
		\pause
		\begin{note}{Why not using \ldots}
			\begin{itemize}
				\item the numbers from 0 to 10?
				\item concrete time intervals (e.g., days)?
			\end{itemize}
		\end{note}
	\end{fancycolumns}
\end{frame}

\subsection{Burndown Chart}
\begin{frame}{\insertsubsection}
	\begin{fancycolumns}[widths={75}]
		\begin{exampletight}{}
			\picDark[width=\linewidth]{management/burndown-chart}
		\end{exampletight}
	\nextcolumn
		\begin{definition}{Burndown Chart}
			\begin{itemize}
				\item visualization of the development progress
				\item used by project management to track progress
			\end{itemize}
		\end{definition}
	\end{fancycolumns}
\end{frame}

\subsection{Recap: Software Pricing}
\begin{frame}<4>{\insertsubsection\ \mytitlesource{\sommerville}}
	\begin{fancycolumns}
		\begin{note}{At the Proposal Stage}
			\begin{itemize}
				\item when bidding for a contract
				\item enough resources?
				\item price for the bidding?
				\item not all requirements known (i.e., system requirements) $\Rightarrow$ inevitable speculative
			\end{itemize}
		\end{note}
		\pause
		\begin{example}{Software Pricing}
			\begin{itemize}
				\item effort costs (software engineers / managers)
				\item hardware and software costs (incl.\ hardware maintenance and software support)
				\item travel and training costs
				\item price = estimated costs + profit + contingency (extra effort, 30--50\%)
			\end{itemize}
		\end{example}
		\nextcolumn
		\pause
		\begin{note}{On Project Startup}
			\begin{itemize}
				\item who will work on the project?
				\item how to split into increments?
				\item refine initial estimates
			\end{itemize}
		\end{note}
		\pause
		\begin{note}{Throughout the Project}
			\begin{itemize}
				\item update plan based on new insights
				\item learn about the software and team capabilities
				\item estimates get more accurate
			\end{itemize}
		\end{note}
	\end{fancycolumns}
\end{frame}

