% TODO several ideas illustrating impact with consequences, but then hard to distinguish from next part

\subsection{Software (Products)}
\begin{frame}{\insertsubsection}
	\begin{fancycolumns}
		\begin{definition}{Software \mysource{adapted from \sommerville}}
			Software stands for one or several computer programs and all associated documentation, libraries, support websites, and configuration data that are needed to make these programs useful.
		\end{definition}
		\begin{example}{Explanation}
			The term program is used in a broader sense here. Software may also include source code, software models, or binaries.
		\end{example}
	\nextcolumn
		\begin{definition}{Software Product and Professional Software}
			A \emph{software product} is a software that can be sold to a customer. \emph{Professional software} is software intended for use by someone apart from its developer and it is usually developed by teams rather than individuals. \mysource{adapted from \sommerville}
		\end{definition}
		\pic[width=\linewidth]{misc/ms-office-cropped}
	\end{fancycolumns}
\end{frame}

\subsection{Characteristics of Software}
\begin{frame}{\insertsubsection}
	\centering\tikz[grow cyclic,
	mindmap, every node/.style=concept,concept color=red!10!background,
	%text width=20mm,align=flush center,
	level 1/.append style={level distance=27mm,sibling angle=360/8}]
	\node {Characteristics of Software}
	child { node {abstract\\(no physical\\laws)} }
	child { node {intangible \deutsch{immateriell}} }
	child { node {hard to measure} }
	child { node {aging} }
	child { node {no \phantom{abc}deterioration} }
	child { node {no spare parts} }
	child { node {easier to adapt than hardware} }
	child { node {frequent adaptation\phantom{;}} }
	;
\end{frame}
%   abstract and intagible (immatriell): not constrained by the properties of materials, nor are they governed by physical laws or by manufacturing processes. [\sommerville]
%   good as what we can built is a matter of our imagination (and computing power)
%   bad as software can get arbitrarily complex [\sommerville]

\subsection{Application and System Software}
\begin{frame}{\insertsubsection}
	\begin{fancycolumns}[T]
		\begin{definition}{Application Software or Application}
			Software that is designed for end users and applied for certain purposes. \deutsch{Anwendungssoftware oder Anwendung}
		\end{definition}

		\begin{example}{Examples}
			web browsers, media players, email or chat clients, text or photo editors, games
		\end{example}
	\nextcolumn
		\begin{definition}{System Software}
			Software that is not application software and typically being designed to provide a platform for other software.
		\end{definition}

		\begin{example}{Examples}
			operating systems, firmware, basic input/output system (BIOS), device drivers, game engines, GUI frameworks
		\end{example}
	\end{fancycolumns}

	\uncover<3->{
		\begin{note}{Classification Not Always Unique}
			e.g., web browsers and chat clients take over more and more features of operating systems
		\end{note}
	}
\end{frame}
% mention windows being in court because IE could not be uninstalled?

\subsection{Where Does Software Run?}
\begin{frame}{\insertsubsection}
	\begin{fancycolumns}[widths={43}]
		\begin{example}{World-Wide PC Sales}
			\picDark[width=\linewidth]{history/PCsales}
		\end{example}
		\nextcolumn
		\begin{example}{World-Wide Mobile Phone Subscriptions}
			\picDark[width=\linewidth]{history/mobile-phone-subscriptions}
		\end{example}
	\end{fancycolumns}	
\end{frame}
% hardware getting smaller and smaller, portable even for bees https://www.bbc.com/news/articles/cj9jzv27lv2o

\begin{frame}{Application Software}
	\begin{fancycolumns}[T,widths={46}]
		\begin{example}{Desktop Application or Desktop App}
			Windows 1.0 released in 1985
		\end{example}
		\pic[width=\linewidth]{history/windows1.0}
	\nextcolumn
		\begin{example}{Web Application or Web App}
			Ebay was born in 1995
		\end{example}
		\pic[height=32mm]{history/flohmarkt}
		\hfill\pause
		\pic[height=32mm]{history/iPhone1stGen}
		\begin{example}{Mobile Application or Mobile App or App}
			First iPhone released in 2007
		\end{example}
	\end{fancycolumns}
\end{frame}
% history of computing/software
% * easy delivery + distribution by the internet, internet of things
% * as it is so "easy" to change software, it is changed more frequently and is typically getting more complex with every change until it dies
%		\myexample{Progressive Web Application}{since 2016, not yet popular}
% "The "brain" [computer] may one day come down to our level [of the common people] and help with our income-tax and book-keeping calculations. But this is speculation and there is no sign of it so far." 1949 Quelle "Tutorial Guide to the EDSAC Simulator" (PDF). The EDSAC Replica Project.
%  ENIAC which became operational in 1946 could be run by a single, albeit highly trained, person.

\xkcdframe{2212}
% cash vs credit/debit card vs paypal/mobile phone: more and more software involved

\subsection{Downloads of Android Apps}
\begin{frame}{Android Apps with 5 Billion Downloads \deutschertitel{5 Milliarden}}
	\centering\picDark[height=65mm]{diagrams/android-downloads-5b}
\end{frame}

\begin{frame}{Android Apps with 5 Billion Downloads \deutschertitel{5 Milliarden}}
	\centering\picDark[height=65mm]{diagrams/android-downloads-5b-np}
\end{frame}
% TODO add facebook,whatsapp,instagram outtage from October 2021?

\begin{frame}{Android Apps with 1 Billion Downloads \deutschertitel{1 Milliarde}}
	\centering\picDark[height=65mm]{diagrams/android-downloads-1b-p}
\end{frame}

\begin{frame}{Android Apps with 1 Billion Downloads \deutschertitel{1 Milliarde}}
	\centering\picDark[height=65mm]{diagrams/android-downloads-1b-np}
\end{frame}

\begin{frame}{Android Apps with 500 Million Downloads}
	\centering\picDark[height=65mm]{diagrams/android-downloads-500m-np}
\end{frame}

\subsection{Properties of Software}
\begin{frame}{\insertsubsection}
	\centering\tikz[grow cyclic,
	mindmap, every node/.style=concept,concept color=blue!10!background,
	%text width=20mm,align=flush center,
	level 1/.append style={level distance=27mm,sibling angle=360/8}]
	\node {Properties of Software}
	child { node {system vs\phantom{;} application software\phantom{ab}} }
	child { node {generic vs customized} }
	child { node {proprietary\\vs open-source} }
	child { node {freeware\\vs commercial} }
	child { node {data-intensive vs computation-intensive} }
	child { node {monolithic vs distributed} }
	child { node {stand-alone vs integrated} }
	child { node {user interface vs API\phantom{ab}} }
	;
\end{frame}
% TODO parts of this diagram not readible

% software is eating the world: https://a16z.com/2011/08/20/why-software-is-eating-the-world/
% data is the new oil

% German E rezept is facing technical challenges
