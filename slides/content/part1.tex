% TODO L01 INTRODUCTION

\ifuniversity{tubs}{\date{October 15, 2024}}

\author{Thomas Thüm}
\lecture{Introduction}{introduction}
% goal of this lecture?
% learn what software is and distinguish term from program, software product, app, ...
% understand the importance/role of software
% understand what software engineering is good for

\inputuniversity{content/01-prelude}

\section{The Impact of Software}
% TODO several ideas illustrating impact with consequences, but then hard to distinguish from next part

\subsection{Software (Products)}
\begin{frame}{\insertsubsection}
	\begin{fancycolumns}
		\begin{definition}{Software \mysource{adapted from \sommerville}}
			Software stands for one or several computer programs and all associated documentation, libraries, support websites, and configuration data that are needed to make these programs useful.
		\end{definition}
		\begin{example}{Explanation}
			The term program is used in a broader sense here. Software may also include source code, software models, or binaries.
		\end{example}
	\nextcolumn
		\begin{definition}{Software Product and Professional Software}
			A \emph{software product} is a software that can be sold to a customer. \emph{Professional software} is software intended for use by someone apart from its developer and it is usually developed by teams rather than individuals. \mysource{adapted from \sommerville}
		\end{definition}
		\pic[width=\linewidth]{misc/ms-office-cropped}
	\end{fancycolumns}
\end{frame}

\subsection{Characteristics of Software}
\begin{frame}{\insertsubsection}
	\centering\tikz[grow cyclic,
	mindmap, every node/.style=concept,concept color=red!10!background,
	%text width=20mm,align=flush center,
	level 1/.append style={level distance=27mm,sibling angle=360/8}]
	\node {Characteristics of Software}
	child { node {abstract\\(no physical\\laws)} }
	child { node {intangible \deutsch{immateriell}} }
	child { node {hard to measure} }
	child { node {aging} }
	child { node {no \phantom{abc}deterioration} }
	child { node {no spare parts} }
	child { node {easier to adapt than hardware} }
	child { node {frequent adaptation\phantom{;}} }
	;
\end{frame}
%   abstract and intagible (immatriell): not constrained by the properties of materials, nor are they governed by physical laws or by manufacturing processes. [\sommerville]
%   good as what we can built is a matter of our imagination (and computing power)
%   bad as software can get arbitrarily complex [\sommerville]

\subsection{Application and System Software}
\begin{frame}{\insertsubsection}
	\begin{fancycolumns}[T]
		\begin{definition}{Application Software or Application}
			Software that is designed for end users and applied for certain purposes. \deutsch{Anwendungssoftware oder Anwendung}
		\end{definition}

		\begin{example}{Examples}
			web browsers, media players, email or chat clients, text or photo editors, games
		\end{example}
	\nextcolumn
		\begin{definition}{System Software}
			Software that is not application software and typically being designed to provide a platform for other software.
		\end{definition}

		\begin{example}{Examples}
			operating systems, firmware, basic input/output system (BIOS), device drivers, game engines, GUI frameworks
		\end{example}
	\end{fancycolumns}

	\uncover<3->{
		\begin{note}{Classification Not Always Unique}
			e.g., web browsers and chat clients take over more and more features of operating systems
		\end{note}
	}
\end{frame}
% mention windows being in court because IE could not be uninstalled?

\subsection{Where Does Software Run?}
\begin{frame}{\insertsubsection}
	\begin{fancycolumns}[widths={43}]
		\begin{example}{World-Wide PC Sales}
			\picDark[width=\linewidth]{history/PCsales}
		\end{example}
		\nextcolumn
		\begin{example}{World-Wide Mobile Phone Subscriptions}
			\picDark[width=\linewidth]{history/mobile-phone-subscriptions}
		\end{example}
	\end{fancycolumns}	
\end{frame}
% hardware getting smaller and smaller, portable even for bees https://www.bbc.com/news/articles/cj9jzv27lv2o

\begin{frame}{Application Software}
	\begin{fancycolumns}[T,widths={46}]
		\begin{example}{Desktop Application or Desktop App}
			Windows 1.0 released in 1985
		\end{example}
		\pic[width=\linewidth]{history/windows1.0}
	\nextcolumn
		\begin{example}{Web Application or Web App}
			Ebay was born in 1995
		\end{example}
		\pic[height=32mm]{history/flohmarkt}
		\hfill\pause
		\pic[height=32mm]{history/iPhone1stGen}
		\begin{example}{Mobile Application or Mobile App or App}
			First iPhone released in 2007
		\end{example}
	\end{fancycolumns}
\end{frame}
% history of computing/software
% * easy delivery + distribution by the internet, internet of things
% * as it is so "easy" to change software, it is changed more frequently and is typically getting more complex with every change until it dies
%		\myexample{Progressive Web Application}{since 2016, not yet popular}
% "The "brain" [computer] may one day come down to our level [of the common people] and help with our income-tax and book-keeping calculations. But this is speculation and there is no sign of it so far." 1949 Quelle "Tutorial Guide to the EDSAC Simulator" (PDF). The EDSAC Replica Project.
%  ENIAC which became operational in 1946 could be run by a single, albeit highly trained, person.

\xkcdframe{2212}
% cash vs credit/debit card vs paypal/mobile phone: more and more software involved

\subsection{Downloads of Android Apps}
\begin{frame}{Android Apps with 5 Billion Downloads \deutschertitel{5 Milliarden}}
	\centering\picDark[height=65mm]{diagrams/android-downloads-5b}
\end{frame}

\begin{frame}{Android Apps with 5 Billion Downloads \deutschertitel{5 Milliarden}}
	\centering\picDark[height=65mm]{diagrams/android-downloads-5b-np}
\end{frame}
% TODO add facebook,whatsapp,instagram outtage from October 2021?

\begin{frame}{Android Apps with 1 Billion Downloads \deutschertitel{1 Milliarde}}
	\centering\picDark[height=65mm]{diagrams/android-downloads-1b-p}
\end{frame}

\begin{frame}{Android Apps with 1 Billion Downloads \deutschertitel{1 Milliarde}}
	\centering\picDark[height=65mm]{diagrams/android-downloads-1b-np}
\end{frame}

\begin{frame}{Android Apps with 500 Million Downloads}
	\centering\picDark[height=65mm]{diagrams/android-downloads-500m-np}
\end{frame}

\subsection{Properties of Software}
\begin{frame}{\insertsubsection}
	\centering\tikz[grow cyclic,
	mindmap, every node/.style=concept,concept color=blue!10!background,
	%text width=20mm,align=flush center,
	level 1/.append style={level distance=27mm,sibling angle=360/8}]
	\node {Properties of Software}
	child { node {system vs\phantom{;} application software\phantom{ab}} }
	child { node {generic vs customized} }
	child { node {proprietary\\vs open-source} }
	child { node {freeware\\vs commercial} }
	child { node {data-intensive vs computation-intensive} }
	child { node {monolithic vs distributed} }
	child { node {stand-alone vs integrated} }
	child { node {user interface vs API\phantom{ab}} }
	;
\end{frame}
% TODO parts of this diagram not readible

% software is eating the world: https://a16z.com/2011/08/20/why-software-is-eating-the-world/
% data is the new oil

% German E rezept is facing technical challenges

\lessonslearned{
	\item What is software?
	\item What is the difference between program, software product, professional software, desktop/web/mobile app?
	\item What are characteristics and properties of software?
	\item What is the impact of software?
	\item Next: What can go possibly wrong with software?
}{
	\item \sommerville\mychapter{1.1}\mypages{19--28} % TODO check
}{
	\begin{enumerate}
		\item<+-> Form groups of 2-3 students
		\item<+-> Introduce yourselves to each other (5 min)
		\item<+-> Share your (recent) experiences with software (5 min)
		\item<+-> Survey: What is your experience with software?\\
		\slideonly{~
			\item<+->[A] Warst du jemals von einer Softwarepanne betroffen?
			\item<+->[B] Hast du dich im letzten Semester über Softwarefehler geärgert?
			\item<+->[C] Hast du selbst Programmiererfahrung?
			\item<+->[D] Hast du bereits Software getestet?
			\item<+->[E] Hast du schon mal Programmierfehler verursacht?
		}
			%\fancyqr{https://studip.tu-braunschweig.de/dispatch.php/questionnaire/answer/dc864612bd163af167a09edabdf058f3?cid=c37620ed3e63513af37fbe31060fcf69&range_type=course&range_id=c37620ed3e63513af37fbe31060fcf69}
	\end{enumerate}
}

\section{Software and Its Engineering}
% Why is Engineering Good Software so Hard?
% How to Develop Good Software?
% Why is Software Production so hard?
\subsection{Therac-25} % motivation for ?
\begin{frame}{\insertsubsection}
	\slideTherac
\end{frame}

\subsection{Xiaomi SU7 Crash} % motivation for ?
\begin{frame}{\insertsubsection}
	\slideXiaomiCrash
\end{frame}

\xkcdframe{2961}

\subsection{CrowdStrike Outage} % motivation for 04-maintenance, 06-management
\begin{frame}{\insertsubsection}
	\slideCrowdStrike
\end{frame}

\subsection{Saxony Election Glitch} % motivation for 11-reuse
\begin{frame}{\insertsubsection}
	\slideSaxonyElection
\end{frame}

\subsection{Ariane 5 Software Failure} % motivation for 11-reuse
\begin{frame}{\insertsubsection}
	\slideArianeFailure
\end{frame}

\subsection{The Project Cartoon}
\begin{frame}[b]{\insertsubsection}
	\vspace{-20mm}\small\renewcommand{\projectcartoonwidth}{.18}
	\begin{fancycolumns}[widths={30},animation=none]
		\uncover<11->{\begin{definition}{Why Do Software Projects Fail?}
			\begin{itemize}
				\item many stakeholders with different background each
				\item miscommunication
				\item implicit or wrong assumptions
				\item time pressure
				\item high complexity
				\item \ldots
				\item numerous reasons specific to certain phases and roles
			\end{itemize}
		\end{definition}}
		\uncover<12->{\begin{note}{}\centering{}software engineering aims to\\reduce such problems\end{note}}
	\nextcolumn
		\centering\hprojectcartoon{01}{how the customer explained it} % se02-10 requirements
		\uncover<2->{\hprojectcartoon{02}{how the project leader understood it}} % se02-07 modeling
		\uncover<3->{\hprojectcartoon{03}{how the analyst designed it}} % se04-10 architecture anddesign
		\uncover<4->{\hprojectcartoon{04}{how the programmer implemented it}} % se02-10 implementation
		\uncover<5->{\hprojectcartoon{05}{what the beta testers received}} % se08-09 testing
	
		\uncover<6->{\hprojectcartoon{06}{how the business consultant described it}} % se02-03 process
		\uncover<7->{\hprojectcartoon{09}{how the customer was billed}} % se10 management and pricing
		\uncover<8->{\hprojectcartoon{10}{how it was supported}} % se12 maintenance
		\uncover<9->{\hprojectcartoon{12}{when it was delivered}} % se13 continous integration/delivery
		\uncover<10->{\hprojectcartoon{13}{what the customer really needed}} % se02-05
		%\hspace{-7mm}
		%\projectcartoon{07}{how the project was documented} % code documentation
		%\hprojectcartoon{18}{how patches were applied} % se11 evolution
		%\hprojectcartoon{17}{how it performed under load} % se14 compilation/quality assurance/performance
		%\hprojectcartoon{11}{what marketing advertised} % se15 reuse/product lines
		%\hprojectcartoon{16}{how open source version} % se16 open source/licensing
		%\projectcartoon{08}{what operations installed} % devops?
		%\projectcartoon{14}{what the digg effect can do to your site} % micro services?
		%\projectcartoon{15}{the disaster recover plan}
		%\hspace{-7mm}
	\end{fancycolumns}
\end{frame}

\subsection{Software Engineering}
\begin{frame}{\insertsubsection}
	\begin{fancycolumns}[widths={46}]
		\begin{definition}{Software Engineering \mysource{\sommerville}}
			\mycite{Software engineering is an engineering discipline that is concerned with all aspects of software production from initial conception to operation and maintenance. [...] Software engineering is not just concerned with the technical processes of software development. It also includes activities such as software project management and the development of tools, methods, and theories to support software development.}
		\end{definition}
		\nextcolumn
		\begin{definition}{Engineering \mysource{\sommerville}}
			\mycite{Engineering is about getting results of the required \emph{quality} within \emph{schedule} and \emph{budget}. [...] Engineers make things work. They apply theories, methods, and tools where these are appropriate. However, they use them selectively and always try to discover solutions to problems even when there are no applicable theories and methods. Engineers also recognize that they must work within organizational and financial constraints, and they must look for solutions within these constraints.}
		\end{definition}
	\end{fancycolumns}
\end{frame}

\begin{frame}[b]{Software Engineering vs Programming}
	\slideSEvsProgramming
\end{frame}

\begin{frame}{Software Engineering vs ...}
	\begin{fancycolumns}
		\begin{definition}{Computer Science \mysource{\sommerville}}
			\mycite{Computer science focuses on theory and fundamentals; software engineering is concerned with the practicalities of developing and delivering useful software. [...] Computer science theory, however, is often most applicable to relatively small programs. Elegant theories of computer science are rarely relevant to large, complex problems that require a software solution.}
		\end{definition}
	\nextcolumn
		\begin{definition}{System Engineering \mysource{\sommerville}}
			\mycite{System engineering is concerned with all aspects of computer-based systems development including hardware, software and process engineering. Software engineering is part of this more general process.}
		\end{definition}
		% "System engineering is therefore concerned with hardware development, policy and process design, and system deployment, as well as software engineering." [\sommerville]
		\pic[width=\linewidth]{misc/lawn-mower-cropped} % copied to process lecture
	\end{fancycolumns}
\end{frame}

%\subsection{Relevance of Software for This Course}
%\begin{frame}{\insertsubsection}
%	\begin{exampletight}{}
%		\picDark[width=\linewidth]{failures/google-scholar2}
%	\end{exampletight}
%\end{frame}
%
%\begin{frame}{\insertsubsection}
%	\centering\picDark[height=\textheightwithtitle]{failures/texstudio}
%\end{frame}
%
%\begin{frame}{\insertsubsection}
%	\begin{fancycolumns}[widths={54},animation=none]
%		\pic[width=\linewidth]{failures/obs-freeze}
%	\nextcolumn
%		\pic[width=\linewidth]{failures/obs-freeze2}
%	\end{fancycolumns}
%\end{frame}
%
%\begin{frame}{\insertsubsection}
%	\begin{fancycolumns}
%		\pic[width=\linewidth]{failures/acrobat-correct-green}
%		\nextcolumn
%		\pic[width=\linewidth]{failures/acrobat-wrong-green}
%	\end{fancycolumns}
%\end{frame}
%
%\xkcdframe{1197}
%
%\begin{frame}{\insertsubsection}
%	\centering\pic[height=\textheightwithtitle]{failures/flash}
%\end{frame}
%
%\begin{frame}{\insertsubsection}
%	\centering\pic[height=\textheightwithtitle]{failures/thunderbird}
%\end{frame}
%

\lessonslearned{
	\item What is software engineering?
	\item Which trade-off is crucial to software engineering?
	\item Next: What is expected from you during this term?
}{
	\item \sommerville\mychapter{1.1}\mypages{19--28} % TODO check
}{
	\begin{enumerate}
		\item<+-> Form groups of 2-3 students
		\item<+-> Discuss: What will you do for a living in 10 years? What will be your connection to software? (5 min)
		\item<+-> Survey: What is your connection to software in 10 years?
		\slideonly{~
			\item<+->[A] Wirst du in 10 Jahren Software entwickeln?
			\item<+->[B] Wirst du in 10 Jahren Software testen?
			\item<+->[C] Wirst du in 10 Jahren Software beauftragen?
			\item<+->[D] Wirst du in 10 Jahren Softwareanforderungen ermitteln?
			\item<+->[E] Wirst du in 10 Jahren die Entwicklung von Software leiten?
		}
	\end{enumerate}
}

\section{Course Overview}
\subsection{What You Should Know}

\begin{frame}{\myframetitle{}}
	\begin{fancycolumns}
		\begin{note}{Fundamentals of Software Engineering}
			\begin{itemize}
				\item development processes: waterfall, v-model, scrum
				%\item project management
				\item analysis, design, implementation, testing, maintanance
				\item UML diagrams (esp. class and component diagrams)
				\item version control (esp. git)
			\end{itemize}
			\ifuniversity{tubs}{$\Rightarrow$ \emph{Software Engineering 1}}
		\end{note}
	\nextcolumn
		\begin{note}{Fundamentals of Theoretical Computer Science}
			\begin{itemize}
				%\item set theory
				\item propositional logic
				\item predicate logic
			\end{itemize}
		\end{note}
		\begin{note}{Exercise}
			skills in object-oriented programming (preferably Java)

			%\ifuniversity{tubs}{$\Rightarrow$ \emph{Software Engineering 1}}
		\end{note}
	\end{fancycolumns}
\end{frame}

\subsection{What You Will Learn}

\begin{frame}{\myframetitle{}}
	\frameCourseQuestions
\end{frame}

\begin{frame}{\myframetitle{}}
	\lectureseriesoverview[1]
\end{frame}

\subsection{What You Might Need}

\begin{frame}{\insertsubsection}
	\begin{fancycolumns}[animation=none]
		\centering\pic[height=50mm]{books/sommerville-softwarenegineering}
		\nextcolumn
		\begin{definition}{\mysource{\sommerville}}
			\begin{itemize}
				\item \sommervillelink{Ian Sommerville. Software Engineering, 10. Edition, Pearson, 2018.}
				\begin{itemize}
					\item German, English, and earlier versions
					\item \href{https://software-engineering-book.com/videos/}{Videos by Ian Sommerville and others available online}
				\end{itemize}
				\item More literature announced in each lecture
			\end{itemize}
		\end{definition}
	\end{fancycolumns}
\end{frame}

%\subsection{Credit for the Slides}
%
%\begin{frame}{\myframetitle{}}
%	\ifuniversity{anonymous}{\mynote{}{\centering\huge Anonymous Authors}}
%	\unlessuniversity{anonymous}{
%		\myframeicon{\href{https://github.com/SoftVarE-Group/Course-on-Software-Product-Lines}{\pic[scale=.75]{cc-by-sa}}}
%	}
%	\begin{fancycolumns}[columns=3,animation=none]
%	\nextcolumn
%		\unlessuniversity{anonymous}{
%			\begin{note}{Thomas Thüm}
%				\centering
%				\href{https://www.uni-ulm.de/en/in/sp/team/thuem/}{\adjincludegraphics[height=.45\textheight,trim={.125\width} 0 {.125\width} 0,clip]{thomas-thuem}}
%
%				\small Professor at Paderborn University
%
%				software engineering
%
%				FeatureIDE team leader
%			\end{note}
%		}
%	\nextcolumn
%	\end{fancycolumns}
%\end{frame}

\inputuniversity{content/01c-course}
\lessonslearned{
	\item How is this course organized?
	\item Next Lecture: Why do programming languages matter?
}{
	\item Main Book: \sommervillelink{Ian Sommerville. Software Engineering, 10. Edition, Pearson, 2018.}
}{
	\begin{itemize}
		\item Any questions?
	\end{itemize}
}

%\faq{
%	\item
%}{
%	\item
%}{
%	\item
%}

% TODO L02 IMPLEMENTATION

\ifuniversity{tubs}{\date{October 22, 2024}}

\author{Thomas Thüm}
\lecture{Implementation}{implementation}

\begin{frame}{\insertsubtitle}
	\renewcommand{\projectcartoonwidth}{.19}
	\alt<2->{
		\projectcartoon{01}{how the customer explained it}
		\projectcartoon{02}{how the project leader understood it}
		\projectcartoon{03}{how the analyst designed it} % architecture/design
	}{%
		\hprojectcartoon{01}{how the customer explained it}
		\hprojectcartoon{02}{how the project leader understood it}
		\hprojectcartoon{03}{how the analyst designed it} % architecture/design
	}%
	\hprojectcartoon{04}{how the programmer implemented it}
	%\hprojectcartoon{05}{what the beta testers received} % testing
	%\hprojectcartoon{06}{how the business consultant described it}
	%\hprojectcartoon{07}{how the project was documented} % code documentation
	%\hprojectcartoon{08}{what operations installed} % devops/continuous integration
	%\hprojectcartoon{09}{how the customer was billed} % pricing
	%\hprojectcartoon{10}{how it was supported}
	%\hprojectcartoon{11}{what marketing advertised}
	%\hprojectcartoon{12}{when it was delivered} % continous delivery
	\uncover<-0|handout:0>{\projectcartoon{13}{what the customer really needed}}
	%\hprojectcartoon{14}{what the digg effect can do to your site} % ???
	%\hprojectcartoon{15}{the disaster recover plan}
	%\hprojectcartoon{16}{how open source version} % open source/licensing
	%\hprojectcartoon{17}{how it performed under load} % quality assurance/performance
	%\hprojectcartoon{18}{how patches were applied} % software maintenance
\end{frame}

\section{Choosing a Programming Language}
\subsection{History of Programming Languages}
\begin{frame}{\insertsubsection}
	\begin{fancycolumns}
		\begin{note}{Milestones\mysource{\jonesbestpractice}}
			\begin{itemize}
				\setlength\itemsep{.1em}
				\item controlling behavior of mechanical devices by wiring or with punchcards \deutsch{Lochkarten}
				\item machine languages used during World War II
				\uncover<2->{
					\item assembly languages: distinction between human-readable instructions (source code) and executable instructions (object code)
					\item birth of compilers and interpreters having a one-to-many mapping between source and object code (opposed to one-to-one mapping in assemblers)
				}
				\uncover<3->{
					\item structured programming pioneered by David Parnas and Edsger Dijkstra
					\item high-level programming languages: high number of executable for each human-readable instruction
					\item domain-specific languages, later general-purpose programming languages
				}
			\end{itemize}
		\end{note}
	\nextcolumn
		\uncover<4->{
		\begin{example}{Languages\mysource{\jonesbestpractice\ + \handbuch}}
			\begin{itemize}
				\setlength\itemsep{.1em}
				\item 1945: first high-level language Plankalkül by Konrad Zuse (compiler written in 1998)
				\item 1954: first professional high-level language FORTRAN (Formula Translator) by IBM
				\item 1963: Basic as general-purpose language
				\uncover<5->{
					\item 1959: functional language Lisp
					\item 1970: first object-oriented lang.\ Smalltalk-80
					\item 1970: declarative language SQL
				}
				\uncover<6->{
					\item 1971: Pascal by Niklaus Wirth for teaching
					\item 1974: very common procedural language C
					\item 1977: logical language Prolog
				}
				\uncover<7->{
					\item 1980: C++ as object-oriented extension of C
					\item 1990: object-oriented language Java
					\item 1990: functional language Haskell
				}
				\uncover<8->{
					\item 1991: multi-paradigm language Python % TODO for machine learning
					\item 1995: scripting language JavaScript % TODO for web applications
				}
			\end{itemize}
		\end{example}
	}
	\end{fancycolumns}
\end{frame}
% TODO revise and talk about paradigms and their languages

\subsection{Programming Languages Today}
\begin{frame}{\insertsubsection}
	\slideProgrammingLanguagesToday
\end{frame}

\subsection{Choice of Programming Languages}
\begin{frame}{\insertsubsection}
	\begin{fancycolumns}
		\begin{definition}{{Desired Properties\mysource{\ludewiglichter}}}
			\begin{itemize}
				\item modular implementation
				\item separation of inferfaces and implementations
				\item type system: strongly/weakly typed languages
				\item readable syntax (FORTRAN vs ALGOL60) % designed for fewer characters vs readability
				\item automatic pointer management (C vs Java)
				\item exception handling
			\end{itemize}
		\end{definition}
	\nextcolumn
		\begin{example}{Criteria in Practice}
			\begin{itemize}
				\item language required by the company or customer?
				\item existing infrastructure?
				\item domain-specific languages available?
				\item language known/liked by developers?
				\item available libraries?
				\item available tool support?
				\item language popularity?
				\item what may change in the future?
			\end{itemize}
		\end{example}
	\end{fancycolumns}
\end{frame}

\subsection{Popularity of Programming Languages}
\begin{frame}[b]{\insertsubsection}
	\picDark[width=\linewidth]{history/tiobe}
\end{frame}
\begin{frame}{\insertsubsection}
	\centering\picDark[width=.8\linewidth]{history/tiobe-longterm}
\end{frame}

\begin{frame}
	\begin{fancycolumns}[height=8.5cm]
		\pic[width=\linewidth,trim=0 20 0 20,clip]{people/patrick-mckenzie}
		\vspace{-7mm}
		
		\mynote{Patrick McKenzie \mysource{\href{https://twitter.com/CodeWisdom/status/1182702520696803329}{twitter.com}}}{\mycite{Every great developer you know got there by solving problems they were unqualified to solve until they actually did it.}}
		% computer scientist, entrepreneur, influencer
	\nextcolumn
		\pic[width=\linewidth,trim=0 55 0 10,clip]{people/bill-gates}
		\vspace{-7mm}
		
		\mynote{Bill Gates \mysource{\href{https://code.org/quotes}{code.org}}}{\mycite{Learning to write programs stretches your mind, and helps you think better, creates a way of thinking about things that I think is helpful in all domains.}}
		% richest person in 15 years between 1994 and 2014
	\end{fancycolumns}
\end{frame}


\lessonslearned{
	\item Historical perspective on programming
	\item Criteria for choosing languages
	\item Popularity of programming languages
	\item Next: What has programming to do with reading?
}{
	\item \jonesbestpractice\mychapter{8 Programming and Code Development}
	\item \handbuch\mychapter{2.4 Programming Languages}
}{
	\begin{enumerate}
		\item Enter 1+2*3 into the calculator app of your mobile device or laptop
		\item Compare the results with your colleagues
		\item Discuss why results (could) differ
	\end{enumerate}
}

\begin{frame}{Calc on Windows 10\ \mytitlesource{\href{https://youtu.be/LKDbQfzzGJo?t=1544}{youtube.de}}}
	\begin{fancycolumns}
		\centering\picDark[width=.66\linewidth]{failures/win10-calc-scientific}
	\nextcolumn
		\centering\picDark[width=.66\linewidth]{failures/win10-calc-standard}
	\end{fancycolumns}
\end{frame}

\section{Coding Conventions}
\begin{frame}
	\begin{fancycolumns}[height=8.5cm]
		\pic[width=\linewidth,trim=40 0 100 0,clip]{people/douglas-crockford}
		\vspace{-7mm}
		
		\begin{note}{Douglas Crockford \mysource{\javascript}}
			\mycite{It turns out that style matters in programming for the same reason that it matters in writing. It makes for better reading.}
		\end{note}
		% known for JavaScript, JSON, works at Paypal
	\nextcolumn
		\pic[width=\linewidth,trim=0 20 0 25,clip]{people/francois-chollet}
		\vspace{-7mm}[height=8.5cm]
		
		\begin{note}{François Chollet \mysource{\href{https://twitter.com/fchollet/status/1038200379605798912}{twitter.com}}}
			\mycite{In software, naming matters, because names reflect how you think about a problem. Code is also communication, and naming is a big part of making it work.}
		\end{note}
		% AI researcher at Google
	\end{fancycolumns}
\end{frame}

\xkcdframe{2021}

\subsection{Code Formatting}
\begin{frame}{\insertsubsection}
	\begin{fancycolumns}
		\begin{note}{Motivation}
			\begin{itemize}
				\item code is read much more often and by more developers than written
				\item avoid differences by each programmer
			\end{itemize}
		\end{note} % TODO add citation
		\begin{definition}{Code Formatting}
			\begin{itemize}
				\item indentation: typically 4 characters per level
				\item length of a line: often 80 or 100 characters
				\item extra indentation: typically 8 characters when breaking extra long lines
				\item empty lines between members (e.g., methods and attributes)
			\end{itemize}
		\end{definition} % TODO add citation
	\nextcolumn
		\begin{example}{Code Formatting in Practice}
			\begin{itemize}
				\item automated code formatters available (on demand or when saving the editor)
				\item typical formatting rules for each language
				\item automated code formatters are configurable (handle with care)
			\end{itemize}
		\end{example} % TODO add citation
	\end{fancycolumns}
\end{frame}

\subsection{Rules on Naming}
\begin{frame}{\insertsubsection}
	\begin{fancycolumns}
		\begin{example}{Unwanted Names}
			\begin{itemize}
				\item single character as a name
				\item very long names
				\item names consisting only of special chars
				\item synonyms: delete, remove, clear
				\item abbreviations (unless very common)
			\end{itemize}
		\end{example} % TODO add citation
	\nextcolumn
		\begin{definition}{Wanted Names}
			\begin{itemize}
				\item nouns for class names: Calculator
				\item nouns for attribute names: calculateButton
				\item verbs for method names: getCalculator(), evaluate(), isZero(), hasChildren(), setValue()
				\item PascalCaseNotation for classes, interfaces (not consistent in C/C++)
				\item camelCaseNotation for attributes, methods, local variables, parameters (exception: Python)
				\item UPPER\_SNAKE\_CASE\_NOTATION for constants (exception: Go)
				\item lowercasenotation for package names (exception: C\#)
			\end{itemize}
		\end{definition} % TODO add citation
	\end{fancycolumns}
\end{frame}

\begin{frame}
	\begin{fancycolumns}[height=8.5cm]
		\pic[width=\linewidth,trim=0 275 0 25,clip]{people/martin-fowler}
		\vspace{-7mm}
		
		\begin{note}{Martin Fowler (1999) \mysource{\refactoring}}
			\mycite{Any fool can write code that a computer can understand. Good programmers write code that humans can understand.}
		\end{note}
		% known for refactoring and agile development
	\nextcolumn
		\pic[width=\linewidth,trim=0 0 0 10,clip]{people/cory-house}
		\vspace{-7mm}
		
		\begin{note}{Cory House \mysource{\href{https://twitter.com/housecor/status/400479246713229312}{twitter.com}}}
			\mycite{Code is like humor. When you have to explain it, it’s bad.}
		\end{note}
		% influencer known for teaching React and JavaScript
	\end{fancycolumns}
\end{frame}

\subsection{Code Documentation}
\begin{frame}{\insertsubsection}
	\begin{fancycolumns}
		\begin{definition}{Comments \ldots}
			\begin{itemize}
				\item in source code are easier to maintain (than in external documents)
				\item should be written while editing the code
				\item can be used to generate documentation (e.g., JavaDoc, Doxygen)
				\item are used to specify classes and public methods (e.g., parameters, exceptions, dependencies)
				\item document hacks, side effects and unfinished parts (e.g., TODO)
				\item should not paraphrase the code
			\end{itemize}
		\end{definition}
	\end{fancycolumns}
\end{frame}

\begin{frame}
	\begin{fancycolumns}[height=8.5cm]
		\pic[width=\linewidth,trim=0 0 0 60,clip]{people/ryan-campbell}
		\vspace{-7mm}
		
		\begin{note}{Ryan Campbell \mysource{\href{https://handbook.problemsolving.io/01-model/03-code.html}{problemsolving.io}}}
			\mycite{Commenting your code is like cleaning your bathroom - you never want to do it, but it really does create a more pleasant experience for you and your guests.}
		\end{note}
		% software developer
	\nextcolumn
		\pic[width=\linewidth,trim=90 15 10 20,clip]{people/steve-mcconnell}
		\vspace{-7mm}
		
		\begin{note}{Steve McConnell (2004) \mysource{\codecomplete}}
			\mycite{Good code is its own best documentation. As you're about to add a comment, ask yourself, \mycite{How can I improve the code so that this comment isn't needed?} Improve the code and then document it to make it even clearer.}
		\end{note}
		% author of several textbooks on software development and project management
	\end{fancycolumns}
\end{frame}




\lessonslearned{
	\item Coding conventions \deutsch{Programmierrichtlinien}
	\item Formatting, naming, comments, and documentation
	\item Next: What tools are available for programming?
}{
	\item \javastyleguide
	% TODO add literature? \refactoring \codecomplete \javascript
}{
	\begin{enumerate}
		\item<+-> Form groups of 2-3 students
		\item What happens if each programmer applies different coding conventions?
		\item Would it help to agree on coding conventions for each file individually?
	\end{enumerate}
}

\section{Tools and Environments}
\subsection{Computer-Aided Software Engineering}
\begin{frame}{\insertsubsection}
	\begin{fancycolumns}[widths={55}]
		\begin{definition}{Terms \mysource{adapted from \ghezzi}}
			A \emph{tool} is an application that supports a particular activity. An \emph{environment} is a collection of related tools. Tools and environments aim at automating some of the activities that are involved in software engineering. The generic term for this field of study is \emph{computer-aided software engineering}.
		\end{definition}
	\end{fancycolumns}
\end{frame}

\widexkcdframe{378} % real programmers

\subsection{Overview on Development Tools}
\begin{frame}{\insertsubsection}
	\begin{fancycolumns}[widths={57}]
		\begin{note}{Variety of Tools \mysource{\ghezzi}}
			\begin{itemize}
				\item text(ual) editors: emacs, vim, ed, Word, \ldots
				\item graphical editors: UML editors, Powerpoint, \ldots
				\item assembler, compiler, interpreter
				\uncover<2->{
				\item configuration management tools: git, SVN, CVS, \ldots
				\item tracking tools (issue trackers): Github, Gitlab, \ldots
				\item tools for code navigation and refactoring
				\item tools for test specification, generation, execution, reporting
				\item tools for static and dynamic code analysis (e.g., debugger), reverse/reengineering, project management
				}
				\uncover<3->{
				\item integrated development environments (IDEs): Eclipse, Visual Studio, IntelliJ, Android Studio
				}
			\end{itemize}
		\end{note}
	\end{fancycolumns}
\end{frame}

\subsection{Demo on Tool Support in Eclipse}
\begin{frame}{\insertsubsection\ \mytitlesource{\href{https://youtu.be/Jxt77kTbFZ0?si=oGjCcdfji7NMmbM7&t=3318}{youtube.de}}}
	\centering\pic[width=.7\linewidth,trim=0 76 0 76,clip]{demo/livecoding2} % TODO update picture to recent version
\end{frame}


\lessonslearned{
	\item Tool support for numerous activities
	\item By means of tools, environments, and IDEs
	\item Next: How to find errors beyond compiler errors?
}{
	\item \ghezzi\mychapter{9 Software Engineering Tools and Environments}
}{
	\begin{enumerate}
		\item<+-> Form groups of 2-3 students
		\item<+-> What tools have you used already?
		\item<+-> How does development profit from those tools?
	\end{enumerate}
}

%\faq{
%	\item
%}{
%	\item
%}{
%	\item
%}

% TODO L03 Testing

\ifuniversity{tubs}{\date{October 29, 2024}}

\author{Thomas Thüm}
\lecture{Testing}{testing}

\begin{frame}{\insertsubtitle}
	\renewcommand{\projectcartoonwidth}{.18}
	\alt<2->{
		\projectcartoon{01}{how the customer explained it}
		\projectcartoon{02}{how the project leader understood it}
		\projectcartoon{03}{how the analyst designed it}
		\projectcartoon{04}{how the programmer implemented it}
	}{
		\hprojectcartoon{01}{how the customer explained it}
		\hprojectcartoon{02}{how the project leader understood it}
		\hprojectcartoon{03}{how the analyst designed it}
		\hprojectcartoon{04}{how the programmer implemented it}
	}%
	\hprojectcartoon{05}{what the beta testers received}
\end{frame}

\section{Quality Assurance}
\begin{frame}
	\begin{fancycolumns}[height=8.5cm,reverse]
		\pic[width=\linewidth,trim=0 240 0 300,clip]{people/andy-hunt}
		\vspace{-7mm}
		
		\begin{note}{Andy Hunt \mysource{\thepragmaticprogrammer}}
			\mycite{No one in the brief history of computing has ever written a piece of perfect software. It's unlikely that you'll be the first.}
		\end{note}
		% co-authored The Pragmatic Programmer, known for the Agile Manifesto
		\nextcolumn
		\pic[width=\linewidth,trim=425 0 400 0,clip]{people/donald-trump}
		\vspace{-7mm}
		
		\begin{note}{Donald Trump (May 2020) \mysource{\href{https://www.huffpost.com/entry/trump-testing-claim-pennsylvania_n_5ebdf19bc5b6c9c187419778}{huffpost.com}}}
			\mycite{If we didn’t do any testing, we would have very few cases.}
		\end{note}
	\end{fancycolumns}
\end{frame}

\subsection{Software Quality}
\begin{frame}{\insertsubsection}
	\begin{fancycolumns}
		\begin{definition}{Quality \mysource{\ludewiglichter}}
			Quality is the entirety of properties and characteristics of a product or process that indicate adequacy with respect to given requirements.
		\end{definition}
		\begin{definition}{Quality Assurance \mysource{\ludewiglichter}}
			Quality assurance \deutsch{Qualitätssicherung} are all activities with the goal to improve the quality.
		\end{definition}
		\nextcolumn
		\begin{note}{Expectations on Quality \mysource{\sommerville}}
			\mycite{Because of their previous experiences with buggy, unreliable software, users sometimes have low expectations of software quality. They are not surprised when their software fails. When a new system is installed, users may tolerate failures because the benefits of use outweigh the costs of failure recovery. However, as a software product becomes more established, users expect it to become more reliable.\uncover<3->{ [...] If a software product or app is very cheap, users may be willing to tolerate a lower level of reliability.}\uncover<4->{ [...] Customers may be willing to accept the software, irrespective of problems, because the costs of not using the software are greater than the costs of working around the problems.}}
		\end{note}
	\end{fancycolumns}
\end{frame}

\subsection{Product Quality}
\begin{frame}[label=productquality]{\insertsubsection\ \mytitlesource{\isoiectfzoz}}
	\label{slide:productquality}
	\vspace{-9mm}
	\hfill
	\begin{tikzpicture}
		\path[small mindmap,
		every node/.style={concept,font=\scriptsize},
		emph/.style={font=\bfseries\scriptsize},
		hide/.style={visible on=<1->},
		concept color=foreground!20!background,
		level 1/.append style={level distance=25mm,sibling angle=360/8},
		level 2/.append style={level distance=20mm,sibling angle=360/12},
		level 3/.append style={level distance=20mm,sibling angle=360/8},
		]
		node {Product Quality \deutsch{Produktqualität}}
		[clockwise from=0]
		child[concept color=orange!20!background,visible on={<5,7>},level distance=33mm] { node {Maintainability} 
			[clockwise from=30]
			child { node {Modularity} }
			child { node {Reusability} }
			child { node {Analysability} }
			child { node {Modifyability} }
			child { node {Testability} }
		}
		child[concept color=red!20!background,visible on={<4,7>}] { node {Performance Efficiency ...} }
		child[concept color=blue!20!background,visible on={<4,7>}] { node {Compatibility ...} }
		child[concept color=green!20!background,visible on={<4,7>}] { node {Usability ...} }
		child[concept color=red!20!background,visible on={<3,7>},level distance=28mm] { node {Reliability} 
			[clockwise from=255]
			child { node {Maturity} }
			child { node {Availability} }
			child { node {Fault Tolerance} }
			child { node {Recoverability} }
		}
		child[concept color=orange!20!background,visible on={<2,7>}] { node {Security} 
			[clockwise from=190]
			child { node {Confidentiality} }
			child { node {Integrity} }
			child { node {Non-Repudiation} }
			child { node {Accountability} }
			child { node {Authenticity} }
		}
		child[concept color=green!20!background,visible on={<1,7>}] { node {Functional Suitability} 
			[clockwise from=90]
			child { node {Completeness} }
			child { node {Correctness} }
			child { node {Appropriateness} }
		}
		child[concept color=blue!20!background,visible on={<6-7>}] { node {Portability} 
			[clockwise from=50]
			child { node {Adaptability} }
			child { node {Installability} }
			child { node {Replaceability} }
		}
		;
	\end{tikzpicture}
	\hspace{-5mm}
\end{frame}
% omitted categories:
%   performance efficiency: time behaviour, resource utilization, capacity
%   compatibility: co-existence, interoperability
%   usability: appropriateness recognizability, learnability, operability, user error protection,
%              user interface aesthetics, accessibility

\subsection{Quality in Use}
\begin{frame}{\insertsubsection\ \mytitlesource{\isoiectfzoz}}
	\begin{tikzpicture}
		\path[small mindmap,
		every node/.style={concept,font=\scriptsize},
		emph/.style={font=\bfseries\scriptsize},
		hide/.style={visible on=<1->},
		concept color=foreground!20!background,
		level 1/.append style={level distance=25mm,sibling angle=360/7},
		level 2/.append style={level distance=20mm,sibling angle=360/11},
		level 3/.append style={level distance=20mm,sibling angle=360/8},
		]
		node {Quality in Use \deutsch{Gebrauchsqualität}}
		[clockwise from=193]
		child[concept color=blue!20!background,visible on={<1,5>}] { node {Effectiveness} }
		child[concept color=blue!20!background,visible on={<1,5>}] { node {Efficiency} }
		child[concept color=green!20!background,visible on={<2,5>}] { node {Satisfaction} 
			[clockwise from=155]
			child { node {Usefulness} }
			child { node {Trust} }
			child { node {Pleasure} }
			child { node {Comfort} }
		}
		child[concept color=red!20!background,visible on={<3,5>}] { node {Freedom from Risk} 
			[clockwise from=75]
			child { node {Economic Risk Mitigation} }
			child { node {Health and Safety Risk Mitigation} }
			child { node {Environmental Risk Mitigation} }
		}
		child[concept color=orange!20!background,visible on=<4-5>] { node {Context Coverage} 
			[clockwise from=30]
			child { node {Context Completeness} }
			child { node {Flexibility} }
		}
		;
	\end{tikzpicture}
\end{frame}

\xkcdframe{2200} % unreachable state

\subsection{Software Testing}
\begin{frame}{\insertsubsection}
	\begin{fancycolumns}[height=1cm,animation=none]
		\uncover<1->{
			\begin{definition}{Software Testing \mysource{\sommerville}}
				\mycite{Testing is intended to show that a program does what it is intended to do and to discover program defects before it is put into use.}
			\end{definition}
		}
		\uncover<2->{
			\begin{definition}{Validation Testing \mysource{\sommerville}}
				\mycite{Demonstrate to the developer and the customer that the software meets its requirements.}
			\end{definition}
		}
		\uncover<3->{
			\begin{definition}{Defect Testing \mysource{\sommerville}}
				\mycite{Find inputs or input sequences where the behavior of the software is incorrect, undesirable, or does not conform to its specification.}
			\end{definition}
		}
		\nextcolumn
		%\vspace{-12mm}
		\uncover<4->{
			\begin{note}{V\&V \mysource{\seeconomics}}
				\mycite{\emph{Validation}: Are we building the right product?\\\emph{Verification}: Are we building the product right?}
			\end{note}	
		} 
		% TODO better visualization of V&V (see Inas slides). move to V model?
		\vspace{-.7mm}
		\uncover<5->{
			\begin{note}{Stages of Testing \mysource{\sommerville}}
				\begin{itemize}
					\setlength\itemsep{.1em}
					\item[1.] \mycite{\emph{Development testing}, where the system is tested during development to discover
						bugs and defects}
					\item[2.] \mycite{\emph{Release testing}, where a separate testing team tests a complete version of the
						system before it is released to users}
					\item[3.] \mycite{\emph{User testing}, where users or potential users of a system test the system in their
						own environment}
				\end{itemize}
			\end{note}
		}
		\vspace{-.7mm}
		\uncover<6->{
			\begin{note}{}
				\mycite{In \emph{manual testing}, a tester runs the program with some test data and
					compares the results to their expectations. [...] In \emph{automated testing}, the tests are encoded in a program that is run each time the system under development is to be tested.} \mysource{\sommerville}
			\end{note}
		}
	\end{fancycolumns}
\end{frame}

\subsection{Quality Assurance}
\begin{frame}{\insertsubsection\ \mytitlesource{\ludewiglichter}}
	\slideMindmapQualityAssurance{}{}{}{visible on={<2->}}{visible on={<4->}}{visible on={<5->}}{visible on={<3->}}
\end{frame}

\subsection{Code Reviews} % TODO add literature on code reviews
\begin{frame}{\insertsubsection}
	\begin{fancycolumns}
		\begin{definition}{}
			\begin{itemize}
				\item Idea: improve quality by asking other programmers for feedback
				\item Typically applied with quality checklist
				\item Quality criteria: functionality, comprehensibility, maintainability, coding guidelines, design patterns, \ldots
				\item Reviewer selection: based on familiarity with code, availability, expertise
			\end{itemize}
		\end{definition}
		\pic[width=\linewidth,trim=0 40 0 10,clip]{codereview/codereview1}
		\nextcolumn
		\begin{note}{}
			\begin{itemize}
				\item Cannot be done by yourself
				\item Reviewers need programming experience and knowledge of the code (mutual feedback)
				\item Feedback should be timely and constructive
				\item Only changes reviewed, not too many
			\end{itemize}
		\end{note}
		\uncover<3->{\pic[width=.49\linewidth,trim=83 0 210 0,clip]{codereview/codereview3}}\hfill
		\uncover<4->{\pic[width=.49\linewidth,trim=50 0 450 0,clip]{codereview/codereview2}}
	\end{fancycolumns}
\end{frame}

\begin{frame}{\insertsubsection\ on Github}
	\picDark[width=\linewidth]{../pics/codereview/github}
	%\href{https://github.com/tthuem/FeatureIDE/commit/73df3fd463487c3adb17ca9cb39ba647deddc5ba}{\includegraphics[height=.9\textheight]{../pics/codereview/github2}}
\end{frame}

\lessonslearned{
	\item What is quality, quality assurance, product quality, quality in use?
	\item Terms: software testing, validation / defect testing, validation / verification, development / release / user testing, manual / automated testing, constructive / analytical / organizational quality assurance
	\item Code reviews
	\item Next: How to cover all parts of a program?
}{
	\item \sommerville, Chapter 8 Software Testing
	\item \ludewiglichter, Chapter 5 \deutsch{Software-Qualität} and Chapter 13 \deutsch{Software-Qualitätssicherung und -Prüfung}
}{
	\begin{enumerate}
		\item<+-> Form in groups of 2--3 students
		\item<+-> Discuss own examples for insufficient product quality
		\item<+-> Do you find an example for every kind of product quality?\\~\\functional suitability, security, reliability, usability, compatibility, performance efficiency, maintainability, portability (see Slide~\ref{slide:productquality})
	\end{enumerate}
}
	%	\item Open your favorite IDE and write a small example program illustrating one error detected by the Java compiler
	%	\item Upload a screenshot of your program in Moodle including a brief description of the error: \url{https://moodle.uni-ulm.de/mod/moodleoverflow/discussion.php?d=2127}
	%	\item Do a code review for another example (i.e., comment on possible improvements)
	%	\item See \href{https://moodle.uni-ulm.de/mod/moodleoverflow/discussion.php?d=3866}{Moodle} % TODO in 2023 update link
	%	\item Think of a change to the modulo method (cf.\ previous slide) that would be classified as error or warning by the Java compiler
	%	\item Do a code review of the \href{https://github.com/tthuem/2020WS-SWT-Calculator/blob/testinglecturev0/ModuloExample/src/de/uulm/sp/swt/moduloexample/ModuloExample.java}{modulo method} (i.e., comment on possible improvements) % TODO in 2023 update link
	%	\item Submit your results in Moodle and inspect other submissions
	%\item[] Answer the quiz in Moodle to track your learning progress of this chapter
	%\\\hfill\fancyqr{https://moodle.uni-ulm.de/course/view.php?id=42562\#section-7}

\slideonly{
	\addtocounter{framenumber}{-1}
	\againframe<7>{productquality}
}

\section{White-Box Testing}
\begin{frame}
	\begin{fancycolumns}[height=8.5cm]
		\pic[width=\linewidth,trim=0 425 0 75,clip]{people/edsger-dijkstra}
		\vspace{-7mm}
		
		\begin{note}{Edsger W. Dijkstra (1972) \mysource{\thehumbleprogrammer}}
			\mycite{Program testing can be a very effective way to show the presence of bugs, but it is hopelessly inadequate for showing their absence.}
		\end{note}
		% 1930-2002, ACM Turing Award winner
		\nextcolumn
		%\href{}{\includegraphics[width=\linewidth,trim=0 0 0 0,clip]{burt-rutan}}
		\vspace{56mm}
		
		\begin{note}{Burt Rutan \mysource{\href{https://www.routledge.com/Design-of-Biomedical-Devices-and-Systems-4th-edition/King-Fries-Johnson/p/book/9781138723061}{King~et~al.\ 2018}}}
			\mycite{Testing leads to failure, and failure leads to understanding.}
		\end{note}
		% unclear
	\end{fancycolumns}
\end{frame}

\subsection{Test Cases}
\begin{frame}{\insertsubsection\ \deutschertitel{Testfälle}}
	\begin{fancycolumns}[animation=none]
		\uncover<1->{
			\begin{definition}{Systematic Test \mysource{\ludewiglichter}}
				A systematic test is a test, in which
				\begin{itemize}
					\item[1.] the setup is defined,
					\item[2.] the inputs are chosen systematically,
					\item[3.] the results are documented and evaluated by criteria being defined prior to the test. 
				\end{itemize}
			\end{definition}
		}
		\uncover<2->{
			\begin{definition}{Test Case \mysource{\ludewiglichter}}
				In a test, a number of test cases are executed, whereas each test case consists \emph{input values} for a single execution and \emph{expected outputs}. An \emph{exhaustive test} refers a test in which the test cases exercise all the possible inputs.
			\end{definition}
		}
		\nextcolumn
		\uncover<3->{
			\begin{example}{Exhaustive Testing in Practice?}
				\texttt{boolean a, b, c;}\\			
				\texttt{int i, j;}\\~\\
				\texttt{bla(a,b,c)} \uncover<4->{has $2^3 = 8$ possible inputs}\\
				\texttt{blub(i,j)} \uncover<5->{has $(2^{32})^2 = 2^{64} \approx 10^{19}$ inputs}
				\uncover<6->{\begin{itemize}
						\item assuming $10^9$ test cases can be executed in 1~second (cf.\ CPU with more than 1 GHz)
						\item exhaustive test of \texttt{blub} takes $\approx 585$ years
						\item testing for a day would cover less than 0.0005 \% of the inputs
				\end{itemize}}
				\uncover<7->{How to test thousands of such methods several times a day?}
			\end{example}
		}
	\end{fancycolumns}
\end{frame}

\subsection{Test-Case Design}
\begin{frame}{\insertsubsection\ \deutschertitel{Testfallentwurf}}
	\begin{fancycolumns}
		\begin{note}{Goal \mysource{\ludewiglichter}}
			Detect a large number of failures with a low number of test cases. A test case (execution) is \emph{positive}, if it detects a failure, and \emph{successful} if it detects an unknown failure.
		\end{note}
		\begin{definition}{An ideal test case is \ldots \mysource{\ludewiglichter}}
			\begin{itemize}
				\item representative: represents a large number of feasible test cases
				\item failure sensitive: has a high probability to detect a failure
				\item non-redundant: does not check what other test cases already check
			\end{itemize}
		\end{definition}
		\nextcolumn
		\\\vspace{-7mm}
		\hfill
		\begin{tikzpicture}
			\path[small mindmap,
			every node/.style={concept,font=\scriptsize},
			emph/.style={font=\bfseries\scriptsize},
			concept color=foreground!20!background]
			node {Test-Case Design}
			child[concept color=blue!20!background,visible on={<2->}] { node {experience-based} }
			child[concept color=green!20!background,visible on={<3->}] { node {structure-based} 
				child { node {white-box testing} }
			}
			child[concept color=orange!20!background,visible on={<4->}] { node {specification-based} 
				child { node {black-box testing} }
			}
			;
		\end{tikzpicture}
	\end{fancycolumns}
\end{frame}

\subsection{Modulo in Different Programming Languages}
\begin{frame}{\insertsubsection}
	\begin{fancycolumns}[widths={55}]
		\begin{exampletight}{Overview by Torsten Curdt:}
			\centering\picDark[height=60mm]{testing/modulo-negative}
		\end{exampletight}
		\nextcolumn
		\begin{exampletight}{An Own Modulo Implementation}
			\centering\pic[height=60mm]{testing/modulo-with-javadoc-small-nohint} % TODO in 2023 update link
		\end{exampletight}
	\end{fancycolumns}
\end{frame}

\tikzset{emph/.style={thick,draw=blue,fill=blue!10!background},e1/.style={},e2/.style={},e3/.style={},e4/.style={},e5/.style={},e6/.style={},e7/.style={},e8/.style={}}
\renewcommand{\cfg}{ % TODO hack as command already exists, rename consistently
	\begin{tikzpicture}[yscale=-.5,xscale=.866,every node/.style={draw,semithick,circle,fill=background},to/.style={->,>={Stealth[round]},semithick,}]
		\node[e1,e2,e3,e4,e5,e6,e7,e8] (1) at (0,0) {1};
		\node[e1,e4,e5] (2) at (1,1) {2};
		\node[e1,e2,e4,e5,e6,e7,e8] (3) at (0,2) {3};
		\node[e1,e2,e4,e5,e6,e7,e8] (4) at (0,4) {4};
		\node[e1,e5,e7,e8] (5) at (1,5) {5};
		\node[e1,e2,e8] (6) at (0,6) {6};
		\draw[to,e1,e4,e5] (1) to (2);
		\draw[to,e2,e6,e7,e8] (1) to (3);
		\draw[to,e1,e4,e5] (2) to (3);
		\draw[to,e1,e2,e4,e5,e6,e7,e8] (3) to (4);
		\draw[to,e1,e5,e7,e8] (4) to[bend right=20] (5);
		\draw[to,e1,e2,e8] (4) to (6);
		\draw[to,e1,e5,e7,e8] (5) to[bend right=20] (4);
	\end{tikzpicture}
}

\subsection{White-Box Testing}
\begin{frame}{\insertsubsection\ \deutschertitel{Strukturtest}}
	\begin{fancycolumns}[animation=none]
		\begin{definition}{White-Box Testing \mysource{\ludewiglichter}}
			\begin{itemize}
				\setlength\itemsep{.1em}
				\item inner structure of test object is used
				\item idea: coverage of structural elements
				\item code translated into control flow graph
				\item specific test case (concrete inputs)\\derived from logical test case (conditions)\\derived from path in control flow graph
			\end{itemize}
		\end{definition}
		\uncover<2->{
			\begin{exampletight}{}
				\pic[width=\linewidth]{testing/modulo}
			\end{exampletight}
		}
		\nextcolumn
		\\\vspace{33mm}
		{\only<4->{\tikzset{e2/.append style={emph}}}\only<3->{\cfg}}
		%\only<3->{\tikzset{e1/.append style={emph}}\cfg}
	\end{fancycolumns}
\end{frame}
% control flow graphs as activity diagrams: \umluserguide, Chapter~20, Modeling an Operation

\subsection{Coverage Criteria}
\begin{frame}{\insertsubsection\ \deutschertitel{Überdeckungskriterien}}
	\begin{fancycolumns}[animation=none]
		\begin{definition}{Coverage Criteria \mysource{\ludewiglichter}}
			\begin{itemize}
				\item[1.] statement coverage \deutsch{Anweisungsüberdeck.}: all statements are executed for at least one test case
				\uncover<3->{\item[2.] branching coverage \deutsch{Zweigüberdeckung}: statement coverage and for each branching statement all branches have been exercised} % TODO not so easy to define as percentage
				\uncover<4->{\item[3.] term coverage \deutsch{Termüberdeckung}: branching coverage and terms ($n$) used in a branching statement are combined exhaustively ($2^n$)\hfill(simplified)}
				% TODO discuss path coverage?
			\end{itemize}
		\end{definition}
		\vspace{-1mm}
		\uncover<2->{\small
			\begin{example}{In Practice}
				100\% statement coverage not feasible in presence of dead code or some unreachable error handling
				
				\uncover<5->{100\% term coverage not feasible for certain dependencies between choices: \texttt{even() || odd()}}
			\end{example}
		}
		\nextcolumn
	\end{fancycolumns}
\end{frame}

\subsection{Statement Coverage}
\begin{frame}
	\begin{fancycolumns}[t,animation=none]
		\begin{exampletight}{Statement Coverage for Modulo Example}
			\pic[width=\linewidth]{testing/modulo}
		\end{exampletight}
		\begin{example}{First Test Case}
			\setlength\tabcolsep{.5mm}
			\begin{tabularx}{\textwidth}{rl}				
				path: & \uncover<2->{1}\uncover<3->{, 2, 3, 4}\uncover<4->{, 5, 4}\uncover<5->{, 6}\\
				logical test case: & \uncover<3->{$b < 0$}\uncover<4->{ $\wedge$ $a > -b$}\uncover<5->{ $\wedge$ $a+b \leq -b$}\\
				specific test case: & \uncover<6->{$a = 5$, $b = -3$}\\
				expected result: & \uncover<7->{$m = 2$}
			\end{tabularx}
		\end{example}
		\nextcolumn
		\\[5mm]
		\only<1|handout:0>{\cfg}%
		\only<2|handout:0>{\tikzset{e3/.append style={emph}}\cfg}%
		\only<3|handout:0>{\tikzset{e4/.append style={emph}}\cfg}%
		\only<4|handout:0>{\tikzset{e5/.append style={emph}}\cfg}%
		\only<5->{\tikzset{e1/.append style={emph}}\cfg}%
		\uncover<8->{\correct}
	\end{fancycolumns}
\end{frame}

\subsection{Branching Coverage}
\begin{frame}[plain]
	\begin{fancycolumns}[t,animation=none]
		\begin{exampletight}{Branching Coverage for Modulo Example}
			\pic[width=\linewidth]{testing/modulo}
		\end{exampletight}
		\vspace{-1mm}
		\begin{example}{First Test Case}
			\setlength\tabcolsep{.5mm}
			\begin{tabularx}{\textwidth}{rl}				
				path: & 1, 2, 3, 4, 5, 4, 6\\
				logical test case: & $b < 0$ $\wedge$ $a > -b$ $\wedge$ $a+b \leq -b$\\
				specific test case: & $a = 5$, $b = -3$ and $m = 2$
			\end{tabularx}
		\end{example}
		\vspace{-1mm}
		\begin{example}{Second Test Case}
			\setlength\tabcolsep{.5mm}
			\begin{tabularx}{\textwidth}{rl}				
				path: & \uncover<2->{1}\uncover<3->{, 3, 4}\uncover<4->{, 6}\\
				logical test case: & \uncover<3->{$b \geq 0$}\uncover<4->{ $\wedge$ $0 \leq a \leq b$}\\
				specific test case: & \uncover<5->{$a = 0$, $b = 5$}\uncover<6->{ and $m = 0$}
			\end{tabularx}
		\end{example}
		\nextcolumn
		\\[5mm]
		{\tikzset{e1/.append style={emph}}\cfg}
		\only<1|handout:0>{\cfg}%
		\only<2|handout:0>{\tikzset{e3/.append style={emph}}\cfg}%
		\only<3|handout:0>{\tikzset{e6/.append style={emph}}\cfg}%
		\only<4->{\tikzset{e2/.append style={emph}}\cfg}%
		\uncover<7->{\correct}
		\\[15mm]
		\uncover<8->{\begin{note}{Ternary Operator in Statement 5}
				\begin{itemize}
					\item could also be treated as branching statement
					\item how to adapt the control flow graph then?
					\item are the two test cases still sufficient?
				\end{itemize}
		\end{note}}
	\end{fancycolumns}
\end{frame}

\subsection{Term Coverage}
\begin{frame}[plain]
	\begin{fancycolumns}[t,animation=none]
		\begin{exampletight}{Term Coverage for Modulo Example}
			\pic[width=\linewidth]{testing/modulo}
		\end{exampletight}
		\vspace{-1mm}
		\begin{example}{}
			\setlength\tabcolsep{.5mm}
			\begin{tabularx}{\textwidth}{rl}				
				logical test case: & $b < 0$ $\wedge$ $a > -b$ $\wedge$ $a+b \leq -b$\\
				specific test case: & $a = 5$, $b = -3$ and $m = 2$\uncover<14->{\correct}
			\end{tabularx}
		\end{example}
		\vspace{-1mm}
		\begin{example}{}
			\setlength\tabcolsep{.5mm}
			\begin{tabularx}{\textwidth}{rl}				
				logical test case: & $b \geq 0$ $\wedge$ $0 \leq a \leq b$\\
				specific test case: & $a = 0$, $b = 5$ and $m = 0$\uncover<14->{\correct}
			\end{tabularx}
		\end{example}
		\vspace{-1mm}
		\begin{example}{}
			\setlength\tabcolsep{.5mm}
			\begin{tabularx}{\textwidth}{rl}				
				path: & \uncover<8->{1}\uncover<9->{, 3, 4}\uncover<10->{, 5, 4}\uncover<11->{, 6}\\
				logical test case: & \uncover<9->{$b \geq 0$}\uncover<10->{ $\wedge$ $a < 0$}\uncover<11->{ $\wedge$ $0 \leq a+b \leq b$}\\
				specific test case: & \uncover<12->{$a = -2$, $b = 5$}\uncover<13->{ and $m = 3$}\uncover<14->{\correct}
			\end{tabularx}
		\end{example}
		\nextcolumn
		\\[5mm]
		{\tikzset{e1/.append style={emph}}\cfg}
		{\tikzset{e2/.append style={emph}}\cfg}
		\only<7|handout:0>{\cfg}%
		\only<8|handout:0>{\tikzset{e3/.append style={emph}}\cfg}%
		\only<9|handout:0>{\tikzset{e6/.append style={emph}}\cfg}%
		\only<10|handout:0>{\tikzset{e7/.append style={emph}}\cfg}%
		\only<11->{\tikzset{e8/.append style={emph}}\cfg}\\[5mm]
		\uncover<2->{$\neg(m < 0)$}\uncover<3->{ and $m > b$\correct}\\[5mm]
		\uncover<2->{$\neg(m < 0)$}\uncover<4->{ and $\neg(m > b)$\correct}\\[5mm]
		\uncover<5->{$m < 0$ and $m > b$}\uncover<6->{ impossible\only<15->{*}\alsocorrect}\\[5mm]
		\uncover<5->{$m < 0$ and $\neg(m > b)$}\uncover<13->{\correct}\\[3mm]
		\uncover<15->{\tiny*\,see third part of the lecture}
	\end{fancycolumns}
\end{frame}

\lessonslearned{
	\item Systematic test, exhaustive testing
	\item Test case: representative, failure sensitive, non-redundant
	\item Test-case design: experience-/structure-/specification-based
	\item Coverage in white-box testing: statement/branching/term coverage
	\item Next: How to cover all parts of the specification?
}{
	\item \ludewiglichter, Chapter 19 \deutsch{Programmtest}
}{
	\pic[width=\linewidth]{testing/modulo}
	
	Determine the coverage of a test suite.
}
	%	\item See \href{https://moodle.uni-ulm.de/mod/moodleoverflow/discussion.php?d=3867}{Moodle} % TODO in 2023 update link
	%	\item Watch screencast on how to do white-box testing with JUnit
	%	\item Implement a new operation in the calculator and update the white-box tests accordingly: \url{https://github.com/tthuem/2020WS-SWT-Calculator/tree/testinglecturev1} % TODO in 2023 update link
	%	\item Post your code changes and tests in Moodle
	%\item[] Determine the coverages of certain Modulo Test Cases in Moodle
	%\\\hfill\fancyqr{https://moodle.uni-ulm.de/course/view.php?id=42562\#section-7}

\section{Black-Box Testing}
\subsection{Black-Box Testing}
\begin{frame}{\insertsubsection\ \mytitlesource{\ludewiglichter}}
	\begin{fancycolumns}
		\begin{note}{Motivation}
			\begin{itemize}
				\item source code not always available (e.g., outsourced components, obfuscated code)
				\item specific test cases derived from logical ones using arbitrary values
				\item specification not incorporated so far (only for expected results)
				\item invalid inputs not tested
				\item errors are not equally distributed
			\end{itemize}
		\end{note}
		\begin{definition}{Black-Box Testing \deutsch{Funktionstest}}
			\begin{itemize}
				\item test-case design based on specification
				\item source code and its inner structure is ignored (assumed as a black-box)
			\end{itemize}
		\end{definition}
		\nextcolumn
		\vspace{-5mm}
		\begin{definition}{1. Equivalence Class Testing}
			\begin{itemize}
				\item idea: classify inputs and outputs into equivalence classes
				\item assumption: equivalent test cases detect the same faults, one test case is sufficient
			\end{itemize}
		\end{definition}
		\begin{definition}{2. Boundary Testing}
			\begin{itemize}
				\item extension of equivalence class testing
				\item goal: use experience (e.g., off-by-one errors)
				\item for every equivalence class: consider smallest, typical, and largest value
			\end{itemize}
		\end{definition}
		\begin{example}{In Practice}
			\begin{itemize}
				\item often combinations of white-box and black-box testing
				\item more techniques with requirements or design
			\end{itemize}
		\end{example}
	\end{fancycolumns}
\end{frame}

\subsection{Equivalence Class Testing}
\begin{frame}{\insertsubsection}
	\begin{fancycolumns}[columns=3,widths={46,4,50},animation=none]
		\begin{exampletight}{JavaDoc Specification for Modulo Example}
			\picWhite[width=\linewidth]{testing/modulo-only-javadoc-small-nohint}
		\end{exampletight}
		\begin{example}{Equivalence Classes}
			\begin{itemize}
				\item input a: $a < 0$, $a \geq 0$
				\item input b: $b < 0$, $b \geq 0$
				\item output: $m = 0$, $m > 0$, exception
			\end{itemize}
		\end{example}
		\nextcolumn
		\nextcolumn
		\uncover<2->{
			\begin{example}{Test Cases}
				\begin{tabular}{c c c c}	% temp
					% TODO: fix M{c}{num-}
					%\begin{tabular}{M{c}{1-} M{c}{4-} M{c}{5-} M{c}{6-}}
					& TC1 & TC2 & TC3 \\
					\toprule
					$a < 0$ & X &  &  \\
					$a \geq 0$ &  & X & X \\
					\midrule
					$b < 0$ & X &  &  \\
					$b > 0$ &  & X &  \\
					$b = 0$ &  &  & X \\
					\midrule
					$m = 0$ & X &  &  \\
					$m > 0$ &  & X &  \\
					exception &  &  & X \\
					\midrule
					\uncover<3->{input a & -3 & 1 & 2} \\
					\uncover<3->{input b & -3 & 2 & 0} \\
					\uncover<3->{expected output & 0 & 1 & exception} \\
					\midrule
					\uncover<4->{result & 0\correct & 1\correct & timeout\wrong} \\
					\bottomrule
				\end{tabular}
			\end{example}
		}
	\end{fancycolumns}
\end{frame}

\subsection{Boundary Testing}
\begin{frame}{\insertsubsection}
	\begin{fancycolumns}[columns=3,widths={4,92}]
		\nextcolumn
		\begin{example}{Test Cases}
			\begin{tabular}{c c c c c c c c}	% temp
				% TODO: fix M{c}{num-}
				%\begin{tabular}{M{c}{1-} M{c}{1-} M{c}{1-} M{c}{1-} M{c}{2-} M{c}{3-} M{c}{4-} M{c}{5-}}
				& TC1 & TC2 & TC3 & TC4 & TC5 & TC6 & TC7 \\
				\toprule
				$a < 0$ & X &  &  & min & max &  &  \\
				$a \geq 0$ &  & X & X &  &  & min & max \\
				\midrule
				$b < 0$ & X &  &  & max &  & min &  \\
				$b > 0$ &  & X &  &  & max &  & min \\
				$b = 0$ &  &  & X &  &  &  &  \\
				\midrule
				$m = 0$ & X &  &  & X &  & X & X \\
				$m > 0$ &  & min &  &  & max &  &  \\
				exception &  &  & X &  &  &  &  \\
				\midrule
				\uncover<1->{input a & -3 & 1 & 2 & minInt & -1 & 0 & maxInt }\\
				\uncover<1->{input b & -3 & 2 & 0 & -1 & maxInt & minInt & 1 }\\
				\uncover<1->{expected output & 0 & 1 & exception & 0 & maxInt-1 & 0 & 0 }\\
				\midrule
				\uncover<2->{result & 0\correct & 1\correct & timeout\wrong & 0\correct & maxInt-1\correct & timeout\wrong & 1\wrong }\\
				\bottomrule
			\end{tabular}
		\end{example}
		\nextcolumn
	\end{fancycolumns}
\end{frame}
%// a smallest negative, b largest negative, smallest result
%assertEquals(0,ModuloExample.modulo(Integer.MIN_VALUE,-1));
%// a largest negative, b largest positive, largest result
%assertEquals(Integer.MAX_VALUE-1,ModuloExample.modulo(-1,Integer.MAX_VALUE));
%// a smallest positive, b smallest negative, smallest result
%//assertEquals(0,ModuloExample.modulo(0,Integer.MIN_VALUE)); // leads to endless loop for every a
%assertEquals(0,ModuloExample.modulo(0,Integer.MIN_VALUE+1));
%// a largest positive, b smallest positive, smallest result
%assertEquals(0,ModuloExample.modulo(Integer.MAX_VALUE,1));

\subsection{Detected Faults in Modulo Example}
\begin{frame}
	\begin{fancycolumns}[animation=none]
		\uncover<1->{
			\begin{exampletight}{Detected Faults in Modulo Example}
				\picWhite[width=\linewidth]{testing/modulo}
			\end{exampletight}
			\vspace{-1mm}
			\begin{example}{}
				\wrong~~~~TC3: infinite loop for $b = 0$,\\ missing exception compared to JavaDoc
			\end{example}
		}
		\vspace{-1mm}
		\uncover<2->{
			\begin{example}{}
				\wrong~~~~TC6: $b$ remains negative as\\ \texttt{-Integer.MIN\_VALUE == Integer.MIN\_VALUE}\\ and the loop condition is fulfilled for any integer
			\end{example}
		}
		\vspace{-1mm}
		\uncover<3->{
			\begin{example}{}
				\wrong~~~~TC7: indicates that $m > b$ in the loop condition should be fixed to $m \geq b$
			\end{example}
		}
		\nextcolumn
		\uncover<4->{
			\begin{exampletight}{Improved Modulo Example}
				\picWhite[width=\linewidth]{testing/modulo-fixed}
			\end{exampletight}
		}
		\vspace{-1mm}
		\uncover<5->{
			\begin{example}{Passes All Test Cases}
				\vspace{3mm}
				\correct~~~~~\correct~~~~~\correct~~~~~\correct~~~~~\correct~~~~~\correct~~~~~\correct~~~~~\correct~~~~~\correct~~~~~\correct~~~~~ 
			\end{example}
		}
	\end{fancycolumns}
\end{frame}

\subsection{Reasons for Positive Test Cases}
\begin{frame}{\insertsubsection}
	\begin{fancycolumns}
		\begin{note}{Reasons for Positive Test Cases}
			\begin{itemize}
				\item actual fault
				\uncover<2->{\item wrong test case (input and expected results do not match)}
				\uncover<3->{\item interaction with other programs/libraries
					\item fault in the compiler
					\item fault in the operating system / device drivers
					\item fault in the hardware or hardware defect
					\item not enough memory
					\item does not halt (cf.\ undecidability of the halting problem)
					\item bitflip due to cosmic ray
					\item \ldots}
			\end{itemize}
		\end{note}
		%		\mynote{Reasons for Negative Test Cases}{
			%			\begin{itemize}
				%				\setlength\itemsep{.5em}
				%				\item program correct
				%				\item see above
				%			\end{itemize}
			%		}
	\end{fancycolumns}
\end{frame}

\begin{frame}
	\begin{fancycolumns}[height=8.5cm]
		\pic[width=\linewidth,trim=0 425 0 75,clip]{people/edsger-dijkstra}
		\vspace{-7mm}
		
		\begin{note}{Edsger W. Dijkstra \mysource{\goodliffe}}
			\mycite{If debugging is the process of removing software bugs, then programming must be the process of putting them in.}
		\end{note}
		% 1930-2002, ACM Turing Award winner
		\nextcolumn
		\pic[width=\linewidth,trim=250 0 0 0,clip]{people/brian-kernighan}
		\vspace{-7mm}
		
		\begin{note}{Brian Kernighan (1978) \mysource{\elementsofprogrammingstyle}}
			\mycite{Everyone knows that debugging is twice as hard as writing a program in the first place. So if you're as clever as you can be when you write it, how will you ever debug it?}
		\end{note}
		% known for books on C and Go
	\end{fancycolumns}
\end{frame}

\lessonslearned{
	\item Black-box testing: equivalence class testing and boundary testing
	\item Even systematic testing cannot ensure finding all faults
	\item Next: Never change a running system! How realistic is that?
}{
	\item \ludewiglichter, Chapter 19 \deutsch{Programmtest}
}{
	Any questions?
}
	%	\item See \href{https://moodle.uni-ulm.de/mod/moodleoverflow/discussion.php?d=3868}{Moodle} % TODO in 2023 update link
	%	\item Watch screencast on how to do black-box testing with JUnit
	%	\item Extend the black-box tests of the calculator for your new operation: \url{https://github.com/tthuem/2020WS-SWT-Calculator/tree/testinglecturev2}
	%	\item Post your changed code and tests in Moodle
	%	\item[] Live Coding Exercise for JUnit, see Moodle for instructions and solutions
	%\\\hfill\fancyqr{https://moodle.uni-ulm.de/course/view.php?id=42562\#section-7}
%	\item[] Answer the quiz in \panda{} to track your learning progress of this chapter
%	\\\hfill\fancyqr{https://panda.uni-paderborn.de/mod/quiz/view.php?id=2744505} 

%\faq{
%	\item
%}{
%	\item
%}{
%	\item
%}

% TODO L04 SOFTWARE CHANGES

\author{Thomas Thüm}
\lecture{Software Changes}{changes}

\section{}
%\input{content/}
\lessonslearned{
	\item Next: ...
}{
	\item ...
}{
	\begin{enumerate}
		\item<+-> Form groups of 2-3 students
	\end{enumerate}
}

\section{}
%\input{content/}
\lessonslearned{
	\item Next: ...
}{
	\item ...
}{
	\begin{enumerate}
		\item<+-> Form groups of 2-3 students
	\end{enumerate}
}

\section{}
%\input{content/}
\lessonslearned{
	\item Next: Why is it not sufficient to store the latest version of each software?
}{
	\item ...
}{
	\begin{enumerate}
		\item<+-> Form groups of 2-3 students
	\end{enumerate}
}

%\faq{
%	\item
%}{
%	\item
%}{
%	\item
%}

% TODO L05 VERSION CONTROL

\author{Thomas Thüm}
\lecture{Version Control}{versioncontrol}

\section{}
%\input{content/}
\lessonslearned{
	\item Next: ...
}{
	\item ...
}{
	\begin{enumerate}
		\item<+-> Form groups of 2-3 students
	\end{enumerate}
}

\section{}
%\input{content/}
\lessonslearned{
	\item Next: ...
}{
	\item ...
}{
	\begin{enumerate}
		\item<+-> Form groups of 2-3 students
	\end{enumerate}
}

\section{}
%\input{content/}
\lessonslearned{
	\item Next: ...
}{
	\item ...
}{
	\begin{enumerate}
		\item<+-> Form groups of 2-3 students
	\end{enumerate}
}

%\faq{
%	\item
%}{
%	\item
%}{
%	\item
%}

\input{template/footer}
