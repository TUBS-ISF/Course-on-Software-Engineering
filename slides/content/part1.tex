% TODO L01 INTRODUCTION

\ifuniversity{tubs}{\date{October 15, 2024}}

\author{Thomas Thüm}
\lecture{Introduction}{introduction}
% goal of this lecture?
% learn what software is and distinguish term from program, software product, app, ...
% understand the importance/role of software
% understand what software engineering is good for

\inputuniversity{content/01-prelude}

\section{The Impact of Software}
% TODO several ideas illustrating impact with consequences, but then hard to distinguish from next part

\subsection{Software (Products)}
\begin{frame}{\insertsubsection}
	\begin{fancycolumns}
		\begin{definition}{Software \mysource{adapted from \sommerville}}
			Software stands for one or several computer programs and all associated documentation, libraries, support websites, and configuration data that are needed to make these programs useful.
		\end{definition}
		\begin{example}{Explanation}
			The term program is used in a broader sense here. Software may also include source code, software models, or binaries.
		\end{example}
	\nextcolumn
		\begin{definition}{Software Product and Professional Software}
			A \emph{software product} is a software that can be sold to a customer. \emph{Professional software} is software intended for use by someone apart from its developer and it is usually developed by teams rather than individuals. \mysource{adapted from \sommerville}
		\end{definition}
		\pic[width=\linewidth]{misc/ms-office-cropped}
	\end{fancycolumns}
\end{frame}

\subsection{Characteristics of Software}
\begin{frame}{\insertsubsection}
	\centering\tikz[grow cyclic,
	mindmap, every node/.style=concept,concept color=red!10!background,
	%text width=20mm,align=flush center,
	level 1/.append style={level distance=27mm,sibling angle=360/8}]
	\node {Characteristics of Software}
	child { node {abstract\\(no physical\\laws)} }
	child { node {intangible \deutsch{immateriell}} }
	child { node {hard to measure} }
	child { node {aging} }
	child { node {no \phantom{abc}deterioration} }
	child { node {no spare parts} }
	child { node {easier to adapt than hardware} }
	child { node {frequent adaptation\phantom{;}} }
	;
\end{frame}
%   abstract and intagible (immatriell): not constrained by the properties of materials, nor are they governed by physical laws or by manufacturing processes. [\sommerville]
%   good as what we can built is a matter of our imagination (and computing power)
%   bad as software can get arbitrarily complex [\sommerville]

\subsection{Application and System Software}
\begin{frame}{\insertsubsection}
	\begin{fancycolumns}[T]
		\begin{definition}{Application Software or Application}
			Software that is designed for end users and applied for certain purposes. \deutsch{Anwendungssoftware oder Anwendung}
		\end{definition}

		\begin{example}{Examples}
			web browsers, media players, email or chat clients, text or photo editors, games
		\end{example}
	\nextcolumn
		\begin{definition}{System Software}
			Software that is not application software and typically being designed to provide a platform for other software.
		\end{definition}

		\begin{example}{Examples}
			operating systems, firmware, basic input/output system (BIOS), device drivers, game engines, GUI frameworks
		\end{example}
	\end{fancycolumns}

	\uncover<3->{
		\begin{note}{Classification Not Always Unique}
			e.g., web browsers and chat clients take over more and more features of operating systems
		\end{note}
	}
\end{frame}
% mention windows being in court because IE could not be uninstalled?

\subsection{Where Does Software Run?}
\begin{frame}{\insertsubsection}
	\begin{fancycolumns}[widths={43}]
		\begin{example}{World-Wide PC Sales}
			\picDark[width=\linewidth]{history/PCsales}
		\end{example}
		\nextcolumn
		\begin{example}{World-Wide Mobile Phone Subscriptions}
			\picDark[width=\linewidth]{history/mobile-phone-subscriptions}
		\end{example}
	\end{fancycolumns}	
\end{frame}
% hardware getting smaller and smaller, portable even for bees https://www.bbc.com/news/articles/cj9jzv27lv2o

\begin{frame}{Application Software}
	\begin{fancycolumns}[T,widths={46}]
		\begin{example}{Desktop Application or Desktop App}
			Windows 1.0 released in 1985
		\end{example}
		\pic[width=\linewidth]{history/windows1.0}
	\nextcolumn
		\begin{example}{Web Application or Web App}
			Ebay was born in 1995
		\end{example}
		\pic[height=32mm]{history/flohmarkt}
		\hfill\pause
		\pic[height=32mm]{history/iPhone1stGen}
		\begin{example}{Mobile Application or Mobile App or App}
			First iPhone released in 2007
		\end{example}
	\end{fancycolumns}
\end{frame}
% history of computing/software
% * easy delivery + distribution by the internet, internet of things
% * as it is so "easy" to change software, it is changed more frequently and is typically getting more complex with every change until it dies
%		\myexample{Progressive Web Application}{since 2016, not yet popular}
% "The "brain" [computer] may one day come down to our level [of the common people] and help with our income-tax and book-keeping calculations. But this is speculation and there is no sign of it so far." 1949 Quelle "Tutorial Guide to the EDSAC Simulator" (PDF). The EDSAC Replica Project.
%  ENIAC which became operational in 1946 could be run by a single, albeit highly trained, person.

\xkcdframe{2212}
% cash vs credit/debit card vs paypal/mobile phone: more and more software involved

\subsection{Downloads of Android Apps}
\begin{frame}{Android Apps with 5 Billion Downloads \deutschertitel{5 Milliarden}}
	\centering\picDark[height=65mm]{diagrams/android-downloads-5b}
\end{frame}

\begin{frame}{Non-Preinstalled Android Apps with 5 Billion Downloads \deutschertitel{5 Milliarden}}
	\centering\picDark[height=65mm]{diagrams/android-downloads-5b-np}
\end{frame}
% TODO add facebook,whatsapp,instagram outtage from October 2021?

%\begin{frame}{Android Apps with 1 Billion Downloads \deutschertitel{1 Milliarde}}
%	\centering\picDark[height=65mm]{diagrams/android-downloads-1b-p}
%\end{frame}

%\begin{frame}{Non-Preinstalled Android Apps with 1 Billion Downloads \deutschertitel{1 Milliarde}}
%	\centering\picDark[height=65mm]{diagrams/android-downloads-1b-np}
%\end{frame}

%\begin{frame}{Android Apps with 500 Million Downloads}
%	\centering\picDark[height=65mm]{diagrams/android-downloads-500m-np}
%\end{frame}

\subsection{Categories of Software}
\begin{frame}{\insertsubsection}
	\centering\tikz[grow cyclic,
	mindmap, every node/.style=concept,concept color=blue!10!background,
	%text width=20mm,align=flush center,
	level 1/.append style={level distance=27mm,sibling angle=360/8}]
	\node {Categories of Software}
	child { node {system vs\phantom{;} application software\phantom{ab}} }
	child { node {generic vs customized} }
	child { node {proprietary\\vs open-source} }
	child { node {freeware\\vs commercial} }
	child { node {data-intensive vs computation-intensive} }
	child { node {monolithic vs distributed} }
	child { node {stand-alone vs integrated} }
	child { node {user interface vs API\phantom{ab}} }
	;
\end{frame}
% TODO parts of this diagram not readible

% software is eating the world: https://a16z.com/2011/08/20/why-software-is-eating-the-world/
% data is the new oil

% German E rezept is facing technical challenges



\lessonslearned{
	\item What is software?
	\item What is the difference between program, software product, professional software, desktop/web/mobile app?
	\item What are characteristics and properties of software?
	\item What is the impact of software?
	\item Next: What can go possibly wrong with software?
}{
	\item \sommerville\mychapter{1.1}\mypages{19--28} % TODO check
}{
	\begin{enumerate}
		\item<+-> Form groups of 2-3 students
		\item<+-> Introduce yourselves to each other (5 min)
		\item<+-> Share your (recent) experiences with software (5 min)
		\item<+-> Survey: What is your experience with software?\\
		\slideonly{~
			\item<+->[A] Warst du jemals von einer Softwarepanne betroffen?
			\item<+->[B] Hast du dich im letzten Semester über Softwarefehler geärgert?
			\item<+->[C] Hast du selbst Programmiererfahrung?
			\item<+->[D] Hast du bereits Software getestet?
			\item<+->[E] Hast du schon mal Programmierfehler verursacht?
		}
			%\fancyqr{https://studip.tu-braunschweig.de/dispatch.php/questionnaire/answer/dc864612bd163af167a09edabdf058f3?cid=c37620ed3e63513af37fbe31060fcf69&range_type=course&range_id=c37620ed3e63513af37fbe31060fcf69}
	\end{enumerate}
}

\section{Software and Its Engineering}
% Why is Engineering Good Software so Hard?
% How to Develop Good Software?
% Why is Software Production so hard?
\subsection{The Project Cartoon}
\begin{frame}[b]{\insertsubsection}
	\vspace{-20mm}\small\renewcommand{\projectcartoonwidth}{.18}
	\begin{fancycolumns}[widths={30},animation=none]
		\uncover<11->{\begin{definition}{Why Do Software Projects Fail?}
			\begin{itemize}
				\item many stakeholders with different background each
				\item miscommunication
				\item implicit or wrong assumptions
				\item time pressure
				\item high complexity
				\item \ldots
				\item numerous reasons specific to certain phases and roles
			\end{itemize}
		\end{definition}}
		\uncover<12->{\begin{note}{}\centering{}software engineering aims to\\reduce such problems\end{note}}
	\nextcolumn
		\centering\hprojectcartoon{01}{how the customer explained it} % se02-10 requirements
		\uncover<2->{\hprojectcartoon{02}{how the project leader understood it}} % se02-07 modeling
		\uncover<3->{\hprojectcartoon{03}{how the analyst designed it}} % se04-10 architecture anddesign
		\uncover<4->{\hprojectcartoon{04}{how the programmer implemented it}} % se02-10 implementation
		\uncover<5->{\hprojectcartoon{05}{what the beta testers received}} % se08-09 testing
	
		\uncover<6->{\hprojectcartoon{06}{how the business consultant described it}} % se02-03 process
		\uncover<7->{\hprojectcartoon{09}{how the customer was billed}} % se10 management and pricing
		\uncover<8->{\hprojectcartoon{10}{how it was supported}} % se12 maintenance
		\uncover<9->{\hprojectcartoon{12}{when it was delivered}} % se13 continous integration/delivery
		\uncover<10->{\hprojectcartoon{13}{what the customer really needed}} % se02-05
		%\hspace{-7mm}
		%\projectcartoon{07}{how the project was documented} % code documentation
		%\hprojectcartoon{18}{how patches were applied} % se11 evolution
		%\hprojectcartoon{17}{how it performed under load} % se14 compilation/quality assurance/performance
		%\hprojectcartoon{11}{what marketing advertised} % se15 reuse/product lines
		%\hprojectcartoon{16}{how open source version} % se16 open source/licensing
		%\projectcartoon{08}{what operations installed} % devops?
		%\projectcartoon{14}{what the digg effect can do to your site} % micro services?
		%\projectcartoon{15}{the disaster recover plan}
		%\hspace{-7mm}
	\end{fancycolumns}
\end{frame}

\subsection{Software Engineering}
\begin{frame}{\insertsubsection}
	\begin{fancycolumns}
		\begin{definition}{Software Engineering \mysource{\sommerville}}
			\mycite{Software engineering is an engineering discipline that is concerned with all aspects of software production from initial conception to operation and maintenance. [...] Software engineering is not just concerned with the technical processes of software development. It also includes activities such as software project management and the development of tools, methods, and theories to support software development.}
		\end{definition}
	\end{fancycolumns}
\end{frame}

%\subsection{Software Engineering vs Programming}
\begin{frame}[b]{Software Engineering vs Programming}
	\slideSEvsProgramming
\end{frame}

\xkcdframe{2021}

%\subsection{}
\begin{frame}{Software Engineering vs Computer Science}
	\begin{fancycolumns}[animation=none]
		\begin{definition}{SE vs CS \mysource{\sommerville}}
			\mycite{Computer science focuses on theory and fundamentals; software engineering is concerned with the practicalities of developing and delivering useful software. [...] Computer science theory, however, is often most applicable to relatively small programs. Elegant theories of computer science are rarely relevant to large, complex problems that require a software solution.}
		\end{definition}
	\nextcolumn
	\end{fancycolumns}
\end{frame}
% add picture illustrating other related areas

\subsection{Software and System Engineering}
\begin{frame}{\insertsubsection}
	\begin{fancycolumns}[animation=none]
		\begin{definition}{System Engineering \mysource{\sommerville}}
			\mycite{System engineering is concerned with all aspects of computer-based systems development including hardware, software and process engineering. Software engineering is part of this more general process.}
		\end{definition}
	\nextcolumn
		% "System engineering is therefore concerned with hardware development, policy and process design, and system deployment, as well as software engineering." [\sommerville]
		\pic[width=\linewidth]{misc/lawn-mower-cropped} % copied to process lecture
	\end{fancycolumns}
\end{frame}

\subsection{(Software) Engineering}
\begin{frame}{\insertsubsection}
	\begin{fancycolumns}[animation=none]
		\begin{definition}{Engineering \mysource{\sommerville}}
			\mycite{Engineering is about getting results of the required \emph{quality} within \emph{schedule} and \emph{budget}. [...] Engineers make things work. They apply theories, methods, and tools where these are appropriate. However, they use them selectively and always try to discover solutions to problems even when there are no applicable theories and methods. Engineers also recognize that they must work within organizational and financial constraints, and they must look for solutions within these constraints.}
		\end{definition}
		\nextcolumn
			%\posthandout{\pic[width=\linewidth,trim={0\width} {.25\height} {0\width} {.1\height},clip]{blackboard/spannungsdreieck_240408}} % TODO check blackboard
	\end{fancycolumns}
\end{frame}

% TODO in lecture: Magisches Dreieck aufmalen: cost, time, quality

\subsection{Relevance of Software for This Course}
\begin{frame}{\insertsubsection}
	\begin{exampletight}{}
		\picDark[width=\linewidth]{failures/google-scholar2}
	\end{exampletight}
\end{frame}

\begin{frame}{\insertsubsection}
	\centering\picDark[height=\textheightwithtitle]{failures/texstudio}
\end{frame}

\begin{frame}{\insertsubsection}
	\begin{fancycolumns}[widths={54},animation=none]
		\pic[width=\linewidth]{failures/obs-freeze}
	\nextcolumn
		\pic[width=\linewidth]{failures/obs-freeze2}
	\end{fancycolumns}
\end{frame}

\begin{frame}{\insertsubsection}
	\begin{fancycolumns}
		\pic[width=\linewidth]{failures/acrobat-correct-green}
		\nextcolumn
		\pic[width=\linewidth]{failures/acrobat-wrong-green}
	\end{fancycolumns}
\end{frame}

\xkcdframe{1197}

\begin{frame}{\insertsubsection}
	\centering\pic[height=\textheightwithtitle]{failures/flash}
\end{frame}

\begin{frame}{\insertsubsection}
	\centering\pic[height=\textheightwithtitle]{failures/thunderbird}
\end{frame}


\lessonslearned{
	\item What is software engineering?
	\item Which trade-off is crucial to software engineering?
	\item Next: What is expected from you during this term?
}{
	\item \sommerville\mychapter{1.1}\mypages{19--28} % TODO check
}{
	\begin{enumerate}
		\item<+-> Form groups of 2-3 students
		\item<+-> Discuss: What will you do for a living in 10 years? What will be your connection to software? (5 min)
		\item<+-> Survey: What is your connection to software in 10 years?
		\slideonly{~
			\item<+->[A] Wirst du in 10 Jahren Software entwickeln?
			\item<+->[B] Wirst du in 10 Jahren Software testen?
			\item<+->[C] Wirst du in 10 Jahren Software beauftragen?
			\item<+->[D] Wirst du in 10 Jahren Softwareanforderungen ermitteln?
			\item<+->[E] Wirst du in 10 Jahren die Entwicklung von Software leiten?
		}
	\end{enumerate}
}

\section{Course Overview}
\subsection{What You Should Know}

\begin{frame}{\myframetitle{}}
	\begin{fancycolumns}
%		\begin{note}{Fundamentals of Software Engineering}
%			\begin{itemize}
%				\item development processes
%				\item object-oriented programming
%				\item design patterns
%				\item UML class diagrams
%				\item modularity
%			\end{itemize}
%			\ifuniversity{magdeburg}{$\Rightarrow$ \emph{Software Engineering}}
%		\end{note}
	\nextcolumn
%		\begin{note}{Fundamentals of Theoretical Computer Science}
%			\begin{itemize}
%				\item set theory
%				\item propositional logic
%				\item complexity theory
%			\end{itemize}
%			\ifuniversity{magdeburg}{
%				$\Rightarrow$ \emph{Logik}\\
%				$\Rightarrow$ \emph{Grundlagen der Theoretischen Informatik I}
%			}
%		\end{note}
%		\begin{note}{Exercise}
%			solid programming skills in Java
%
%			\ifuniversity{magdeburg}{
%				$\Rightarrow$ \emph{Einführung in die Informatik}\\
%				$\Rightarrow$ \emph{Algorithmen und Datenstrukturen}
%			}
%		\end{note}
	\end{fancycolumns}
\end{frame}

\subsection{What You Will Learn}

\begin{frame}{\myframetitle{}}
	\lectureseriesoverview[1]
\end{frame}

\subsection{What You Might Need}

\begin{frame}{\myframetitle{}}
%	\myframeicon{\mytitlesource{\fospl, \featureide}}
%	\begin{fancycolumns}
%		\begin{exampletight}{Recommended Literature for Lecture \& Exercise}
%			\centering
%			\parbox{0.49\linewidth}{
%				\centering
%				\pic[width=\linewidth]{cover-fospl}
%				\emph{theory-focused}
%			}
%			\parbox{0.475\linewidth}{
%				\centering
%				\pic[width=\linewidth]{cover-featureide}
%				\emph{practice-oriented}
%			}
%		\end{exampletight}
%	\nextcolumn
%		\begin{exampletight}{Recommended Tool Support for the Exercise}
%			\centering
%			\picDark[width=\linewidth]{featureide-feature-model-editor}\\[.5ex]
%			\pic[width=0.25\linewidth]{featureide-logo}
%		\end{exampletight}
%	\end{fancycolumns}
\end{frame}

%\subsection{Credit for the Slides}
%
%\begin{frame}{\myframetitle{}}
%	\ifuniversity{anonymous}{\mynote{}{\centering\huge Anonymous Authors}}
%	\unlessuniversity{anonymous}{
%		\myframeicon{\href{https://github.com/SoftVarE-Group/Course-on-Software-Product-Lines}{\pic[scale=.75]{cc-by-sa}}}
%	}
%	\begin{fancycolumns}[columns=3,animation=none]
%	\nextcolumn
%		\unlessuniversity{anonymous}{
%			\begin{note}{Thomas Thüm}
%				\centering
%				\href{https://www.uni-ulm.de/en/in/sp/team/thuem/}{\adjincludegraphics[height=.45\textheight,trim={.125\width} 0 {.125\width} 0,clip]{thomas-thuem}}
%
%				\small Professor at Paderborn University
%
%				software engineering
%
%				FeatureIDE team leader
%			\end{note}
%		}
%	\nextcolumn
%	\end{fancycolumns}
%\end{frame}

\inputuniversity{content/01c-course}
\lessonslearned{
	\item How is this course organized?
	\item Next Lecture: Why do programming languages matter?
}{
	\item Main Book: \sommervillelink{Ian Sommerville. Software Engineering, 10. Edition, Pearson, 2018.}
}{
	\begin{itemize}
		\item Any questions?
	\end{itemize}
}

%\faq{
%	\item
%}{
%	\item
%}{
%	\item
%}

% TODO L02 IMPLEMENTATION

\ifuniversity{tubs}{\date{October 22, 2024}}

\author{Thomas Thüm}
\lecture{Implementation}{implementation}

\begin{frame}{\insertsubtitle}
	\renewcommand{\projectcartoonwidth}{.19}
	\alt<2->{
		\projectcartoon{01}{how the customer explained it}
		\projectcartoon{02}{how the project leader understood it}
		\projectcartoon{03}{how the analyst designed it} % architecture/design
	}{%
		\hprojectcartoon{01}{how the customer explained it}
		\hprojectcartoon{02}{how the project leader understood it}
		\hprojectcartoon{03}{how the analyst designed it} % architecture/design
	}%
	\hprojectcartoon{04}{how the programmer implemented it}
	%\hprojectcartoon{05}{what the beta testers received} % testing
	%\hprojectcartoon{06}{how the business consultant described it}
	%\hprojectcartoon{07}{how the project was documented} % code documentation
	%\hprojectcartoon{08}{what operations installed} % devops/continuous integration
	%\hprojectcartoon{09}{how the customer was billed} % pricing
	%\hprojectcartoon{10}{how it was supported}
	%\hprojectcartoon{11}{what marketing advertised}
	%\hprojectcartoon{12}{when it was delivered} % continous delivery
	\uncover<-0|handout:0>{\projectcartoon{13}{what the customer really needed}}
	%\hprojectcartoon{14}{what the digg effect can do to your site} % ???
	%\hprojectcartoon{15}{the disaster recover plan}
	%\hprojectcartoon{16}{how open source version} % open source/licensing
	%\hprojectcartoon{17}{how it performed under load} % quality assurance/performance
	%\hprojectcartoon{18}{how patches were applied} % software maintenance
\end{frame}

\section{Choosing a Programming Language}
\subsection{History of Programming Languages}
\begin{frame}{\insertsubsection}
	\begin{fancycolumns}
		\begin{note}{Milestones\mysource{\jonesbestpractice}}
			\begin{itemize}
				\setlength\itemsep{.1em}
				\item controlling behavior of mechanical devices by wiring or with punchcards \deutsch{Lochkarten}
				\item machine languages used during World War II
				\uncover<2->{
					\item assembly languages: distinction between human-readable instructions (source code) and executable instructions (object code)
					\item birth of compilers and interpreters having a one-to-many mapping between source and object code (opposed to one-to-one mapping in assemblers)
				}
				\uncover<3->{
					\item structured programming pioneered by David Parnas and Edsger Dijkstra
					\item high-level programming languages: high number of executable instructions for each human-readable instruction
					\item domain-specific languages, later general-purpose programming languages
				}
			\end{itemize}
		\end{note}
	\nextcolumn
		\uncover<4->{
		\begin{example}{Languages\mysource{\jonesbestpractice\ + \handbuch}}
			\begin{itemize}
				\setlength\itemsep{.1em}
				\item 1945: first high-level language Plankalkül by Konrad Zuse (compiler written in 1998)
				\item 1954: first professional high-level language FORTRAN (Formula Translator) by IBM
				\item 1963: Basic as general-purpose language
				\uncover<5->{
					\item 1959: functional language Lisp
					\item 1970: first object-oriented lang.\ Smalltalk-80
					\item 1970: declarative language SQL
				}
				\uncover<6->{
					\item 1971: Pascal by Niklaus Wirth for teaching
					\item 1974: very common procedural language C
					\item 1977: logical language Prolog
				}
				\uncover<7->{
					\item 1980: C++ as object-oriented extension of C
					\item 1990: object-oriented language Java
					\item 1990: functional language Haskell
				}
				\uncover<8->{
					\item 1991: multi-paradigm language Python (today's default for machine learning)
					\item 1995: scripting language JavaScript (web apps)
				}
			\end{itemize}
		\end{example}
	}
	\end{fancycolumns}
\end{frame}
% TODO revise and talk about paradigms and their languages

\subsection{Programming Languages Today}
\begin{frame}{\insertsubsection}
	\slideProgrammingLanguagesToday
\end{frame}

\subsection{Choice of Programming Languages}
\begin{frame}{\insertsubsection}
	\begin{fancycolumns}
		\begin{definition}{{Desired Properties\mysource{\ludewiglichter}}}
			\begin{itemize}
				\item modular implementation
				\item separation of inferfaces and implementations
				\item type system: strongly/weakly typed languages
				\item readable syntax (FORTRAN vs ALGOL60) % designed for fewer characters vs readability
				\item automatic pointer management (C vs Java)
				\item exception handling
			\end{itemize}
		\end{definition}
	\nextcolumn
		\begin{example}{Criteria in Practice}
			\begin{itemize}
				\item language required by the company or customer?
				\item existing infrastructure?
				\item domain-specific languages available?
				\item language known/liked by developers?
				\item available libraries?
				\item available tool support?
				\item language popularity?
				\item what may change in the future?
			\end{itemize}
		\end{example}
	\end{fancycolumns}
\end{frame}

\subsection{Popularity of Programming Languages}
\begin{frame}{\insertsubsection}
	\slideTiobeDiagram
\end{frame}
\begin{frame}{\insertsubsection}
	\slideTiobeTable
\end{frame}

\begin{frame}
	\begin{fancycolumns}[height=8.5cm]
		\pic[width=\linewidth,trim=0 20 0 20,clip]{people/patrick-mckenzie}
		\vspace{-7mm}
		
		\mynote{Patrick McKenzie \mysource{\href{https://twitter.com/CodeWisdom/status/1182702520696803329}{twitter.com}}}{\mycite{Every great developer you know got there by solving problems they were unqualified to solve until they actually did it.}}
		% computer scientist, entrepreneur, influencer
	\nextcolumn
		\pic[width=\linewidth,trim=0 55 0 10,clip]{people/bill-gates}
		\vspace{-7mm}
		
		\mynote{Bill Gates \mysource{\href{https://code.org/quotes}{code.org}}}{\mycite{Learning to write programs stretches your mind, and helps you think better, creates a way of thinking about things that I think is helpful in all domains.}}
		% richest person in 15 years between 1994 and 2014
	\end{fancycolumns}
\end{frame}

\subsection{Computer-Aided Software Engineering}
\begin{frame}{\insertsubsection}
	\begin{fancycolumns}[widths={55}]
		\begin{definition}{Terms \mysource{adapted from \ghezzi}}
			A \emph{tool} is an application that supports a particular activity. An \emph{environment} is a collection of related tools. Tools and environments aim at automating some of the activities that are involved in software engineering. The generic term for this field of study is \emph{computer-aided software engineering}.
		\end{definition}
	\end{fancycolumns}
\end{frame}

\widexkcdframe{378} % real programmers

\subsection{Overview on Development Tools}
\begin{frame}{\insertsubsection}
	\begin{fancycolumns}[widths={57}]
		\begin{note}{Variety of Tools \mysource{\ghezzi}}
			\begin{itemize}
				\item text(ual) editors: emacs, vim, ed, Word, \ldots
				\item graphical editors: UML editors, Powerpoint, \ldots
				\item assembler, compiler, interpreter
				\uncover<2->{
					\item configuration management tools: git, SVN, CVS, \ldots
					\item tracking tools (issue trackers): Github, Gitlab, \ldots
					\item tools for code navigation and refactoring
					\item tools for test specification, generation, execution, reporting
					\item tools for static and dynamic code analysis (e.g., debugger), reverse/reengineering, project management
				}
				\uncover<3->{
					\item integrated development environments (IDEs): Eclipse, Visual Studio, IntelliJ, Android Studio
				}
			\end{itemize}
		\end{note}
	\end{fancycolumns}
\end{frame}

%\subsection{Demo on Tool Support in Eclipse}
%\begin{frame}{\insertsubsection\ \mytitlesource{\href{https://youtu.be/Jxt77kTbFZ0?si=oGjCcdfji7NMmbM7&t=3318}{youtube.de}}}
%	\centering\pic[width=.7\linewidth,trim=0 76 0 76,clip]{demo/livecoding2} % TODO update picture to recent version
%\end{frame}

\lessonslearned{
	\item Historical perspective on programming
	\item Criteria for choosing languages
	\item Popularity of programming languages
	\item Next: What has programming to do with reading?
}{
	\item \jonesbestpractice\mychapter{8 Programming and Code Development}
	\item \handbuch\mychapter{2.4 Programming Languages}
}{
	\begin{enumerate}
		\item Enter 1+2*3 into the calculator app of your mobile device or laptop
		\item Compare the results with your colleagues
		\item Discuss why results (could) differ
	\end{enumerate}
}

\begin{frame}{Calc on Windows 10\ \mytitlesource{\href{https://youtu.be/LKDbQfzzGJo?t=1544}{youtube.de}}}
	\begin{fancycolumns}
		\centering\picDark[width=.66\linewidth]{failures/win10-calc-scientific}
	\nextcolumn
		\centering\picDark[width=.66\linewidth]{failures/win10-calc-standard}
	\end{fancycolumns}
\end{frame}

\section{Coding Conventions}
\begin{frame}
	\begin{fancycolumns}[height=8.5cm]
		\pic[width=\linewidth,trim=40 0 100 0,clip]{people/douglas-crockford}
		\vspace{-7mm}
		
		\begin{note}{Douglas Crockford \mysource{\javascript}}
			\mycite{It turns out that style matters in programming for the same reason that it matters in writing. It makes for better reading.}
		\end{note}
		% known for JavaScript, JSON, works at Paypal
	\nextcolumn
		\pic[width=\linewidth,trim=0 20 0 25,clip]{people/francois-chollet}
		\vspace{-7mm}[height=8.5cm]
		
		\begin{note}{François Chollet \mysource{\href{https://twitter.com/fchollet/status/1038200379605798912}{twitter.com}}}
			\mycite{In software, naming matters, because names reflect how you think about a problem. Code is also communication, and naming is a big part of making it work.}
		\end{note}
		% AI researcher at Google
	\end{fancycolumns}
\end{frame}

\xkcdframe{2021}

\subsection{Code Formatting}
\begin{frame}{\insertsubsection}
	\begin{fancycolumns}
		\begin{note}{Motivation}
			\begin{itemize}
				\item code is read much more often and by more developers than written
				\item avoid differences by each programmer
			\end{itemize}
		\end{note} % TODO add citation
		\begin{definition}{Code Formatting}
			\begin{itemize}
				\item indentation: typically 4 characters per level
				\item length of a line: often 80 or 100 characters
				\item extra indentation: typically 8 characters when breaking extra long lines
				\item empty lines between members (e.g., methods and attributes)
			\end{itemize}
		\end{definition} % TODO add citation
	\nextcolumn
		\begin{example}{Code Formatting in Practice}
			\begin{itemize}
				\item automated code formatters available (on demand or when saving the editor)
				\item typical formatting rules for each language
				\item automated code formatters are configurable (handle with care)
			\end{itemize}
		\end{example} % TODO add citation
	\end{fancycolumns}
\end{frame}

\subsection{Rules on Naming}
\begin{frame}{\insertsubsection}
	\begin{fancycolumns}
		\begin{example}{Unwanted Names}
			\begin{itemize}
				\item single character as a name
				\item very long names
				\item names consisting only of special chars
				\item synonyms: delete, remove, clear
				\item abbreviations (unless very common)
			\end{itemize}
		\end{example} % TODO add citation
	\nextcolumn
		\begin{definition}{Wanted Names}
			\begin{itemize}
				\item nouns for class names: Calculator
				\item nouns for attribute names: calculateButton
				\item verbs for method names: getCalculator(), evaluate(), isZero(), hasChildren(), setValue()
				\item PascalCaseNotation for classes, interfaces (not consistent in C/C++)
				\item camelCaseNotation for attributes, methods, local variables, parameters (exception: Python)
				\item UPPER\_SNAKE\_CASE\_NOTATION for constants (exception: Go)
				\item lowercasenotation for package names (exception: C\#)
			\end{itemize}
		\end{definition} % TODO add citation
	\end{fancycolumns}
\end{frame}

\begin{frame}
	\begin{fancycolumns}[height=8.5cm]
		\pic[width=\linewidth,trim=0 275 0 25,clip]{people/martin-fowler}
		\vspace{-7mm}
		
		\begin{note}{Martin Fowler (1999) \mysource{\refactoring}}
			\mycite{Any fool can write code that a computer can understand. Good programmers write code that humans can understand.}
		\end{note}
		% known for refactoring and agile development
	\nextcolumn
		\pic[width=\linewidth,trim=0 0 0 10,clip]{people/cory-house}
		\vspace{-7mm}
		
		\begin{note}{Cory House \mysource{\href{https://twitter.com/housecor/status/400479246713229312}{twitter.com}}}
			\mycite{Code is like humor. When you have to explain it, it’s bad.}
		\end{note}
		% influencer known for teaching React and JavaScript
	\end{fancycolumns}
\end{frame}

\subsection{Code Documentation}
\begin{frame}{\insertsubsection}
	\begin{fancycolumns}
		\begin{definition}{Comments \ldots}
			\begin{itemize}
				\item in source code are easier to maintain (than in external documents)
				\item should be written while editing the code
				\item can be used to generate documentation (e.g., JavaDoc, Doxygen)
				\item are used to specify classes and public methods (e.g., parameters, exceptions, dependencies)
				\item document hacks, side effects and unfinished parts (e.g., TODO)
				\item should not paraphrase the code
			\end{itemize}
		\end{definition}
	\end{fancycolumns}
\end{frame}

\begin{frame}
	\begin{fancycolumns}[height=8.5cm]
		\pic[width=\linewidth,trim=0 0 0 60,clip]{people/ryan-campbell}
		\vspace{-7mm}
		
		\begin{note}{Ryan Campbell \mysource{\href{https://handbook.problemsolving.io/01-model/03-code.html}{problemsolving.io}}}
			\mycite{Commenting your code is like cleaning your bathroom - you never want to do it, but it really does create a more pleasant experience for you and your guests.}
		\end{note}
		% software developer
	\nextcolumn
		\pic[width=\linewidth,trim=90 15 10 20,clip]{people/steve-mcconnell}
		\vspace{-7mm}
		
		\begin{note}{Steve McConnell (2004) \mysource{\codecomplete}}
			\mycite{Good code is its own best documentation. As you're about to add a comment, ask yourself, \mycite{How can I improve the code so that this comment isn't needed?} Improve the code and then document it to make it even clearer.}
		\end{note}
		% author of several textbooks on software development and project management
	\end{fancycolumns}
\end{frame}




\lessonslearned{
	\item Coding conventions \deutsch{Programmierrichtlinien}
	\item Formatting, naming, comments, and documentation
	\item Next: What tools are available for programming?
}{
	\item \javastyleguide
	% TODO add literature? \refactoring \codecomplete \javascript
}{
	\begin{enumerate}
		\item<+-> Form groups of 2-3 students
		\item What happens if each programmer applies different coding conventions?
		\item Would it help to agree on coding conventions for each file individually?
	\end{enumerate}
}

\section{Tools and Environments}
\subsection{Computer-Aided Software Engineering}
\begin{frame}{\insertsubsection}
	\begin{fancycolumns}[widths={55}]
		\begin{definition}{Terms \mysource{adapted from \ghezzi}}
			A \emph{tool} is an application that supports a particular activity. An \emph{environment} is a collection of related tools. Tools and environments aim at automating some of the activities that are involved in software engineering. The generic term for this field of study is \emph{computer-aided software engineering}.
		\end{definition}
	\end{fancycolumns}
\end{frame}

\widexkcdframe{378} % real programmers

\subsection{Overview on Development Tools}
\begin{frame}{\insertsubsection}
	\begin{fancycolumns}[widths={57}]
		\begin{note}{Variety of Tools \mysource{\ghezzi}}
			\begin{itemize}
				\item text(ual) editors: emacs, vim, ed, Word, \ldots
				\item graphical editors: UML editors, Powerpoint, \ldots
				\item assembler, compiler, interpreter
				\uncover<2->{
				\item configuration management tools: git, SVN, CVS, \ldots
				\item tracking tools (issue trackers): Github, Gitlab, \ldots
				\item tools for code navigation and refactoring
				\item tools for test specification, generation, execution, reporting
				\item tools for static and dynamic code analysis (e.g., debugger), reverse/reengineering, project management
				}
				\uncover<3->{
				\item integrated development environments (IDEs): Eclipse, Visual Studio, IntelliJ, Android Studio
				}
			\end{itemize}
		\end{note}
	\end{fancycolumns}
\end{frame}

\subsection{Demo on Tool Support in Eclipse}
\begin{frame}{\insertsubsection\ \mytitlesource{\href{https://youtu.be/Jxt77kTbFZ0?si=oGjCcdfji7NMmbM7&t=3318}{youtube.de}}}
	\centering\pic[width=.7\linewidth,trim=0 76 0 76,clip]{demo/livecoding2} % TODO update picture to recent version
\end{frame}


\lessonslearned{
	\item Tool support for numerous activities
	\item By means of tools, environments, and IDEs
	\item Next: How to find errors beyond compiler errors?
}{
	\item \ghezzi\mychapter{9 Software Engineering Tools and Environments}
}{
	\begin{enumerate}
		\item<+-> Form groups of 2-3 students
		\item<+-> What tools have you used already?
		\item<+-> How does development profit from those tools?
	\end{enumerate}
}

%\faq{
%	\item
%}{
%	\item
%}{
%	\item
%}

% TODO L03 Testing

\author{Thomas Thüm}
\lecture{Testing}{testing}

\section{}
%\input{content/}
\lessonslearned{
	\item Next: ...
}{
	\item ...
}{
	\begin{enumerate}
		\item<+-> Form groups of 2-3 students
	\end{enumerate}
}

\section{}
%\input{content/}
\lessonslearned{
	\item Next: ...
}{
	\item ...
}{
	\begin{enumerate}
		\item<+-> Form groups of 2-3 students
	\end{enumerate}
}

\section{}
%\input{content/}
\lessonslearned{
	\item Next: ...
}{
	\item ...
}{
	\begin{enumerate}
		\item<+-> Form groups of 2-3 students
	\end{enumerate}
}

%\faq{
%	\item
%}{
%	\item
%}{
%	\item
%}

% TODO L04 SOFTWARE CHANGES

\author{Thomas Thüm}
\lecture{Software Changes}{changes}

\section{}
%\input{content/}
\lessonslearned{
	\item Next: ...
}{
	\item ...
}{
	\begin{enumerate}
		\item<+-> Form groups of 2-3 students
	\end{enumerate}
}

\section{}
%\input{content/}
\lessonslearned{
	\item Next: ...
}{
	\item ...
}{
	\begin{enumerate}
		\item<+-> Form groups of 2-3 students
	\end{enumerate}
}

\section{}
%\input{content/}
\lessonslearned{
	\item Next: Why is it not sufficient to store the latest version of each software?
}{
	\item ...
}{
	\begin{enumerate}
		\item<+-> Form groups of 2-3 students
	\end{enumerate}
}

%\faq{
%	\item
%}{
%	\item
%}{
%	\item
%}

% TODO L05 VERSION CONTROL

\author{Thomas Thüm}
\lecture{Version Control}{versioncontrol}

\section{}
%\input{content/}
\lessonslearned{
	\item Next: ...
}{
	\item ...
}{
	\begin{enumerate}
		\item<+-> Form groups of 2-3 students
	\end{enumerate}
}

\section{}
%\input{content/}
\lessonslearned{
	\item Next: ...
}{
	\item ...
}{
	\begin{enumerate}
		\item<+-> Form groups of 2-3 students
	\end{enumerate}
}

\section{}
%\input{content/}
\lessonslearned{
	\item Next: ...
}{
	\item ...
}{
	\begin{enumerate}
		\item<+-> Form groups of 2-3 students
	\end{enumerate}
}

%\faq{
%	\item
%}{
%	\item
%}{
%	\item
%}

\mode<beamer>{
	\addtocounter{framenumber}{-1}
	\begin{frame}{\inserttitle}
		\lectureseriesoverview[\insertlecturenumber]
	\end{frame}

	%\addtocounter{framenumber}{-1}
	%\againtitle % TODO does not work as we have redefined maketitle
}

