% TODO replace links each year
\newcommand{\StudIPLectureLink}{https://studip.tu-braunschweig.de/dispatch.php/course/details?sem_id=c37620ed3e63513af37fbe31060fcf69&again=yes}
\newcommand{\StudIPExerciseLink}{https://studip.tu-braunschweig.de/dispatch.php/course/details?sem_id=c9e139ced277e63816048cb443f843fe&again=yes}
\newcommand{\StudIPLecture}{\href{\StudIPLectureLink}{Stud.IP (Lecture)}}
\newcommand{\StudIPExercise}{\href{\StudIPExerciseLink}{Stud.IP (Exercise)}}

\subsection{About This Course} 
\begin{frame}{\insertsubsection}
	\begin{fancycolumns}[widths={55}]
		\begin{definition}{Software Engineering 1}
			\begin{itemize}
				\item Abbreviation: SE1
				\item Credits: 5 ECTS %(45h+75h+30h=150h)
				\item Semester hours: 2+1
				\item Courses of studies:
				\begin{itemize}
					\item B.\,Sc.: Informatik, Wirtschaftsinformatik, \ldots
					\item M.\,Sc.: \ldots
					%\item \ldots
				\end{itemize}
			\end{itemize}
		\end{definition}
		\begin{note}{Disclaimer}
			Passing this course is required to participate in the SEP \deutsch{Softwareentwicklungspraktikum} in summer term!
		\end{note}
		\nextcolumn
		\begin{definition}{Course Organization}
			\begin{itemize}
				\item 2 hours: typically one lecture per week, see schedule in \StudIPLecture
				\item 1 hour: one exercise (2 hours) every two weeks, see schedule in \StudIPExercise
				\item Written exam at the end of the term
			\end{itemize}
		\end{definition}
	\end{fancycolumns}
\end{frame}

\subsection{The Lectures}
\begin{frame}{\insertsubsection}
	\begin{fancycolumns}[widths={60}]
		\begin{definition}{The Lectures}
			\begin{itemize}
				\item weekly, Tuesday 11:30--13:00
%				\item except for holidays \deutsch{Pfingstmontag}
				\item lecture follows the Sandwich model
				\item three lecture parts
				\item interactive tasks in between (some require smartphone, tablet, notebook)
				\item lecture recordings planned (but without any guarantee)
			\end{itemize}
		\end{definition}
	\nextcolumn
		\begin{note}{Adam Osborne} % TODO add picture and source
			\mycite{The most valuable thing you can make is a mistake -- you can't learn anything from being perfect.}
		\end{note}
	\end{fancycolumns}
\end{frame}

\subsection{The Exercises}
\begin{frame}{\insertsubsection}
	\small
	\begin{fancycolumns}[widths={30}]
		\begin{definition}{The Exercises}
			\begin{itemize}
				\item goal: deeper understanding of lecture topics, preparation for the exam
				\item bi-weekly, concrete dates in \StudIPExercise
				\item starting in \emph{third} week (special exercise in \emph{second} week)
				\item at most \emph{15 participants} in each exercise
				\item assignment to time slots via \StudIPExercise
				\item switching between or attending multiple exercises is \emph{not allowed}
				\item holidays: alternative dates discussed in affected exercises
			\end{itemize}
		\end{definition}
		\nextcolumn
		\begin{definition}{Voting System \deutsch{Votierungssystem}}
			\begin{itemize}
				\item 7 sheets with 5-8 tasks each
				\item participants prepare solutions for the tasks \emph{before} the exercise
				\item in the 15 minutes before the exercise, participants can vote \deutsch{votieren} tasks they prepared in a dedicated list \deutsch{Votierungsliste}
				\item during the exercise one or multiple prepared participants are selected to present their solution at the whiteboard (or beamer)
				\item at least 50\,\% votes \deutsch{Votierungspunkte} and at least 3 presentations \deutsch{Vortragspunkte} required to pass the exercise \deutsch{Studienleistung}
				%\item at least 75\,\% votes required to get the grade improvement \deutsch{Notenbonus für eine Notenstufe}
				\item students may prepare solutions in groups but every member needs to be able to explain the solution if voted
			\end{itemize}
		\end{definition}\pause\pause
		\begin{note}{Exception Handling}
			\begin{itemize}
				\item tasks without prepared participants will not be discussed
				\item unprepared/cheating participants risk to loose one or all votes for the current sheet (incl. presentation points)
			\end{itemize}
		\end{note}
	\end{fancycolumns}
\end{frame}

\begin{frame}{\insertsubsection}
	\begin{fancycolumns}[animation=none]
		\begin{definition}{Special Exercises in Second Week}
			\begin{itemize}
				\item following rules only apply to the second week!
				\item special offer for students with limited or no programming skills
				\item good preparation for later lecture topics on programming and testing
				\item come unprepared (i.e., no voting)
				\item participation not necessary, but recommended
				\item time slot of both alternating weeks merged
			\end{itemize}
		\end{definition}
		\nextcolumn
		\begin{definition}{How to Find Answers}
			\begin{enumerate}
				\item<+-> Ask questions during the lecture (e.g., during interactive parts)
				\item<+-> Check information in StudIP
				\begin{itemize}
					\item Check already answered questions
					\item Ask your own questions and answer questions of fellow students
				\end{itemize}
				\item<+-> Ask questions in your exercise
				\item<+-> Meet Thomas in his consultation hour:\\
					Wednesday 2pm in IZ 348\\(preregistration useful)
				\item<+-> Slowest option: Contact us via  \href{mailto:sebastian.krieter@uni-paderborn.de?cc=t.thuem@tu-braunschweig.de&subject=[SE1]}{e-mail}
			\end{enumerate}
		\end{definition}
	\end{fancycolumns}
\end{frame}

\subsection{Literature for This Course}
\begin{frame}{\insertsubsection}
	\begin{fancycolumns}[animation=none]
		\centering\pic[height=50mm]{books/sommerville-softwarenegineering}
		\nextcolumn
		\begin{definition}{\mysource{\sommerville}}
			\begin{itemize}
				\item \sommervillelink{Ian Sommerville. Software Engineering, 10. Edition, Pearson, 2018.}
				\begin{itemize}
					\item German, English, and earlier versions
					\item \href{https://software-engineering-book.com/videos/}{Videos by Ian Sommerville and others available online}
				\end{itemize}
				\item More literature announced in each lecture
			\end{itemize}
		\end{definition}
	\end{fancycolumns}
\end{frame}
