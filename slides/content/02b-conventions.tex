\begin{frame}
	\begin{fancycolumns}[height=8.5cm]
		\pic[width=\linewidth,trim=40 0 100 0,clip]{people/douglas-crockford}
		\vspace{-7mm}
		
		\begin{note}{Douglas Crockford \mysource{\javascript}}
			\mycite{It turns out that style matters in programming for the same reason that it matters in writing. It makes for better reading.}
		\end{note}
		% known for JavaScript, JSON, works at Paypal
	\nextcolumn
		\pic[width=\linewidth,trim=0 20 0 25,clip]{people/francois-chollet}
		\vspace{-7mm}[height=8.5cm]
		
		\begin{note}{François Chollet \mysource{\href{https://twitter.com/fchollet/status/1038200379605798912}{twitter.com}}}
			\mycite{In software, naming matters, because names reflect how you think about a problem. Code is also communication, and naming is a big part of making it work.}
		\end{note}
		% AI researcher at Google
	\end{fancycolumns}
\end{frame}

\subsection{Code Formatting}
\begin{frame}{\insertsubsection}
	\begin{fancycolumns}
		\begin{note}{Motivation}
			\begin{itemize}
				\item code is read much more often and by more developers than written
				\item avoid differences by each programmer
			\end{itemize}
		\end{note} % TODO add citation
		\begin{definition}{Code Formatting}
			\begin{itemize}
				\item indentation: typically 4 characters per level
				\item length of a line: often 80 or 100 characters
				\item extra indentation: typically 8 characters when breaking extra long lines
				\item empty lines between methods and attributes
			\end{itemize}
		\end{definition} % TODO add citation
	\nextcolumn
		\begin{example}{Code Formatting in Practice}
			\begin{itemize}
				\item automated code formatters available (on demand or when saving the editor)
				\item typical formatting rules for each language
				\item automated code formatters are configurable (handle with care)
			\end{itemize}
		\end{example} % TODO add citation
	\end{fancycolumns}
\end{frame}

\subsection{Rules on Naming}
\begin{frame}{\insertsubsection}
	\begin{fancycolumns}
		\begin{example}{Unwanted Names}
			\begin{itemize}
				\item single character as a name
				\item very long names
				\item names consisting only of special chars
				\item synonyms: delete, remove, clear
				\item abbreviations (unless very common)
			\end{itemize}
		\end{example} % TODO add citation
	\nextcolumn
		\begin{definition}{Wanted Names}
			\begin{itemize}
				\item nouns for class names: Calculator
				\item nouns for attribute names: calculateButton
				\item verbs for method names: getCalculator(), evaluate(), isZero(), hasChildren(), setValue() % TODO consistent use of "method" opposed to "operation"?
				\item CamelCaseNotation for classes, attributes, methods, local variables, parameters
				\item UPPER\_CASE\_NOTATION for constants
				\item lowercasenotation for package names
			\end{itemize}
		\end{definition} % TODO add citation
	\end{fancycolumns}
\end{frame}

\begin{frame}
	\begin{fancycolumns}[height=8.5cm]
		\pic[width=\linewidth,trim=0 275 0 25,clip]{people/martin-fowler}
		\vspace{-7mm}
		
		\begin{note}{Martin Fowler (1999) \mysource{\refactoring}}
			\mycite{Any fool can write code that a computer can understand. Good programmers write code that humans can understand.}
		\end{note}
		% known for refactoring and agile development
	\nextcolumn
		\pic[width=\linewidth,trim=0 0 0 10,clip]{people/cory-house}
		\vspace{-7mm}
		
		\begin{note}{Cory House \mysource{\href{https://twitter.com/housecor/status/400479246713229312}{twitter.com}}}
			\mycite{Code is like humor. When you have to explain it, it’s bad.}
		\end{note}
		% influencer known for teaching React and JavaScript
	\end{fancycolumns}
\end{frame}

\subsection{Code Documentation}
\begin{frame}{\insertsubsection}
	\begin{fancycolumns}
		\begin{definition}{Comments \ldots}
			\begin{itemize}
				\item in source code are easier to maintain (than in external documents)
				\item should be written while editing the code
				\item can be used to generate documentation (e.g., JavaDoc, Doxygen)
				\item are used to specify classes and public methods (e.g., parameters, exceptions, dependencies)
				\item document hacks, side effects and unfinished parts (e.g., TODO)
				\item should not paraphrase the code
			\end{itemize}
		\end{definition}
	\end{fancycolumns}
\end{frame}

\begin{frame}
	\begin{fancycolumns}[height=8.5cm]
		\pic[width=\linewidth,trim=0 0 0 60,clip]{people/ryan-campbell}
		\vspace{-7mm}
		
		\begin{note}{Ryan Campbell \mysource{\href{https://handbook.problemsolving.io/01-model/03-code.html}{problemsolving.io}}}
			\mycite{Commenting your code is like cleaning your bathroom - you never want to do it, but it really does create a more pleasant experience for you and your guests.}
		\end{note}
		% software developer
	\nextcolumn
		\pic[width=\linewidth,trim=90 15 10 20,clip]{people/steve-mcconnell}
		\vspace{-7mm}
		
		\begin{note}{Steve McConnell (2004) \mysource{\codecomplete}}
			\mycite{Good code is its own best documentation. As you're about to add a comment, ask yourself, \mycite{How can I improve the code so that this comment isn't needed?} Improve the code and then document it to make it even clearer.}
		\end{note}
		% author of several textbooks on software development and project management
	\end{fancycolumns}
\end{frame}



