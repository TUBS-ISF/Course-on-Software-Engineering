\subsection{Motivating Example}
\begin{frame}{\insertsubsection}
	\begin{fancycolumns}[widths={66}]
		\picDark[width=\linewidth,trim=200 100 200 200,clip]{versioncontrol/dieselgate}
	\nextcolumn
		\begin{example}{Which products are affected?}
			\begin{itemize}
				\item by manufacturing error,\\ security issues,\\ safety issues,\\ illegal functionality (cf. Dieselgate),\\ \ldots{}
				\item version needs to be available without access to the product
			\end{itemize}
		\end{example}
	\end{fancycolumns}
\end{frame}

\subsection{Configuration Management}
\begin{frame}{\insertsubsection\ \mytitlesource{\sommerville}}
	\begin{fancycolumns}
		\begin{definition}{Configuration Management}
			\mycite{\emph{Configuration management} is concerned with the policies, processes, and tools for managing changing software systems.} % TODO fix source Aiello and Sachs 2011
		\end{definition}
		\begin{definition}{Four Activities in Configuration Management}
			\begin{itemize}
				\item version control / management: keeping track of multiple versions, enabling simultaneous changes
				\item system building: collecting, compiling, and linking components into executable systems
				\item change management: tracking change requests and planning if/when realized
				\item release management: preparing new and managing old releases
			\end{itemize}
		\end{definition}
		\nextcolumn
		\begin{definition}{Development Stages \deutsch{Entwicklungsphasen}}
			\begin{itemize}
				\item \emph{development phase}: adding new functionality
				\item \emph{system testing phase}: internal release, bug fixes, performance improvements, security fixes, but no new functionality
				\item \emph{release phase}: bug reports and feature requests by users or customers
				\item often several versions co-exist at different stages
			\end{itemize}
		\end{definition}
	\end{fancycolumns}
\end{frame}

\subsection{Version Control}
\begin{frame}{\insertsubsection}
	\begin{fancycolumns}
		\begin{note}{Goals of Version Control}
			\begin{itemize}
				\item collaborative work: synchronize files and folders with other users
				\item compare own version with other versions
				\item merge files edited by several users
				\item history: access old versions, log changes
			\end{itemize}
		\end{note}
		\begin{definition}{Version Control Systems}
			\begin{itemize}
				\item local only: SCCS (1972), RCS (1982), \ldots
				\item centralized: CVS (1986), SVN (2000), \ldots
				\item distributed: \emph{git} (2005), mercurial (2005), \ldots
			\end{itemize}
			\vspace{2mm}
			invented for software, useful for most files %(cf.\ agile methods)
		\end{definition}
		\begin{example}{}
			Alternatives? Why not use locks (cf.\ databases)?
		\end{example}
		\nextcolumn
		\picDark[width=\linewidth]{versioncontrol/trends-git}
		\vspace{-4mm}
		\begin{example}{Personal Experience}
			\begin{itemize}
				\item 2004: started working with CVS in industry
				\item 2007: initiated research protoype FeatureIDE with SVN
				\item 2009: published FeatureIDE open source (history neglected)
				\item 2014: migrated FeatureIDE's code to git
			\end{itemize}
		\end{example}
	\end{fancycolumns}
\end{frame}

\subsection{Git vs Mercurial}
\begin{frame}{\insertsubsection}
	\begin{fancycolumns}[animation=none]
		\picDark[width=\linewidth]{versioncontrol/bitbucket-mercurial1}
		\nextcolumn
		\picDark[width=\linewidth]{versioncontrol/bitbucket-mercurial2}
	\end{fancycolumns}
\end{frame}

\xkcdframe{1597} % git

\subsection{Centralized Version Control}
\begin{frame}{\insertsubsection}
	\vspace{-5mm}
	\begin{fancycolumns}
		\clientserver{computer}{computer}{computer}{Server}{Client X}{Client Y}{
			\draw[kante] (client1) to (server);
			\draw[kante] (server) to (client1);
			\draw[kante] (client2) to (server);
			\draw[kante] (server) to (client2);
		}
		\begin{definition}{Centralized Version Control Systems}
			\begin{itemize}
				\item client-server architecture
				\item server manages the main copy
				\item clients use server to synchronize files/folders
			\end{itemize}
		\end{definition}
		\nextcolumn
		\only<-2|handout:0>{\clientserver{computer}{computer}{computer}{Repository: Revision 1-10}{}{}{
		}}
		\only<3|handout:0>{\clientserver{computer}{computer}{computer}{Repository: Revision 1-10}{Working Copy: Revision 10}{}{
				\draw[kante] (server) to (client1);
		}}
		\only<4|handout:0>{\clientserver{computer}{computer}{computer}{Repository: Revision 1-10}{Working Copy: Revision 10}{Working Copy: Revision 5}{
				\draw[kante] (server) to (client2);
		}}
		\only<5|handout:0>{\clientserver{computer}{computer}{computer}{Repository: Revision 1-10}{Working Copy: 10 (changed)}{Working Copy: Revision 5}{
		}}
		\only<6>{\clientserver{computer}{computer}{computer}{Repository: Revision 1-11}{Working Copy: Revision 11}{Working Copy: Revision 5}{
				\draw[kante] (client1) to (server);
		}}
		\only<7|handout:0>{\clientserver{computer}{computer}{computer}{Repository: Revision 1-11}{Working Copy: Revision 11}{Working Copy: Revision 11}{
				\draw[kante] (server) to (client2);
		}}
		\begin{note}{Centralized Version Control Systems}
			\begin{itemize}
				\item repository on server
				\item working copy on each client
				\item new revision for every change on the server
				\item revisions used to undo changes, merge files
			\end{itemize}
		\end{note}
	\end{fancycolumns}
\end{frame}

\subsection{Distributed Version Control}
\begin{frame}{\insertsubsection}
	\vspace{-2mm}
	\begin{fancycolumns}
		\peertopeer{computer}{tonne}{computer}{computer}{GIT Server}{GIT Server/Client X}{GIT Server/Client Y}{
			\draw[kante] (client1) to (server);
			\draw[kante] (server) to (client1);
			\draw[kante] (client2) to (server);
			\draw[kante] (server) to (client2);
			\draw[innerekante] (client1.80) to (client1.100);
			\draw[innerekante] (client1.-100) to (client1.-80);
			\draw[innerekante] (client2.80) to (client2.100);
			\draw[innerekante] (client2.-100) to (client2.-80);
		}
		\begin{definition}{Distributed Version Control Systems}
			\begin{itemize}
				\item peer-to-peer architecture
				\item clients use repository clone to synchronize
				\item version history available on each client
				%\item comparing client version with server version / history of repository clone
				%\item merging files edited by several clients using an updated version of the repository clone
			\end{itemize}
		\end{definition}
		\nextcolumn
		\only<2|handout:0>{\peertopeer{computer}{tonne}{computer}{computer}{Remote Repository: Revision 1-10}{~}{~}{
		}}
		\only<3|handout:0>{\peertopeer{computer}{tonne}{computer}{computer}{Remote Repository: Revision 1-10}{Repository Clone: 1-10}{Repository Clone: 1-10}{
				\draw[kante] (server) to (client1);
				\draw[kante] (server) to (client2);
		}}
		\only<4|handout:0>{\peertopeer{computer}{tonne}{computer}{computer}{Remote Repository: Revision 1-10}{Working Copy: 10\\Repository Clone: 1-10}{Working Copy: 5\\Repository Clone: 1-10}{
				\draw[innerekante] (client1.80) to (client1.100);
				\draw[innerekante] (client2.80) to (client2.100);
		}}
		\only<5|handout:0>{\peertopeer{computer}{tonne}{computer}{computer}{Remote Repository: Revision 1-10}{Working Copy: 10 (changed)\\Repository Clone: 1-10}{Working Copy: 5\\Repository Clone: 1-10}{
		}}
		\only<6|handout:0>{\peertopeer{computer}{tonne}{computer}{computer}{Remote Repository: Revision 1-10}{Working Copy: 11\\Repository Clone: 1-11}{Working Copy: 5\\Repository Clone: 1-10}{
				\draw[innerekante] (client1.-100) to (client1.-80);
		}}
		\only<7|handout:1>{\peertopeer{computer}{tonne}{computer}{computer}{Remote Repository: Revision 1-11}{Working Copy: 11\\Repository Clone: 1-11}{Working Copy: 5\\Repository Clone: 1-10}{
				\draw[kante] (client1) to (server);
		}}
		\only<8|handout:0>{\peertopeer{computer}{tonne}{computer}{computer}{Remote Repository: Revision 1-11}{Working Copy: 11\\Repository Clone: 1-11}{Working Copy: 5\\Repository Clone: 1-11}{
				\draw[kante] (server) to (client2);
		}}
		\only<9|handout:0>{\peertopeer{computer}{tonne}{computer}{computer}{Remote Repository: Revision 1-11}{Working Copy: 11\\Repository Clone: 1-11}{Working Copy: 11\\Repository Clone: 1-11}{
				\draw[innerekante] (client2.80) to (client2.100);
		}}
		\begin{note}{Distributed Version Control Systems}
			\begin{itemize}
				\item main repository on one or several servers
				\item clone of the repository on each client 
				\item new revision on every change on the clients
				%\item Revisions used to undo changes and to merge files
			\end{itemize}
		\end{note}
	\end{fancycolumns}
\end{frame}
