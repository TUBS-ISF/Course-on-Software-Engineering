\subsection{System Building}
\begin{frame}{\insertsubsection\ \mytitlesource{\sommerville}}
	\begin{fancycolumns}
		\begin{definition}{System Building}
			\mycite{\emph{System building} is the process of creating a complete, executable system by compiling and linking the system components, external libraries, configuration files, and other information.}
		\end{definition}
		\begin{definition}{Building Involves Three Platforms}
			\begin{itemize}
				\item \emph{development system}: compilers and editors used on the developer's system to test prior to commit
				\item \emph{build server}: server to build and distribute executable versions, triggered by commits or schedule (i.e., nightly builds)
				\item \emph{target environment}: intended platform for executable system (e.g., ECU in a car)
			\end{itemize}
		\end{definition}
		\nextcolumn
		\begin{definition}{Tooling for Building and System Integration}
			\begin{itemize}
				\item build script generation: identify dependent components, automated generation or tool support for creation and editing
				\item version control system integration: checkout required versions of components
				\item minimal recompilation: determine which parts need to be recompiled
				\item executable system creation: compilation and linking
				\item test automation: run automated tests (e.g., unit tests)
				\item reporting: reports about success or failure of builds and tests
				\item documentation generation: release notes, help pages
			\end{itemize}
		\end{definition}
	\end{fancycolumns}
\end{frame}

\subsection{Continuous Integration}
\begin{frame}{\insertsubsection\ \mytitlesource{\sommerville}}
	\begin{fancycolumns}
		\begin{note}{Continuous Integration}
			\mycite{Agile methods recommend that very frequent system builds should be carried out, with automated testing used to discover software problems. Frequent builds are part of a process of continuous integration [...].}
		\end{note}
		\begin{definition}{Continuous Integration Tools}
			\begin{itemize}
				\item \href{https://en.wikipedia.org/wiki/CruiseControl}{CruiseControl} (2001--2010) --- first open-source tool
				\item \href{https://en.wikipedia.org/wiki/TeamCity}{TeamCity} (2006--)
				\item \href{https://en.wikipedia.org/wiki/Hudson_(software)}{Hudson} (2008--2017)
				\item \href{https://en.wikipedia.org/wiki/Jenkins_(software)}{Jenkins} (2011--) --- fork of Hudson
				\item \href{https://en.wikipedia.org/wiki/Travis_CI}{Travis CI} (2011--) --- made in Germany
				\item \href{https://en.wikipedia.org/wiki/GitLab}{GitLab} (2014--)
				% TODO add more tools? go continuous delivery, codehsip, circleci, codefresh, bamboo
			\end{itemize}
		\end{definition}
		\nextcolumn
		\begin{definition}{Steps in Continuous Integration}
			\begin{itemize}
				\item clone/fetch from version control
				\item if feasible: build and run automated tests, if it fails others are responsible
				\item apply changes
				\item build and run automated tests locally, if it fails continue editing
				\item if local tests pass, commit to feature branch in version control
				\item commit triggers build server, if it fails continue editing
				\item if tests pass (and code review approves changes), merge branch into main development branch
			\end{itemize}
		\end{definition}
	\end{fancycolumns}
\end{frame}

\xkcdframe{2224} % versioning

\subsection{DevOps}
\begin{frame}{\insertsubsection\ \mytitlesource{\handbuch}}
	\begin{fancycolumns}
		\begin{note}{Motivation}
			\begin{itemize}
				\item if software fails: 
				\begin{itemize}
					\item programmers blame administrators for misconfiguration
					\item administrators blame programmers for erroneous software
				\end{itemize}
				\item programmers want frequent updates
				\item administrators follow the slogan: \mycite{never change a running system}
				\item customers and users want a single responsibility
				\item shorter and shorter update cycles
			\end{itemize}
		\end{note}
		\nextcolumn
		\begin{definition}{DevOps}
			\begin{itemize}
				\item promoted in agile development
				\item \emph{Dev}: development by programmers
				\item \emph{Ops}: operation \deutsch{Betrieb} by administrators
				\item \emph{DevOps}: teams that are responsible for both, development and operations
				\item goal: avoid blaming each other by shared responsibility
			\end{itemize}
		\end{definition}
	\end{fancycolumns}
\end{frame}

\begin{frame}{\insertsubsection}
	\centering\picDark[width=.75\linewidth]{versioncontrol/devops}
\end{frame}

