% introduced: introduction
% reused: ?
\newcommand{\slideSEvsProgramming}{
	\begin{fancycolumns}[animation=none]
		\centering
		\pic[height=60mm]{misc/ulm-muenster}
	\nextcolumn
		\centering\pic[height=60mm]{misc/tarp-tent-cropped}
	\end{fancycolumns}
	\begin{note}{}
		\centering what is needed besides programming will be motivated and shown throughout this course
	\end{note}
}

% introduced: implementation
% reused: ?
\newcommand{\slideProgrammingLanguagesToday}{
	\begin{fancycolumns}
		\begin{example}{Today\mysource{\jonesbestpractice\ + \handbuch}}
			\begin{itemize}
				\item 2002: C\# by Microsoft
				\item 2009: Go by Google
				\item 2010: Rust by Mozilla Research
				\item 2014: Swift by Apple
				\item thousands of programming languages
				\item very few programming languages used for more than 10 years
				\item languages used for more than 25 years: Ada, C, C++, COBOL, Java, Objective C, PL/I, SQL, Visual Basic, \ldots
			\end{itemize}
		\end{example}
	\nextcolumn
		\begin{note}{Many Languages\mysource{\jonesbestpractice}}
			\begin{itemize}
				\item good: fit for every use case
				\item bad: developer training for new and dead languages, expensive tool support
			\end{itemize}
		\end{note}
	\end{fancycolumns}
}

% introduced: testing
% reused: ?
\newcommand{\slideMindmapQualityAssurance}[7]{
	\vspace{-7mm}\hfill
	\begin{tikzpicture}
		\path[small mindmap,
		every node/.style={concept,font=\scriptsize},
		emph/.style={font=\bfseries\scriptsize},
		concept color=foreground!20!background,
		level 1/.append style={level distance=25mm,sibling angle=360/6},
		level 2/.append style={level distance=20mm,sibling angle=360/6},
		level 3/.append style={level distance=20mm,sibling angle=360/8},
		]
		node {Quality Assurance \deutsch{Qualitätssicherung}}
		[counterclockwise from=210]
		child[#1] { node {constructive} 
			[clockwise from=225]
			child[concept color=blue!20!background,#4] { node {Coding Guidelines} }
		}
		child[#2] { node {analytical} 
			[counterclockwise from=240]
			child[concept color=green!20!background,#5] { node {analysis}
				[counterclockwise from=180]
				child { node {Compilation} }
				child { node {Code Reviews} }
			}
			child[concept color=red!20!background,#6] { node {execution}
				[counterclockwise from=315]
				child { node {White-Box Testing} }
				child { node {Black-Box Testing} }
			}
		}
		child[#3] { node {organizational} 
			[clockwise from=-45]
			child[concept color=orange!20!background,#7] { node {Software Project Management} }
		}
		;
	\end{tikzpicture}
}

% introduced: changes
% reused: ?
\newcommand{\slideEvolutionAndMaintenance}{
	\begin{fancycolumns}[t]
		\begin{note}{Evolution}
			\begin{itemize}
				\item new or removed functionality
				\item typically larger changes
				\item often foreseen changes
				\item results in upgrades, service packs, or cumulative updates
			\end{itemize}
		\end{note}
		\begin{example}{Minor Release}
			new minor version: 2.3.1 $\Rightarrow$ 2.4.0
		\end{example}
		\begin{example}{Major Release}
			new major version: 2.3.1 $\Rightarrow$ 3.0.0
		\end{example}
		\nextcolumn
		\begin{note}{Maintenance}
			\begin{itemize}
				\item mostly corrections
				\item typically smaller changes
				\item often unforeseen changes
				\item results in patches and hot fixes
			\end{itemize}
		\end{note}
		\begin{example}{Patch Release}
			new patch version: 2.3.1 $\Rightarrow$ 2.3.2
		\end{example}
	\end{fancycolumns}
}

% introduced: versioncontrol
% reused: ?
\newcommand{\slideBranchingAndMerging}{
	\begin{fancycolumns}[animation=none,b,widths={58}]
		\only<1|handout:0>{\trunkbranch{}{}}%
		\only<2|handout:0>{\trunkbranch{,draw=green}{}}%
		\only<3-|handout:1>{\trunkbranch{,draw=green}{,draw=orange}}%
		\only<1->{
			\begin{note}{}
				\begin{itemize}
					\item simultaneous, independent development
					\item option to merge in the future
					\item main development on branch \texttt{main} (formerly \texttt{master})
					\item parallel developments on branches for new features, bug fixes, multiple versions, \ldots
					\item avoiding name \texttt{master} for branches:\\\texttt{git config --global init.defaultBranch main}
				\end{itemize}
			\end{note}
		}
		\nextcolumn
		\uncover<4-|handout:1>{\picDark[width=\linewidth]{versioncontrol/master-main-github}}%
	\end{fancycolumns}
}

% introduced: management
% reused: process
\newcommand{\definitionuserstory}[1]{
	\begin{definition}{#1User Story \mysource{\sommerville}}
		\begin{itemize}
			\item a scenario of use that might be experienced by a system user
			\item aka.\ \emph{story card} as user stories are sometimes written on physical cards
			\item user stories are typically prioritized by the customer
			\item subset of all user stories is chosen for the next release
		\end{itemize}
	\end{definition}
}

% introduced: process
% reused: 
\newcommand{\slideWaterfallModel}{
	\begin{fancycolumns}
		\begin{definition}{Waterfall Model}
			\begin{itemize}
				\item first process model, motivated by practice
				\item by \href{https://scholar.google.de/scholar?cluster=8624196209257442707}{Winston W. Royce 1970}
				\item each development phase ends by the approval of one or more documents (document-driven process model)
				\item phases do not overlap
				\item numerous variants with varying number of phases: 5--7
				\item here: simplified variant by Sommerville
			\end{itemize}
		\end{definition}
		\nextcolumn
		\diagramWaterfallModel
	\end{fancycolumns}
}
\newcommand{\diagramWaterfallModel}{
	{\color{black}\begin{tikzpicture}[yscale=-1.2,xscale=.75,phase/.style={draw,thick,rounded rectangle,fill=blue!10,align=center,text width=21mm},label/.style={auto,bend right,align=left},to/.style={->,>={Stealth[round]},thick}]
			\node[phase,visible on=<6->] (1) at (0,0) {Requirements Analysis};
			\node[phase,visible on=<7->] (2) at (1,1) {System and Software Design};
			\node[phase,visible on=<2->] (3) at (2,2) {Implementation and Unit Testing};
			\node[phase,visible on=<4->] (4) at (3,3) {Integration and System Testing};
			\node[phase,visible on=<3->] (5) at (4,4) {Operation and Maintenance};
			\draw[to,visible on=<8->,blue] (1) -| node[label] {System\\Specification} (2.30);
			\draw[to,visible on=<8->,blue] (2) -| node[label] {Design\\Specification} (3.30);
			\draw[to,visible on=<5->,blue] (3) -| node[label] {Program\\Documentation} (4.30);
			\draw[to,visible on=<5->,blue] (4) -| node[label] {Test\\Documentation} (5.30);
			\draw[to,visible on=<9|handout:0>,red] (5) -| (1.210);
			\draw[to,visible on=<9|handout:0>,red] (5) -| (2.210);
			\draw[to,visible on=<9|handout:0>,red] (5) -| (3.210);
			\draw[to,visible on=<9|handout:0>,red] (5) -| (4.210);
			\draw[to,visible on=<10->,orange] (2) -| (1.210);
			\draw[to,visible on=<10->,orange] (3) -| (2.210);
			\draw[to,visible on=<10->,orange] (4) -| (3.210);
			\draw[to,visible on=<10->,orange] (5) -| (4.210);
	\end{tikzpicture}}
}

% introduced: process
% reused: ?
\newcommand{\diagramVModel}{
	{\color{black}\begin{tikzpicture}[yscale=-1.45,xscale=.45,phase/.style={draw,thick,rounded rectangle,fill=blue!10,align=center,text width=17mm},label/.style={midway,anchor=west},to/.style={->,>={Stealth[round]},thick}]
			\node[phase,visible on=<2->] (1) at (0,0) {Requirements Elicitation};
			\node[phase,visible on=<3->] (2) at (1,1) {System Modeling};
			\node[phase,visible on=<4->] (3) at (2,2) {Architecture Design};
			\node[phase,visible on=<5->] (4) at (3,3) {Software Design};
			\node[phase,visible on=<6->] (5) at (6,4) {Implemen-tation};
			\node[phase,visible on=<7->] (6) at (9,3) {Unit\\Testing};
			\node[phase,visible on=<8->] (7) at (10,2) {Integration Testing};
			\node[phase,visible on=<9->] (8) at (11,1) {System Testing};
			\node[phase,visible on=<10->] (9) at (12,0) {Acceptance Testing};
			\draw[to,visible on=<3->,blue] (1) to node[label] {Requirements Specification} (2);
			\draw[to,visible on=<4->,blue] (2) to node[label] {System Specification} (3);
			\draw[to,visible on=<5->,blue] (3) to node[label] {Architecture Spec.} (4);
			\draw[to,visible on=<6->,blue] (4) to node[label] {Design Spec.} (5.165);
			\draw[to,visible on=<7->,blue] (5.15) to node[label] {Component} (6);
			\draw[to,visible on=<8->,blue] (6) to node[label] {Components} (7);
			\draw[to,visible on=<9->,blue] (7) to node[label] {System} (8);
			\draw[to,visible on=<10->,blue] (8) to node[label] {System} (9);
			\draw[to,visible on=<11->,orange,dashed] (1) to node[auto] {test cases} (9);
			\draw[to,visible on=<12->,orange,dashed] (2) to (8);
			\draw[to,visible on=<12->,orange,dashed] (3) to (7);
			\draw[to,visible on=<12->,orange,dashed] (4) to (6);
	\end{tikzpicture}}
}

% introduced: process
% reused: process
\newcommand{\diagramScrum}{
	\pic[width=\linewidth]{process/scrum}
}

% introduced: requirements
% reused: summary
\newcommand{\slideMindmapNonFunctionalRequirements}{
	{\newcommand{\requirements}{requirements}%
		\begin{tikzpicture}
			\path[%grow cyclic,
			small mindmap,
			%every node/.style=concept,
			concept color=green!20!background,
			%	text width=20mm,align=flush center,
			%	level 1/.append style={level distance=27mm,sibling angle=360/3},
			]
			node[concept] {non-functional \requirements}
			[counterclockwise from=195]
			child[concept color=blue!20!background] { node[concept] {product \requirements} 
				[counterclockwise from=75]
				child[visible on=<2->] { node[concept] {usability \requirements} }
				child[visible on=<3->] { node[concept] {efficiency \requirements} 
					[counterclockwise from=165]
					child { node[concept] {performance \requirements} }
					child { node[concept] {space \requirements} }
				}
				child[visible on=<4->] { node[concept] {dependability \requirements} }
				child[visible on=<5->] { node[concept] {security \requirements} }
			}
			child[concept color=red!20!background] { node[concept] {organizational \requirements} 
				[counterclockwise from=210]
				child[visible on=<6->] { node[concept] {environmental \requirements} }
				child[visible on=<7->] { node[concept] {operational \requirements} }
				child[visible on=<8->] { node[concept] {development \requirements} }
			}
			child[concept color=orange!20!background] { node[concept] {external \requirements} 
				[counterclockwise from=285]
				child[visible on=<9->] { node[concept] {regulatory \requirements} }
				child[visible on=<10->] { node[concept] {ethical \requirements} }
				child[visible on=<11->] { node[concept] {legislative \requirements}
					[counterclockwise from=-15]
					child { node[concept] {accounting \requirements} }
					child { node[concept] {safety/security \requirements} }
				}
			}
			;
	\end{tikzpicture}}
}

% introduced: modeling
% reused: architecture, design, summary
% 
% legend: gray (will not be taught), bold (taught in this lecture), normal font (will be taught)
% 02-requirements: use case diagrams
% 03-modeling: UML, activity diagrams, state machine diagrams
% 04-architecture: component diagrams
% 05-design: class diagrams, sequence diagrams
%
% parameters: use case, activity, state machine, component, class, sequence, structure, behavior, interactions
\tikzset{notTaughtUMLDiagrams/.style={text=gray}}
\newcommand{\slideMindmapUMLdiagrams}[9]{
	{\centering
		\begin{tikzpicture}
			\path[small mindmap,
			every node/.style={concept,font=\footnotesize},
			emph/.style={font=\bfseries\footnotesize},
			blue/.style={emph,text=blue},
			red/.style={emph,text=red},
			concept color=green!20!background,
			level 1/.append style={level distance=35mm,sibling angle=360/3},
			level 2/.append style={level distance=20mm,sibling angle=360/8},
			level 3/.append style={level distance=20mm,sibling angle=360/8},
			]
			node {UML Diagrams}
			[clockwise from=150]
			child[concept color=blue!20!background,#7] { node {Structure Diagrams} 
				[counterclockwise from=15]
				child { node[#4] {Component Diagram} }
				child { node[#5] {Class Diagram} }
				child[notTaughtUMLDiagrams] { node {Profile Diagram} }
				child[notTaughtUMLDiagrams] { node {Composite Structure Diagram} }
				child[notTaughtUMLDiagrams] { node {Deployment Diagram} }
				child[notTaughtUMLDiagrams] { node {Object Diagram} }
				child[notTaughtUMLDiagrams] { node {Package Diagram} }
			}
			child[concept color=red!20!background,#8] { node {Behavior Diagrams} 
				[clockwise from=135]
				child { node[#1] {Use Case Diagram} }
				child { node[#2] {Activity Diagram} }
				child { node[#3] {State Machine Diagram} }
				child[concept color=orange!20!background,#9] { node {Interaction Diagrams} 
					[clockwise from=30]
					child { node[#6] {Sequence Diagram} }
					child[notTaughtUMLDiagrams] { node {Communi-cation Diagram} }
					child[notTaughtUMLDiagrams] { node {Interaction Overview Diagram} }
					child[notTaughtUMLDiagrams] { node {Timing Diagram} }
				}
			}
			;
	\end{tikzpicture}}
}

% introduced: architecture
% reused: summary
\newcommand{\slideArchitecturalPattern}{
	\begin{fancycolumns}
		\centering
		\begin{definition}{Architectural Pattern\mysource{\sommerville}}
			\mycite{Architectural patterns capture the essence of an architecture that has been used in different software systems. [...] Architectural patterns are a means of reusing knowledge about generic system architectures.}
		\end{definition}
		\nextcolumn
		\begin{note}{Goals}
			\partofpage{35}{
				\pic[width=\textwidth]{architecture/database}
			}
			\partofpage{60}{
				\begin{itemize}
					\item preserve knowledge of software architects
					\item reuse of established architectures
					\item enable efficient communication
				\end{itemize}
			}
		\end{note}
	\end{fancycolumns}
}

% introduced: architecture
% reused: compilation?
\newcommand{\slidePipeAndFilter}{
	\begin{fancycolumns}[animation=none]
		\begin{definition}{Pipe-and-Filter Architecture \mysource{\sommerville}}
			\begin{itemize}
				\item \emph{Problem}: data is processed in numerous processing steps, which are prone to change
				\item \emph{Idea}: modularization of each processing step into a component
				\item filter components process a stream of data continously
				\item pipes transfer data unchanged from filter output to filter input
			\end{itemize}
		\end{definition}
		\uncover<2->{
			\begin{example}{Pipe Operator in UNIX}
				\mycite{\texttt{ls -al | grep '2020' | grep -v 'Nov' | more}} searches files in a folder from the year 2020 except those from November and delivers the results in pages.
			\end{example}
		}
		\nextcolumn%
		\uncover<3->{\hfill\picDark[width=.8\textwidth]{architecture/pipe-and-filter}}%
	\end{fancycolumns}
}

% introduced: reuse
% reused: summary
%
% legend: gray (will not be taught), bold (taught in this lecture), normal font (will be taught)
% 
% parameters: object adapter, composite, decorator, singleton, abstract factory, observer, structural, creational, behavioral
\tikzset{notTaughtDesignPatterns/.style={text=gray}}
\newcommand{\slideMindmapDesignPatterns}[9]{
	{\centering
		\begin{tikzpicture}
			\path[small mindmap,
			every node/.style={concept,font=\scriptsize},
			emph/.style={font=\bfseries\scriptsize},
			blue/.style={emph,text=blue},
			red/.style={emph,text=red},
			concept color=green!20!background,
			level 1/.append style={level distance=35mm,sibling angle=65},
			level 2/.append style={level distance=20mm,sibling angle=360/12},
			level 3/.append style={level distance=20mm,sibling angle=360/8},
			]
			node {Design Patterns}
			[clockwise from=155]
			child[concept color=blue!20!background,#7] { node {Structural Patterns} 
				[clockwise from=215]
				child[#1] { node {Object Adapter} }
				child[notTaughtDesignPatterns] { node {Class Adapter} }
				child[notTaughtDesignPatterns] { node {Bridge} }
				child[#2] { node {Composite} }
				child[#3] { node {Decorator} }
				child[notTaughtDesignPatterns] { node {Facade} }
				child[notTaughtDesignPatterns] { node {Flyweight} }
				child[notTaughtDesignPatterns] { node {Proxy} }
			}
			child[concept color=red!20!background,#8] { node {Creational Patterns} 
				[clockwise from=150]
				child[#5] { node {Abstract Factory} }
				child[notTaughtDesignPatterns] { node {Builder} }
				child[notTaughtDesignPatterns] { node {Factory Method} }
				child[notTaughtDesignPatterns] { node {Prototype} }
				child[#4] { node {Singleton} }
			}
			child[concept color=orange!20!background,#9] { node {Behavioral Patterns} 
				[clockwise from=175]
				child[notTaughtDesignPatterns] { node {Chain of Responsibility} }
				child[notTaughtDesignPatterns] { node {Command} }
				child[notTaughtDesignPatterns] { node {Interpreter} }
				child[notTaughtDesignPatterns] { node {Iterator} }
				child[notTaughtDesignPatterns] { node {Mediator} }
				child[notTaughtDesignPatterns] { node {Memento} }
				child[#6] { node {Observer} }
				child[notTaughtDesignPatterns] { node {State} }
				child[notTaughtDesignPatterns] { node {Strategy} }
				child[notTaughtDesignPatterns] { node {Template Method} }
				child[notTaughtDesignPatterns] { node {Visitor} }
			}
			;
	\end{tikzpicture}}
}

