\subsection{Software Development Project}
\begin{frame}{\insertsubsection\ \mytitlesource{\ludewiglichter}}
	\begin{fancycolumns}
		\begin{definition}{Software Development Project}
			\begin{itemize}
				\item aka.\ software engineering project
				\item temporary activity with start and end date
				\item has goals
				\begin{itemize}
					\item creation / modification of a software product
					\item creation / modification of components for future projects
					\item gain experience / knowledge
					\item capacity utilization \deutsch{Mitarbeiterauslastung}
					\item \ldots
				\end{itemize}
				\item is successful if goals are largely fulfilled
			\end{itemize}
		\end{definition}
	\end{fancycolumns}
\end{frame}

\subsection{Project Management}
\begin{frame}{\insertsubsection\ \mytitlesource{\sommerville}}
	\begin{fancycolumns}
		\begin{note}{Motivation}
			\mycite{Good management cannot guarantee project success. However,
				bad management usually results in project failure: The software may be delivered late,
				cost more than originally estimated, or fail to meet the expectations of customers.}
		\end{note}
		\pause
		\begin{definition}{Goals of Project Management}
			\begin{itemize}
				\item \mycite{deliver the software to the customer at the agreed \emph{time}
					\item keep overall \emph{costs} within budget
					\item deliver software that meets the customer’s \emph{expectations}
					\item maintain a coherent and well-functioning development \emph{team}}
			\end{itemize}
		\end{definition}
		\nextcolumn
		\pause
		\begin{example}{Project Management Depends on \ldots}
			\begin{tabularx}{\textwidth}{rX}				
				\emph{company size} & large companies have management hierarchies and reporting / budgeting / approval processes\\
				\emph{customers} & external customers (i.e., government agencies) usually have policies\\
				\emph{software size} & large systems require multiple development teams in different companies / locations\\
				\emph{software type} & safety-critical systems require all design decisions to be documented\\
				%\emph{org.\ culture} & some companies take higher risks\\
				\emph{dev.\ process} & project management heavily depends on process model
			\end{tabularx}
		\end{example}
	\end{fancycolumns}
\end{frame}

\xkcdframe{1319} % planning

\subsection{Activities in Project Management}
\begin{frame}{\insertsubsection\ \mytitlesource{\sommerville}}
	\small
	\begin{fancycolumns}
		\begin{definition}{Project Planning \deutsch{Projektplanung}}
			\mycite{Project managers are responsible for \emph{planning, estimating, and scheduling} project development and assigning people to tasks. They supervise the work to ensure that it is carried out to the required standards, and they \emph{monitor progress} to check that the development is on time and within budget.}
		\end{definition}
		\pause
		\begin{definition}{Risk Management \deutsch{Risikomanagement}}
			\mycite{Project managers have to \emph{assess the risks} that may affect a project, monitor these risks, and take action when problems arise.}
		\end{definition}
		\pause
		\begin{definition}{People Management \deutsch{Mitarbeiterführung}}
			\mycite{Project managers are responsible for \emph{managing a team} of people. They have to choose people for their team and establish ways of working that lead to effective team performance.}
		\end{definition}
		\nextcolumn
		\pause
		\begin{definition}{Reporting \deutsch{Berichterstattung}}
			\mycite{Project managers are usually responsible for \emph{reporting on the progress} of a project to customers and to the managers of the company developing the software. They have to be able to communicate at a range of levels, from detailed technical information to management summaries.}% They have to write concise, coherent documents that abstract critical information from detailed project reports. They must be able to present this information during progress reviews.
		\end{definition}
		\pause
		\begin{definition}{Proposal Writing \deutsch{Projektbeantragung}}
			\mycite{The first stage in a software project may involve writing a proposal to \emph{win a contract} to carry out an item of work. The proposal describes the objectives of the project and how it will be carried out. It usually includes \emph{cost and schedule estimates} and justifies why the project contract should be awarded to a particular organization or team. Proposal writing is a critical task as the survival of many software companies depends on having enough proposals accepted and contracts awarded.}
		\end{definition}
	\end{fancycolumns}
\end{frame}

\subsection{Risk Management}
\begin{frame}{\insertsubsection\ \mytitlesource{\sommerville}}
	\begin{fancycolumns}
		%\mynote{Motivation}{\mycite{Risk management is one of the most important jobs for a project manager. You can think of a risk as something that you’d prefer not to have happen.}}
		\begin{definition}{Risk}
			\setlength\tabcolsep{1mm}
			\begin{tabularx}{\textwidth}{rX}				
				\emph{Probability} & insignificant, low, moderate, high, very high\\
				\emph{Severity} & insignificant, tolerable, serious, catastrophic
			\end{tabularx}
		\end{definition}
		\pause
		\setlength\tabcolsep{1mm}
		\begin{note}{Classification of Risks}
			\begin{tabularx}{\textwidth}{rX}				
				\emph{Project Risks} & affect project schedule or resources: loss of an experienced system architect may result in longer development time\\
				\emph{Product Risks} & affect software quality: purchased component may not scale\\
				\emph{Business Risks} & affect organization / company: product of a competitor may reduce number of sales
			\end{tabularx}
		\end{note}
		\nextcolumn
		\pause
		\begin{definition}{Stages in Risks Management}
			\setlength\tabcolsep{1mm}
			\begin{tabularx}{\textwidth}{rX}				
				\emph{1. Risk Identification} & identify possible project, product, and business risks\\
				\emph{2. Risk Analysis} & assess likelihood and consequences\\
				\emph{3. Risk Planning} & plan how to address risks: avoidance or minimization of effects\\
				\emph{4. Risk Monitoring} & regularly assess risks and revise plans if needed
			\end{tabularx}
		\end{definition}
		\pause
		\begin{example}{Risks in Agile Development}
			reduced risks for requirements changes, increased risks for loss of staff due to fewer documentation
		\end{example}
	\end{fancycolumns}
\end{frame}

\subsection{People Management}
\begin{frame}{\insertsubsection\ \mytitlesource{\sommerville}}
	\begin{fancycolumns}
		\begin{note}{Motivation}
			\mycite{The people working in a software organization are its \emph{greatest assets}. It is expensive to recruit and retain good people, and it is up to software managers to ensure that the engineers working on a project are as \emph{productive} as possible. In successful companies and economies, this productivity is achieved when people are respected by the organization and are assigned responsibilities that reflect their skills and experience.}
		\end{note}
		\pause
		\begin{example}{In Practice}
			\mycite{Software engineers often have strong \emph{technical skills} but may lack the softer skills that enable them to \emph{motivate and lead a project development team}.}
		\end{example}
		\nextcolumn
		\pause
		\begin{definition}{Critical Factors}
			\setlength\tabcolsep{1mm}
			\begin{tabularx}{\textwidth}{rX}
				\emph{1. Consistency} & treat people comparably with similar rewards\\
				\emph{2. Respect} & let all people contribute and respect their differences in skills\\
				\emph{3. Inclusion} & consider views of least experienced peoples\\
				\emph{4. Honesty} & manager is honest about own skills and team performance
			\end{tabularx}
		\end{definition}
		\pause
		\begin{note}{Teamwork}
			\small\mycite{Most professional software is developed by project teams that range in size from two to several hundred people. However, as it is impossible for everyone in a large group to work together on a single problem, \emph{large teams are usually split} into a number of smaller groups. Each group is responsible for developing part of the overall system.}% The best size for a software engineering group is 4 to 6 members, and they should never have more than 12 members. When groups are small, communication problems are reduced. Everyone knows everyone else, and the whole group can get around a table for a meeting to discuss the project and the software that they are developing.
		\end{note}
	\end{fancycolumns}
\end{frame}
