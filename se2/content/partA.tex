% TODO L01 Introduction

\ifuniversity{tubs}{\date{April 8, 2025}}

\author{Thomas Thüm}
\lecture{Introduction}{introduction}

\section{Recap on Software Engineering 1}
\subsection{Software Engineering}
\begin{frame}{Software (Engineering) vs Program(ming)}
	\slideSEvsProgramming
\end{frame}

\subsection{Management}
\begin{frame}[2]{How to manage software development?}
	\slideGanttAndNetwork
\end{frame}
\begin{frame}[10]{How to structure software development?}
	\begin{fancycolumns}[widths={45}]
		\diagramWaterfallModel
	\nextcolumn
		\diagramVModel
	\end{fancycolumns}
\end{frame}

\subsection{Analysis}
\begin{frame}[1]{What software is required?}
	\slideMindmapNonFunctionalRequirements
\end{frame}
\begin{frame}{How to model the planned software?}
	\centering\slideMindmapUMLdiagrams{}{}{}{}{}{}{}{}{}
\end{frame}

\subsection{Design}
\begin{frame}{What is the high-level structure?}
	\begin{fancycolumns}[animation=none]
		\pic[page=16,width=\linewidth,trim=225 20 25 40,clip]{printedslides/2021wt/architecture}
	\nextcolumn
		\pic[page=26,width=\linewidth,trim=225 20 25 40,clip]{printedslides/2021wt/architecture}
	\end{fancycolumns}
\end{frame}
\begin{frame}{What is the low-level structure?}
	\begin{fancycolumns}[animation=none]
		\pic[page=13,width=\linewidth,trim=225 20 20 40,clip]{printedslides/2021wt/design}
	\nextcolumn
		\objectadapter{width=\linewidth}
	\end{fancycolumns}
\end{frame}

\subsection{Implementation}
\begin{frame}[2]{How to implement the software?}
	\picDark[width=\linewidth]{history/tiobe}
\end{frame}

\subsection{Testing}
\begin{frame}{How to test the software?}
	\slideMindmapQualityAssurance{}{}{}{}{}{}{}
\end{frame}

\subsection{Maintenance}
\begin{frame}[2]{How to change the software?}
	\slideEvolutionAndMaintenance
\end{frame}
\begin{frame}{How to control changes?}
	\centering\picDark[width=.75\linewidth]{versioncontrol/devops}
\end{frame}

% TODO introduce project cartoon

%\begin{frame}{Software Engineering I}
%	\vspace{-5mm}
%	\renewcommand{\projectcartoonwidth}{.13}
%	\small\hfill
%	\hprojectcartoon{01}{requirements}
%	\hprojectcartoon{02}{modeling}
%	\hprojectcartoon{03}{architecture and design} % architecture/design
%	\hprojectcartoon{04}{implementation}\\\hfill
%	\hprojectcartoon{05}{testing} % testing
%	\hprojectcartoon{06}{process}
%	%\hprojectcartoon{07}{how the project was documented} % code documentation
%	%\hprojectcartoon{08}{what operations installed} % devops/continuous integration
%	\hprojectcartoon{09}{management and pricing} % pricing
%	%\hprojectcartoon{10}{how it was supported}
%	%\hprojectcartoon{11}{what marketing advertised}
%	%\hprojectcartoon{12}{when it was delivered} % continous delivery
%	\uncover<2->{\hprojectcartoon{13}{Software Engineering II}}
%	%\hprojectcartoon{14}{what the digg effect can do to your site} % ???
%	%\hprojectcartoon{15}{the disaster recover plan}
%	%\hprojectcartoon{16}{how open source version} % open source/licensing
%	%\hprojectcartoon{17}{how it performed under load} % quality assurance/performance
%	%\hprojectcartoon{18}{how patches were applied} % software maintenance
%	
%\end{frame}

\lessonslearned{
	\item How to manage / structure software development?
	\item What software is required?
	\item How to model / design / implement / test / change / maintain software?
	\item Next: What are motivating examples for SE2?
}{
	\item[] See literature pointers in SE1
}{
	Quiz in \StudIP:
	
	~
	
	\centering\fancyqr{color=black,height=30mm}{https://studip.tu-braunschweig.de/dispatch.php/course/courseware/courseware/18333?cid=635f5186b364a369979d0f45ab2ace1d\#/structural\_element/421044}
}

\section{Motivating Examples}
% TODO add new Tesla example: https://www.bbc.com/news/articles/c93dqpkwx4xo.amp

\subsection{Saxony Election Glitch} % motivation for 10-opensource
\begin{frame}{\insertsubsection}
	\slideSaxonyElection
\end{frame}

\subsection{Switzerland Goes Open Source} % motivation for 10-opensource
\begin{frame}{\insertsubsection}
	\begin{fancycolumns}
		\begin{definition}{EMBAG\mysource{\href{https://www.zdnet.com/article/switzerland-now-requires-all-government-software-to-be-open-source/}{zdnet.com}, \href{https://datenrecht.ch/en/bundesgesetz-ueber-den-einsatz-elektronischer-mittel-zur-erfuellung-von-behoerdenaufgaben-embag-in-schlussabstimmung-angenommen/}{datenrecht.ch}}}
			\begin{itemize}
				\item Federal Law on the Use of Electronic Means for the Fulfillment of Government Tasks \deutsch{Einsatz elektronischer Mittel zur Erfüllung von Behördenaufgaben}
				\item official since March 13, 2023
				\item goal: enable collaboration within Switzerland and to other countries
				\item vision: public money, public code
				\item means: source code and data is published
				\item envisioned: transparency, security, efficiency
			\end{itemize}
		\end{definition}
		\nextcolumn
		\begin{note}{EMBAG\hfill\tikz[overlay] \node[anchor=-10,xshift=2mm] {\pic[width=.2\linewidth]{misc/switzerland}};}
			\begin{itemize}
				\item \mycite{the use of Open Source Software - to the extent possible, the federal government should \emph{disclose the source code} of software that it develops or has developed \emph{for free reuse}}
				\item \mycite{the gradual public \emph{provision of data} obtained or generated for the fulfillment of statutory tasks (Open Government Data)}
				\item \mycite{the provision or purchase of \emph{Shared Services} by the federal authorities}
				\item \mycite{the implementation of \emph{Pilot testing}} \mysource{\href{https://datenrecht.ch/en/bundesgesetz-ueber-den-einsatz-elektronischer-mittel-zur-erfuellung-von-behoerdenaufgaben-embag-in-schlussabstimmung-angenommen/}{datenrecht.ch}} 
			\end{itemize}
		\end{note}
	\end{fancycolumns}
\end{frame}

\xkcdframe{2961}

\subsection{CrowdStrike Outage} % motivation for 02-evolution, 03-maintance, 07-dynamic analysis, 09-configuration
\begin{frame}{\insertsubsection}
	\slideCrowdStrike
\end{frame}

\subsection{Ariane 5 Software Failure} % motivation for 08-designbycontract
\begin{frame}{\insertsubsection}
	\slideArianeFailure
\end{frame}

\subsection{Xiaomi SU7 Crash} % motivation for 12-automotive
\begin{frame}{\insertsubsection}
	\slideXiaomiCrash
\end{frame}

%TODO include Therac-25 as motivation for dynamic/testing

% motivation for 04-designpatterns, 05-compilation, 06-staticanalysis, 11-productlines
%\subsection{Example}
%\begin{frame}{\insertsubsection}
%	\begin{fancycolumns}
%		\begin{definition}{\insertsubsection}
%			\begin{itemize}
%				\item 
%			\end{itemize}
%		\end{definition}
%		\nextcolumn
%		% TODO add picture
%	\end{fancycolumns}
%\end{frame}

% 2021 Facebook outage https://en.wikipedia.org/wiki/2021_Facebook_outage

% Xerox

% Heartbleed

% too high x-ray

% 4096 more votes

% Why the Vasa sank: 10 problems and some antidotes for software projects
% https://ieeexplore.ieee.org/document/1184161

% obama care website: motivation for agil, operations 
% https://www.nuna.com/wp-content/uploads/2022/08/TimeMagazine-Code-Red.pdf
% https://content.time.com/time/covers/0,16641,20140310,00.html

% Software update mission critical https://www.space.com/spacex-crew-6-dragon-docking-space-station\

% Cyber Resilience Act (CRA)  

% Log4J

% CVE-2021-44228

% https://www.heise.de/news/Sicherheitsbehoerden-draengen-Softwarehersteller-zu-sicheren-Programmiersprachen-9573379.html

% Microsoft Calculator showing wrong formulas for years
% https://github.com/Microsoft/calculator

% TVs watch what you watch
% https://maestro.acm.org/trk/click?ref=z16l2snue3_2-3170a_0x342a63x0103451

% https://apnews.com/article/v2x-infrastructure-cars-streets-utah-michigan-texas-4609b3f43768f6726bbb4e5442f40f21

% https://maestro.acm.org/trk/click?ref=z16l2snue3_2-31717_0x342b62x0101340

\lessonslearned{
	\item software failures are everywhere:\\Saxony election, CrowdStrike outage, Ariane 5 failure, Xiaomi SU7 crash
	\item and may have severe consequences
	\item software engineering aims to reduce them
	\item good software engineering practices required by law: Switzerland's government
	\item Next: How is SE2 organized?
}{
	\item[] see links on previous slides
}{
	\begin{enumerate}
		\item Form groups of 2--3 students
		\item Familiarize with each other: name, field of study, term, other common courses, \ldots
		\item Do you know any further examples of software glitches?
		\item Could better software engineering have prevented those?
	\end{enumerate}
}

\section{Course Organization}
\subsection{What You Should Know}

\begin{frame}{\myframetitle{}}
	\begin{fancycolumns}
		\begin{note}{Fundamentals of Software Engineering}
			\begin{itemize}
				\item development processes: waterfall, v-model, scrum
				%\item project management
				\item analysis, design, implementation, testing, maintanance
				\item UML diagrams (esp. class and component diagrams)
				\item version control (esp. git)
			\end{itemize}
			\ifuniversity{tubs}{$\Rightarrow$ \emph{Software Engineering 1}}
		\end{note}
	\nextcolumn
		\begin{note}{Fundamentals of Theoretical Computer Science}
			\begin{itemize}
				%\item set theory
				\item propositional logic
				\item predicate logic
			\end{itemize}
		\end{note}
		\begin{note}{Exercise}
			skills in object-oriented programming (preferably Java)

			%\ifuniversity{tubs}{$\Rightarrow$ \emph{Software Engineering 1}}
		\end{note}
	\end{fancycolumns}
\end{frame}

\subsection{What You Will Learn}

\begin{frame}{\myframetitle{}}
	\frameCourseQuestions
\end{frame}

\begin{frame}{\myframetitle{}}
	\lectureseriesoverview[1]
\end{frame}

\subsection{What You Might Need}

\begin{frame}{\insertsubsection}
	\begin{fancycolumns}[animation=none]
		\centering\pic[height=50mm]{books/sommerville-softwarenegineering}
		\nextcolumn
		\begin{definition}{\mysource{\sommerville}}
			\begin{itemize}
				\item \sommervillelink{Ian Sommerville. Software Engineering, 10. Edition, Pearson, 2018.}
				\begin{itemize}
					\item German, English, and earlier versions
					\item \href{https://software-engineering-book.com/videos/}{Videos by Ian Sommerville and others available online}
				\end{itemize}
				\item More literature announced in each lecture
			\end{itemize}
		\end{definition}
	\end{fancycolumns}
\end{frame}

%\subsection{Credit for the Slides}
%
%\begin{frame}{\myframetitle{}}
%	\ifuniversity{anonymous}{\mynote{}{\centering\huge Anonymous Authors}}
%	\unlessuniversity{anonymous}{
%		\myframeicon{\href{https://github.com/SoftVarE-Group/Course-on-Software-Product-Lines}{\pic[scale=.75]{cc-by-sa}}}
%	}
%	\begin{fancycolumns}[columns=3,animation=none]
%	\nextcolumn
%		\unlessuniversity{anonymous}{
%			\begin{note}{Thomas Thüm}
%				\centering
%				\href{https://www.uni-ulm.de/en/in/sp/team/thuem/}{\adjincludegraphics[height=.45\textheight,trim={.125\width} 0 {.125\width} 0,clip]{thomas-thuem}}
%
%				\small Professor at Paderborn University
%
%				software engineering
%
%				FeatureIDE team leader
%			\end{note}
%		}
%	\nextcolumn
%	\end{fancycolumns}
%\end{frame}

\inputuniversity{content/01c-course}
\lessonslearned{
	\item What you: should know, will learn, might need
	\item The course: lectures, exercises
	% TODO \item Next Lecture: 
}{
	\item[] Main Book: \sommervillelink{Ian Sommerville. Software Engineering, 10. Edition, Pearson, 2018.}
}{
	\begin{itemize}
		\item Any questions?
		\item Optional: repeat SE1 quizzes from last term, all collected in one quiz for SE2 in \StudIP

			~

			\centering\fancyqr{color=black,height=30mm}{https://studip.tu-braunschweig.de/dispatch.php/course/courseware/courseware/18333?cid=635f5186b364a369979d0f45ab2ace1d\#/structural\_element/421044}
	\end{itemize}
}

\faq{
	\item Recap on SE1
	\item How is software engineering different from programming?
	\item How are software projects managed with Gantt charts and network diagrams?
	\item How are software projects structured with process models (e.g., Waterfall model, V model)?
	\item What are functional and non-functional requirements?
	\item How to model software with UML (incl. high-level and low-level structure)?
	\item How to implement and test software?
	\item How to change software and control changes?
}{
	\item How can a software glitch effect elections?
	\item How can politians enforce the use and development of open-source software?
	\item How can a software update cause the largest IT outage?
	\item How can software reuse postpone research by one year?
	\item How can advanced driver assistance crash a car?
	% TODO Raphael: collect other examples throughout the course and add all here
}{
	\item What you should know before taking this course?
	\item What you will learn during this course?
	\item Which book is recommended for this course?
	\item How is this course organized?
}

\input{../se1/template/footer}

% TODO L02 Evolution

\ifuniversity{tubs}{\date{April 14, 2025}}

\author{Thomas Thüm}
\lecture{Evolution}{evolution}

\section{Laws on Software Evolution}
\subsection{Change is Inevitable}
\begin{frame}{\insertsubsection}
	\begin{fancycolumns}
		\begin{definition}{\sommerville:}
			\mycite{Change is inevitable in all large software projects. The \textbf{system requirements change} as businesses respond to external pressures, competition, and changed management priorities. As \textbf{new technologies become available}, new approaches to design and implementation become possible. Therefore whatever software process model is used, it is essential that it can \textbf{accommodate changes} to the software being developed. [...] It may then be necessary to \textbf{redesign} the system to deliver the new requirements, \textbf{change any programs} that have been developed, and \textbf{retest} the system.}
		\end{definition}
		\nextcolumn
		\pic[width=\linewidth]{failures/swu-evolution}
	\end{fancycolumns}
\end{frame}

\begin{frame}{\insertsubsection}
	\begin{fancycolumns}
		\begin{definition}{Reasons for Changes \mysource{\sommerville}}
			\begin{itemize}
				\item costly implementation, longer usage that costs pay off, make better use of it by small improvements
				\item changes to the environment (e.g., other systems) require changes to the software
				\item new laws may require adaptations
				\item corrections of errors
				\item iterative development (e.g., with Scrum)
				\item \ldots
			\end{itemize}
		\end{definition}
		\begin{note}{Alternatives to Changes}
			\begin{itemize}
				\item stick to old version (e.g., feature freeze)
				\item replace system by new implementation
				\item abandom the system without replacement
			\end{itemize}
		\end{note}
	\nextcolumn
		\begin{definition}{Change Activities \mysource{\sommerville}}
			\begin{itemize}
				\item change specification
				\item analysis of change consequences
				\item release planning
				\item implementation and testing of changes
				\item release of the changed version
			\end{itemize}
		\end{definition}
	\end{fancycolumns}
\end{frame}

\pictureframe{
	\pic[width=\paperwidth]{emotions/walking-on-water}
}{
	\vspace{65mm}
	\begin{note}{Edward V. Berard (1993) \mysource{\href{https://en.wikiquote.org/wiki/Edward_V._Berard}{wikiquote.org}}}
		Recap: \mycite{Walking on water and developing software from a specification are easy if both are frozen.}
	\end{note}
}

\subsection{Changes to Requirements}
\begin{frame}{\insertsubsection}
	\begin{fancycolumns}
		\begin{note}{Motivation \mysource{\sommerville}}
			\begin{itemize}
				\item changed problem understanding for all stakeholders
				\item new or updated hardware
				\item interfacing to other systems
				\item new legislation and regulations
				\item new compromise on conflicting requirements of different stakeholders
			\end{itemize}
		\end{note}
	\end{fancycolumns}
\end{frame}

\subsection{Recap: Process Models}
\begin{frame}<8>{\insertsubsection}
	\begin{fancycolumns}
		\begin{exampletight}{Waterfall Model}
			\centering
			\diagramWaterfallModel
		\end{exampletight}
		\nextcolumn
		\begin{exampletight}{Scrum}
			\diagramScrum
		\end{exampletight}
	\end{fancycolumns}
\end{frame}

\subsection{Recap: Project Management}
\begin{frame}{\insertsubsection}
	\begin{fancycolumns}[animation=none]
		\nextcolumn
		\begin{exampletight}{Gantt Charts}
			\centering\picDark[width=\linewidth]{management/gantt-chart}
		\end{exampletight}
	\end{fancycolumns}
\end{frame}




% TODO \widexkcdframe{1667} % complexity comes with changes over time, copied from se1-04b

\subsection{Increasing Complexity}
\begin{frame}{\insertsubsection} % slided copied from se1-04b. reuse?
	\begin{fancycolumns}
		\pic[width=\linewidth,trim=0 200 0 600,clip]{people/tony-hoare}
		\vspace{-7mm}
		
		\begin{note}{Tony Hoare (born 1934) \mysource{\href{https://twitter.com/CodeWisdom/status/893532377586302977}{twitter.com}}}
			\mycite{Inside every large program, there is a small program trying to get out.}
		\end{note}
		% quicksort, hoare logic
		\nextcolumn
		\pic[width=\linewidth,trim=50 85 200 120,clip]{people/manny-lehman}
		\vspace{-7mm}
		
		\begin{note}{{Meir \mycite{Manny} Lehman (1925--2010) \mysource{\href{https://twitter.com/CodeWisdom/status/921139649661284352}{twitter.com}}}}
			\mycite{An evolving system increases its complexity unless work is done to reduce it.}
		\end{note}
		% known for laws on software evolution
	\end{fancycolumns}
\end{frame}

\subsection{Lehman's Laws of Software Evolution}
\begin{frame}{\insertsubsection}
	\begin{fancycolumns}
		\begin{definition}{Law of Continuing Change\mysource{\lehmanslaws}}
			\begin{itemize}
				\item \deutsch{Kontinuierliche Veränderung}
				\item systems must be continually adapted to stay satisfactory % E-type systems must be continually adapted else they become progressively less satisfactorv.
				\item explanation: an unchanged system will loose its benefits when the real environment changes
			\end{itemize}
		\end{definition}
		\begin{example}{Driver Assistance vs New Traffic Signs}
			\begin{fancycolumns}[b,columns=3,animation=none]
				\centering\pic[width=\linewidth]{automotive/StVO-2019-escooter}
				2019-06-06
			\nextcolumn
				\centering\pic[width=\linewidth]{automotive/StVO-2020-carsharing}
				2020-08-31
			\nextcolumn
				\centering\pic[width=\linewidth]{automotive/StVO-2022-charging}
				2022-08-15
			\end{fancycolumns}
		\end{example}
		\nextcolumn
		\begin{definition}{Law of \mysource{\lehmanslaws}}
			\begin{itemize}
				\item \deutsch{}
				\item 
				\item explanation: 
			\end{itemize}
		\end{definition}
		\begin{example}{}
		\end{example}
	\end{fancycolumns}
\end{frame}

\begin{frame}{\insertsubsection}
	\begin{fancycolumns}
		\begin{definition}{Law of \mysource{\lehmanslaws}}
			\begin{itemize}
				\item \deutsch{}
				\item 
				\item explanation: 
			\end{itemize}
		\end{definition}
		\begin{example}{}
		\end{example}
		\nextcolumn
		\begin{definition}{Law of \mysource{\lehmanslaws}}
			\begin{itemize}
				\item \deutsch{}
				\item 
				\item explanation: 
			\end{itemize}
		\end{definition}
		\begin{example}{}
		\end{example}
	\end{fancycolumns}
\end{frame}

\begin{frame}{\insertsubsection}
	\begin{fancycolumns}
		\begin{definition}{Law of \mysource{\lehmanslaws}}
			\begin{itemize}
				\item \deutsch{}
				\item 
				\item explanation: 
			\end{itemize}
		\end{definition}
		\begin{example}{}
		\end{example}
		\nextcolumn
		\begin{definition}{Law of \mysource{\lehmanslaws}}
			\begin{itemize}
				\item \deutsch{}
				\item 
				\item explanation: 
			\end{itemize}
		\end{definition}
		\begin{example}{}
		\end{example}
	\end{fancycolumns}
\end{frame}

\begin{frame}{\insertsubsection\ \mytitlesource{\href{https://ieeexplore.ieee.org/iel3/5031/13795/00637156.pdf}{Lehman et al.\ 1997}}} % slided copied from se1-04b. reuse?
	\begin{fancycolumns}
		\begin{definition}{Lehman's Laws (excerpt)}
			\begin{itemize}
				\item Continuing Change: systems must be continually adapted to stay satisfactory % E-type systems must be continually adapted else they become progressively less satisfactorv.
				\item Increasing Complexity: complexity increases during evolution unless work is done to maintain or reduce it % As an E-type system evolves its complexity increases unless work is done to maintain or reduce it.
				%\item Self Regulation: %E-type system evolution process is self regulating with distribution of product and process measures close to normal.
				%\item Conservation of Organizational Stability (invariant work rate): %The average effective global activity rate in an evolving E-type system is invariant over product lifetime.
				\item Conservation of Familiarity: satisfactory evolution excludes excessive growth %As an E-type system evolves all associated with it, developers, sales personnel, users, for example, must maintain mastery of its content and behaviour to achieve satisfactory evolution. Excessive growth diminishes that mastery. Hence the average incremental growth remains invariant as the system evolves.
				\item Continuing Growth: functionality must be continually increased to maintain user satisfaction %The functional content of E-type systems must be continually increased to maintain user satisfaction over their lifetime.
				\item Declining Quality: quality will decline unless rigorously maintained and adapted to operational environment changes %The quality of E-type systems will appear to be declining unless they are rigorously maintained and adapted to operational environment changes.
				%\item Feedback System: %E-type evolution processes constitute multi-level, multi-loop, multi-agent feedback systems and must be treated as such to achieve significant improvement over any reasonable base.
			\end{itemize}
		\end{definition}
		\nextcolumn
		\begin{note}{Essence of the Laws}
			\begin{itemize}
				\item software that is used will be modified
				\item when modified, its complexity will increase (unless one does actively work against it)
			\end{itemize}
		\end{note}
		\begin{example}{Consequences}
			\begin{itemize}
				\item functional changes are inevitable
				\item changes are not necessarily a consequence of errors (e.g., in requirements engineering or programming)
				\item there are limits to what a development team can achieve (cf.\ Continuing Growth)
			\end{itemize}
		\end{example}
	\end{fancycolumns}
\end{frame}
% TODO read paper again? separate slide on each law? see Lehman-LawsSoftwareEvolution, Herraiza-EvolutionLawsSoftwareEvolution

% TODO add Elias' graphs on Linux increase

% TODO counter example: massive removal of functionality: see Rust-DefeatingFeatureFatigue
% Antoine de Saint-Exupéry's "It seems that perfection is reached not when there is nothing left to add, but when there is nothing left to take away";















% TODO alpha, beta, ... versions

% TODO stages: software maturity, phase-out, close-down (see Vaclav and Keith)

% TODO read \ludewiglichter, 23.1 Software-Evolution

% TODO read Harry M. Sneed, Richard Seidl: Softwareevolution. 1. Auflage. dpunkt.verlag, Heidelberg 2013, ISBN 978-3-86490-041-9, S. 284.

% TODO read Tom Mens, Serge Demeyer: Software Evolution. 1. Auflage. Springer-Verlag, Berlin / Heidelberg 2008, ISBN 978-3-540-76439-7, S. 347 (englisch).

\lessonslearned{
	\item Incremental development and changed requirements cause evolution
	\item Lehman's laws: software is modified and gets more complex\\
		Continuing Change, Increasing Complexity, Conservation of Familiarity, Continuing Growth, Declining Quality, Feedback System
	\item Next: How to reduce the complexity?
}{
	\item \sommerville\mychapter{2.3 Coping with Change}
	\item \sommerville\mychapter{4.6 Requirements Change}
	\item \sommerville\mychapter{9 Software Evolution}
	\item \lehmanslaws
	%\item \ludewiglichter, 23.1 \deutsch{Software-Evolution}
}{
	\begin{enumerate}
		\item Form groups of 2--3 students
		\item Try to find a (counter) example for each of Lehman's laws
		\item Discuss the example with your colleagues -- does it fulfill further laws?
	\end{enumerate}
}

\section{Smells and Refactorings}
\subsection{Writing Requires Reading}
\begin{frame}{\insertsubsection}
	\begin{fancycolumns}
		\pic[width=\linewidth,trim=800 800 800 600,clip]{people/robert-martin}
		\vspace{-7mm}
		
		\begin{note}{{Robert C.\ Martin (Uncle Bob, born 1952)}}
			\mycite{So if you want to go fast, if you want to get done quickly, if you want your code to be easy to write, make it easy to read.} \mysource{\href{https://learning.oreilly.com/library/view/clean-code-a/9780136083238/}{oreilly.com}}
		\end{note}
		% agile manifesto, book author
		\nextcolumn
		\pic[width=\linewidth,trim=100 0 250 0,clip]{architecture/source-code2}
		\vspace{-7mm}
		
		\begin{note}{Eagleson's Law \mysource{\href{https://www.comp.nus.edu.sg/~damithch/pages/SE-quotes.htm?type=maintenanceQuotes}{nus.edu.sg}}}
			\mycite{Any code of your own that you haven't looked at for six or more months might as well have been written by someone else.}
		\end{note}
		% TODO find complete author name, picture, source
	\end{fancycolumns}
\end{frame}

% TODO software aging (see Parnas-SoftwareAging, GodfreyGerman-LehmansLaws)

% TODO software erosion/rot/decay/entropy:  https://en.wikipedia.org/wiki/Software_rot
% examples: y2k/2038 problem, 1989 zip code union, 1999/2002 euro, 
% What happens on January 19, 2038? On this date the Unix Time Stamp will cease to work due to a 32-bit overflow. https://www.unixtimestamp.com/

%\widexkcdframe{2038} % y2k bug vs 2038 bug

% TODO add technical debt: https://en.wikipedia.org/wiki/Technical_debt
% related to John Carmack's quote in lecture on product lines

% "Shipping first time code is like going into debt. A little debt speeds development so long as it is paid back promptly with a rewrite... The danger occurs when the debt is not repaid. Every minute spent on not-quite-right code counts as interest on that debt. Entire engineering organizations can be brought to a stand-still under the debt load of an unconsolidated implementation, object-oriented or otherwise." Ward Cunningham, 1992

% Fowler's quadrant: https://martinfowler.com/bliki/TechnicalDebtQuadrant.html

% technical depth in numbers, see TornhillBorg-CodeRed, Krasner-CostPoorSoftware

% causes, consequences, limitations of the metapher

% giving devs more time?
% Adding manpower to a late software project makes it later. https://en.wikipedia.org/wiki/Brooks%27s_law
% "work expands so as to fill the time available for its completion" https://en.wikipedia.org/wiki/Parkinson%27s_law

% code smells: https://en.wikipedia.org/wiki/Code_smell
% Fowler, Martin (1999). Refactoring. Improving the Design of Existing Code. Addison-Wesley. ISBN 978-0-201-48567-7.
% Martin, Robert C. (2009). "17: Smells and Heuristics". Clean Code: A Handbook of Agile Software Craftsmanship. Prentice Hall. ISBN 978-0-13-235088-4.

% TODO add cyclomatic complexity https://en.wikipedia.org/wiki/Cyclomatic_complexity




% anti-patterns: https://en.wikipedia.org/wiki/Anti-pattern
% spaghetti code: https://en.wikipedia.org/wiki/Spaghetti_code

\widexkcdframe{292} % goto

\subsection{Spaghetti or Lasagne?}
\begin{frame}{\insertsubsection}
	\begin{fancycolumns}
		\pic[width=\linewidth,trim=0 500 0 150,clip]{people/edsger-dijkstra}
		\vspace{-7mm}
		
		\begin{note}{Edsger W. Dijkstra (1968) \mysource{\href{https://dl.acm.org/doi/pdf/10.1145/362929.362947}{acm.org}}}
			\mycite{Go-To Statement Considered Harmful}\\(today known as spaghetti code)
		\end{note}
		% 1930-2002, ACM Turing Award winner
		\nextcolumn
		\pic[width=\linewidth,trim=0 240 0 310,clip]{emotions/lasagne}
		\vspace{-7mm}
		
		\begin{note}{Roberto Waltman \mysource{\href{https://twitter.com/codewisdom/status/1105462704947580928}{twitter.com}}}
			\mycite{In the one and only true way. The object-oriented version of 'Spaghetti code' is, of course, 'Lasagna code'. (Too many layers).} % TODO check citation within https://www.google.de/books/edition/Introduction_to_Programming_Using_Proces/el2jBgAAQBAJ?hl=en&gbpv=0
		\end{note}
	\end{fancycolumns}
\end{frame}

\subsection{Simple and Clean}
\begin{frame}{\insertsubsection} % slided copied from se1-04b. reuse?
	\begin{fancycolumns}
		\pic[width=\linewidth,trim=0 35 0 40,clip]{people/grady-booch}
		\vspace{-7mm}
		
		\begin{note}{Grady Booch (born 1955) \mysource{\href{https://twitter.com/Grady_Booch/status/1035409406068813824}{twitter.com}}}
			\mycite{The function of good software is to make the complex appear to be simple.}
		\end{note}
		% known for UML
		\nextcolumn
		\pic[width=\linewidth,trim=800 700 800 600,clip]{people/robert-martin}
		\vspace{-7mm}
		
		\begin{note}{{Robert C.\ Martin (Uncle Bob, born 1952)}}
			Boy Scouts Rule: \mycite{Leave the campground cleaner than the way you found it.} \mysource{\href{https://learning.oreilly.com/library/view/97-things-every/9780596809515/ch08.html}{oreilly.com}}
		\end{note}
		% agile manifesto, book author
	\end{fancycolumns}
\end{frame}

\subsection{Code Refactoring}
\begin{frame}{\insertsubsection\ \mytitlesource{\sommerville}} % slided copied from se1-04b. reuse?
	\begin{fancycolumns}
		\begin{note}{Motivation}
			\begin{itemize}
				\item anticipating changes is typically infeasible
				\item anticipated changes may not materialize and unanticipated changes are required
				\item make changes easier by constantly refactoring the code
			\end{itemize}
		\end{note}
		\begin{definition}{Refactoring}
			\mycite{Refactoring means that the programming team looks for possible improvements to the software and implements them immediately. When team members see code that can be improved, they make these improvements even in situations where there is no immediate need for them.}
		\end{definition}
		\nextcolumn
		\begin{definition}{Structure of the Software}
			\mycite{A fundamental problem of incremental development is that local changes tend to \textbf{degrade the software structure}. Consequently, further changes to the software become harder and harder to implement. Essentially, the development proceeds by finding workarounds to problems, with the result that code is often duplicated, parts of the software are reused in inappropriate ways, and the overall structure degrades as code is added to the system. \textbf{Refactoring improves the software structure and readability and so avoids the structural deterioration that naturally occurs when software is changed.}}
		\end{definition}
	\end{fancycolumns}
\end{frame}

\subsection{Refactoring in Practice}
\begin{frame}{\insertsubsection} % slided copied from se1-04b. reuse?
	\begin{fancycolumns}
		\begin{example}{Theory vs Practice \mysource{\sommerville}}
			\mycite{In principle, when refactoring is part of the development process, the software should always be easy to understand and change as new requirements are proposed. In practice, this is not always the case. Sometimes \textbf{development pressure means that refactoring is delayed} because the time is devoted to the implementation of new functionality. Some new features and changes cannot readily be accommodated by code-level refactoring and \textbf{require that the architecture of the system be modified}.}
		\end{example}
		\nextcolumn
		\begin{note}{{Grandma Beck's Child-Rearing Philosophy}}
			\mycite{If it stinks, change it.} \mysource{\href{https://learning.oreilly.com/library/view/refactoring-improving-the/9780134757681/ch03.xhtml}{oreilly.com}}
		\end{note}
	\end{fancycolumns}
\end{frame}

\subsection{Smells and Refactorings} % TODO add examples for refactorings and smells
\begin{frame}{\insertsubsection\ \mytitlesource{\refactoring}} % slided copied from se1-04b. reuse?
	\begin{fancycolumns}
		\begin{definition}{Mysterious Name}
			\mycite{Puzzling over some text to understand what’s going on is a great thing if you’re reading a detective novel, but not when you’re reading code. [...] One of the most important parts of clear code is good names, so we put a lot of thought into naming functions, modules, variables, classes, so they clearly communicate what they do and how to use them.}\\--- Rename Method/Field/Variable Refactoring
		\end{definition}
		\nextcolumn
		\begin{definition}{Duplicated Code (aka.\ Code Clones)}
			\mycite{If you see the same code structure in more than one place, you can be sure that your program will be better if you find a way to unify them. Duplication means that every time you read these copies, you need to read them carefully to see if there’s any difference. If you need to change the duplicated code, you have to find and catch each duplication.}\\--- Extract/Pull-Up Method Refactoring
		\end{definition}
	\end{fancycolumns}
\end{frame}

\begin{frame}{\insertsubsection\ \mytitlesource{\refactoring}} % slided copied from se1-04b. reuse?
	\begin{fancycolumns}
		\begin{definition}{Long Method}
			\mycite{Since the early days of programming, people have realized that the longer a function is, the more difficult it is to understand. Older languages carried an overhead in subroutine calls, which deterred people from small functions. [...] [T]he real key to making it easy to understand small functions is good naming. If you have a good name for a function, you mostly don’t need to look at its body. [...] A heuristic we follow is that whenever we feel the need to comment something, we write a function instead.}\\--- Extract Method Refactoring
		\end{definition}
		\nextcolumn
		%\mydefinition{}{\mycite{}\\---  Refactoring}
		\pic[width=\linewidth,trim=800 800 800 600,clip]{people/robert-martin}
		\vspace{-7mm}
		
		\begin{note}{{Robert C.\ Martin (Uncle Bob, born 1952)}}
			\mycite{Functions should do one thing. They should do it well. They should do it only.} \mysource{\href{https://learning.oreilly.com/library/view/clean-code-a/9780136083238/}{oreilly.com}}
		\end{note}
		% agile manifesto, book author
	\end{fancycolumns}
\end{frame}


\lessonslearned{
	\item Smells / antipatterns: structures to avoid
	\item Examples: mysterious name, duplicated code, long method
	\item Refactorings: clean-up of structures before and after every change
	\item Examples: rename X refactoring, extract/pull-up method refactoring
	\item Next: What are principles against smells?
}{
	\item \refactoring
	\item \cleancode
	% book on antipatterns
}{
	\begin{enumerate}
		\item Form groups of 2--3 students
		\item What further smells and refactorings do you know?
		\item Enter these smells and refactorings using the link in \StudIP

		~
		
		\centering\fancyqr{color=black,height=30mm}{https://studip.tu-braunschweig.de/dispatch.php/course/courseware/courseware/18333?cid=635f5186b364a369979d0f45ab2ace1d\#/structural\_element/421045}
	\end{enumerate}
}

\section{Principles for Software Evolution}
\subsection{Avoiding Complexity}
\begin{frame}{\insertsubsection} % slided copied from se1-04b. reuse?
	\begin{fancycolumns}
		\pic[width=\linewidth,trim=10 0 10 0,clip]{people/richard-pattis}
		\vspace{-7mm}
		
		\begin{note}{Richard E.\ Pattis \mysource{\href{https://www.cs.cmu.edu/~pattis/quotations.html}{cmu.edu}}}
			\mycite{When debugging, novices insert corrective code; experts remove defective code.}
		\end{note}
		% American professor, author of the educational programming language Karel
		\nextcolumn
		\pic[width=\linewidth,trim=0 120 0 30,clip]{people/ken-thompson}
		\vspace{-7mm}
		
		\begin{note}{Ken Thompson (born 1943) \mysource{\href{https://twitter.com/CodeWisdom/status/1386049811800109056}{twitter.com}}}
			\mycite{One of my most productive days was throwing away 1,000 lines of code.}
		\end{note}
		% Turing Award 1983, inventor of the programming language B (predecessor of C) and Go, UTF-8 encoding
	\end{fancycolumns}
\end{frame}

\subsection{KISS Principle}
\begin{frame}{\insertsubsection}
	\begin{fancycolumns}
		\pic[width=\linewidth,trim=0 0 0 0,clip]{people/kelly-johnson}
		\vspace{-7mm}
		
		\begin{note}{Kelly Johnson (1910--1990)\mysource{\href{https://www.nasonline.org/wp-content/uploads/2024/06/johnson-clarence.pdf}{nasaonline.org}}}
			\mycite{KISS --- Keep it simple, stupid!}
		\end{note}
	\nextcolumn
		\begin{definition}{\insertsubsection}
			\begin{itemize}
				\item Johnson leading engineers building jets
				\item goal was to make jets easily repairable by average mechanic
				\item analogy: build software systems that are easy to maintain and repair
			\end{itemize}
		\end{definition}
		\uncover<3->{\begin{example}{Russian Invasion of Ukraine (since 2022)}
			German weapons hard to repair; broken for long time before they can be used again \mysource{\href{https://www.tagesschau.de/multimedia/video/video-1454812.html}{tagesschau.de}}
		\end{example}}
	\end{fancycolumns}
\end{frame}

\subsection{DRY Principle}
\begin{frame}{\insertsubsection{} \mytitlesource{\thepragmaticprogrammer}}
	\begin{fancycolumns}[widths={35},animation=none]
		\pic[width=\linewidth,trim=0 0 0 0,clip]{books/the-pragmatic-programmer}
		\nextcolumn
		\begin{note}{DRY --- Don't Repeat Yourself}
			\mycite{Every piece of knowledge must have a single, unambiguous, authoritative, representation within a system.}
		\end{note}
		\uncover<2->{\begin{definition}{\insertsubsection}
			\begin{itemize}
				\item knowledge documented in specifications, code, and tests
				\item knowledge not stable, changes frequently
				\item duplicate representations are altered inconsistently
				\item means: extract method refactoring
			\end{itemize}
		\end{definition}}
		\uncover<3->{\begin{example}{Reasons for Duplication}
			\begin{itemize}
				\item imposed duplication: devs feel that they have no other choice
				\item inadvertent duplication: devs do not realize that they duplicate
				\item impatient duplication: duplication seems easier
				\item interdeveloper duplication: multiple devs duplicate
			\end{itemize}
		\end{example}}
	\end{fancycolumns}
\end{frame}

% TODO add principles
% SOLID: https://en.wikipedia.org/wiki/SOLID
% RERO: https://en.wikipedia.org/wiki/Release_early,_release_often
%    "Release early. Release often. And listen to your customers". This philosophy was popularized by Eric S. Raymond in his 1997 essay The Cathedral and the Bazaar,
% more here: https://en.wikipedia.org/wiki/List_of_software_development_philosophies

% bus factor: https://en.wikipedia.org/wiki/Bus_factor

\subsection{Simplicity over Performance}
\begin{frame}{\insertsubsection}
	\slideSimplicityOverPerformance
\end{frame}


\lessonslearned{
	\item KISS --- Keep It Simple, Stupid!
	\item DRY --- Don't Repeat Yourself
	\item RERO --- Release Early, Release Often
	\item Next: How to maintain software?
}{
	\item \thepragmaticprogrammer
}{
	Quiz in \StudIP

	~
	
	\centering\fancyqr{color=black,height=30mm}{https://studip.tu-braunschweig.de/dispatch.php/course/courseware/courseware/18333?cid=635f5186b364a369979d0f45ab2ace1d\#/structural\_element/421045}
}

\faq{
	\item What are reasons to change existing software?
	\item What are alternative to software changes?
	\item What activities are involved when changing software?
	\item What is the law of \ldots continuing change, increasing complexity, conservation of familiarity, continuing growth, decling quality, feedback system?
	\item What follows from Lehman's laws?
	\item What are (counter) examples for Lehman's laws?
}{
	\item Why does the readibility of code matter?
	\item What is Eagleson's law?
	\item Why are go-to's problematic?
	\item What is spaghetti or lasagna code?
	\item What is the boy scouts rule and what are its implications for software engineering?
	\item How does a child-rearing philosophy help with software development?
	\item What is the motivation for refactorings?
	\item What are (examples for) refactorings and code smells?
	\item What is a mysterious name, code clone, duplicated code, or long method?
	\item What is a rename, extract method, or pull-up method refactoring?
}{
	\item Why can it be productive to remove code?
	\item What is the KISS, DRY, RERO principle?
	\item Which refactorings or smells are related to those principles and why?
	\item Which principle is violated in a given example code?
	\item What is better correct or clean or consise or fast code?
}

\input{../se1/template/footer}

% TODO L03 Maintenance

\ifuniversity{tubs}{\date{April 15, 2025}}

\author{Thomas Thüm}
\lecture{Maintenance}{maintenance}

\section{Kinds of Maintenance}
\subsection{Wisdom on Maintenance}
\begin{frame}{\insertsubsection} % slided copied from se1-04a. reuse?
	\begin{fancycolumns}
		\pic[width=\linewidth,trim=0 360 0 50,clip]{people/karolina-szczur}
		\vspace{-7mm}
		
		\begin{note}{Karolina Szczur \mysource{\href{https://twitter.com/codewisdom/status/1141686386103332864}{twitter.com}}}
			\mycite{Writing software as if we are the only person that ever has to comprehend it is one of the biggest mistakes and false assumptions that can be made.}
		\end{note}
		\nextcolumn
		\pic[width=\linewidth,trim=450 235 450 120,clip]{changes/psychopath}
		\vspace{-7mm}
		
		\begin{note}{John F. Woods \mysource{\href{https://groups.google.com/g/comp.lang.c++/c/rYCO5yn4lXw/m/oITtSkZOtoUJ}{google.com}}}
			\mycite{Always code as if the guy who ends up maintaining your code will be a violent psychopath who knows where you live.}
		\end{note}
	\end{fancycolumns}
\end{frame}

\subsection{Software Maintenance}
\begin{frame}{\insertsubsection\ \mytitlesource{\ludewiglichter}} % slided copied from se1-04a. reuse?
	\begin{fancycolumns}
		\begin{note}{Motivation}
			\begin{itemize}
				\item for software: no compensation of deterioration, repair, spare parts
				\item corrections (especially shortly after first delivery)
				\item modification and reconstruction
			\end{itemize}
		\end{note}
		\nextcolumn
		\begin{definition}{Operation and Maintenance Phase \mysource{\ieeesixten}}
			\mycite{The period of time in the software life cycle during which a software product is employed in its operational environment, monitored for satisfactory performance, and modified as necessary to correct problems or to respond to changing requirements.}
		\end{definition}
		\begin{definition}{Maintenance \mysource{\ieeesixten}}
			\mycite{The process of modifying a software system or component after delivery to correct faults, improve performance or other attributes, or adapt to a changed environment.}
		\end{definition}
	\end{fancycolumns}
\end{frame}
% TODO Operations

\subsection{Adaptive Maintenance}
\begin{frame}{\insertsubsection}
	\begin{fancycolumns}
		\begin{definition}{\insertsubsection\mysource{\lientzswanson}}
			\mycite{Software maintenance performed to make a computer program usable in a changed environment.} \hfill \deutsch{adaptive Wartung}
		\end{definition}
		\begin{example}{}
			desktop application for a new version of an operating system (e.g., from Windows 8.1 to 10)
		\end{example}
		\nextcolumn
	\end{fancycolumns}
\end{frame}

\subsection{Corrective Maintenance}
\begin{frame}{\insertsubsection}
	\begin{fancycolumns}
		\begin{definition}{\insertsubsection\mysource{\lientzswanson}}
			\mycite{Maintenance performed to correct faults in software.} \hfill \deutsch{korrektive Wartung}
		\end{definition}
		\begin{example}{}
			Windows calculator showing wrong formulas
		\end{example}
		\nextcolumn
	\end{fancycolumns}
\end{frame}

\subsection{Perfective Maintenance}
\begin{frame}{\insertsubsection}
	\begin{fancycolumns}
		\begin{definition}{\insertsubsection\mysource{\lientzswanson}}
			\mycite{Software maintenance performed to improve the performance, maintainability, or other attributes of a computer program.} \hfill \deutsch{perfektive Wartung}
		\end{definition}
		\begin{example}{}
			better handling of very large files in a text editor
		\end{example}
		\nextcolumn
	\end{fancycolumns}
\end{frame}
% TODO it is annoying that perfective maintenance has an overlap with preventive maintenance, as better maintainability typically also results in fewer problems

% TODO software aging (see Parnas-SoftwareAging, GodfreyGerman-LehmansLaws)

\subsection{Y2K and 2038}
\widexkcdframe{2038} % y2k bug vs 2038 bug

\begin{frame}{Y2K Problem \mytitlesource{\href{https://en.wikipedia.org/wiki/Year_2000_problem}{wikipedia.org}}}
	\begin{fancycolumns}[b]
		\begin{definition}{Y2K Problem}
			\begin{itemize}
				\item years were often stored by only two digits
				\item high awareness in media
				\item about 500 billion dollars spent
				\item about 90\% before 2000
				\item most major problems could be prevented
			\end{itemize}
		\end{definition}
		\centering\picDark[width=.8\linewidth]{failures/y2k-javascript}
	\nextcolumn
		\vspace{-15mm}
		
		\centering\pic[width=.5\linewidth]{failures/y2k}
		\begin{example}{Selected Y2K Failures}
			\begin{itemize}
				\item 1999-01-01: computers for temporary passports in Sweden crashed
				\item 1999-12-28: credit and debit card transactions rejected because they could not be completed within four working days
				%\item 2000-01-01: date 2036-02-06 shown in nuclear power plant in Fukushima
				\item 2000-01-01: German bank accidentially transferred 6 million euro issued 1899-12-30
				\item 2000-01-03: french school showing wrong date
				\item 2020-01-01: New York City parking meters refuse credit cards
			\end{itemize}
		\end{example}
	\end{fancycolumns}
\end{frame}

\begin{frame}{Y2038 Problem \mytitlesource{\href{https://en.wikipedia.org/wiki/Year_2038_problem}{wikipedia.org}}}
	\begin{fancycolumns}
		\begin{definition}{Y2038 Problem}
			\begin{itemize}
				\item 2038-01-19 at 03:14:07 potential overflow
				\item systems affected storing Unix time as signed 32-bit integer
				\item similarly: 2106-02-07 for unsigned integers
				\item solution: use signed 64-bit integer
				\item 292 billion years to next overflow
			\end{itemize}
		\end{definition}
		\centering\picDark[width=\linewidth]{failures/y2038}
	\nextcolumn
		\begin{example}{Y2038 Failures}
			2006-05-13: AOL server software crashed
			\begin{itemize}
				\item it used 1 billion seconds extra to state that database transactions do not timeout
				\item \emph{temporary} solution: set the timeout to a lower value in configuration file
			\end{itemize}
			embedded systems without over-the-air updates are likely to be affected (e.g., ECUs in most automotive systems)
		\end{example}
	\end{fancycolumns}
\end{frame}

\subsection{Preventive Maintenance}
\begin{frame}{\insertsubsection}
	\begin{fancycolumns}
		\begin{definition}{\insertsubsection\mysource{\lientzswanson}}
			\mycite{Maintenance performed for the purpose of preventing problems before they occur.} \hfill \deutsch{präventive Wartung}
		\end{definition}
		\begin{example}{Y2K Problem}
			was very expensive, still not over
		\end{example}
		\begin{example}{Y2038 Problem}
			first failures already happen, will be expensive
		\end{example}
	\nextcolumn
		\begin{example}{Leap Years in Gregorian Calendar}
			\begin{itemize}
				\item solar year is 365.24 days
				\item every 4 years, but not every 100 years, but again every 400 years
				\item 2000 was a leap year
				\item one day extra: February 29, 2028
				\item common problem when durations are computed
				\item example: meat sold beginning of March
			\end{itemize}
		\end{example}
		\begin{example}{Leap Seconds}
			\begin{itemize}
				\item solar day longer than a day
				\item last leap second on 2016-12-31
				\item more details in SE1
			\end{itemize}
			\centering\pic[width=.5\linewidth,trim=0 0 0 0,clip]{changes/leap-second}
		\end{example}
	\end{fancycolumns}
\end{frame}

\subsection{Kinds of Maintenance}
\begin{frame}{\insertsubsection\ \mytitlesource{\ludewiglichter}}
	\begin{fancycolumns}
		\nextcolumn
	\end{fancycolumns}
\end{frame}

\xkcdframe{1172} % No more CPU heating as breaking change

\subsection{Evolution and Maintenance}
\begin{frame}{\insertsubsection\ \mytitlesource{\ludewiglichter}}
	\slideEvolutionAndMaintenance
\end{frame}
% Changelog (honest?)
% vgl. Release Management, Semantic Versioning
% Problems: time pressure, results in new failures, unpopular task

% TODO 22.4.2 Die Behandlung von Problemmeldungen \ludewiglichter

% TODO software erosion/rot/decay/entropy:  https://en.wikipedia.org/wiki/Software_rot

% TODO software estimation (bewertung)
% 	motivation, metrics for OOP
% 14 Metriken und Bewertungen \ludewiglichter

% New Content

% TODO APIs: every observable behavior will eventually be used by users

% TODO new example of Tesla: https://www.heise.de/news/Tesla-Autos-duerfen-nicht-mehr-furzen-Rueckruf-in-den-USA-6441936.html


\lessonslearned{
	\item Maintenance (Phase)
	\item Kinds: adaptive, corrective, perfective, preventive
	\item Examples for maintenance
	\item Maintenance vs evolution
	\item Next: What are common reengineering task?
}{
	\item \ludewiglichter\mychapter{22 \deutsch{Software-Wartung}}
}{
	\begin{enumerate}
		\item Form groups of 2--3 students for Think-Pair-Share
		\item Think (2 min): find examples for two maintenance categories
		\item Pair (5 min): explain example to your group and let colleagues guess its category
		\item Share: what examples were hard to classify?
	\end{enumerate}
}

\section{Reengineering}
\subsection{Forward Engineering}

\subsection{Reverse Engineering}

\subsection{Refactoring}

\subsection{Reengineering}

\subsection{Reengineering Tasks}
\begin{frame}[b]{\insertsubsection\ \mytitlesource{\ludewiglichter}}
	\vspace{-10mm}
	\begin{fancycolumns}[b]
		\begin{definition}{Reverse Engineering \mysource{Chikofsky und Cross}}
			\mycite{Reverse engineering is the process of analyzing a system to identify the system’s components and their interrelationships and create representations of the system in another form or at a higher level of abstraction.}
		\end{definition}
		\begin{example}{}
			%			updating UML diagrams from source code
			exploring what are the classes of a calculator and what is their purpose
		\end{example}
		\begin{definition}{Forward Engineering \mysource{Chikofsky und Cross}}
			\mycite{Forward engineering is the traditional process of moving from high-level abstractions and logical, implementation-independent designs to the physical implementation of a system.}
		\end{definition}
		\begin{example}{}
			%			see Software Engineering I
			implement a new feature in a calculator
		\end{example}
		\nextcolumn
		\begin{definition}{Restructuring \mysource{Chikofsky und Cross}}
			\mycite{Restructuring is a transformation from one form of representation to another at the same relative level of abstraction. The new representation is meant to preserve the semantics and external behavior of the original.}
		\end{definition}
		\begin{example}{}
			refactorings as introduced in lecture on evolution
		\end{example}
		\begin{definition}{Reengineering \mysource{Chikofsky und Cross}}
			\mycite{Reengineering is the examination and alteration of a subject system to reconstitute it in a new form and the subsequent implementation of the new form.}
		\end{definition}
		\begin{example}{}
			combination of reverse engineering, refactoring, and forward engineering
		\end{example}
	\end{fancycolumns}
\end{frame}
% TODO add links to original literature
% TODO Refactoring is the process of changing a software system in such a way that it does not alter the external behavior of the code yet improves its internal structure. It is a disciplined way to clean up code that minimizes the chances of introducing bugs. In essence when you refactor you are improving the design of the code after it has been written. Martin Fowler

\subsection{Wisdom on Reengineering}
\begin{frame}{\insertsubsection}
	\begin{fancycolumns}
		\pic[width=\linewidth,trim=50 0 0 0,clip]{people/kyle-simpson}
		\vspace{-7mm}
		
		\begin{note}{Kyle Simpson \mysource{\href{https://twitter.com/codewisdom/status/857988273674883072}{twitter.com}}}
			\mycite{There's nothing more permanent than a temporary hack.} \deutsch{Sprichwort: Nichts ist so beständig wie das Provisorium.} % TODO Quelle für deutsches Sprichwort suchen
		\end{note}
		\nextcolumn
		\vspace{11mm}
		\centering\pic[width=.5\linewidth,trim=0 0 0 0,clip]{people/jeff-sickel}
		\vspace{-3mm}
		
		\begin{note}{Jeff Sickel \mysource{\href{https://twitter.com/codewisdom/status/710046205317873664}{twitter.com}}}
			\mycite{Deleted code is debugged code.}
		\end{note}
	\end{fancycolumns}
\end{frame}


\lessonslearned{
	\item Reengineering =
	\item reverse engineering +
	\item refactoring +
	\item forward engineering
	\item Next: How to migrate legacy software?
}{
	\item \ludewiglichter\mychapter{23 \deutsch{Reengineering}}
}{
	\begin{enumerate}
		\item Form groups of 2--3 students for Think-Pair
		\item Think (2 min): find examples for two reengineering categories
		\item Pair (5 min): explain example to your group and let colleagues guess its category 
	\end{enumerate}
}

\section{Migration of Legacy Systems}
\subsection{Legacy Software}
\begin{frame}{\insertsubsection\ \deutsch{Altsysteme}}
	\begin{fancycolumns}
		\begin{definition}{Legacy Software \mysource{\ludewiglichter}}
			\mycite{A large software system that we don’t know how to cope with but that is vital to our organization.}
		\end{definition}
		\begin{exampletight}{Winamp 5.623 on 4k Display in Windows 10}
			\pic[width=\linewidth,trim=0 0 0 0,clip]{changes/winamp}
		\end{exampletight}
		\nextcolumn
		\begin{note}{Typical Properties \mysource{\ludewiglichter}}
			\begin{itemize}
				\item larger than 100k LOC
				\item older than 10 years
				\item original developers and architects not available anymore
				\item outdated programming languages and development concepts
				\item business-critical
				\item outdated or missing documentation
				\item based on outdated hardware and system software
				\item subject to numerous iterations of corrective and adaptive maintenance
				\item high costs for maintenance
			\end{itemize}
		\end{note}
	\end{fancycolumns}
\end{frame}

\xkcdframe{2221} % emulated software, long-living

\subsection{Software Migration}
\begin{frame}{\insertsubsection\ \mytitlesource{\ludewiglichter}}
	\begin{fancycolumns}
		\begin{definition}{Software Migration}
			\begin{itemize}
				\item opposed to data and hardware migration
				\item renewing or replacing legacy software
				\item new software needs to be downward compatible
				\item new software meets current functional and non-functional requirements
				\item data often needs to be migrated
				\item outage as short as possible
			\end{itemize}
		\end{definition}
		\nextcolumn
		\begin{note}{Migration Strategies}
			\begin{itemize}
				\item big bang migration: whole legacy software is migrated in a single step
				\item incremental migration: stepwise renewal and replacement
				\item wrapping: building a new software around a stable version of the legacy software
				\item redevelopment: functionality implemented again and replaces legacy software
				\item combinations feasible (except for incremental and big bang)
			\end{itemize}
		\end{note}
	\end{fancycolumns}
\end{frame}

\subsection{Big Bang Migration}

\subsection{Migration by Wrapping}

\subsection{Incremental Migration}

\subsection{Redevelopment?}

% TODO reasons/requirements/process for migration

\subsection{End of Service}
\begin{frame}{\insertsubsection}
	\begin{fancycolumns}
		\begin{exampletight}{{End of Life: December 31, 2020}}
			\centering\pic[width=.85\linewidth,trim=0 0 0 0,clip]{changes/flash-eol}
		\end{exampletight}
		\nextcolumn
		\begin{exampletight}{Online Game Fuchstreff Discontinued}
			\pic[width=\linewidth,trim=0 0 0 0,clip]{changes/fuchstreff}
		\end{exampletight}
	\end{fancycolumns}
\end{frame}

% TODO 32bit architectures discontinued in many Linux distributions
% https://www.debugpoint.com/32-bit-linux-distributions/

% TODO release, maintenance, end of life / alternatives to maintenance https://en.wikipedia.org/wiki/Software_maintenance

% TODO end of life: "the perfect race car crosses the finish line in first place and immediately falls into pieces" Ferdinand Porsche, https://www.fahrtraum.at/en/the-best-quotes-and-sayings-from-ferdinand-porsche/


\lessonslearned{
	\item Legacy software: typical properties, examples
	\item Software migration
	\item Migration strategies: big bang migration, incremental migration, wrapping, redevelopment
	\item Next: How to design systems simplifying particular future changes?
}{
	\item \ludewiglichter\mychapter{23 \deutsch{Reengineering}}
}{
	Quiz in \StudIP

	~
	
	\centering\fancyqr{color=black,height=30mm}{https://studip.tu-braunschweig.de/dispatch.php/course/courseware/courseware/18333?cid=635f5186b364a369979d0f45ab2ace1d\#/structural\_element/421046}
}

\faq{
	\item What is the operations and maintenance phase?
	\item What is software maintenance and how is it different from evolution?
	\item What are the different kinds of maintenance?
	\item What are (examples for) corrective, perfective, adaptive, preventive maintenance?
	\item Given an example maintenance activity, what kind(s) of maintenance is it?
	\item Give examples that belong to several kinds of maintenance!
}{
	\item What is forward engineering, reverse engineering, restructuring, reengineering?
	\item Give examples for the above mentioned reengineering task!
	\item Given an example, classify what kind of reengineering it is!
	\item Explain the difference between X and Y by means of examples!
}{
	\item What is legacy software?
	\item What are typical properties of legacy software?
	\item Given an example description, argue whether it is legacy software or not!
	\item What is software migration?
	\item What are migration strategies?
	\item Given an example migration, identify which migration strategy was applied!
	\item Give an example where migration X is better suited than Y!
	\item What is the end of service/life?
	\item What are reasons to discontinue software projects?
}

\input{../se1/template/footer}

% TODO L04 Design Patterns

\ifuniversity{tubs}{\date{April 28, 2025}}

\author{Thomas Thüm}
\lecture{Design Patterns}{designpatterns}

\newcommand{\userstoriesA}{
	\item User Story 1: Implement subtraction and division as two new operations.
	\item User Story 2: Fix that the error label does not disappear after wrong followed by correct input.
	\item User Story 3: Replace first occurrence of existing value (e.g., 3) in the formula by a new value (e.g., 5).
}
\newcommand{\userstoriesB}{
	\item User Story 4: Enable or disable debug outputs with an enum constant (no debug output, only warnings, all debug output).
	\item User Story 5: Layout all buttons in the GUI horizontally instead of vertically.
	\item User Story 6: Implement a dark mode of the calculator by means of a feature flag.
}
\newcommand{\userstoriesC}{
	\item User Story 7: Display brackets only when needed.
	\item User Story 8: If there is no error message, show seconds elapsed since last input.
	\item User Story 9: Toggle in GUI to let users choose between brackets and evaluation of subterms.
}

\section{Structural Patterns}
\subsection{Gang of Four}
\begin{frame}{\insertsubsection\ \mytitlesource{\gof}} % TODO copied from SE1. reuse?
	\begin{fancycolumns}[animation=none]
		\centering\pic[width=.66\linewidth]{books/gof}
		\nextcolumn
		\centering\pic[height=27mm,trim=185 0 184 0,clip]{people/erich-gamma}
		\pic[height=27mm]{people/richard-helm}
		
		\pic[height=27mm,trim=0 0 5 0,clip]{people/ralph-johnson}
		\pic[height=27mm,trim=0 0 160 0,clip]{people/john-vlissides}
	\end{fancycolumns}
\end{frame}

\subsection{Design Patterns}
\begin{frame}{\insertsubsection\ \mytitlesource{\gofen}} % TODO copied from SE1. reuse?
	\begin{fancycolumns}
		\begin{note}{Motivation}
			\mycite{Designing object-oriented software is hard, and designing \emph{reusable} object-oriented software is even harder. [...]  It takes a long time for novices to learn what good object-oriented design is all about. Experienced designers evidently know something inexperienced ones don't. What is it?}
		\end{note}
		\nextcolumn
		\begin{definition}{Design Patterns \deutsch{Entwurfsmuster}}
			\setlength\tabcolsep{1mm}
			\begin{tabularx}{\textwidth}{rX}				
				pattern name & for communication and high-level abstraction\\
				problem & when to apply the pattern\\
				solution & template on how to arrange classes and objects\\
				consequences & trade-offs of applying the pattern
			\end{tabularx}
		\end{definition}
		\begin{note}{Kinds of Patterns}
			\mycite{Creational patterns \deutsch{Erzeugungsmuster} concern the process of object creation. Structural patterns \deutsch{Strukturmuster} deal with the composition of classes or objects. Behavioral patterns \deutsch{Verhaltensmuster} characterize the ways in which classes or objects interact and distribute responsibility.}
		\end{note}
	\end{fancycolumns}
\end{frame}

\subsection{Structural Design Patterns}
\begin{frame}{\insertsubsection{} \mytitlesource{\gof}} % TODO copied from SE1. reuse?
	\centering\slideMindmapDesignPatterns{}{notTaughtDesignPatterns}{notTaughtDesignPatterns}{}{notTaughtDesignPatterns}{}{visible on={<2->}}{}{visible on={<-0|handout:0>}}
\end{frame}

\subsection{Object Adapter Pattern}
\begin{frame}{\insertsubsection} % TODO copied from SE1. reuse?
	\begin{fancycolumns}
		\begin{definition}{Object Adapter \mysource{\gofen}}
			\setlength\tabcolsep{1mm}
			\begin{tabularx}{\textwidth}{rX}				
				intent & \mycite{Convert the interface of a class into another interface clients expect. Adapter lets classes work together that couldn't otherwise because of incompatible interfaces.}\\
				aka. & wrapper\\
				motivation & enable reuse of classes even though incompatible interfaces cannot be made compatible\\
				idea & create a new class with a compatible interface that forwards all requests
			\end{tabularx}
		\end{definition}
		\nextcolumn
		\objectadapter{width=\linewidth}
	\end{fancycolumns}
\end{frame}

\begin{frame}[t]{\insertsubsection{} \mytitlesource{\umlrefman\mypages{91--92,198--199,387--388}; \umluserguide\mypages{318--323}; \umlspec\mypages{217--218}}} % TODO copied from SE1. reuse?
	\alt<2->{\profcalculatorcollaboration{width=\linewidth,page=2}}{\profcalculatorcollaboration{width=\linewidth,page=6}}
	
	\vspace{-75mm}
	\begin{fancycolumns}[widths={65},animation=none]
		\nextcolumn
		\mynotetight{Object Adapter Pattern}{\objectadapter{width=\linewidth}}
	\end{fancycolumns}
\end{frame}


\lessonslearned{
	\item Design patterns by the Gang of Four
	\item Object adapter, composite pattern, decorator pattern
	\item Next: How to solve problems with the creation of objects?
}{
	\item \gof\mychapter{4}
}{
	\begin{itemize}
		\item Download the FancyCalculator from \StudIP
		\item Install Java and IDE (Eclipse recommended)
		\userstoriesA
		\item Hint: Apply knowledge on coding conventions, smells, refactorings, and design patterns where applicable.
	\end{itemize}
}

\section{Creational Patterns}
\subsection{Creational Design Patterns}
\begin{frame}{\insertsubsection{} \mytitlesource{\gof}} % TODO copied from SE1. reuse?
	\centering\slideMindmapDesignPatterns{}{notTaughtDesignPatterns}{notTaughtDesignPatterns}{}{notTaughtDesignPatterns}{}{visible on={<-0|handout:0>}}{}{visible on={<-0|handout:0>}}
\end{frame}

\subsection{Singleton Pattern}
\begin{frame}{\insertsubsection} % TODO copied from SE1. reuse?
	\begin{fancycolumns}
		\begin{definition}{Singleton \mysource{\gofen}}
			\setlength\tabcolsep{1mm}
			\begin{tabularx}{\textwidth}{rX}				
				intent & \mycite{Ensure a class [has only] one instance, and provide a global point of access to it.}\\
				% aka. & \\
				motivation & avoid the uncontrolled creation of multiple instances\\
				idea & a private constructor is called on class initialization and a public static method is used for access to that single instance
			\end{tabularx}
		\end{definition}
		\nextcolumn
		\centering\only<2|handout:0>{\singleton{width=.1\linewidth}}
		\only<3->{\singleton{width=.5\linewidth}}
	\end{fancycolumns}
\end{frame}

\begin{frame}{\insertsubsection: Example} % TODO copied from SE1. reuse?
	\begin{fancycolumns}
		\begin{definitiontight}{Pattern}
			\centering\singleton{width=.5\linewidth}		
		\end{definitiontight}
		\nextcolumn
		\begin{exampletight}{Example Application}
			\centering\singletonexample{width=.5\linewidth}
		\end{exampletight}
	\end{fancycolumns}
\end{frame}

\begin{frame}[t]{\insertsubsection{} \mytitlesource{\umlrefman\mypages{91--92,198--199,387--388}; \umluserguide\mypages{318--323}; \umlspec\mypages{217--218}}} % TODO copied from SE1. reuse?
	\alt<2->{\profcalculatorcollaboration{width=\linewidth,page=4}}{\profcalculatorcollaboration{width=\linewidth,page=6}}
	
	\vspace{-70mm}
	\begin{fancycolumns}[widths={70},animation=none]
		\nextcolumn
		\mynotetight{Singleton Pattern}{\singleton{width=\linewidth}}
	\end{fancycolumns}
\end{frame}


\lessonslearned{
	\item Singleton pattern
	\item Abstract factory pattern
	\item Next: How to solve problems with the behavior of objects?
}{
	\item \gof\mychapter{3}
}{
	Continue implementing the FancyCalculator:
	\begin{itemize}
		{\color{gray} \userstoriesA }
		\userstoriesB
	\end{itemize}
}

\section{Behavioral Patterns}
\subsection{Behavioral Design Patterns}
\begin{frame}{\insertsubsection{} \mytitlesource{\gof}} % TODO copied from SE1. reuse?
	\centering\slideMindmapDesignPatterns{}{notTaughtDesignPatterns}{notTaughtDesignPatterns}{}{notTaughtDesignPatterns}{}{}{}{visible on={<2->}}
\end{frame}

\subsection{Observer Pattern}
\begin{frame}{\insertsubsection} % TODO copied from SE1. reuse?
	\begin{fancycolumns}
		\begin{definition}{Observer \mysource{\gofen}}
			\setlength\tabcolsep{1mm}
			\begin{tabularx}{\textwidth}{rX}				
				intent & \mycite{Define a one-to-many dependency between objects so that when one object changes state, all its dependents are notified and updated automatically.}\\
				aka. & dependents, publish-subscribe\\
				motivation & avoid coupling of classes (i.e., explicit references such as imports)\\
				example & no coupling between model and view in model-view-controller architectures\\
				idea & data object (model) has a list of observers (views) that are notified about changes
			\end{tabularx}
		\end{definition}
		\nextcolumn
		\observer{width=\linewidth}
	\end{fancycolumns}
\end{frame}

%\subsection{Overview on Design Patterns}
%\begin{frame}{\insertsubsection{} \mytitlesource{\gof}}
%	\only<-4|handout:0>{\tikzset{notTaughtDesignPatterns/.style={}}}%
%	\slideMindmapDesignPatterns{}{notTaughtDesignPatterns}{notTaughtDesignPatterns}{}{notTaughtDesignPatterns}{}{visible on={<1,4,5->}}{visible on={<2,4->}}{visible on={<3->}}
%\end{frame}


\lessonslearned{
	\item Observer pattern
	\item Visitor pattern
	\item Next lecture: What do software engineers need to know about compilers?
}{
	\item \sommerville\mychapter{5}
}{
	Continue implementing the FancyCalculator:\\see next slide
}

\begin{frame}
	\begin{fancycolumns}[widths={65}]
		\begin{example}{FancyCalculator - 3rd Exercise on Design Patterns}
			\begin{itemize}
				\userstoriesA
				\userstoriesB
				\userstoriesC
			\end{itemize}
		\end{example}
	\nextcolumn
		\begin{definition}{Design Patterns}
			\begin{itemize}
				\item Structural patterns:\\object adapter, composite, decorator
				\item Creational patterns:\\singleton, abstract factory
				\item Behavioral patterns:\\observer
			\end{itemize}
		\end{definition}
		\begin{note}{Hints}
			\begin{itemize}
				\item User Story 1/2: fix DRY first
				\item User Story 3: try composite
				\item User Story 4: try singleton
			\end{itemize}
		\end{note}
	\end{fancycolumns}
\end{frame}

%\faq{
%	\item
%}{
%	\item
%}{
%	\item
%}

\input{../se1/template/footer}
