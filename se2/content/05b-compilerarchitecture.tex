\begin{frame}<3>{Recap: Pipe-and-Filter Architecture}
	\slidePipeAndFilter
\end{frame}

\begin{frame}{Compiler Architecture}
	\begin{fancycolumns}[animation=none]
		\uncover<3->{%
			\begin{definition}{Application of the Chomsky Hierarchy}
				\begin{itemize}
					\item scanning/lexing: regular expression for each token class
					\item parsing: context-free grammars (e.g., a class has fields, methods, and inner classes)
					\item name and type analysis: context-sensitive analysis (e.g., lookup type for identifier)
				\end{itemize}
				\begin{center}
					\begin{tikzpicture}[xscale=.7,yscale=.5]
						\draw[color=foreground, fill=background, very thick](0,1.7) ellipse (3.5 and 2.9);
						\draw[color=foreground, fill=background, very thick](0,1) ellipse (2.6 and 2.1);
						\draw[color=foreground, fill=background, very thick](0,0.5) ellipse (1.8 and 1.5);
						\draw[color=foreground, fill=background, very thick](0,0) ellipse (1.1 and 0.9);
						
						\node[] at (0,0) {regular};
						\node[] at (0,1.3) {context-free};
						\node[] at (0,2.4) {context-sensitive};
						\node[] at (0,3.6) {recursively enumerable};
					\end{tikzpicture}
				\end{center}
			\end{definition}
		}
		\nextcolumn
		\vspace{3mm}
		\only<1|handout:0>{\figcompilerarchbase}%
		\only<2->{\figcompilerarchnoop}%
	\end{fancycolumns}
\end{frame}

\begin{frame}{Compilation and Performance}
	\vspace{-10mm}
	\begin{fancycolumns}
		\vspace{15mm}
		\begin{note}{Goals of Compiler Optimizations}
			\begin{itemize}
				\item fast execution
				\item low memory / energy consumption
				\item small binaries (fast start/download/updates)
				\item desired for both compiler (compile time) and compiled program (run time)
			\end{itemize}
		\end{note}
		
		%~
		
		\begin{definition}{Compile Time vs Run Time}
			run time: when program or software is executed
			
			compile time: during \uncover<3->{(ahead-of-time) }compilation
		\end{definition}
		\nextcolumn
		\hfill\resizebox{60mm}{!}{\figJIT}
		
		\begin{definition}{Just-in-Time Compilation}
			\begin{itemize}
				\item often executed code is compiled at run time
				\item warm-up time: execution is slower when new code is executed (partly also for new data)
			\end{itemize}
		\end{definition}
	\end{fancycolumns}
\end{frame}

\subsection{Compiler Optimizations}
\begin{frame}{\insertsubsection}
	\begin{fancycolumns}[widths={64},animation=none]
		%\only<1|handout:0>{\figcompilerarchbase}%
		\only<1|handout:0>{\figcompilerarchnoop}%
		\only<2->{\figcompilerarchfull}%
		\nextcolumn
		\uncover<3->{
			\begin{definition}{Kinds of Optimizations}
				\begin{itemize}
					
					\item machine-dependent optimizations: exploit properties of a particular machine
					\item machine-independent optimizations: applicable to several machines
					\item local optimizations: e.g., switch order of two statements
					\item intra-procedural optimizations: affect only one method
					\item inter-procedural optimizations: affect several methods or require global knowledge
				\end{itemize}
			\end{definition}
		}
	\end{fancycolumns}
\end{frame}

\begin{frame}{The Power of Intermediate Languages}
	\begin{fancycolumns}[widths={40}]
		\begin{note}{Motivation}
			\begin{itemize}
				\item hard to establish \emph{new programming languages} as compilers needed for many architectures
				\item hard to establish \emph{new architectures} as as compilers needed for many programming languages
				\item high-level programming languages are complex to analyze and translated
				\item idea: reuse effort across languages and architectures
			\end{itemize}
		\end{note}
		\nextcolumn
		\small
		\begin{tikzpicture}[every node/.style = {align = center, text depth = 0, fill=background}, node distance = 15mm and 2mm]
			\node[align = center, fill=red!10!background, inner sep=4pt, draw = red!30, rounded corners=4pt] (M) {Intermediate Language\\(e.g., Java bytecode)};
			
			\node[above = of M] (H1) {$\ldots$};
			
			\node[left = of H1] (u2) {Source Language $2$\\(e.g., Scala)};
			\node[left = of u2] (u1) {Source Language $1$\\(e.g., Java)};
			\node[right = of H1] (u3) {Source Language $m$\\(e.g., Clojure)};
			
			\node[below = of M] (H2) {$\ldots$};
			\node[left = of H2] (l2) {Target Language $2$\\(e.g., Linux)};
			\node[left = of l2] (l1) {Target Language $1$\\(e.g., Windows)};
			\node[right = of H2] (l3) {Target Language $n$\\(e.g., MacOS)};
			
			\path[-latex, in = 90, out = -90, looseness = 0.5] 
			(u1) edge[looseness = 0.3] (M.north)
			(u2) edge (M)
			(u3) edge (M.north)
			;
			
			\path[-latex, in = 90, out = -90, looseness = 0.5] 
			(M) edge [looseness = 0.3] (l1)
			(M) edge (l2)
			(M) edge (l3)
			;
		\end{tikzpicture}
	\end{fancycolumns}
\end{frame}

