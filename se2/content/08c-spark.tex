\newcommand{\introtoada}{\href{https://learn.adacore.com/courses/intro-to-ada/}{Introduction to Ada}}

\newcommand{\introtospark}{\href{https://learn.adacore.com/courses/intro-to-spark/}{Introduction to SPARK}}

\subsection{Ada}
\begin{frame}{\insertsubsection}
	\begin{fancycolumns}
		\begin{definition}{Ada \mysource{\introtoada}}
			\begin{itemize}
				\item United States Department of Defense suffered from an explosion of programming languages and dialects
				\item call for proposals lead to Ada
				\item 1983: first Ada standard
				\item revisions: 1995, 2005, 2012
				\item most similiarities with C++ and Rust
				\item used for embedded real-time systems
			\end{itemize}
		\end{definition}
		\nextcolumn
		\begin{example}{Example Domains of Ada \mysource{\introtoada}}
			\begin{itemize}
				\item aerospace
				\item defense
				\item civil aviation
				\item rail
				\item video games
				\item real-time audio
				\item kernel modules
			\end{itemize}
		\end{example}
	\end{fancycolumns}
\end{frame}

\subsection{SPARK Language}
\begin{frame}{\insertsubsection}
	\begin{fancycolumns}[animation=none,widths={60}]
		\begin{definition}{SPARK Language \mysource{\introtospark}}
			\begin{itemize}
				\item programming language targeted at functional specification and static verification
				\item based on subset of the Ada language
				\item adding language constructs to specify contracts
				\item removing language constructs from Ada that would be hard to specify with contracts:
					\begin{itemize}
						\item complex control flow (e.g., goto)
						\item expressions with side-effects
						\item aliasing (multiple variables pointing to same object)
					\end{itemize}
			\end{itemize}
		\end{definition}
	\nextcolumn
		\vspace*{-5mm}
		\picDark[width=\linewidth]{designbycontract/spark-ada}
		\vspace*{-4mm}
		\uncover<2->{\begin{exampletight}{The SPARK Mode}
			\centering\makebox{\usebox{\sparkmode}}
		\end{exampletight}}
	\end{fancycolumns}
\end{frame}

\subsection{SPARK Tools}
\begin{frame}{\insertsubsection}
	\begin{fancycolumns}
		\begin{definition}{SPARK Tools \mysource{\introtospark}}
			\begin{itemize}
				\item development and verification tools
				\item GNAT Static Analysis Suite: abstract interpretation
				\item flow analysis: checks initializations of variables, unused assignments, unmodified variables.
				\item proof: checks for absence of runtime errors, conformance of program and specifications
			\end{itemize}
		\end{definition}
	\nextcolumn
		\begin{example}{SPARK Online Course \mysource{\introtospark}}
			\begin{itemize}
				\item Run: compile the code with assertions enabled and run the executable produced.
				\item Examine: perform the flow analysis stage of formal verification
				\item Prove: perform the proof stage of formal verification (which includes flow analysis)
			\end{itemize}
		\end{example}
	\end{fancycolumns}
\end{frame}

\subsection{Design by Contract in SPARK}
\begin{frame}{\insertsubsection}
	\begin{fancycolumns}[widths={33}]
		\begin{exampletight}{A Trivial Example in SPARK}
			\centering\makebox{\usebox{\sparkincrement}}
		\end{exampletight}
		\nextcolumn
		\begin{definition}{\insertsubsection{} \mysource{\introtospark}}
			\begin{itemize}
				\item principle: the SPARK language extends Ada by a behavioral interface specification language
				\item \emph{\lstinline|Global|}: read + written global variables
				\item \emph{\lstinline|Depend|}: value of X depends only on previous value of X
				\item \emph{\lstinline|Pre|}: precondition that needs to hold when method is called
				\item \emph{\lstinline|Post|}: postcondition that needs to hold when method returns
			\end{itemize}
		\end{definition}
	\end{fancycolumns}
\end{frame}

% https://learn.adacore.com/courses/intro-to-spark/chapters/03_Proof_Of_Program_Integrity.html#contracts

%\begin{frame}{\insertsubsection}
%	\begin{fancycolumns}[t]
	%		\begin{exampletight}{Logical Error}
		%			\centering\makebox{\usebox{\edgelogicalerror}}
		%		\end{exampletight}
	%		\uncover<2->{\begin{note}{No explicit error during compilation or execution}
			%				error can be identified by inspecting the program's output or by means of assertions
			%		\end{note}}
	%		\nextcolumn
	%		\uncover<3->{\begin{exampletight}{Fixed Program}
			%				\centering\makebox{\usebox{\edgelogicalerrorfix}}
			%		\end{exampletight}}
	%		\uncover<4->{\begin{definition}{Explanation}
			%				\begin{itemize}
				%					\item no explicit exception as for runtime errors
				%					\item tests need comparison with expected outcomes
				%				\end{itemize}
			%		\end{definition}}
	%	\end{fancycolumns}
%\end{frame}

