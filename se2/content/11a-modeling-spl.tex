\subsection{Motivation and Recap}
%\begin{frame}{Greenfield Development?\ \deutsch{Auf der grünen Wiese?}}
%	\pic[width=.6\linewidth,trim=0 0 0 0,clip]{may21-ulm}
%\end{frame}

\begin{frame}{Who Produces Only One Product?}
	\pic[width=.6\linewidth,trim=0 0 0 0,clip]{productlines/car-tower}
\end{frame}

% TODO zoo of animals/tools

% TODO 30 printers per year, more examples

\begin{frame}<8>{Recap: Process Models}
	\begin{fancycolumns}
		\begin{exampletight}{Waterfall Model}
			\centering
			\diagramWaterfallModel
		\end{exampletight}
		\nextcolumn
		\begin{exampletight}{Scrum}
			\diagramScrum
		\end{exampletight}
	\end{fancycolumns}
\end{frame}

\subsection{Software Product Lines}
\begin{frame}{\insertsubsection\ \mytitlesource{\fospl}}
	\begin{fancycolumns}
		\begin{definition}{Mass Customization}
			\begin{itemize}
				\item opposed to mass production
				\item opposed to individual customization
				\item idea: produce customized goods as efficient as with mass production
			\end{itemize}
		\end{definition}
		\pic[width=\linewidth,trim=0 50 0 240,clip]{productlines/car-manufacturing}
		\nextcolumn
		\begin{definition}{Software Product Line}
			\begin{itemize}
				\item set of software-intensive systems (aka. products)
				\item sharing a common, managed set of features
				\item satisfying the needs of a domain
				\item developed from a common set of core assets (reuse)
			\end{itemize}
		\end{definition}
		\begin{definition}{Feature}
			\begin{itemize}
				\item domain abstraction
				\item used for communication by stakeholders (e.g., manager, developer, tester, customer, marketing)
				\item specifies differences between products
			\end{itemize}
		\end{definition}
	\end{fancycolumns}
\end{frame}

\xkcdframe{2369} % paper processor 25s

\subsection{Dependencies Between Features}
\begin{frame}{\insertsubsection}
	\begin{fancycolumns}[widths={70},animation=none]
		\begin{exampletight}{Thomas Configuring a Notebook}
			\only<1,3->{\picDark[width=\linewidth]{productlines/configurators/thinkpad-x1-yoga-display}}\only<2|handout:0>{\picDark[width=\linewidth]{productlines/configurators/thinkpad-x1-yoga-display-invalidconf}}
		\end{exampletight}
	\end{fancycolumns}
\end{frame}

\begin{frame}{\insertsubsection}
	\begin{fancycolumns}[widths={70},animation=none]
		\begin{exampletight}{Thomas Still Configuring a Notebook}
			\picDark[width=\linewidth]{productlines/configurators/thinkpad-x1-yoga-office}
		\end{exampletight}
	\end{fancycolumns}
\end{frame}

\begin{frame}{\insertsubsection}
	\begin{fancycolumns}[widths={35},animation=none]
		\nextcolumn
		\begin{exampletight}{Thomas Configuring a Car}
			\pic[width=\linewidth]{productlines/configurators/toyota-aygo-wheels}
		\end{exampletight}
	\end{fancycolumns}
\end{frame}

\begin{frame}{\insertsubsection}
	\begin{fancycolumns}[widths={35},animation=none]
		\nextcolumn
		\begin{exampletight}{Thomas Configuring a Car with a Weird Price}
			\centering\picDark[width=.55\linewidth]{productlines/configurators/toyota-aygo-costs}
		\end{exampletight}
	\end{fancycolumns}
\end{frame}

\begin{frame}{\insertsubsection}
	\begin{fancycolumns}[widths={35},animation=none]
		\nextcolumn
		\begin{exampletight}{Thomas Configuring a Car with 8 Wheels!}
			\picDark[width=\linewidth]{productlines/configurators/toyota-aygo-costs3}
		\end{exampletight}
	\end{fancycolumns}
\end{frame}

\begin{frame}{\insertsubsection}
	\begin{fancycolumns}[widths={35},animation=none]
		\nextcolumn
		\begin{exampletight}{Thomas Configuring a German Car}
			\picDark[width=\linewidth]{productlines/configurators/bmw-series1-confassistant-bluetooth}
		\end{exampletight}
	\end{fancycolumns}
\end{frame}

\subsection{Feature Models and Configurations}
\begin{frame}{Feature Modeling\ \mytitlesource{\fospl}}
	\begin{fancycolumns}
		\begin{definition}{Feature Model}	%TODO: Features haben einzigartige Namen
			\begin{itemize}
				\item hierarchy of features
				\item dependencies between features modeled by tree and cross-tree constraints
				\item \emph{tree constraints}: defined by the hierarchy
				\item \emph{cross-tree constraints}: propositional formulas over features
				\item \emph{concrete features} have an implementation
				\item \emph{abstract features} are used to group other features
			\end{itemize}
		\end{definition}
		\nextcolumn
		\vspace{-12mm}
		\begin{exampletight}{A Database Management Product Line}
			\centering\pic[width=\linewidth]{productlines/featuremodels/db-constraint}
		\end{exampletight}
		\begin{definition}{Tree Constraints}
			\begin{itemize}
				\item each feature requires its parent
				\item an \emph{optional feature} has no further constraints % TODO rephrase
				\item a \emph{mandatory feature} is required by its parent
				\item \emph{or group}: at least one feature must be selected when the parent is selected
				\item \emph{alternative group}: exactly one feature must be selected when the parent is selected
			\end{itemize}
		\end{definition}
	\end{fancycolumns}
\end{frame}
% TODO translate Basis into Base

% TODO add other example showing groups/mandator/optional at other positions

\begin{frame}{Enumerating All Configurations of a Feature Model}
	\begin{fancycolumns}
		\begin{exampletight}{}
			\centering\pic[width=.75\textwidth]{productlines/featuremodels/db-constraint}
		\end{exampletight}
		\begin{example}{26 Valid Configurations}
			\footnotesize
			\leftandright{ % TODO transform to new template (deprecated) once nested fancycolumns work
				$\{B,G,W\}$\\
				$\{B,P,W\}$\\
				$\{B,G,P,W\}$\\
				$\{B,D,W\}$\\
				$\{B,G,D,W\}$\\
				$\{B,P,D,W\}$\\
				$\{B,G,P,D,W\}$\\
				$\{B,P,T,W\}$\\
				$\{B,G,P,T,W\}$\\
				$\{B,D,T,W\}$\\
				$\{B,G,D,T,W\}$\\
				$\{B,P,D,T,W\}$\\
				$\{B,G,P,D,T,W\}$
			}{
				$\{B,G,U\}$\\
				$\{B,P,U\}$\\
				$\{B,G,P,U\}$\\
				$\{B,D,U\}$\\
				$\{B,G,D,U\}$\\
				$\{B,P,D,U\}$\\
				$\{B,G,P,D,U\}$\\
				$\{B,P,T,U\}$\\
				$\{B,G,P,T,U\}$\\
				$\{B,D,T,U\}$\\
				$\{B,G,D,T,U\}$\\
				$\{B,P,D,T,U\}$\\
				$\{B,G,P,D,T,U\}$
			}
		\end{example}
	\end{fancycolumns}
\end{frame}

%\subsection{Feature Modeling in Practice}
%\begin{frame}{A Feature Model for the Linux Kernel}
%	%\slideLinuxFeatureModel
%\end{frame}
%
%\begin{frame}{Dependencies Modeled in Excel}
%	\picDark[width=.7\linewidth,trim=10 10 0 10,clip]{productlines/variabilitymodels/constraints-in-excel}
%\end{frame}

