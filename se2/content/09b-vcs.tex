% !TeX spellcheck = en_US

\subsection{git}

\begin{frame}[b]
	\begin{fancycolumns}[animation=none]
		\begin{center}
			\pic[width=.9\linewidth]{people/isaac-wolkerstorfer}
		\end{center}
		\vspace{-7mm}
		
		\begin{note}{Isaac Wolkerstorfer (2014) \mysource{\href{https://twitter.com/agnoster/status/44636629423497217}{Wolkerstorfer 2014}}}
			\mycite{git gets easier once you get the basic idea that branches are homeomorphic endofunctors mapping submanifolds of a Hilbert space.}
		\end{note} \pause
		\nextcolumn
		\begin{note}{Why it's false}
			\begin{itemize}
				\item The git space is discrete, not continuous
				\item There is no closeness metric
				\item Homeomorphic mappings also require continuity
			\end{itemize}
		\end{note} \pause
		\pic[width=\columnwidth]{versioncontrol/git-spatial}
	\end{fancycolumns}
\end{frame}

\xkcdframe{1597} % git

\subsection{Distributed Version Control}
\begin{frame}{Recap -- \insertsubsection}
	\slideDistributedVC
\end{frame}

\subsection{Recap -- Common Operations}

\slideCloneFetch{}

\slideCommitPush{}

\xkcdframe{1296} % commit messages

\slideMergePull{}

\slideIgnore{}

\begin{frame}{Branching and Merging} %TODO add git dags (see rebase) for other frames
	\slideBranchingAndMerging{}
\end{frame}

\begin{frame}[fragile]{Automatic Merge} %copied from se1-versioncontrol, command does not work with fragile
	\begin{columns}[onlytextwidth]
		\begin{column}{0.6\linewidth}
			\begin{columns}[T]
				\begin{column}{0.5\linewidth}
					Working Copy: Revision \only<1|handout:0>{10}\only<2-3|handout:0>{\emph{10$^*$}}\only<4>{\emph{11$^*$}}\\[2mm]
					
					\begin{onlyenv}<1|handout:0>
						\begin{lstlisting}[style=java,basicstyle=\fontfamily{pcr}\small\selectfont,numbers=none,escapechar=|]
class Foo {
	void bar() {
		print("Bar");
	}
}	
						\end{lstlisting}
					\end{onlyenv}
					\begin{onlyenv}<2-3|handout:0>
						\begin{lstlisting}[style=java,basicstyle=\fontfamily{pcr}\small\selectfont,numbers=none,escapechar=|]
class Foo {
	void bar() {
		print("Bar");
		print("\n");
	}
}	
						\end{lstlisting}
					\end{onlyenv}
					\begin{onlyenv}<4-|handout:1>
						\begin{lstlisting}[style=java,basicstyle=\fontfamily{pcr}\small\selectfont,numbers=none,escapechar=|]
class Foo {
	void bar() {
		print("Foo");
		print("Bar");
		print("\n");
	}
}	
						\end{lstlisting}
					\end{onlyenv}
				\end{column}
				\begin{column}{0.5\linewidth}					
					Repository: Revision \only<1-2|handout:0>{10}\only<3->{\emph{11}}
					
					\begin{onlyenv}<1-2|handout:0>
						\begin{lstlisting}[style=java,basicstyle=\fontfamily{pcr}\small\selectfont,numbers=none,escapechar=|]
class Foo {
	void bar() {
		print("Bar");
	}
}
						\end{lstlisting}
					\end{onlyenv}
					\begin{onlyenv}<3-|handout:1>
						\begin{lstlisting}[style=java,basicstyle=\fontfamily{pcr}\small\selectfont,numbers=none,escapechar=|]
class Foo {
	void bar() {
		print("Foo");
		print("Bar");
	}
}
						\end{lstlisting}
					\end{onlyenv}
				\end{column}
			\end{columns}
			
			\begin{example}{}
				\begin{itemize}
					\item<1-> Checkout of or update to revision 10
					\item<2-> Changing the working copy (\emph{10$^*$})
					\item<3-> In the meantime: New commit to repository (\emph{11})
					\item<4-> Update to head revision, automatically merged
				\end{itemize}			
			\end{example}
		\end{column}
		\begin{column}{0.5\linewidth}
		\end{column}
	\end{columns}
\end{frame}

\begin{frame}[fragile]{Merge Conflicts}	%copied from se1-versioncontrol, command does not work with fragile
	\begin{columns}[T, onlytextwidth]
		\begin{column}{0.6\linewidth}
			\begin{columns}[T]
				\begin{column}{0.5\linewidth}
					Working Copy: Revision \only<1|handout:0>{10}\only<2-3|handout:0>{\emph{10$^*$}}\only<4>{\emph{11$^*$}}\\[2mm]
					
					\begin{onlyenv}<1|handout:0>
						\begin{lstlisting}[style=java,basicstyle=\fontfamily{pcr}\small\selectfont,numbers=none,escapechar=|]
class Foo {
	void bar() {
		print("Bar");
	}
}	
						\end{lstlisting}
					\end{onlyenv}
					\begin{onlyenv}<2-3|handout:0>
						\begin{lstlisting}[style=java,basicstyle=\fontfamily{pcr}\small\selectfont,numbers=none,escapechar=|]
class Foo {
	void bar() {
		print("Bar\n");
	}
}
						\end{lstlisting}
					\end{onlyenv}
					\begin{onlyenv}<4-|handout:1>
						\begin{lstlisting}[style=java,basicstyle=\fontfamily{pcr}\small\selectfont,numbers=none,escapechar=|]
class Foo {
	void bar() {		
		<<<<<<< .mine
		print("Bar\n");
		=======
		print("FooBar");
		>>>>>>> .r11
	}
}	
						\end{lstlisting}
					\end{onlyenv}
				\end{column}
				\begin{column}{0.5\linewidth}					
					Repository: Revision \only<1-2|handout:0>{10}\only<3->{\emph{11}}
					
					\begin{onlyenv}<1-2|handout:0>
						\begin{lstlisting}[style=java,basicstyle=\fontfamily{pcr}\small\selectfont,numbers=none,escapechar=|]
class Foo {
	void bar() {
		print("Bar");
	}
}
						\end{lstlisting}
					\end{onlyenv}
					\begin{onlyenv}<3-|handout:1>
						\begin{lstlisting}[style=java,basicstyle=\fontfamily{pcr}\small\selectfont,numbers=none,escapechar=|]
class Foo {
	void bar() {
		print("FooBar");
	}
}
						\end{lstlisting}
					\end{onlyenv}
				\end{column}
			\end{columns}
			
			\begin{example}{}
				\begin{itemize}
					\item<1-> Checkout of or update to revision 10
					\item<2-> Changing the working copy (\emph{10$^*$})
					\item<3-> In the meantime: New commit to repository (\emph{11})
					\item<4-> Update to head revision results in conflict
					\item<4-> Automatic merge fails: user has to provide a fix
				\end{itemize}			
			\end{example}
		\end{column}
		\begin{column}{0.5\linewidth}
		\end{column}
	\end{columns}
\end{frame}

\subsection{Merge Algorithms}
\begin{frame}[fragile]{\insertsubsection}
	\begin{fancycolumns}[animation=none, widths={30}]
		
		\begin{lstlisting}[language=bash, basicstyle=\small, breaklines=true, numbers=none, escapechar=*]
git checkout main
>> Switched to branch 'main'
git merge feature
>> Merge made by the 'ost' strategy.
>> example.file 20 +
>> 1 file changed, 20 insertion(+)

git merge feature
>> Auto-merging example.file
>> CONFLICT (content): Merge conflict in example.file
>> Automatic merge failed; fix conflicts and then commit the result.
		\end{lstlisting}
		
		\nextcolumn
		\begin{center}
			\begin{tikzpicture}
				\gitDAG[grow right sep = 2em]{
					A -- B -- { 
						C -- D -- E,
						{ F -- G },
					} -- M
				};
				% Tag reference
				\gittag
				[ca]       % node name
				{Common Ancestor}       % node text
				{above=of B} % node placement
				{B}          % target
				% Remote branch
				% Branch
				\gitbranch
				{main}      % node name and text 
				{above=of M} % node placement
				{M}          % target
				\gitbranch
				{feature}      % node name and text 
				{below=of G} % node placement
				{G}          % target
				% HEAD reference
				\gitHEAD
				{above=of main} % node placement
				{main}          % target
			\end{tikzpicture}
		\end{center}
	\end{fancycolumns}
\end{frame}

\subsection{Patching, Cherry-Picking, and Rebasing}
\begin{frame}{\insertsubsection}
	\begin{fancycolumns}[animation=none]
		{\small 
		\begin{definition}{Patch}
			\begin{itemize}
				\item Text file containing code changes and git metadata
				\item Can be applied to a codebase
				\item Create patch: \texttt{git format-patch}
				\item Apply patch as commit: \texttt{git am}
			\end{itemize}
		\end{definition}\pause
		\begin{definition}{Cherry-Picking}
			\begin{itemize}
				\item Take commit and apply as patch
				\item Multiple commits $\rightarrow$ sequentially applied
				\item Transfers single changes between branches
			\end{itemize}
		\end{definition}\pause
		\begin{definition}{Rebase}
			\begin{itemize}
				\item Transfer sequence of commits to a new base 
				\item Internal: applies every commit to new base as new commit
				\item Creates a linear project history
			\end{itemize}
		\end{definition}\pause
		}
		\nextcolumn
		\begin{center}
			\begin{tikzpicture}
			% Commit DAG
			\gitDAG[grow right sep = 2em]{
				A -- B -- { 
					C,
					D -- E,
				}
			};
			% Tag reference
			\gittag
			[v0p1]       % node name
			{v0.1}       % node text
			{above=of A} % node placement
			{A}          % target
			% Remote branch
			\gitremotebranch
			[origmain]    % node name
			{origin/main} % node text
			{above=of C}    % node placement
			{C}             % target
			% Branch
			\gitbranch
			{main}     % node name and text 
			{above=of E} % node placement
			{E}          % target
			% HEAD reference
			\gitHEAD
			{above=of main} % node placement
			{main}          % target
		\end{tikzpicture}
		
		\pause
				
		\begin{tikzpicture}
			\gitDAG[grow right sep = 2em]{
				A -- B -- { 
					C -- D' -- E',
					{[nodes=unreachable] D -- E },
				}
			};
			% Tag reference
			\gittag
			[v0p1]       % node name
			{v0.1}       % node text
			{above=of A} % node placement
			{A}          % target
			% Remote branch
			\gitremotebranch
			[origmain]    % node name
			{origin/main} % node text
			{above=of C}    % node placement
			{C}             % target
			% Branch
			\gitbranch
			{main}      % node name and text 
			{above=of E'} % node placement
			{E'}          % target
			% HEAD reference
			\gitHEAD
			{above=of main} % node placement
			{main}          % target
		\end{tikzpicture}
		\end{center}
	\end{fancycolumns}
\end{frame}


\subsection{Branching Strategies}
\begin{frame}{\insertsubsection \mytitlesource{\href{https://gitlab.com/gitlab-org/gitlab-foss/-/blob/0fdb03ee16f0ccd7f122a4f0af23ee628d1de3c9/doc/workflow/gitlab\_flow.md}{GitLab Documentation}}}
	\begin{fancycolumns}[animation=none]
		\begin{definition}{Git Flow}
			\begin{itemize}
				\item First approach to use branches
				\item Different branches for different aims
				\begin{itemize}
					\item Main
					\item Development
					\item Individual features
					\item Individual releases
					\item Hotfixes
				\end{itemize}
				\item Feature $\rightarrow$ Dev $\rightarrow$ Release $\rightarrow$ Main
			\end{itemize}
		\end{definition}\pause
		\begin{note}{Problems}
			\begin{itemize}
				\item Very complex
				\item Does not work with shorter release cycles
				\item Many long diverging branches $\rightarrow$ many merge conflicts
			\end{itemize}
		\end{note}
		\nextcolumn
		\onslide<1->{\centering\pic[width=.7\columnwidth]{versioncontrol/git-flow}}
	\end{fancycolumns}
\end{frame}

\begin{frame}{\insertsubsection \mytitlesource{\href{https://docs.github.com/en/get-started/using-github/github-flow}{GitHub Documentation}}}
	\begin{fancycolumns}[animation=none]
		\begin{definition}{GitHub Flow}
			\begin{itemize}
				\item Easier alternative for agile development
				\item Main for deployment
				\item Small feature branches for development
				\item Tags for releases
				\item Hotfixes are also feature branches
			\end{itemize}
		\end{definition}\pause
		\begin{note}{Advantages}
			\begin{itemize}
				\item Frequent merges $\rightarrow$ small merge conflicts
				\item Supports fast release cycles
				\item Minimizes complexity: less branches, Main most important
				\item Code fast in Main $\rightarrow$ minimizes \enquote{dead} code
			\end{itemize}
		\end{note}
		\nextcolumn
		\onslide<1->{\centering\pic[width=.7\columnwidth]{versioncontrol/github-flow}}
	\end{fancycolumns}
\end{frame}
	


       	