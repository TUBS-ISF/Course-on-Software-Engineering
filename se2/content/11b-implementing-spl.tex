\subsection{Motivation and Recap}
\begin{frame}<3>{Recap: Branching \& Merging}
	\slideBranchingAndMerging
\end{frame}
% clone-and-own
% “ Copy and paste is a design error. ” - David Parnas

\begin{frame}{The Value of Feature Modeling}
	\begin{fancycolumns}
		\begin{note}{Interview with Practitioners \mysource{\href{https://link.springer.com/chapter/10.1007/978-3-319-11653-2_19}{Berger~et~al.~2014}}}
			\mycite{I think the best [about feature modeling] is you can see relationships, to actually know what configurations are allowed and what are not allowed. That was also not so easy to express in the past [\ldots] This is from the developer’s point of view. But it’s also, we can see that from the, say project development, it’s also important, because before we noticed that \emph{the same functionality was implemented twice} within the same project, basically they haven’t realized that. They implemented the same features.}
		\end{note}
	\end{fancycolumns}
\end{frame}

\subsection{Runtime Parameters and Runtime Properties}
\begin{frame}{Runtime Parameters}
	\begin{fancycolumns}
		\pic[width=\linewidth]{productlines/runtime-parameters-win10-cmd-dir}
		\nextcolumn
		\pic[width=\linewidth]{productlines/runtime-parameters-win10-cmd-dir2}
	\end{fancycolumns}
\end{frame}

\begin{frame}{\insertsubsection\ \mytitlesource{\featureide}}
	\begin{fancycolumns}
		\picDark[width=\linewidth]{productlines/implementation/runtime-parameters}
		\nextcolumn
		\picDark[width=\linewidth]{productlines/implementation/runtime-properties}
	\end{fancycolumns}
\end{frame}

\xkcdframe{619} % linux features 20s

\subsection{Preprocessors for C and Java}
\begin{frame}{The C Preprocessor\ \mytitlesource{\featureide}}
	\begin{fancycolumns}
		\begin{exampletight}{Example Input to the Preprocessor}
			\picDark[scale=.3]{productlines/implementation/preprocessor-c}
		\end{exampletight}
		\nextcolumn
		\begin{exampletight}{Example Output (Simplified)}
			\picDark[scale=.15]{productlines/implementation/preprocessor-c-output}
		\end{exampletight}
	\end{fancycolumns}
\end{frame}

\begin{frame}{Munge -- A Preprocessor for Java\ \mytitlesource{\featureide}}
	\begin{fancycolumns}
		\begin{exampletight}{Example Input and Output}
			\picDark[width=\linewidth]{productlines/implementation/preprocessor-munge}
		\end{exampletight}
		\nextcolumn
		\begin{exampletight}{Calling the Preprocessor}
			\centering
			\picDark[width=\linewidth]{productlines/implementation/preprocessor-munge-call}
			
			~
			
			\picDark[width=.7\linewidth]{productlines/implementation/preprocessor-munge-idea}
		\end{exampletight}
	\end{fancycolumns}
\end{frame}

\begin{frame}{Antenna -- An In-Place Preprocessor\ \mytitlesource{\featureide}}
	\begin{fancycolumns}
		\begin{exampletight}{Example Input and Output}
			\picDark[width=\linewidth]{productlines/implementation/preprocessor-antenna}
		\end{exampletight}
		\nextcolumn
		\begin{exampletight}{Calling the Preprocessor}
			\centering
			\picDark[width=\linewidth]{productlines/implementation/preprocessor-antenna-call}
			
			~
			
			\picDark[width=.7\linewidth]{productlines/implementation/preprocessor-antenna-idea}
		\end{exampletight}
	\end{fancycolumns}
\end{frame}

\begin{frame}{Discussion of Preprocessors\ \mytitlesource{\featureide}}
	\begin{fancycolumns}
		\begin{exampletight}{A Slightly More Complex Example}
			\picDark[width=\linewidth]{productlines/implementation/preprocessor-antenna-elevator}
		\end{exampletight}
		\nextcolumn
		\begin{exampletight}{Tool Support for Feature Traceability}
			\picDark[width=\linewidth]{productlines/implementation/feature-traceability}
		\end{exampletight}
	\end{fancycolumns}
\end{frame}

\subsection{Frameworks with Plug-Ins}
\begin{frame}{\insertsubsection\ \mytitlesource{\featureide}}
	\begin{fancycolumns}
		\picDark[width=\linewidth]{productlines/implementation/framework-with-plugins}
		\nextcolumn
		\picDark[width=\linewidth]{productlines/implementation/framework-a-plugin}
	\end{fancycolumns}
\end{frame}

% car configurator: they will not build each car from scratch!

% “ Before software can be reusable it first has to be usable. ” - Ralph Johnson
% \sommerville, Chapter 7.3.1 Reuse
% Reuse Most modern software is constructed by reusing existing components or systems. When you are developing software, you should make as much use as possible of existing code. \sommerville
% \sommerville, Chapter 15--17

% meta: implementation technique, example/illustration, characteristics, typical examples

% use picture from legoland

% components/services: was kam schon dran? pointer zu standardisierung im sopra, show component diagram for the current sopra, illustrate extensibility

% micro services?

% android: preprocessors, plugins (apps), forks
