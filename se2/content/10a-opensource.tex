\subsection{Open-Source Development}
\begin{frame}{\insertsubsection\ \mytitlesource{\sommerville}} % copied from se1
	\begin{fancycolumns}
		\begin{definition}{Open-Source Development}
			\mycite{\emph{Open-source development} is an approach to software development in which the source code of a software system is published and volunteers are invited to participate in the development process.} % Raymond 2001
		\end{definition}
		\begin{note}{Open-Source Development}
			\begin{itemize}
				\setlength\itemsep{.0em}
				\item assumption in the early days: code developed by small core group
				\item today: Internet to recruit volunteer developers (often users)
				\item volunteers contribute with bug reports, feature requests, changes to the software (cf.\ pull requests)
				\item core group controls changes to main repo
				\item opposite: proprietary software / closed-source development
			\end{itemize}
		\end{note}
		\nextcolumn
		\begin{note}{Internet iff Open Source}
			\begin{itemize}
				\setlength\itemsep{.0em}
				\item Internet used to distribute source code and recruit developers
				\item Internet builds on open-source software
			\end{itemize}
		\end{note}
		\begin{example}{Examples} % TODO separate slides with images?
			\begin{itemize}
				\setlength\itemsep{.0em}
				\item operating system Linux (for most servers)
				\item operating system Android (for most clients)
				\item web server Apache
				\item database management system mySQL
				\item browser Firefox
				% TODO \item e-mail client Thunderbird
				\item image editor GIMP
				\item media player VLC
				\item screen recording software OBS
				\item video editing software Shotcut
				\item runtime env.\ and standard library OpenJDK
				\item development environment Eclipse
			\end{itemize} % 
		\end{example}
	\end{fancycolumns}
\end{frame}

\subsection{Central Questions for Each Software Project}
\begin{frame}{\insertsubsection\ \mytitlesource{\sommerville}} % copied from se1
	\begin{fancycolumns}
		\begin{note}{First Question}
			\mycite{Should the product that is being developed make use of open-source components?}
		\end{note}
		\begin{example}{Example Criteria}
			\begin{itemize}
				\item are related open-source components available?
				\item is their quality sufficient?
				\item will they be maintained?
				\item what is the effort to integrate them?
				\item is it cheaper to modify or redevelop them?
				\item how are they licensed?
			\end{itemize}
		\end{example}
		\nextcolumn
		\begin{note}{Second Question}
			\mycite{Should an open-source approach be used for its own software development?}
		\end{note}
		\begin{example}{Example Criteria}
			\begin{itemize}
				\item business model: how to earn money then? % TODO does not mean it is free
				\item support? consulting? proprietary extensions? -- \textit{rich relatives or friends? unconditional basic income \deutsch{bedingungsloses Grundeinkommen}?}
				\item what additional value do you have from opening the source?
				\item more customers, recognition, collaborations?
				\item does it reveal confidential business knowledge?
			\end{itemize}
		\end{example}
	\end{fancycolumns}
\end{frame}

\xkcdframe{1810} % chat clients

\subsection{Richard Stallman's Four Freedoms}
\begin{frame}{\insertsubsection\ \mytitlesource{\href{https://www.youtube.com/watch?v=Ag1AKIl_2GM}{Stallman's 2014 TEDx Talk}}}
	\begin{fancycolumns}
		\centering\pic[width=.6\linewidth,trim=0 0 0 0,clip]{people/richard-stallman}
		\vspace{-7mm}
		
		\begin{note}{}
			Richard Matthew Stallman aka.\ RMS (born 1953)
		\end{note}
		\nextcolumn
		\begin{definition}{Four Freedoms by Richard Stallman}
			\begin{itemize}
				\item 0. freedom to run it for whatever purpose
				\item 1. freedom to study and modify the program (source code is available)
				\item 2. freedom to redistribute it (needed for non-programmers)
				\item 3. freedom to redistribute or even sell modified versions
				\item all four needed that the user controls the program
				\item otherwise the program controls the users
			\end{itemize}
		\end{definition}
	\end{fancycolumns}
\end{frame}

\subsection{Free Software}
\begin{frame}{\insertsubsection\ \mytitlesource{\href{https://www.youtube.com/watch?v=Ag1AKIl_2GM}{Stallman's 2014 TEDx Talk}}}
	\begin{fancycolumns}
		\begin{definition}{Free Software}
			\begin{itemize}
				\item free stands for freedom (often confused with \mycite{free of charge})
				\item free software: demand for freedom
				\item open-source software: better code quality (promoted by Eric S.\ Raymond)
				\item today largely replaced with term open-source software
				\item more inclusive term?\\free and open-source software (FOSS) % TODO wikipedia defines this term to be the subset of both worlds https://en.wikipedia.org/wiki/Free_and_open-source_software
			\end{itemize}
		\end{definition}
		\begin{note}{Free Software vs Freeware}
			\begin{itemize}
				\item free software: free as in free speech
				\item freeware: free as in free beer
			\end{itemize}
			free software can be adware or shareware
		\end{note}
		\nextcolumn
		\begin{definition}{Free Software Foundation \mysource{\href{https://www.fsf.org/}{fsf.org}}}
			\begin{itemize}
				\item founded by Richard Stallman in 1985
				\item first ten years: employ developers for the operating system GNU
				
				Linux = GNU + Linux kernel
				\item later: promoting free software, working on legal issues
			\end{itemize}
		\end{definition}
		\begin{exampletight}{}
			\centering\pic[width=.5\linewidth]{opensource/gnu-logo}
		\end{exampletight}
	\end{fancycolumns}
\end{frame}

\subsection{Buying Free Software}
\begin{frame}{\insertsubsection}
	\begin{fancycolumns}
		\centering\pic[height=65mm,trim=0 500 0 600,clip]{opensource/buying-free-software-gimp}
		\nextcolumn
		\centering\pic[height=65mm,trim=0 500 0 600,clip]{opensource/buying-free-software-vlc}
	\end{fancycolumns}
\end{frame}
