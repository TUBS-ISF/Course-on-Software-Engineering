\subsection{Wisdom on Maintenance}
\begin{frame}{\insertsubsection} % slided copied from se1-04a. reuse?
	\begin{fancycolumns}
		\pic[width=\linewidth,trim=0 360 0 50,clip]{people/karolina-szczur}
		\vspace{-7mm}
		
		\begin{note}{Karolina Szczur \mysource{\href{https://twitter.com/codewisdom/status/1141686386103332864}{twitter.com}}}
			\mycite{Writing software as if we are the only person that ever has to comprehend it is one of the biggest mistakes and false assumptions that can be made.}
		\end{note}
		\nextcolumn
		\pic[width=\linewidth,trim=450 235 450 120,clip]{changes/psychopath}
		\vspace{-7mm}
		
		\begin{note}{John F. Woods \mysource{\href{https://groups.google.com/g/comp.lang.c++/c/rYCO5yn4lXw/m/oITtSkZOtoUJ}{google.com}}}
			\mycite{Always code as if the guy who ends up maintaining your code will be a violent psychopath who knows where you live.}
		\end{note}
	\end{fancycolumns}
\end{frame}

\subsection{Software Maintenance}
\begin{frame}{\insertsubsection\ \mytitlesource{\ludewiglichter}} % slided copied from se1-04a. reuse?
	\begin{fancycolumns}
		\begin{note}{Motivation}
			\begin{itemize}
				\item for software: no compensation of deterioration, repair, spare parts
				\item corrections (especially shortly after first delivery)
				\item modification and reconstruction
			\end{itemize}
		\end{note}
		\nextcolumn
		\begin{definition}{Operation and Maintenance Phase \mysource{\ieeesixten}}
			\mycite{The period of time in the software life cycle during which a software product is employed in its operational environment, monitored for satisfactory performance, and modified as necessary to correct problems or to respond to changing requirements.}
		\end{definition}
		\begin{definition}{Maintenance \mysource{\ieeesixten}}
			\mycite{The process of modifying a software system or component after delivery to correct faults, improve performance or other attributes, or adapt to a changed environment.}
		\end{definition}
	\end{fancycolumns}
\end{frame}
% TODO Operations

\subsection{Adaptive Maintenance}
\begin{frame}{\insertsubsection}
	\begin{fancycolumns}
		\begin{definition}{\insertsubsection\mysource{\lientzswanson}}
			\mycite{Software maintenance performed to make a computer program usable in a changed environment.} \hfill \deutsch{adaptive Wartung}
		\end{definition}
		\begin{example}{}
			desktop application for a new version of an operating system (e.g., from Windows 8.1 to 10)
		\end{example}
		\nextcolumn
	\end{fancycolumns}
\end{frame}

\subsection{Corrective Maintenance}
\begin{frame}{\insertsubsection}
	\begin{fancycolumns}
		\begin{definition}{\insertsubsection\mysource{\lientzswanson}}
			\mycite{Maintenance performed to correct faults in software.} \hfill \deutsch{korrektive Wartung}
		\end{definition}
		\begin{example}{}
			Windows calculator showing wrong formulas
		\end{example}
		\nextcolumn
	\end{fancycolumns}
\end{frame}

\subsection{Perfective Maintenance}
\begin{frame}{\insertsubsection}
	\begin{fancycolumns}
		\begin{definition}{\insertsubsection\mysource{\lientzswanson}}
			\mycite{Software maintenance performed to improve the performance, maintainability, or other attributes of a computer program.} \hfill \deutsch{perfektive Wartung}
		\end{definition}
		\begin{example}{}
			better handling of very large files in a text editor
		\end{example}
		\nextcolumn
	\end{fancycolumns}
\end{frame}
% TODO it is annoying that perfective maintenance has an overlap with preventive maintenance, as better maintainability typically also results in fewer problems

% TODO software aging (see Parnas-SoftwareAging, GodfreyGerman-LehmansLaws)

\subsection{Y2K and 2038}
\widexkcdframe{2038} % y2k bug vs 2038 bug

\begin{frame}{Y2K Problem \mytitlesource{\href{https://en.wikipedia.org/wiki/Year_2000_problem}{wikipedia.org}}}
	\begin{fancycolumns}[b]
		\begin{definition}{Y2K Problem}
			\begin{itemize}
				\item years were often stored by only two digits
				\item high awareness in media
				\item about 500 billion dollars spent
				\item about 90\% before 2000
				\item most major problems could be prevented
			\end{itemize}
		\end{definition}
		\centering\picDark[width=.8\linewidth]{failures/y2k-javascript}
	\nextcolumn
		\vspace{-15mm}
		
		\centering\pic[width=.5\linewidth]{failures/y2k}
		\begin{example}{Selected Y2K Failures}
			\begin{itemize}
				\item 1999-01-01: computers for temporary passports in Sweden crashed
				\item 1999-12-28: credit and debit card transactions rejected because they could not be completed within four working days
				%\item 2000-01-01: date 2036-02-06 shown in nuclear power plant in Fukushima
				\item 2000-01-01: German bank accidentially transferred 6 million euro issued 1899-12-30
				\item 2000-01-03: french school showing wrong date
				\item 2020-01-01: New York City parking meters refuse credit cards
			\end{itemize}
		\end{example}
	\end{fancycolumns}
\end{frame}

\begin{frame}{Y2038 Problem \mytitlesource{\href{https://en.wikipedia.org/wiki/Year_2038_problem}{wikipedia.org}}}
	\begin{fancycolumns}
		\begin{definition}{Y2038 Problem}
			\begin{itemize}
				\item 2038-01-19 at 03:14:07 potential overflow
				\item systems affected storing Unix time as signed 32-bit integer
				\item similarly: 2106-02-07 for unsigned integers
				\item solution: use signed 64-bit integer
				\item 292 billion years to next overflow
			\end{itemize}
		\end{definition}
		\centering\picDark[width=\linewidth]{failures/y2038}
	\nextcolumn
		\begin{example}{Y2038 Failures}
			2006-05-13: AOL server software crashed
			\begin{itemize}
				\item it used 1 billion seconds extra to state that database transactions do not timeout
				\item \emph{temporary} solution: set the timeout to a lower value in configuration file
			\end{itemize}
			embedded systems without over-the-air updates are likely to be affected (e.g., ECUs in most automotive systems)
		\end{example}
	\end{fancycolumns}
\end{frame}

\subsection{Preventive Maintenance}
\begin{frame}{\insertsubsection}
	\begin{fancycolumns}
		\begin{definition}{\insertsubsection\mysource{\lientzswanson}}
			\mycite{Maintenance performed for the purpose of preventing problems before they occur.} \hfill \deutsch{präventive Wartung}
		\end{definition}
		\begin{example}{Y2K Problem}
			was very expensive, still not over
		\end{example}
		\begin{example}{Y2038 Problem}
			first failures already happen, will be expensive
		\end{example}
	\nextcolumn
		\begin{example}{Leap Years in Gregorian Calendar}
			\begin{itemize}
				\item solar year is 365.24 days
				\item every 4 years, but not every 100 years, but again every 400 years
				\item 2000 was a leap year
				\item one day extra: February 29, 2028
				\item common problem when durations are computed
				\item example: meat sold beginning of March
			\end{itemize}
		\end{example}
		\begin{example}{Leap Seconds}
			\begin{itemize}
				\item solar day longer than a day
				\item last leap second on 2016-12-31
				\item more details in SE1
			\end{itemize}
			\centering\pic[width=.5\linewidth,trim=0 0 0 0,clip]{changes/leap-second}
		\end{example}
	\end{fancycolumns}
\end{frame}

\subsection{Kinds of Maintenance}
\begin{frame}{\insertsubsection\ \mytitlesource{\ludewiglichter}}
	\begin{fancycolumns}
		\nextcolumn
	\end{fancycolumns}
\end{frame}

\xkcdframe{1172} % No more CPU heating as breaking change

\subsection{Evolution and Maintenance}
\begin{frame}{\insertsubsection\ \mytitlesource{\ludewiglichter}}
	\slideEvolutionAndMaintenance
\end{frame}
% Changelog (honest?)
% vgl. Release Management, Semantic Versioning
% Problems: time pressure, results in new failures, unpopular task

% TODO 22.4.2 Die Behandlung von Problemmeldungen \ludewiglichter

% TODO software erosion/rot/decay/entropy:  https://en.wikipedia.org/wiki/Software_rot

% TODO software estimation (bewertung)
% 	motivation, metrics for OOP
% 14 Metriken und Bewertungen \ludewiglichter

% New Content

% TODO APIs: every observable behavior will eventually be used by users

% TODO new example of Tesla: https://www.heise.de/news/Tesla-Autos-duerfen-nicht-mehr-furzen-Rueckruf-in-den-USA-6441936.html

