% !TeX spellcheck = en_US
\subsection{Motivating Examples}
\begin{frame}{Recap -- CrowdStrike Outage}
	\slideCrowdStrike
\end{frame}

\begin{frame}{Therac-25}
	\begin{fancycolumns}[b, animation=none]
		\onslide<1->{
			\begin{definition}{Therac-25 \mysource{\href{https://ieeexplore.ieee.org/document/274940}{Investigation}}}
				\begin{itemize}
					\item Linear accelerator for treating cancer
					\item First medical accelerator to primarily use a computer-based control system
					\item 6 deadly incidents between 1985 and 1987
					\item Texas-Bug: race condition
					\item Washington-Bug: variable overflow
				\end{itemize}
			\end{definition}
		}
		\onslide<2->{
			\small\begin{note}{Later Analysis of Thorac-25 Software \mysource{\href{https://ieeexplore.ieee.org/document/274940}{Investigation}}}
				\mycite{We know that the software for the Therac-25 was developed by a single person using POP 11 assembly language, over a period of several years. [\ldots] the AECL [manufacturer] response also seems to point out an apparent lack of documentation on software specifications and a software test plan.}
			\end{note}
		}
		\nextcolumn
		\vspace{-10mm}
		\onslide<1->{
			\pic[width=\columnwidth]{failures/therac-25}\vspace{-2mm}
		}
		\onslide<2->{
			\small\begin{note}{Fifth Review of the Code Base by the FDA \mysource{\href{https://ieeexplore.ieee.org/document/274940}{FDA Review}}}
				\mycite{Amazingly, the test data presented to show that the software changes to handle the edit problems in the Therac-25 are appropriate prove the exact opposite result. [\ldots] The	manufacturer should be admonished for this error. Where is the QC [quality control]	review for the test program?}
			\end{note}
		}
	\end{fancycolumns}
\end{frame}

\subsection{Recap -- Software Quality}
\begin{frame}<4>{\insertsubsection}
	\slideSoftwareQuality
\end{frame}

\subsection{Recap -- Software Testing}
\begin{frame}<6>{\insertsubsection}
	\slideSoftwareTesting
\end{frame}

\subsection{Recap -- Quality Assurance}
\begin{frame}{\insertsubsection\ \mytitlesource{\ludewiglichter}}
	\slideMindmapQualityAssuranceMod{}{}{}{}{}{}{}
\end{frame}

\subsection{Recap -- Test Cases}
\begin{frame}<4>{\insertsubsection}
	\begin{fancycolumns}[animation=none]
		\explTestCases
		\nextcolumn
		\figTestDesign
	\end{fancycolumns}
	
\end{frame}

\subsection{Recap -- Test Case Types}
\begin{frame}<3>{\insertsubsection}
	\small
	\begin{fancycolumns}[animation=none]
		\begin{definition}{White-Box Testing \mysource{\ludewiglichter}} %Copy from SE1-testing
			\begin{itemize}
				\setlength\itemsep{.1em}
				\item inner structure of test object is used
				\item idea: coverage of structural elements
				\item code translated into control flow graph
				\item specific test case (concrete inputs)\\derived from logical test case (conditions)\\derived from path in control flow graph
			\end{itemize}
		\end{definition} \pause
		\begin{definition}{Coverage Criteria \mysource{\ludewiglichter}} %Copy from SE1-testing
			\begin{itemize}
				\item[1.] statement coverage \deutsch{Anweisungsüberdeck.}: all statements are executed for at least one test case
				\item[2.] branching coverage \deutsch{Zweigüberdeckung}: statement coverage and for each branching statement all branches have been exercised
				\item[3.] term coverage \deutsch{Termüberdeckung}: branching coverage and terms ($n$) used in a branching statement are combined exhaustively ($2^n$)\hfill(simplified)
			\end{itemize}
		\end{definition}

	\nextcolumn
	\pause
		\begin{definition}{Black-Box Testing \deutsch{Funktionstest}} %Copy from SE1-testing
			\begin{itemize}
				\item test-case design based on specification
				\item source code and its inner structure is ignored (assumed as a black-box)
			\end{itemize}
		\end{definition}
		\begin{definition}{1. Equivalence Class Testing}
			\begin{itemize}
				\item idea: classify inputs and outputs into equivalence classes
				\item assumption: equivalent test cases detect the same faults, one test case is sufficient
			\end{itemize}
		\end{definition}
		\begin{definition}{2. Boundary Testing}
			\begin{itemize}
				\item extension of equivalence class testing
				\item goal: use experience (e.g., off-by-one errors)
				\item for every equivalence class: consider smallest, typical, and largest value
			\end{itemize}
		\end{definition}
	
	\end{fancycolumns}
\end{frame}

\subsection{Recap -- Stages of Testing}
\begin{frame}<5>{\insertsubsection}
	\slideStagesTesting
\end{frame}