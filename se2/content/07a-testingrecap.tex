% !TeX spellcheck = en_US
\subsection{Motivating Examples}
\begin{frame}{Recap -- CrowdStrike Outage}
	\slideCrowdStrike
\end{frame}

\begin{frame}{Therac-25}
	\slideTherac
\end{frame}

\subsection{Recap -- Software Quality}
\begin{frame}<4>{\insertsubsection}
	\slideSoftwareQuality
\end{frame}

\subsection{Recap -- Software Testing}
\begin{frame}<6>{\insertsubsection}
	\slideSoftwareTesting
\end{frame}

\subsection{Recap -- Quality Assurance}
\begin{frame}{\insertsubsection\ \mytitlesource{\ludewiglichter}}
	\slideMindmapQualityAssuranceMod{}{}{}{}{}{}{}
\end{frame}

\subsection{Recap -- Test Cases}
\begin{frame}<4>{\insertsubsection}
	\begin{fancycolumns}[animation=none]
		\explTestCases
		\nextcolumn
		\figTestDesign
	\end{fancycolumns}
	
\end{frame}

\subsection{Recap -- Test Case Types}
\begin{frame}<3>{\insertsubsection}
	\small
	\begin{fancycolumns}[animation=none]
		\begin{definition}{White-Box Testing \mysource{\ludewiglichter}} %Copy from SE1-testing
			\begin{itemize}
				\setlength\itemsep{.1em}
				\item inner structure of test object is used
				\item idea: coverage of structural elements
				\item code translated into control flow graph
				\item specific test case (concrete inputs)\\derived from logical test case (conditions)\\derived from path in control flow graph
			\end{itemize}
		\end{definition} \pause
		\begin{definition}{Coverage Criteria \mysource{\ludewiglichter}} %Copy from SE1-testing
			\begin{itemize}
				\item[1.] statement coverage \deutsch{Anweisungsüberdeck.}: all statements are executed for at least one test case
				\item[2.] branching coverage \deutsch{Zweigüberdeckung}: statement coverage and for each branching statement all branches have been exercised
				\item[3.] term coverage \deutsch{Termüberdeckung}: branching coverage and terms ($n$) used in a branching statement are combined exhaustively ($2^n$)\hfill(simplified)
			\end{itemize}
		\end{definition}

	\nextcolumn
	\pause
		\begin{definition}{Black-Box Testing \deutsch{Funktionstest}} %Copy from SE1-testing
			\begin{itemize}
				\item test-case design based on specification
				\item source code and its inner structure is ignored (assumed as a black-box)
			\end{itemize}
		\end{definition}
		\begin{definition}{1. Equivalence Class Testing}
			\begin{itemize}
				\item idea: classify inputs and outputs into equivalence classes
				\item assumption: equivalent test cases detect the same faults, one test case is sufficient
			\end{itemize}
		\end{definition}
		\begin{definition}{2. Boundary Testing}
			\begin{itemize}
				\item extension of equivalence class testing
				\item goal: use experience (e.g., off-by-one errors)
				\item for every equivalence class: consider smallest, typical, and largest value
			\end{itemize}
		\end{definition}
	
	\end{fancycolumns}
\end{frame}

\subsection{Recap -- Stages of Testing}
\begin{frame}<5>{\insertsubsection}
	\slideStagesTesting
\end{frame}