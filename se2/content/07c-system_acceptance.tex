% !TeX spellcheck = en_US
\subsection{System Testing}
\begin{frame}<4>{\insertsubsection}
	\slideStagesTesting
\end{frame}

\subsection{End-to-End Testing}
\begin{frame}{\insertsubsection}
	\begin{fancycolumns}[animation=none]
		\begin{definition}{End-to-End Testing}
			\begin{itemize}
				\item Testing of complete system configuration
				\item All components necessary for one system configuration connected
				\item often first interaction of components developed by different teams
				\item $\rightarrow$ check compatibility and compare behavior to specification
				\item often performed by a dedicated team
			\end{itemize}
		\end{definition}\pause
		\nextcolumn
		\begin{definition}{Use-Case Testing}
			\begin{enumerate}
				\item Identify common use-cases
				\item Specify using sequence diagram
				\item Create necessary test cases
				\item Check interactions
			\end{enumerate}
			$\rightarrow$ Disadvantage: uncommon interactions not tested
		\end{definition}\pause
		\begin{note}{Other Strategies}
			\begin{itemize}
				\item recovery testing
				\item safety testing
				\item security testing
				\item performance testing
			\end{itemize}
		\end{note}
	\end{fancycolumns}
\end{frame}

\subsection{Manual Testing}
\begin{frame}{\insertsubsection}
	\begin{fancycolumns}[animation=none]
		\begin{definition}{Manual Testing}
			\begin{itemize}
				\item Not every test can easily be automatized
				\item Test automation has a high initial cost
				\item Easier non-functional testing
				\item Also includes static tests, e.g., code reviews
			\end{itemize}
		\end{definition}\pause
		\begin{note}{Stages of Manual Testing}
			\begin{enumerate}
				\item Create high-level test plan with general methodology, resources, \dots
				\item Create detailed test cases with clear steps and expected outcome
				\item Assign test cases to testers
				\item Create test report with detailed descriptions and protocols
			\end{enumerate}
		\end{note} \pause
		\nextcolumn
		\begin{note}{Trade-Off}
			\textbf{Automated Tests:}
			\begin{itemize}
				\item Easier to execute
				\item Parallelizable
				\item Better reproducibility
				\item Complex creation
			\end{itemize} \pause
			\textbf{Manual tests:}
			\begin{itemize}
				\item Easier to create
				\item Complex interactions testable
				\item Allows for subjective feedback
				\item Expensive to executed $\rightarrow$ limits number of executions
				\item Bad scalability 
			\end{itemize}
		\end{note}
		
	\end{fancycolumns}
\end{frame}

\subsection{Fuzz Testing}
\begin{frame}{\insertsubsection}
	\begin{fancycolumns}[animation=none]
		\begin{definition}{Fuzzing}
			\begin{itemize}
				\item Automated testing of diverse program input
				\item Especially invalid, unexpected and random input
				\item Goal: create crashes, assertion fails, or memory leaks
				\item Input can either be structured or random
				\item Prioritize user-facing components
				\item Especially helpful to find corner cases
			\end{itemize}
		\end{definition}\pause
		\begin{note}{Working Principle}
			\begin{enumerate}
				\item Create test cases
				\item Create input specification
				\item Operate fuzzer with repeated input
				\item (Reduce input for found issues)
				\item Replicate found issues with saved inputs
			\end{enumerate}
		\end{note} \pause
		\nextcolumn
		\pic[width=\columnwidth]{testing/fuzzing} \pause
		\begin{note}{Jazzer} %----LIVE DEMONSTRATION---- https://github.com/TUBS-ISF/SE2-liveDemo-Dynamic
			Tool for fuzz testing in Java
		\end{note}
	\end{fancycolumns}
\end{frame}

\subsection{Acceptance Testing}
\begin{frame}<5>{\insertsubsection}
	\slideStagesTesting
\end{frame}

\begin{frame}{\insertsubsection}
	\begin{fancycolumns}[animation=none]
		\begin{note}{V\&V \mysource{\seeconomics}}
			\mycite{\emph{Validation}: Are we building the right product?\\\emph{Verification}: Are we building the product right?}
		\end{note}	\pause
		\begin{definition}{Release Testing}
			\begin{itemize}
				\item Often dedicated team
				\item Validation $\rightarrow$ against requirements
				\item Many tests per requirement
				\item Requirement-based testing: validation against stated requirements
				\item Scenario-based testing: testing of common usage scenarios
			\end{itemize}
		\end{definition}\pause
		\nextcolumn
		\begin{definition}{User tests}
			\begin{itemize}
				\item Testing using user participation
				\item Three types:
				\begin{itemize}
					\item Alpha tests: small group, direct interaction
					\item Beta tests: publishing of software release to limited user base
					\item Acceptance tests: customer decides on software acceptance
				\end{itemize}
			\end{itemize} 
		\end{definition} \pause
		\begin{note}{Stages of Acceptance \mysource{\sommerville}}
			\begin{enumerate}
				\item Create acceptance criteria
				\item Develop plan for acceptance strategy
				\item Create tests that check developed criteria
				\item Perform tests
				\item Negotiate results
				\item Accept or reject product
			\end{enumerate}
		\end{note}
		
	\end{fancycolumns}
\end{frame}

\subsection{GUI Testing}
\begin{frame}{\insertsubsection}
	\begin{fancycolumns}[animation=none]
		\onslide<1->{
		\begin{note}{Motivation}
			\begin{itemize}
				\item E2E Testing 
				\item User interacts with GUI $\rightarrow$ realistic testing scenario
			\end{itemize}
		\end{note}}
		\onslide<2->{ \small
		\begin{note}{Development of GUI Testing}
			\textbf{1st Generation:}
			\begin{itemize}
				\item Replay of exact screen coordinates
				\item Problem: not robust, high maintenance effort
			\end{itemize}
			\onslide<3->{
			\textbf{2nd Generation:}
			\begin{itemize}
				\item Identify elements based on DOM searches
				\item Actions and expected results encoded in DSL
				\item Problem: expensive and difficult to create
			\end{itemize}}
			\onslide<4->{
			\textbf{3rd Generation:}
			\begin{itemize}
				\item Image recognition and easier DSL to emulate user behavior
			\end{itemize} }
		\end{note} }
		\nextcolumn
		\onslide<3->{ \small
		\begin{definition}{XPath}
			\begin{itemize}
				\item Tree-navigation with additional conditions
				\item \texttt{/bookstore/book[price>35]/title} 
				\item \texttt{//book[@name='Software Engineering']/title}
			\end{itemize}
		\end{definition}}
		\onslide<4->{\centering\pic[width=.8\columnwidth]{testing/selenium_3}}
		\onslide<5->{
		\begin{note}{Example: Selenium}
			Framework for automated testing of web applications
		\end{note}}
	\end{fancycolumns}
\end{frame}

%\subsection{Performance Testing}
%\begin{frame}{\insertsubsection}
%	\begin{fancycolumns}[animation=none]
%		\begin{definition}{}
%			content
%		\end{definition}\pause
%		\nextcolumn
%		
%		
%	\end{fancycolumns}
%\end{frame}

%\subsection{Dependability Testing}
%\begin{frame}{\insertsubsection}
%	\begin{fancycolumns}[animation=none]
%		\begin{definition}{}
%			content
%		\end{definition}\pause
%		\nextcolumn
%		
%		
%	\end{fancycolumns}
%\end{frame}