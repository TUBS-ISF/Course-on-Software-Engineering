\subsection{What You Should Know}

\begin{frame}{\myframetitle{}}
	\begin{fancycolumns}
		\begin{note}{Fundamentals of Software Engineering}
			\begin{itemize}
				\item development processes: waterfall, v-model, scrum
				%\item project management
				\item analysis, design, implementation, testing, maintanance
				\item UML diagrams (esp. class and component diagrams)
				\item version control (esp. git)
			\end{itemize}
			\ifuniversity{tubs}{$\Rightarrow$ \emph{Software Engineering 1}}
		\end{note}
	\nextcolumn
		\begin{note}{Fundamentals of Theoretical Computer Science}
			\begin{itemize}
				%\item set theory
				\item propositional logic
				\item predicate logic
			\end{itemize}
		\end{note}
		\begin{note}{Exercise}
			skills in object-oriented programming (preferably Java)

			%\ifuniversity{tubs}{$\Rightarrow$ \emph{Software Engineering 1}}
		\end{note}
	\end{fancycolumns}
\end{frame}

\subsection{What You Will Learn}

\begin{frame}{\myframetitle{}}
	\lectureseriesoverview[1]
\end{frame}

\begin{frame}{\myframetitle{}}
	\small
	\begin{fancycolumns}[columns=3,t]
		\begin{definition}{Part A: How to \emph{change} software?}
			\begin{itemize}
				\item Evolution: How does software change over time and why?
				\item Maintenance: How is software operated and maintained?
				\item Design Patterns: How to better support future changes?
				\item (Lego Scrum: How to develop agile and in teams?)
			\end{itemize}
		\end{definition}
	\nextcolumn
		\begin{definition}{Part B: How to \emph{verify} software?}
			\begin{itemize}
				\item Compilation: How do compilers work and what is their role in software engineering?
				\item Static Analysis: How can we detect faults without executing the software?
				\item Dynamic Analysis: How can we detect faults by executing the software?
				\item Design by Contract: How to specify the intended behavior of the software?
				\item Configuration Management: How to manage and verify versions of software?
			\end{itemize}
		\end{definition}
	\nextcolumn
		\begin{definition}{Part C: How to \emph{reuse} software?}
			\begin{itemize}
				\item Open-Source Software: When are we allowed to reuse software?
				\item Software Product Lines: How can we reuse software systematically?
				\item Automotive Software: How is reuse is performed for automotive software?
				\item Guest Lectures: How is software engineering done in practice and envisioned in research?
			\end{itemize}
		\end{definition}
	\end{fancycolumns}
\end{frame}

\subsection{What You Might Need}

\begin{frame}{\insertsubsection}
	\begin{fancycolumns}[animation=none]
		\centering\pic[height=50mm]{books/sommerville-softwarenegineering}
		\nextcolumn
		\begin{definition}{\mysource{\sommerville}}
			\begin{itemize}
				\item \sommervillelink{Ian Sommerville. Software Engineering, 10. Edition, Pearson, 2018.}
				\begin{itemize}
					\item German, English, and earlier versions
					\item \href{https://software-engineering-book.com/videos/}{Videos by Ian Sommerville and others available online}
				\end{itemize}
				\item More literature announced in each lecture
			\end{itemize}
		\end{definition}
	\end{fancycolumns}
\end{frame}

%\subsection{Credit for the Slides}
%
%\begin{frame}{\myframetitle{}}
%	\ifuniversity{anonymous}{\mynote{}{\centering\huge Anonymous Authors}}
%	\unlessuniversity{anonymous}{
%		\myframeicon{\href{https://github.com/SoftVarE-Group/Course-on-Software-Product-Lines}{\pic[scale=.75]{cc-by-sa}}}
%	}
%	\begin{fancycolumns}[columns=3,animation=none]
%	\nextcolumn
%		\unlessuniversity{anonymous}{
%			\begin{note}{Thomas Thüm}
%				\centering
%				\href{https://www.uni-ulm.de/en/in/sp/team/thuem/}{\adjincludegraphics[height=.45\textheight,trim={.125\width} 0 {.125\width} 0,clip]{thomas-thuem}}
%
%				\small Professor at Paderborn University
%
%				software engineering
%
%				FeatureIDE team leader
%			\end{note}
%		}
%	\nextcolumn
%	\end{fancycolumns}
%\end{frame}
