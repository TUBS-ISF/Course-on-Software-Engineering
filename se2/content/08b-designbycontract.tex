\subsection{Ideal Documentation of Assumptions}
\begin{frame}{\insertsubsection}
	\begin{definition}{Summary on Techniques to Document Developers' Assumptions}
		\centering\renewcommand{\arraystretch}{1.5}
		\begin{tabular}{cccccc}
			\toprule
			Technique & Accessible & Unambiguous & Automated Checks & Overhead / Side-Effects \\
			\midrule
			Code Documentation & yes & no & no & no \\
			Defensive Programming & no & yes & yes & yes \\
			Runtime Assertions & no & yes & yes & only debugging \\
			Unit Testing & no & yes & yes & no \\
			??? & yes & yes & \only<2->{yes & no} \\
			\bottomrule
		\end{tabular}
	\end{definition}
	\begin{note}{Motivation}
			\begin{itemize}
				\item no technique that is accessible during development and unambiguous
				\item<2-> further wanted: automated checks, no runtime overhead, no (or controllable) side-effects
			\end{itemize}
	\end{note}
\end{frame}

\subsection{Design by Contract}
\begin{frame}{\insertsubsection}
	\begin{fancycolumns}
		\begin{definition}{\insertsubsection\mysource{\lessonsofariane}}
			\mycite{Design by contract [\ldots] is the principle that interfaces between modules of a software system---especially a mission-critical one---should be governed by precise specifications, similar to contracts between humans or companies.}
		\end{definition}
		\nextcolumn
		\begin{example}{Gentlemen in the Bathroom by Loriot\mysource{\href{https://de.wikipedia.org/wiki/Herren_im_Bad}{wikipedia.org}}}
			\centering
			\pic[width=\linewidth,trim=0 500 0 0,clip]{designbycontract/herren-im-bad}
			
			Mr. Müller-Lüdenscheidt to Dr. Klöbner:\\
			\mycite{If you let the duck in, I'll let the water out.}
		\end{example}
	\end{fancycolumns}
\end{frame}

\subsection{Behavioral Interface Specification Languages}
\begin{frame}{\insertsubsection{} \mytitlesource{\hatcliff}}
	\begin{fancycolumns}[widths={30}]
		\begin{note}{Motivation}
			\begin{itemize}
				\item software systems typically composed of reusable components (e.g., libraries)
				%\item software contracts can be used to specify the interfaces of those components
				\item increasing scale of software leads to multiple teams and more reuse
				\item increase reuse comes with challenges when updating libraries
				\item safety-critical systems need intended behavior for verification %and blame assignment
			\end{itemize}
		\end{note}
	\nextcolumn
		\begin{definition}{\insertsubsection}
			\begin{itemize}
				\item behavorial specification: \mycite{precise description of the intended behavior of some computing system or its components}
				\item formal specification: \mycite{mathematically precise notation for recording intended properties of software}
				\item formalization makes specifications unambiguous + \mycite{enables tools to provide automated reasoning about specifications and their relationship to associated code}
				\item interface specification: \mycite{syntactic interfaces of [...] modules, such as the names and types of methods}
				\item behavioral interface specification: formal specification of an interface
			\end{itemize}
		\end{definition}
		\begin{example}{}
			\begin{itemize}
				\item Java Modeling Language (JML) for Java
				\item Spec\# as language extension to C\#
				\item Eiffel has built-in support
				\item SPARK as subset + extension of Ada
			\end{itemize}
		\end{example}
	\end{fancycolumns}
\end{frame}

\subsection{Java Modeling Language (JML)}
\begin{frame}{\insertsubsection}
	\begin{fancycolumns}
		\begin{exampletight}{Documenting Assumptions in JML}
			\centering\makebox{\usebox{\jml}} % TODO listing has weird color for "\r"
		\end{exampletight}
		\nextcolumn
		\begin{definition}{\insertsubsection{} \mysource{\designofjml}}
			\begin{itemize}
				\item principle: a behavioral interface specification language for Java
				\item \emph{\lstinline|requires|}: precondition that needs to hold when method is called
				\item \emph{\lstinline|ensures|}: postcondition that needs to hold when method returns
				\item \emph{\lstinline|invariant|}: condition that needs to hold after object creation as well as before and after each method call
				\item special keywords: \emph{\lstinline|\\result|}, \emph{\lstinline|\\old|}, \ldots
				\item special operators: \emph{\lstinline|<==>|}, \emph{\lstinline|==>|}, \ldots
				\item methods can only be called in JML specification if pure (i.e., free of side effect)
			\end{itemize}
		\end{definition}
	\end{fancycolumns}
\end{frame}

\begin{frame}{A Non-Trivial JML Example}
	\begin{fancycolumns}[widths={45}]
		\begin{exampletight}{A Non-Trivial JML Example \mysource{\designofjml}}
			\centering\makebox{\usebox{\jmlisqrt}}
		\end{exampletight}
		\nextcolumn
		\begin{definition}{\insertsubsection{} cont.{} \mysource{\designofjml}}
			\begin{itemize}
				\item \emph{\lstinline|public|}: visibility of the contract
				\item \emph{\lstinline|normal_behavior|}: method returns normally and no exception is thrown
				\item \emph{\lstinline|assignable|}: which locations can be modified
				\item special keywords: \emph{\lstinline|\\nothing|} (for an empty list)
				\item many more keywords
			\end{itemize}
		\end{definition}
		\begin{note}{JML Tools}
			\begin{itemize}
				\item language differs among tools: OpenJML, KeY
				\item many tools discontinued: KeY's Eclipse extension, ESC/Java2, JMLUnit, Krakatoa/Why, jmlc
				\item no web tools available yet
			\end{itemize}
		\end{note}
	\end{fancycolumns}
\end{frame}

\begin{frame}{\insertsubsection}
	\begin{definition}{Summary on Techniques to Document Developers' Assumptions}
		\centering\renewcommand{\arraystretch}{1.5}
		\begin{tabular}{cccccc}
			\toprule
			Technique & Accessible & Unambiguous & Automated Checks & Overhead / Side-Effects \\
			\midrule
			Code Documentation & yes & no & no & no \\
			Defensive Programming & no & yes & yes & yes \\
			Runtime Assertions & no & yes & yes & only debugging \\
			Unit Testing & no & yes & yes & no \\
			\only<1|handout:0>{??? & yes & yes & yes & no}
			\only<2->{Design by Contract & \only<3->{yes} & \only<4->{yes} & \only<5|handout:0>{?}\only<6->{yes} & \only<7->{only debugging / no}} \\
			\bottomrule
		\end{tabular}
	\end{definition}
	\uncover<5->{\begin{note}{Automatically Checking Contracts?}
			\begin{itemize}
				\item JML compiler: translates JML specifications into Java assertions
				\item JML test-case generation: translates JML specifications into JUnit test cases
				\item formal verification: verifies that code fulfills its specification (e.g., given any valid input, a method produces a valid output)
			\end{itemize}
	\end{note}}
\end{frame}

% TODO add slides on blame assignement?

% TODO add success story of TimSort? https://link.springer.com/article/10.1007/s10817-017-9426-4