\subsection{Ideal Documentation of Assumptions}
\begin{frame}{\insertsubsection}
	\begin{definition}{Summary on Techniques to Document Developers' Assumptions}
		\centering\renewcommand{\arraystretch}{1.5}
		\begin{tabular}{cccccc}
			\toprule
			Technique & Accessible & Unambiguous & Automated Checks & Overhead / Side-Effects \\
			\midrule
			Code Documentation & yes & no & no & no \\
			Defensive Programming & no & yes & yes & yes \\
			Runtime Assertions & no & yes & yes & only debugging \\
			Unit Testing & no & yes & yes & no \\
			??? & yes & yes & yes & no \\
			\bottomrule
		\end{tabular}
	\end{definition}
	\only<2->{\begin{note}{Motivation}
			\begin{itemize}
				\item no technique that is accessible during development and unambiguous
				\item further wanted: automated checks, no runtime overhead, no (or controllable) side-effects
			\end{itemize}
	\end{note}}
\end{frame}

\subsection{Design by Contract}
\begin{frame}{\insertsubsection}
	\begin{fancycolumns}
		\begin{definition}{\insertsubsection\mysource{\lessonsofariane}}
			\mycite{Design by contract [\ldots] is the principle that interfaces between modules of a software system---especially a mission-critical one---should be governed by precise specifications, similar to contracts between humans or companies.}
		\end{definition}
		\nextcolumn
	\end{fancycolumns}
\end{frame}

\subsection{}
\begin{frame}{\insertsubsection}
	\begin{fancycolumns}
		\begin{definition}{...}
			\begin{itemize}
				\item ...
			\end{itemize}
		\end{definition}
		\nextcolumn
	\end{fancycolumns}
\end{frame}

\subsection{}
\begin{frame}{\insertsubsection}
	\begin{fancycolumns}
		\begin{definition}{...}
			\begin{itemize}
				\item ...
			\end{itemize}
		\end{definition}
		\nextcolumn
	\end{fancycolumns}
\end{frame}

\subsection{}
\begin{frame}{\insertsubsection}
	\begin{fancycolumns}
		\begin{definition}{...}
			\begin{itemize}
				\item ...
			\end{itemize}
		\end{definition}
		\nextcolumn
	\end{fancycolumns}
\end{frame}

\begin{frame}{\insertsubsection}
	\begin{fancycolumns}[t]
		\begin{exampletight}{Logical Error}
			\centering\makebox{\usebox{\edgelogicalerror}}
		\end{exampletight}
		\uncover<2->{\begin{note}{No explicit error during compilation or execution}
				error can be identified by inspecting the program's output or by means of assertions
		\end{note}}
		\nextcolumn
		\uncover<3->{\begin{exampletight}{Fixed Program}
				\centering\makebox{\usebox{\edgelogicalerrorfix}}
		\end{exampletight}}
		\uncover<4->{\begin{definition}{Explanation}
				\begin{itemize}
					\item no explicit exception as for runtime errors
					\item tests need comparison with expected outcomes
				\end{itemize}
		\end{definition}}
	\end{fancycolumns}
\end{frame}
