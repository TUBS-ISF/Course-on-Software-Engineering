\pgfsetlayers{very-background,background,main,middle,foreground}
\colorlet{black}{foreground}
\colorlet{white}{background}
\ifdarkmode
\colorlet{darkgray}{black!80!background}
\colorlet{lightgray}{black!40!background}
\fi
\tikzset{
    bottom note/.style={font=\tiny,color=gray},
    path image shift/.style={}, % scale patch just for white padding
    path image/.style={path picture={\node[scale=.98] at ([path image shift]path picture bounding box.center) {#1};}}
}
\colorlet{ca@base@color}{foreground}
\def\I#1#2{\footnotesize\absexpr{\IntCC{#1}{#2}}}

\newsavebox\SimpleSignLattice
\begin{lrbox}{\SimpleSignLattice}
\scriptsize
\begin{tikzpicture}[line cap=round,x=6.5mm,y=6.5mm]
   \node (top) at (0,0) {\absexpr{\top}};
   \node (pos) at (-1,-1) {\absexpr{\geq 0}};
   \node (neg) at (1,-1) {\absexpr{\leq 0}};
   \node (zero) at (0,-2) {\absexpr{0}};
   \node (bot) at (0,-3) {\absexpr{\bot}};
   \draw (top) -- (pos) -- (zero) -- (neg) -- (top) (zero) -- (bot);
\end{tikzpicture}
\end{lrbox}

\mode
<presentation>

\newsavebox\MyImageBox

\begin{frame}{Who is this?}
\savebox\MyImageBox{\pic[width=3cm]{people/florian-sihler}}%
\begin{uncoverenv}<2->   
   \begin{tikzpicture}[overlay,remember picture]
      \draw[path image=\usebox\MyImageBox,rounded corners=3mm] ([xshift=-2mm,yshift=-2mm]current page.north east) rectangle ++(-.9\wd\MyImageBox,-.9\ht\MyImageBox-.9\dp\MyImageBox);
   \end{tikzpicture}
\end{uncoverenv}
\larger
\begin{itemize}
   \itemsep12pt
   \item<3-> Florian Sihler\\\(1\,1/2\) years into my PhD
   \item<4-> Working at Ulm University\\Institute of Softwareengineering and Programming Languages
   \item<5-> Research on hybrid analysis to improve program comprehension of dynamic languages
\end{itemize}
\vspace*{4em}
\begin{center}
   \onslide<6->{Any Questions?}\\
   \onslide<6->{\href{mailto:florian.sihler@uni-ulm.de}{florian.sihler@uni-ulm.de}}
\end{center}
\end{frame}
\mode
<all>

\subsection{The Why}
\subsubsection[Motivation]{Initial Motivation}
\begin{frame}[fragile,label=main-example]{\insertsection}
\lstfs{10}%
\AnimateCode{onslide={o2:{3,...,7},-,-,-,-,-,-},handout=2/1,first slide=2}
\begin{minted}[escapeinside=||,lineskip=1pt]{java}
public static void main(String[] args) {
    int a = 1; |\tikzmarknode{@a1}{\strut}|
    double r = Math.random() * 10; |\tikzmarknode{@r1}{\strut}|
    if (r > 5) { |\tikzmarknode{@r2}{\strut}|
       a = 2; |\tikzmarknode{@a2}{\strut}|
    }
    System.out.println(a); |\tikzmarknode{@a3}{\strut}|
}
\end{minted}
\endAnimateCode
\begin{tikzpicture}[overlay,remember picture]
   \onslide<3->{%
      \coordinate[yshift=2pt] (@a1) at (pic cs:@a1);
      \node[right=4.33cm] (@a1) at (@a1) {\AbstractInfo{a \in \Set{1}}};
   }%
   \onslide<4->{%
      \coordinate[yshift=2pt] (@r1) at (pic cs:@r1);
      \node[right] at (@r1-|@a1.west) {\AbstractInfo{r \in \IntCO{0}{10}}};
   }%
   \onslide<5->{%
      \coordinate[yshift=2pt] (@a2) at (pic cs:@a2);
      \node[right] at (@a2-|@a1.west) {\AbstractInfo{a \in \Set{2}}};
   }%
   \onslide<6->{%
      \coordinate[yshift=2pt] (@a3) at (pic cs:@a3);
      \node[right] (@set) at (@a3-|@a1.west) {\AbstractInfo{a \in \Set{1, 2}}};
      \onslide<7->{%
         \node[right] at (@set.east) {\(\to\)~\;Valid? Ok? Safe?};
      }
   }
\end{tikzpicture}

\begin{uncoverenv}<8->
\begin{definition}{Static Analysis\hfill\small\cite{rival2020introduction}}
   Discover semantic properties of a program without running it.
\end{definition}
\end{uncoverenv}
\end{frame}

\subsubsection{Origins}
\larger\def\Mixin{}%
\tikzset{%
   history-line/.style={line width=1.85mm,gray!30!background\Mixin,line cap=round,rounded corners=2pt},
   history-line skip/.style={history-line, line width=.75mm,loosely dotted},
   history-event/.style={history-line,gray\Mixin,line width=1mm,{Circle[length=1.85mm]}-,shorten <= -1.85mm/2},
   history@box/.style={yshift=.675\baselineskip,foreground\Mixin,text width=5.75cm,font=\small},
   history-range/.style={history-line, gray!60!background\Mixin,line cap=round,-{Triangle Cap}}   
}

% #1 left/right
% #2 when
% #3 what
% #4 optional comment
\def\historybox#1#2#3#4{node[history@box,below #1] (@) {\textbf{#2}: #3\ifx!#4!\else\\\footnotesize\itshape#4\par\fi}}
\begin{frame}[c]{\insertsubsubsection}
\centering\vspace*{-12.5mm}\begin{tikzpicture}
   \only<-4|handout:1>{\draw[history-line] (-.33,-.33) -- ++(.33,.33) -- ++(2,0) coordinate (@2)++(1,0) -- ++(9,0) node[above left,gray] {Static Analysis };}
   \onslide<5|handout:0>{
      \draw[history-line] (-.33,-.33) -- ++(.33,.33) -- ++(2,0) coordinate (@2)++(1,0) -- ++(0.5,0) coordinate (@l) -- ++(1,3) -- ++(7.5,0) node[above left,lightgray,yshift=-3pt] {\vphantom{y}Deductive Methods};
      \draw[history-line] (@l) -- ++(1,1) -- ++(7.5,0)  node[above left,lightgray,yshift=-3pt] {Model Checking};
      \draw[history-line] (@l) -- ++(1,-1) -- ++(7.5,0) node[above left,lightgray,yshift=-3pt] {Symbolic Execution};  % 
      \draw[history-line] (@l) -- ++(1,-3) -- ++(7.5,0) node[above left,lightgray,yshift=-3pt] {Abstract Interpretation};
   }
   \only<6-|handout:2->{
      \draw[history-line] (-.33,-.33) -- ++(.33,.33) -- ++(.5,0) coordinate (@2)++(1,0) -- ++(0.5,0) coordinate (@l) -- ++(1,3) -- ++(9,0) node[above left,lightgray,yshift=-3pt] {\vphantom{y}Deductive Methods};
      \draw[history-line] (@l) -- ++(1,1) -- ++(9,0) node[above left,lightgray,yshift=-3pt] {Model Checking};
      \draw[history-line] (@l) -- ++(1,-1) -- ++(9,0) node[above left,lightgray,yshift=-3pt] {Symbolic Execution};
      \draw[history-line] (@l) -- ++(1,-3) -- ++(9,0) node[above left,lightgray,yshift=-3pt] {Abstract Interpretation};
   }
   \draw[history-line skip] (@2)++(.15,0) -- ++(.85,0);
   \begin{onlyenv}<-5|handout:1>
\pause
   \draw[history-event] (.5,0) -- ++(.25,2) -- ++(.25,0) \historybox{right}{1949}{First Checks}{\citeauthor*{turing1989checking}~\cite{turing1989checking}};
\pause
   \draw[history-event] (1,0) -- ++(.25,1.15) -- ++(.25,0) \historybox{right}{1953}{Rice's Theorem}{Non-trivial Properties\\are undecidable~\cite{rice1953classes}};
\pause
   \draw[history-event] (1.5,0) -- ++(.25,-.45) -- ++(.25,0) \historybox{right}{1967\,\&\,69}{Logical Foundation}{\citeauthor*{floyd1967assigning}~\cite{floyd1967assigning}, \citeauthor*{DBLP:journals/cacm/Hoare69}~\cite{DBLP:journals/cacm/Hoare69}\\But: No Automation};
   \end{onlyenv}
\only<6-|handout:2->{%
   \node[above right,lightgray] at(0,0) {\footnotesize\cite{turing1989checking,rice1953classes}};
   \node[below right,lightgray] at(0,-1pt) {\footnotesize\cite{floyd1967assigning,DBLP:journals/cacm/Hoare69}};
}
% DBLP:conf/cade/OwreRS92
\begin{onlyenv}<7-|handout:2->
   \tikzset{@/.style={}}\only<9-|handout:0>{\tikzset{@/.style={lightgray}}}
   \draw[history-event,@] (4,3) -- ++(.25,1.45) -- ++(.25,0) \historybox{right,@}{1992}{Theorem Prover}{PVS, \citeauthor*{DBLP:conf/cade/OwreRS92}~\cite{DBLP:conf/cade/OwreRS92}};
   \onslide<8->{%
   \draw[history-event,@] (4.4,3) -- ++(.25,.75) -- ++(.25,0) \historybox{right,@}{2004}{Proof Asisstant}{Coq, \citeauthor*{DBLP:series/txtcs/BertotC04}~\cite{DBLP:series/txtcs/BertotC04}}; % isabelle, agda, ...
   }
   \tikzset{@/.style={}}\only<11-|handout:0>{\tikzset{@/.style={lightgray}}}
   % DBLP:journals/toplas/ClarkeES86
   \onslide<9->{%
   \draw[history-event,@] (3.5,1) -- ++(.25,1.45) -- ++(.25,0) \historybox{right,@}{1986}{Foundations}{\citeauthor*{DBLP:journals/toplas/ClarkeES86}~\cite{DBLP:journals/toplas/ClarkeES86}};
   }
   \onslide<10->{%
   \draw[history-event,@] (4.4,1) -- ++(.25,.75) -- ++(.25,0) \historybox{right,@}{2004}{Bounded MC}{\citeauthor*{clarke2004tool}~\cite{clarke2004tool}};
   }
   \tikzset{@/.style={}}\only<13-|handout:0>{\tikzset{@/.style={lightgray}}}
   \onslide<11->{%
   \draw[history-event,@] (3.15,-1) -- ++(.25,1.45) -- ++(.25,0) \historybox{right,@}{1974\,\&\,75}{Foundations}{\citeauthor*{DBLP:conf/relsoft/BoyerEL75}~\cite{DBLP:conf/relsoft/BoyerEL75}, \citeauthor*{DBLP:conf/ibm/King74}~\cite{DBLP:conf/ibm/King74}};
   }
   \onslide<12->{%
   \draw[history-event,@] (4.75,-1) -- ++(.25,.75) -- ++(.25,0) \historybox{right,@}{2008}{Automation}{KLEE, \citeauthor*{DBLP:conf/osdi/CadarDE08}~\cite{DBLP:conf/osdi/CadarDE08}};
   }
   \onslide<13->{%
      \draw[history-event] (3.2,-3) -- ++(.25,1.45) -- ++(.25,0) \historybox{right}{1977}{Fixpoints on Lattices}{\citeauthor{DBLP:conf/popl/CousotC77}~\cite{DBLP:conf/popl/CousotC77}};
   }
   
   \onslide<14->{%
      \draw[history-event] (4.4,-3) -- ++(.25,.75) -- ++(.25,0) \historybox{right}{2004}{Automated Application}{\citeauthor{DBLP:conf/ifip/Mauborgne04}~\cite{DBLP:conf/ifip/Mauborgne04}};
   }
\end{onlyenv}
   \onslide<1->
\end{tikzpicture}
\begin{tikzpicture}[overlay,remember picture]
   \node[above right,gray,yshift=3.5mm,font=\tiny,text width=.9\paperwidth] at (current page.south west) {Based on the amazing \citetitle{DBLP:journals/ftpl/Mine17} by \citeauthor{DBLP:journals/ftpl/Mine17}~\cite{DBLP:journals/ftpl/Mine17}, \href{https://web.archive.org/web/20241208213653/https://www.di.ens.fr/~cousot/AI/}{https://www.di.ens.fr/\textasciitilde cousot/AI/}, and \cite{DBLP:journals/csur/BaldoniCDDF18,DBLP:journals/annals/GiacobazziR22}};
\end{tikzpicture}
\end{frame}

% https://www.youtube.com/watch?v=IBlfJerAcRw&t=2624s
\begin{frame}{Recommended Resources}
\strut\hfill\raisebox{\dimexpr-\height+1cm}{\begin{tikzpicture}
   \onslide<2->{%
      \draw[darkgray,thick,rounded corners=2pt,path image={\pic[width=4cm]{books/itsa-cover}}] (0,0) rectangle ++(3.8,5);
      \node[above] at(current bounding box.north) {\clap{\small\strut Using Analyses~\cite{rival2020introduction}}};
   }
\end{tikzpicture}}\hfill\raisebox{\dimexpr-\height+1cm}{\begin{tikzpicture}
   \onslide<3->{%
      \draw[darkgray,thick,rounded corners=2pt,path image={\pic[width=4cm]{books/poa-cover}}] (0,0) rectangle ++(3.9,5.5);
      \node[above] at(current bounding box.north) {\clap{\small\strut Formal Foundations~\cite{cousout2021principles}}};
   }
\end{tikzpicture}}\hfill\raisebox{\dimexpr-\height+1cm}{\begin{tikzpicture}
   \onslide<4->{%
      \draw[darkgray,thick,rounded corners=2pt,path image={\pic[width=4cm]{books/dfa-cover}}] (0,0) rectangle ++(3.9,6);
      \node[above] at(current bounding box.north) {\clap{\small\strut Dataflow Perspective~\cite{10.5555/1592955}}};
   }
\end{tikzpicture}}\hfill\strut
\begin{tikzpicture}[overlay,remember picture]
   \node[above right=0.5mm,yshift=3mm,gray,font=\tiny] at (current page.south west) {And for an overview: \citetitle{DBLP:journals/ftpl/Mine17}~\cite{DBLP:journals/ftpl/Mine17}};
\end{tikzpicture}
\end{frame}

% \subsection{Areas of Application}

% \begin{frame}{\insertsubsection}
%    \begin{itemize}[<+(1)->]
%       \itemsep12pt
%       \item Linting
%       \item Program Verification
%       \item Code Optimization
%       \item Refactoring
%       \item Program Comprehension
%       \item Program Synthesis
%       \item \ldots
%    \end{itemize}
% \end{frame}



\usetikzlibrary{backgrounds,graphs,arrows.meta,decorations.pathreplacing}

\definecolor{BaseGray}{RGB}{66,66,66} % rgb(66,66,66)

\colorlet{SoftGray}{BaseGray!40}
\colorlet{BackGray}{BaseGray!5}
\colorlet{SoftTextGray}{BackGray!60!SoftGray}


\tikzset{
   Soft/.style={line join=round,line cap=round},
   All Soft/.style={every path/.append style={Soft}},
   FunctionDef/.style={
      draw=BaseGray,
      fill=BaseGray,
      minimum width=1.55cm,
      minimum height=1cm,
      text=foreground,
      font=\bfseries,
      text centered,
      inner sep=0pt,
      rounded corners=1mm,
      outer sep=2pt
   },
   Blob/.style={
      draw=SoftGray,
      fill=SoftGray,
      minimum size=4mm,
      text=foreground,
      circle,
      font=\bfseries,
      text centered,
      inner sep=0pt,
      outer sep=2pt
   },
   Def/.style={
      Blob,
      rectangle, rounded corners=1mm
   },
   ActiveBlob/.style={
      Blob,
      draw=BaseGray!80!foreground, fill=BaseGray!80!foreground
   },
   FunctionBack/.style={
      fill=BackGray,
      % draw=SoftGray,
      rounded corners=2mm,
      rectangle
   },
   href/.style={
      draw=SoftGray,
      line width=1.5pt,
      line cap=round,
      line join=round,
      -%
   },
   Funchref/.style={
      href,
      draw=SoftGray,
      dotted
   },
   Cursor/.style={
      fill=BackGray,
      draw=BaseGray,
      line join=round,
      line cap=round
   },
   Hover-Over/.style={
      fill=BackGray,
      draw=SoftGray,
      opacity=.5,
      draw opacity=1,
      rounded corners=1mm,
      line join=round,
      line cap=round
   },
   Line-Of-Text/.style={
      fill=#1,
      draw=none,
      rounded corners=1.5pt,
      inner sep=1pt,
      minimum width=1cm,
      minimum height=6.5pt
   },
   Input-Base/.style={
      fill=BackGray,
      draw=SoftGray,
      rounded corners=1mm,
      inner sep=1pt,
      minimum width=2cm,
      minimum height=12pt
   },
   % code sub-styles
   A/.style={Line-Of-Text=SoftTextGray},
   B/.style={},
   C/.style={Line-Of-Text=SoftGray},
   path image shift/.style={},
   path image/.style 2 args={path picture={\node at ([path image shift]path picture bounding box.center) {\pic[width=#2]{#1}};}},
}

\newsavebox\CodeFile
\begin{lrbox}{\CodeFile}
\scalebox{1.45}{%
\begin{tikzpicture}
   \draw[rounded corners=1.5pt,fill=foreground] (0,0) |- ++(.6,-.8) [sharp corners] -- ++(0,.6) -- ++(-.2,.2) coordinate (@rl) [rounded corners=2pt] -- cycle (@rl) |- ++(.2,-.2);
   \draw[thick,line cap=round,lightgray] (.1,-.1) -- ++(.2,0)
      % for what have I written random code generation? :C 
      (.1,-.15) -- ++(.1,0) ++(.05,0) -- ++(.1,0)
      (.125,-.2) -- ++(.15,0)++(.05,0)--++(.025,0)
      (.1,-.25) -- ++(.3,0)
      (.125,-.3) --++(.2,0)++(.05,0)--++(.05,0)
      (.125,-.35) --++(.15,0)++(.05,0)--++(.1,0)
      (.15,-.4)--++(.2,0)
      (.15,-.4)--++(.1,0)++(.05,0)--++(.05,0)
      (.125,-.45)--++(.05,0)
      (.1,-.5)--++(.066,0)++(.05,0)--++(.15,0)
      (.1,-.6)--++(.15,0)++(.05,0)--++(.15,0)
      (.1,-.65)--++(.2,0)
      (.1,-.7)--++(.1,0)++(.05,0)--++(.15,0)
   ;
\end{tikzpicture}}
\end{lrbox}

\newcommand\Back[4][]{
   \pgfonlayer{background}
   \fill[FunctionBack,#1] ([xshift=-2mm,yshift=-2mm]#2.south west) rectangle ([xshift=2mm,yshift=2mm]#3.north east);
   \coordinate (#4@north) at([yshift=2mm]0,0|-#3.north);
   \coordinate (#4@west) at([xshift=-2.5mm]0,0-|#2.west);
\endpgfonlayer
}
\newsavebox\UiBox
\begin{lrbox}{\UiBox}
\begin{tikzpicture}
   \node[FunctionDef] (F) at (0,0) {};

   \scope[shift={(F.south)},yshift=-1.5cm]
      \node[Def] (a1) at (-1,0) {};
      \node[Blob] (b1) at (.5,-.25) {};
      \node[ActiveBlob] (c1) at (1.55,-1.35) {};
      \node[Blob] (d1) at (-0.5,-1.6) {};
      \node[Blob] (e1) at (1.8,.25) {};
      \node[Blob] (f1) at (-2.2,-1.5) {};

      \graph[edges={href}] {
         (a1) -> { (b1), (c1) } -> (d1) -> (e1) -> (a1),
         (f1) -> { (a1), (d1) }
      };

      \Back{f1}{e1}{f1}

      \draw[Funchref] (f1@north) -- (F.south);
   \endscope

   \node[Blob] (a) at (2.5,0) {};
   \node[Blob] (b) at (1.5,1) {};
   \node[Blob] (c) at (3,2) {};
   \node[Blob] (d) at (-2.5,0.5) {};
   \node[Def] (e) at (-2.5,1.5) {};
   \node[Blob] (f) at (-3,2.5) {};
   \node[Blob] (g) at (-3.5,-1) {};
   \node[Blob] (h) at (-3,-2.5) {};

   \node[FunctionDef] (F2) at(5,-.5) {};
   \node[Blob] (u) at(5.5,1) {};

   \scope[shift={(F2.east)},xshift=1.5cm]

      \node[Blob] (a2) at (0,1.15) {};
      \node[Def] (b2) at (.75,1.85) {};
      \node[Blob] (c2) at (-.5,-.85) {};

      \node[Blob] (d2) at (1.33,0.25) {};
      \node[FunctionDef] (e2) at (2,-.75) {};

      \node[Blob] (f2) at (3.75,1.5) {};
      \node[Blob] (g2) at (2.75,2.5) {};

      \scope[shift=(e2.south), yshift=-1cm]
         \node[ActiveBlob] (a3) at (-1,0) {};
         \node[Blob] (b3) at (1,0) {};
         \draw[href] (a3) -- (b3);
         \Back[background]{a3}{b3}{a3} % for coordinate
      \endscope

      \coordinate (ll) at ([yshift=-2mm]a3.south west-|c2.west);
      \coordinate (ur) at (f2.north east|-g2.north);

      \Back{ll}{ur}{e2}
      \draw[Funchref] (e2@west) -- (F2.east);

      % draw inner later
      \Back[background]{a3}{b3}{xx} % for the overlay :D
      \draw[Funchref] (a3@north) -- (e2.south);

      \graph[edges={href}] {
         (a2) -> (b2) -> { (c2) , (d2) },
         (d2) -> { (e2), (f2) },
         (f2) -> (g2),
         (c2) -> (a3)
      };
   \endscope

   \graph[edges={href}] {
      (F) -> { (a), (b) },
      (a) -> (e1),
      (b) -> (c) -> (b2),
      (b) -> (d) -> { (F), (e) },
      (e) -> (f) -> (c),
      (d) -> (g) -> (h) -> (f1),
      (u) -> (F2)
   };

   \coordinate (cursor-pos) at ([xshift=-1.65mm,yshift=2mm]F2.south east);
   \draw[Cursor,rotate around={28:(cursor-pos)}]  [rounded corners=2.25pt] (cursor-pos)  [rounded corners=2.25pt] -- ++(4pt,-9.5pt) [rounded corners=2pt] -- ++(-4pt,3pt) [rounded corners=2.25pt]-- ++(-4pt,-3pt) -- cycle;
   \node[below left,xshift=-1mm,yshift=1.15mm,Hover-Over,minimum width=1.25cm,minimum height=6mm] (hoverover) at(cursor-pos) {%
      %
   };
   % \scope[opacity=.95,transparency group]
      \fill[Line-Of-Text=SoftGray] ([shift={(2pt,-2pt)}]hoverover.north west) rectangle ++(7.5mm,-4pt);
      \fill[Line-Of-Text=SoftTextGray] ([shift={(2pt,-7.5pt)}]hoverover.north west) rectangle ++(11mm,-3pt);
      \fill[Line-Of-Text=SoftTextGray] ([shift={(2pt,-7.5pt-4.5pt)}]hoverover.north west) rectangle ++(8mm,-3pt);
   % \endscope

   % TODO: outsource window?
   \coordinate (wul) at ([shift={(-5mm,6mm)}]current bounding box.north west);
   \coordinate (wur) at ([shift={(5mm,6mm)}]current bounding box.north east);
   \draw[thin,rounded corners=3pt,SoftGray] ([shift={(-5mm,-5mm)}]current bounding box.south west) rectangle (wur);
   % just "overlay" the top :D
   \filldraw[SoftGray] ([yshift=-1mm]wul) coordinate (@) -- (wur|-@) [rounded corners] |- ([yshift=2.5mm]wul) [sharp corners] -- cycle;

   \fill[Line-Of-Text=BackGray] ([yshift=1.35mm,xshift=4pt]wul) rectangle ++(2cm,-1.27mm);
   \node[above left,BackGray] at([yshift=-1.33mm,xshift=-1mm]wur) {\smash{\scalebox{.9}{\scriptsize\faAngleDown~~\faAngleUp~~\faTimesCircle}}};

   \coordinate (root control window) at(current bounding box.south west);

   \scope[shift={(current bounding box.south east)},shift={(-7cm,4.5mm)}]
      \coordinate (wul) at (0,0);
      \coordinate (wur) at (8.5,0);

      \draw[thin,rounded corners=3pt,SoftGray,fill=background] (0,-5) rectangle (wur);

      % code line
      \draw[thin,BackGray] ([xshift=1cm,yshift=-1mm]wul) coordinate (@) -- (@|-0,-5);
      % line numbers and code
      \foreach[count=\i from 0] \Code in {
         {1/A,0.5/A,5/C,2/A},
         {1.5/B,3/A,0.5/A,2/A},
         {},
         {1.5/B,0.5/A,3/A},
         {1/A},
         {},
         {5/A,0.5/A,2/A},%
         {0.5/A,5/A,3/A,0.5/A,2/A,5.5/A},
         {},
         {1/C,0.5/A,1/A,0.5/A,1/A,0.5/A,0.15/B,0.5/A},
         {1.5/B,3/A,0.5/A,1/C,2/A,1.5/A,1/C,1.5/A},
         {1.5/B,0.75/A,0.5/A,1.5/A,0.5/A,1/A},
         {1.5/B,3/A,0.5/A,2/A},
         {1.5/B,2/A,0.5/A,2/A},
         {1/A}%
      } {
         \ifnum\i<6 \def\Width{4mm} \else \def\Width{6mm} \fi
         \fill[Line-Of-Text=BackGray] ([yshift=-1mm,xshift=8mm]wul|-0,-\i*0.33*10mm+1mm) rectangle ++(-\Width,-2mm);
         \def\XShift{5}
         \foreach \CW/\Style in \Code {
            \ifstrequal{\Style}{B}{\def\RandomSuffix{0}}{\def\RandomSuffix{(rand*0.4mm+0.75mm)}}
            \pgfmathsetmacro\w{3*\CW mm+\RandomSuffix}
            \path[\Style] ([yshift=-1mm,xshift=1cm+\XShift pt]wul|-0,-\i*0.33*10mm+1mm) rectangle ++(\w pt,-2mm);

            \pgfmathsetmacro{\XShift}{\XShift+\w pt+1.5mm}
            \xdef\XShift{\XShift}
         }
      }

      % slider
      \draw[Line-Of-Text=SoftTextGray,rounded corners=.325mm] ([yshift=-4cm+5mm,xshift=-1mm]wur) rectangle ++(-.75mm,-1cm);

      % just redraw the frame :D
      \draw[thin,rounded corners=3pt,SoftGray] (0,-5) rectangle (wur);

      \filldraw[SoftGray] ([yshift=-1mm]wul) coordinate (@) -- (wur|-@) [rounded corners] |- ([yshift=2.5mm]wul) [sharp corners] -- cycle;

      \fill[Line-Of-Text=BackGray] ([yshift=1.35mm,xshift=4pt]wul) rectangle ++(1.66cm,-1.27mm);
      \node[above left,BackGray] at([yshift=-1.33mm,xshift=-1mm]wur) {\smash{\scalebox{.9}{\scriptsize\faAngleDown~~\faAngleUp~~\faTimesCircle}}};

   \endscope

   % window for controls
   \scope[shift={(root control window)},shift={(-1cm,-5mm)}]
      \draw[thin,rounded corners=3pt,SoftGray,fill=background] (0,-4) rectangle ++(8.33cm,4cm);
      \coordinate (wul) at (0,0);
      \coordinate (wur) at (8.33,0);

      \node[Input-Base,minimum width=6cm,below right=3mm] (slice-criterion-input) at (wul) {};
      \fill[Line-Of-Text=SoftTextGray] ([yshift=1mm,xshift=4pt]slice-criterion-input.west) rectangle ++(1.25cm,-2mm);
      \fill[Line-Of-Text=SoftTextGray] ([yshift=1mm,xshift=4pt+1.25cm+2mm]slice-criterion-input.west) rectangle ++(1cm,-2mm);

      % frame stuff
      \filldraw[SoftGray] ([yshift=-1mm]wul) coordinate (@) -- (wur|-@) [rounded corners] |- ([yshift=2.5mm]wul) [sharp corners] -- cycle;
      \node[right=1mm,SoftGray,] at(slice-criterion-input.east) {\tiny\faChevronRight~~~~~\faPieChart~~\faUpload~~\faDownload};

      % logging window
      \draw[rounded corners=3pt,BackGray,fill=background] ([xshift=3mm,yshift=-1cm]wul) rectangle ([xshift=-3mm,yshift=-3.7cm]wur);

      % log
      \foreach[count=\i from 0] \Code in {
         {1/A,2/A,6/A,1/B,3/A},
         {1.5/B,3/A,2/A,2/A,1/A},
         {1.5/B,2/A,3/A,1/A,2/A,3/A,1/A},
         {1.5/B,4/A,1/A},
         {1.5/B,3/A,2/A,2/A,1/A},
         {1.5/B,3/A,1/A,4/A,1/A},
         {},
         {1/A,2/A,6/A},
         {1/A,2/A,3/A,1.5/A,2.5/A},
      } {
         \def\XShift{5}
         \foreach \CW/\Style in \Code {
            \ifstrequal{\Style}{B}{\def\RandomSuffix{0}}{\def\RandomSuffix{(rand*0.4mm+0.75mm)}}
            \pgfmathsetmacro\w{3*\CW mm+\RandomSuffix}
            \path[\Style] ([yshift=-2.5mm-1cm,xshift=3.25mm+\XShift pt]wul|-0,-\i*0.28*10mm+1mm) rectangle ++(\w pt,-1.75mm);

            \pgfmathsetmacro{\XShift}{\XShift+\w pt+1mm}
            \xdef\XShift{\XShift}
         }
      }

      % slider
      \draw[Line-Of-Text=SoftTextGray,rounded corners=.325mm] ([yshift=-2cm+3mm,xshift=-4mm]wur) rectangle ++(-.75mm,-1cm);

      % head
      \fill[Line-Of-Text=BackGray] ([yshift=1.35mm,xshift=4pt]wul) rectangle ++(5mm,-1.27mm);
      \fill[Line-Of-Text=BackGray] ([yshift=1.35mm,xshift=4pt+6.5mm]wul) rectangle ++(1cm,-1.27mm);
      \node[above left,BackGray] at([yshift=-1.33mm,xshift=-1mm]wur) {\smash{\scalebox{.9}{\scriptsize\faAngleDown~~\faAngleUp~~\faTimesCircle}}};
   \endscope
\end{tikzpicture}
\end{lrbox}

\subsection{Conducting Static Analysis}
\begin{frame}{\strut What do they\ldots~do?}
\larger%
% they take input, textual, syntactical, semantic (call graphs, pdg, ...), metadata, historical information, requirements, annotations (types, contracts), ...
\bigskip

\hspace*{-3.5mm}
\begin{tikzpicture}[o/.style={outer sep=0pt,inner sep=0pt}]
   \onslide<2->{%
      \node[o] (@) at (0,0) {\usebox\CodeFile};
      \node[above=2.5mm,xshift=1.15mm,gray] at(@.north) {\tiny They take \textbf{\normalsize Input}};
   }
   \pgfonlayer{background}
   \onslide<2->{%
   \scope[transparency group,opacity=.4]
   \node[o,rotate around={-30:(@.south east)},anchor=south east] at(@.south east) {\usebox\CodeFile};
   \node[o,rotate around={-12:(@.south east)},anchor=south east] at(@.south east) {\usebox\CodeFile};
   \endscope}
   \endpgfonlayer
   \begin{uncoverenv}<3->
   \coordinate (@) at(@.east);
   \foreach[count=\i] \usecase/\targeti in {{\raisebox{1pt}{Textual}}/4,Syntactical/5, Semantical/6, Historical/7, Annotated/8, {\only<-9|handout:0>{\ldots}\only<10->{\raisebox{-3pt}{Metadata,~\ldots}}}/9} { % program spectra, hardware, contexts, ...
   \pgfmathsetmacro\rot{-24*\i+66}
      \onslide<\targeti->{
         \path ([xshift=.5mm]@.east)++(\rot+10:1mm) coordinate (@a);
         \fill[opacity=.18,gray] (@a.east) -- ++(\rot:1.5cm) arc (\rot:\rot+20:1.5cm) -- cycle;
         \draw[thick,gray] (@a.east)++(\rot:1.5cm) arc (\rot:\rot+20:1.5cm);
         \path (@a.east) -- ++(1.05*\rot+10:1.6cm) node[right,font=\small,darkgray] (@uc-\i) {\vphantom{a}\smash{\usecase}};
      }
   }
   \node[above=1.65mm,xshift=1mm,gray] at(current bounding box.north) {\tiny And use \textbf{\normalsize Perspectives} \rlap{(often combined)}};
   \end{uncoverenv}
   \onslide<11->{%
      \draw[Kite-,gray] ([xshift=6.25mm,yshift=-1mm]current bounding box.south) to[out=-90,in=0] ++(-4.5mm,-5mm) node[below left,yshift=.42\baselineskip,align=right,text width=2.5cm,font=\tiny] {Some of those are the result of other static or dynamic analyses};
   }
   \begin{uncoverenv}<12->
      \node[right,yshift=-2mm,xshift=1cm,align=left,font=\small,darkgray,text width=3.25cm] (@techn) at(current bounding box.east){{\onslide<13->{\subnode{tc-search}{Text/Code Search}\strut}}~\\[4mm]\strut{\onslide<14->{\subnode{clustering}{Clustering}}}~\\[4mm]\strut{\onslide<15->{\subnode{ai}{Abstract Domains}}}~\\[4mm]\strut{\onslide<16->{\subnode{df-constraints}{Dataflow Constraints}}}~\\[1mm]\strut{\onslide<16->{\centerline{\footnotesize\(\vdots\)}}}};
      \onslide<12->{
         \node[above=2.5mm,gray,xshift=-4.5mm] at(@techn.north) {\tiny To apply \textbf{\normalsize Theory}};
      }
   \end{uncoverenv}
   \scope[gray,line cap=round]
   \only<17->{
      \draw ([yshift=1.5pt]@uc-1.east) -- ([yshift=-2.5mm]@techn.north west);
      \draw ([yshift=1.5pt]@uc-2.east) -- ([yshift=-2.5mm]@techn.north west);
      \draw[densely dotted] ([yshift=1.5pt]@uc-3.east) -- ([yshift=-2.5mm]@techn.north west);
   }
   \only<18->{
      \draw ([yshift=1.5pt]@uc-1.east) -- ([yshift=-10.5mm]@techn.north west);
      \draw ([yshift=0pt]@uc-2.east) -- ([yshift=-10.5mm]@techn.north west);
      \draw ([yshift=1.5pt]@uc-4.east) -- ([yshift=-10.5mm]@techn.north west);
      
      \draw ([yshift=-1pt]@uc-2.east) -- ([yshift=-18.5mm]@techn.north west);
      \draw ([yshift=-1pt]@uc-3.east) -- ([yshift=-18.5mm]@techn.north west);
      \draw ([yshift=1.5pt]@uc-5.east) -- ([yshift=-18.5mm]@techn.north west);
      
      \draw ([yshift=-1pt]@uc-2.east) -- ([yshift=-26mm]@techn.north west);
      \draw ([yshift=-1pt]@uc-3.east) -- ([yshift=-26mm]@techn.north west);
   }
   \endscope
   \begin{uncoverenv}<19->
      
      \onslide<20->{\node[right=7.5mm] (@) at(current bounding box.east) {\resizebox*!{2.85cm}{\usebox\UiBox}};
      % coordinate based set fails on tubs slides :/
      \scope[transparency group,opacity=.5,every path/.append style={line cap=round,line width=.5pt}]
         \draw[-Kite,gray] ([yshift=-2.5mm,xshift=-11.5mm]@techn.north east) to[out=0,in=180] ([xshift=1.5mm,yshift=-6.35mm]@.west);
         \draw[-Kite,gray] ([yshift=-8.5mm,xshift=-22.5mm]@techn.north east) to[out=0,in=180] ([xshift=7.15mm,yshift=-1.35mm]@.west);
         \draw[-Kite,gray] ([yshift=-16.5mm,xshift=-10.25mm]@techn.north east) to[out=0,in=180] ([xshift=26.15mm,yshift=-9.35mm]@.west);
         \draw[-Kite,gray] ([yshift=-22mm,xshift=-8.5mm]@techn.north east) to[out=0,in=180] ([xshift=26.15mm,yshift=-9.35mm]@.west);
      \endscope
      }
      \node[above, gray] at(@.north) {\tiny And \textbf{\normalsize Communicate} or \textbf{\normalsize Use} results};
   \end{uncoverenv}
\end{tikzpicture}
\end{frame}


\newsavebox\SimpleSyntacticalPerspective
\begin{lrbox}{\SimpleSyntacticalPerspective}
\begin{forest}
   for tree={draw,rounded corners=3pt, l sep=0, l sep+=-1.5mm, s sep=4pt,execute at begin node={\strut},execute at end node={\kern1pt\null}}
   [Program
      [\bjava{=}
         [\bjava{int}]
         [\bjava{x}]
         [\bjava{0}]
      ]
      [\bjava{while}
         [\bjava{<}
            [\bjava{x}]
            [\bjava{2}]
         ]
         [Block
            [\bjava{=}
               [\bjava{x}]
               [\bjava{+}
                  [\bjava{x}]
                  [\bjava{1}]
               ]
            ]
         ]
      ]
   ]
\end{forest}
\end{lrbox}
\newsavebox\SimpleSemanticalPerspective
\begin{lrbox}{\SimpleSemanticalPerspective}
\begin{tikzpicture}[every node/.append style={execute at begin node={\vphantom{123456789}},circle,draw}]
   \node (1) at (0,0) {1};
   \node[below=3.5mm] (2) at (1.south) {2};
   \node[below=3.5mm] (3) at (2.south) {3};
   \node[right=2.5mm,double] (4) at (2.east) {4};
   \draw[-Kite] (1) -- (2);
   \draw[-Kite] (2) to[bend right=33] (3);
   \draw[-Kite] (3) to[bend right=33] (2);
   \draw[-Kite] (2) -- (4);
\end{tikzpicture}
\end{lrbox}
\newsavebox\SimpleDataflowPerspective
\begin{lrbox}{\SimpleDataflowPerspective}
\begin{tikzpicture}[every node/.append style={execute at begin node={\vphantom{123456789}},circle,draw}]
   \node[rectangle,inner sep=7pt,rounded corners=2pt] (1) at (0,0) {\texttt{x}\textsubscript{1}};
   \node[below=3.5mm] (2) at (1.south) {\texttt{x}\textsubscript{2}};
   \node[below=3.5mm,rectangle,inner sep=7pt,rounded corners=2pt] (3) at (2.south) {\texttt{x}\textsubscript{3}};
   \node[right=3.5mm] (4) at (3.east) {\texttt{x}\textsubscript{4}};
   \draw[-Kite] (2) -- (1);
   \draw[-Kite] (2) -- (3);
   \draw[-Kite] (4) to[bend right=25] (1);
   \draw[-Kite] (4) -- (3);
\end{tikzpicture}
\end{lrbox}
\begin{frame}[fragile]{Perspectives on Programs}
\xlstsetmintedstyle{plain number}
\begin{tikzpicture}[overlay,remember picture]
\begin{uncoverenv}<2->   
\node[right=2cm] (@text) at(current page.west) {%
\begin{minted}[escapeinside=||,lineskip=1pt]{java}
int x = 0;
while(x < 2) {
   x = x + 1;
}
\end{minted}
};
\end{uncoverenv}
\onslide<4->{\node[right=1.5cm,scale=.75] (@ast) at(@text.east) {\usebox\SimpleSyntacticalPerspective};}
\onslide<6->{\node[right=1.5cm,scale=.75,yshift=3cm] (@cfg) at (@ast.east) {\usebox\SimpleSemanticalPerspective};}
\onslide<8->{\node[right=1.5cm,scale=.75,yshift=-1.5cm] (@dfg) at (@ast.east) {\usebox\SimpleDataflowPerspective};}
\onslide<5->{\node[below] at(@ast.south) {\bfseries Syntactical};}
\onslide<3->{\node[below] at(@text.south) {\bfseries Textual};}
\onslide<7->{\node[below] at(@cfg.south) {\bfseries Control-Flow};}
\onslide<9->{\node[below] at(@dfg.south) {\bfseries Data-Flow};}
\end{tikzpicture}
\end{frame}