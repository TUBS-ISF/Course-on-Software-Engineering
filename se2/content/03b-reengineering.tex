% TODO add links to original literature: Chikofsky und Cross

\subsection{Forward Engineering}
\begin{frame}{\insertsubsection\ \mytitlesource{\ludewiglichter}}
	\begin{fancycolumns}[widths={60},animation=none]
		\begin{definition}{Forward Engineering \mysource{Chikofsky und Cross}}
			\mycite{Forward engineering is the traditional process of moving from high-level abstractions and logical, implementation-independent designs to the physical implementation of a system.}
		\end{definition}
		\uncover<2->{\begin{example}{Examples}
			\begin{itemize}
				\item modeling a system with activity diagrams or state machines \emph{based on} the requirements specification
				\item<3-> architecting a system with architectural patterns and component diagrams \emph{based on} the system specification
				\item<4-> designing a system with design patterns, class diagrams, sequence diagrams \emph{based on} the architecture specification
				\item<5-> implement a system in a particular programming language \emph{based on} the design specification
				\item<6-> \ldots
			\end{itemize}
		\end{example}}
	\nextcolumn
		\diagramForwardEngineering
	\end{fancycolumns}
\end{frame}

\subsection{Reverse Engineering}
\begin{frame}{\insertsubsection\ \mytitlesource{\ludewiglichter}}
	\begin{fancycolumns}[widths={40},animation=none]
		\diagramReverseEngineering
	\nextcolumn
		\begin{definition}{Reverse Engineering \mysource{Chikofsky und Cross}}
			\mycite{Reverse engineering is the process of analyzing a system to identify the system’s components and their interrelationships and create representations of the system in another form or at a higher level of abstraction.}
		\end{definition}
		\uncover<2->{\begin{example}{Examples}
			\begin{itemize}
				\item recovering which requirements have already been implemented
				\item<3-> exploring what classes are in the implementation and what is their purpose
				\item<4-> creating UML models for legacy software that was not modeled before
				\item<5-> adapting a UML model \emph{based on} changes in the implementation
				\item<6-> \ldots
			\end{itemize}
		\end{example}}
	\end{fancycolumns}
\end{frame}

\subsection{Refactoring}
\begin{frame}{\insertsubsection\ \mytitlesource{\ludewiglichter}}
	\vspace{-5mm}
	\begin{fancycolumns}
		\begin{definition}{Refactoring (Restructuring) \mysource{Chikofsky und Cross}}
			\mycite{Refactoring is a transformation from one form of representation to another at the same relative level of abstraction. The new representation is meant to preserve the semantics and external behavior of the original.}
		\end{definition}
		\begin{definition}{Refactoring \mysource{\refactoring}}
			\mycite{Refactoring is the process of changing a software system in such a way that it does not alter the external behavior of the code yet improves its internal structure. It is a disciplined way to clean up code that minimizes the chances of introducing bugs. In essence when you refactor you are improving the design of the code after it has been written.}
		\end{definition}
	\nextcolumn
		\begin{example}{Refactorings}
			\begin{itemize}
				\item rename X refactoring
				\item extract/pull-up method refactoring
			\end{itemize}
		\end{example}
		\begin{example}{Smells and Refactorings (see \reflecture{2}\partb)}
			Smells and refactorings that (some of) you know:
			
			~
			
			\pic[width=\linewidth]{2025st/wordcloud-smells-refactorings}
		\end{example}
	\end{fancycolumns}
\end{frame}

\subsection{Reengineering}
\begin{frame}{\insertsubsection\ \mytitlesource{\ludewiglichter}}
	\begin{fancycolumns}
		\begin{definition}{Reengineering \mysource{Chikofsky und Cross}}
			\mycite{Reengineering is the examination and alteration of a subject system to reconstitute it in a new form and the subsequent implementation of the new form.}
		\end{definition}
		\begin{note}{Essence}
			combination of reverse engineering, refactoring, and forward engineering
		\end{note}
	\nextcolumn
		\begin{example}{Examples}
			\begin{itemize}
				\item reimplementing C programs in Rust
				\item modularizing a monolithic application into microservices
			\end{itemize}
		\end{example}
	\end{fancycolumns}
\end{frame}

\subsection{Wisdom on Reengineering}
\begin{frame}{\insertsubsection}
	\begin{fancycolumns}
		\pic[width=\linewidth,trim=50 0 0 0,clip]{people/kyle-simpson}
		\vspace{-7mm}
		
		\begin{note}{Kyle Simpson \mysource{\href{https://twitter.com/codewisdom/status/857988273674883072}{twitter.com}}}
			\mycite{There's nothing more permanent than a temporary hack.} \deutsch{Sprichwort: Nichts ist so beständig wie das Provisorium.} % TODO Quelle für deutsches Sprichwort suchen
		\end{note}
		\nextcolumn
		\vspace{11mm}
		\centering\pic[width=.5\linewidth,trim=0 0 0 0,clip]{people/jeff-sickel}
		\vspace{-3mm}
		
		\begin{note}{Jeff Sickel \mysource{\href{https://twitter.com/codewisdom/status/710046205317873664}{twitter.com}}}
			\mycite{Deleted code is debugged code.}
		\end{note}
	\end{fancycolumns}
\end{frame}

