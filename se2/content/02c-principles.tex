\subsection{Avoiding Complexity}
\begin{frame}{\insertsubsection} % slided copied from se1-04b. reuse?
	\begin{fancycolumns}
		\pic[width=\linewidth,trim=10 0 10 0,clip]{people/richard-pattis}
		\vspace{-7mm}
		
		\begin{note}{Richard E.\ Pattis \mysource{\href{https://www.cs.cmu.edu/~pattis/quotations.html}{cmu.edu}}}
			\mycite{When debugging, novices insert corrective code; experts remove defective code.}
		\end{note}
		% American professor, author of the educational programming language Karel
		\nextcolumn
		\pic[width=\linewidth,trim=0 120 0 30,clip]{people/ken-thompson}
		\vspace{-7mm}
		
		\begin{note}{Ken Thompson (born 1943) \mysource{\href{https://twitter.com/CodeWisdom/status/1386049811800109056}{twitter.com}}}
			\mycite{One of my most productive days was throwing away 1,000 lines of code.}
		\end{note}
		% Turing Award 1983, inventor of the programming language B (predecessor of C) and Go, UTF-8 encoding
	\end{fancycolumns}
\end{frame}

% TODO add principles
% KISS: https://en.wikipedia.org/wiki/KISS_principle
% DRY: https://en.wikipedia.org/wiki/Don%27t_repeat_yourself
% SOLID: https://en.wikipedia.org/wiki/SOLID
% RERO: https://en.wikipedia.org/wiki/Release_early,_release_often
%    "Release early. Release often. And listen to your customers". This philosophy was popularized by Eric S. Raymond in his 1997 essay The Cathedral and the Bazaar,
% more here: https://en.wikipedia.org/wiki/List_of_software_development_philosophies

% bus factor: https://en.wikipedia.org/wiki/Bus_factor

\subsection{Simplicity over Performance}
\begin{frame}{\insertsubsection}
	\slideSimplicityOverPerformance
\end{frame}

