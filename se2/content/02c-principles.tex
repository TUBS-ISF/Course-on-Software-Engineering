\subsection{Avoiding Complexity}
\begin{frame}{\insertsubsection} % slided copied from se1-04b. reuse?
	\begin{fancycolumns}
		\pic[width=\linewidth,trim=10 0 10 0,clip]{people/richard-pattis}
		\vspace{-7mm}
		
		\begin{note}{Richard E.\ Pattis \mysource{\href{https://www.cs.cmu.edu/~pattis/quotations.html}{cmu.edu}}}
			\mycite{When debugging, novices insert corrective code; experts remove defective code.}
		\end{note}
		% American professor, author of the educational programming language Karel
		\nextcolumn
		\pic[width=\linewidth,trim=0 120 0 30,clip]{people/ken-thompson}
		\vspace{-7mm}
		
		\begin{note}{Ken Thompson (born 1943) \mysource{\href{https://twitter.com/CodeWisdom/status/1386049811800109056}{twitter.com}}}
			\mycite{One of my most productive days was throwing away 1,000 lines of code.}
		\end{note}
		% Turing Award 1983, inventor of the programming language B (predecessor of C) and Go, UTF-8 encoding
	\end{fancycolumns}
\end{frame}

\subsection{KISS Principle}
\begin{frame}{\insertsubsection}
	\begin{fancycolumns}
		\pic[width=\linewidth,trim=0 0 0 0,clip]{people/kelly-johnson}
		\vspace{-7mm}
		
		\begin{note}{Kelly Johnson (1910--1990)\mysource{\href{https://www.nasonline.org/wp-content/uploads/2024/06/johnson-clarence.pdf}{nasaonline.org}}}
			\mycite{KISS --- Keep it simple, stupid!}
		\end{note}
	\nextcolumn
		\begin{definition}{\insertsubsection}
			\begin{itemize}
				\item Johnson leading engineers building jets
				\item goal was to make jets easily repairable by average mechanic
				\item analogy: build software systems that are easy to maintain and repair
			\end{itemize}
		\end{definition}
		\uncover<3->{\begin{example}{Russian Invasion of Ukraine (since 2022)}
			German weapons hard to repair; broken for long time before they can be used again \mysource{\href{https://www.tagesschau.de/multimedia/video/video-1454812.html}{tagesschau.de}}
		\end{example}}
	\end{fancycolumns}
\end{frame}

\subsection{DRY Principle}
\begin{frame}{\insertsubsection{} \mytitlesource{\thepragmaticprogrammer}}
	\begin{fancycolumns}[widths={35},animation=none]
		\pic[width=\linewidth,trim=0 0 0 0,clip]{books/the-pragmatic-programmer}
		\nextcolumn
		\begin{note}{DRY --- Don't Repeat Yourself}
			\mycite{Every piece of knowledge must have a single, unambiguous, authoritative, representation within a system.}
		\end{note}
		\uncover<2->{\begin{definition}{\insertsubsection}
			\begin{itemize}
				\item knowledge documented in specifications, code, and tests
				\item knowledge not stable, changes frequently
				\item duplicate representations are altered inconsistently
				\item means: extract method refactoring
			\end{itemize}
		\end{definition}}
		\uncover<3->{\begin{example}{Reasons for Duplication}
			\begin{itemize}
				\item imposed duplication: devs feel that they have no other choice
				\item inadvertent duplication: devs do not realize that they duplicate
				\item impatient duplication: duplication seems easier
				\item interdeveloper duplication: multiple devs duplicate
			\end{itemize}
		\end{example}}
	\end{fancycolumns}
\end{frame}

% TODO add principles
% SOLID: https://en.wikipedia.org/wiki/SOLID
% RERO: https://en.wikipedia.org/wiki/Release_early,_release_often
%    "Release early. Release often. And listen to your customers". This philosophy was popularized by Eric S. Raymond in his 1997 essay The Cathedral and the Bazaar,
% more here: https://en.wikipedia.org/wiki/List_of_software_development_philosophies

% bus factor: https://en.wikipedia.org/wiki/Bus_factor

\subsection{Simplicity over Performance}
\begin{frame}{\insertsubsection}
	\slideSimplicityOverPerformance
\end{frame}

