% !TeX spellcheck = en_US
\subsection{Unit Testing}
\begin{frame}<2>{\insertsubsection}
	\slideStagesTesting
\end{frame}


\subsection{Unit Test Automation}
\begin{frame}{\insertsubsection}
	\begin{fancycolumns}[animation=none, b]
		\begin{definition}{Test Automation \mysource{\cohnagile}}
			\begin{itemize}
				\item Manual tests are expensive $\rightarrow$ infrequent execution
				\item Automated Testing $\rightarrow$ no human intervention during execution
				\item Important Keywords
				\begin{itemize}
					\item System under Test (SuT)
					\item Test case (input + expected result)
					\item Test bed (environment for test execution)
					\item Test procedure (instructions for execution and evaluation)
				\end{itemize}
				\item Not only execution, but also generation, \dots
				\item Regression Testing: rerunning tests after change
				\item Important for CI/CD, test-driven development, \dots
			\end{itemize}
		\end{definition} \pause
		\nextcolumn
		\begin{center}
			\vspace{-15mm}
			\begin{fancycolumns}[animation=none]
				\centering
				\pic[width=.9\columnwidth]{process/test_automation_pyramid}
				\nextcolumn
				\centering
				\pic[width=.9\columnwidth]{testing/junit5-logo}
			\end{fancycolumns}
		\end{center}
		\begin{definition}{Test Automation Tools}
			\begin{itemize}
				\item Java: JUnit, TestNG, Mockito, jqwik, \dots
				\item Haskell: HUnit, QuickCheck, \dots
				\item \dots
			\end{itemize}
		\end{definition} \pause
		\begin{note}{JUnit}
			Biggest test automation framework for Java		
		\end{note}
	\end{fancycolumns}
\end{frame}

\subsection{Test Doubles}
\begin{frame}[fragile]{\insertsubsection}
	\begin{fancycolumns}[animation=none]
		\begin{note}{Motivation}
			\begin{itemize}
				\item Separation of concerns
				\item \mycite{How do we only test one thing at a time?}
				\item Control indirect environment
			\end{itemize}
		\end{note}\pause
		\begin{definition}{Test Doubles}
			\begin{itemize}
				\item Dummy Object: no usage
				\item Test Stub: predefined answers
				\item Fake Object: working implementation with shortcuts
				\item Test Spy: logs usage information
				\item Mock Object: mimics behavior according to specification, runtime modifiable
			\end{itemize}
		\end{definition}\pause
		\nextcolumn
		\begin{note}{Mockito}
			Mocking Framework for Java
		\end{note}
		\begin{minted}[]{java}
@Test
public void shouldReturnGivenValue() {
	Flower flowerMock = mock(Flower.class); 
	when(flowerMock.getNumberOfLeafs()).thenReturn(TEST_NUMBER_OF_LEAFS);
	assertEquals(flowerMock.getNumberOfLeafs(), TEST_NUMBER_OF_LEAFS);
}
		\end{minted} \vspace{-2mm}
		\begin{minted}[]{java}
@Test
void PersonSearch() {
	PersonDB personsDB = mock(PersonDB.class);
	when(personsDB.searchPersonByAll(contains("Alex"),anyInt()).getAge()).thenReturn(43);
}
		\end{minted}
	\end{fancycolumns}
\end{frame}

\subsection{Mutation Testing}
\begin{frame}{\insertsubsection}
	\begin{fancycolumns}[animation=none]
		\begin{note}{Motivation}
			\begin{itemize}
				\item \mycite{Quis custodiet ipsos custodes?}
				\item \mycite{How well do we test?}
			\end{itemize}
		\end{note}\pause
		\begin{definition}{Mutation Testing \mysource{\mutation}}
			\begin{itemize}
				\item Introduce modifications into code base
				\item Mutant: mutated version of program
				\item \mycite{Killing} a mutant: tests detect mutant
				\item RIP model
				\begin{itemize}
					\item Reach mutated statement
					\item Infect program state
					\item Propagate to checked output
				\end{itemize}
				\item requires code knowledge $\rightarrow$ white-box
			\end{itemize}
		\end{definition}\pause
		\nextcolumn
		\small
		\begin{note}{Types of Mutation Testing}
			\begin{itemize}
				\item Statement Mutation: add line, move lines, \dots
				\item Value Mutation: change constant, \dots
				\item Decision Mutation: change logical operators, \dots
			\end{itemize}
		\end{note}\pause
		\begin{note}{Mutation Operators}
			\begin{itemize}
				\item Statement deletion
				\item Statement insertion
				\item Boolean operator replacement
				\item Arithmetic operator replacement
				\item Variable mix up
				\item Constructor calls
				\item \dots
			\end{itemize}
		\end{note}\pause
		\begin{note}{PitTest}
			Mutation Testing on the Java bytecode
		\end{note}
	\end{fancycolumns}
\end{frame}

\subsection{Flaky Tests}
\begin{frame}[fragile]{\insertsubsection}
	\begin{fancycolumns}[animation=none]
		\begin{note}{Common Problems in Regression Testing}
			\begin{itemize}
				\item Tests do not indicate failures
				\item rotten green: assertions not reached
				\item coincidental correctness: assertions not triggered
				\item flaky tests
			\end{itemize}
		\end{note}
		\begin{definition}{Flaky Tests}
			\begin{itemize}
				\item non-deterministically pass or fail
				\item $\rightarrow$ different behavior even without code changes
				\item Problems
				\begin{itemize}
					\item waste developer time
					\item testing trust reduction
					\item very hard to identify
					\item example: 1.5\% flaky tests at Google
				\end{itemize}
			\end{itemize}
		\end{definition}\pause
		\nextcolumn
		\begin{note}{Common Flaky Tests \mysource{\idoft}}
			\begin{itemize}
				\item Order-Dependent: execution order of test suite
				\item Implementation-Dependent: e.g., Java version, \dots
				\item Non-Idempotent-Outcome: pass on first, fail on second execution
				\item Non-Deterministic
				\item Environment-Dependent: OS, time zone, \dots
				\item \dots
			\end{itemize} 
		\end{note}
		\pause
		\begin{note}{Flaky Test Detection}
			\begin{itemize}
				\item Rerunning
				\item Automated Flaky Test Detection
			\end{itemize}
		\end{note}
	\end{fancycolumns}
\end{frame}


\subsection{Integration Testing}
\begin{frame}<3>{\insertsubsection}
	\slideStagesTesting
\end{frame}

\begin{frame}{Incremental \insertsubsection}
	content
\end{frame}

\begin{frame}{Directions of \insertsubsection}
	content
\end{frame}

\subsection{Coverage Computation}
\begin{frame}{\insertsubsection}
	Live Demo
\end{frame}


\subsection{API Evolution}
\begin{frame}{\insertsubsection}
	content
\end{frame}

