\def\DisableCol{\hypersetup{allcolors=.}\setbeamercolor{structure}{fg=black}\setbeamerfont{structure}{series=\mdseries}}
\subsection{The How}

\begin{frame}[fragile]{Abstract Interpretation}
\begin{uncoverenv}<2->
\AnimateCode{onslide={o2:{3,...,7},-,-,-},first slide=2}
\begin{minted}[escapeinside=||,lineskip=1pt]{java}
public static void main(String[] args) {
    int a = 1; |\tikzmarknode{bmark@a1}{\strut}|
    double r = Math.random() * 10; |\tikzmarknode{bmark@r1}{\strut}|
    if (r > 5) { |\tikzmarknode{bmark@r2}{\strut}|
       a = 2; |\tikzmarknode{bmark@a2}{\strut}|
    }
    System.out.println(a); |\tikzmarknode{bmark@a3}{\strut}|
}
\end{minted}
\endAnimateCode
\begin{tikzpicture}[overlay,remember picture]
   \coordinate[yshift=2pt] (bmark@a1) at (pic cs:bmark@a1);
   \node[right=4cm] (bmark@a1) at (bmark@a1) {\AbstractInfo{a \in \Set{1}}};
   \coordinate[yshift=2pt] (bmark@r1) at (pic cs:bmark@r1);
   \node[right] at (bmark@r1-|bmark@a1.west) {\AbstractInfo{r \in \IntCO{0}{10}}};
   \coordinate[yshift=2pt] (bmark@a2) at (pic cs:bmark@a2);
   \node[right] at (bmark@a2-|bmark@a1.west) {\AbstractInfo{a \in \Set{2}}};
   \coordinate[yshift=2pt] (bmark@a3) at (pic cs:bmark@a3);
   \node[right] (bmark@set) at (bmark@a3-|bmark@a1.west) {\AbstractInfo{a \in \Set{1, 2}}};
   \node[right] at (bmark@set.east) {\(\to\)~\;Valid? Ok? Safe?};
   \onslide<3->{%
      \fill[white,opacity=0.9] ([yshift=2cm]current page.west) rectangle ([yshift=-2cm]current page.east);
      \node[text width=.8\paperwidth,align=left] at(current page.center) {\begin{itemize}
         \itemsep10pt
         \item<4-> We want to prove interesting properties of programs
         \begin{itemize}
            \itemsep5pt
            \item<5-> \textit{Dataflow Properties}\\Liveness, Fainting, Reaching Definitions,~\ldots
            \item<6-> \textit{Safety Properties}\\No Null Dereference, No Division by Zero,~\ldots % finite prefix if we find violation
            \item<7-> \textit{Numerical Properties}\\Signs, Intervals, Octagons, Polyhedra,~\ldots
            \item<8-> \ldots
         \end{itemize}
      \end{itemize}};
   }%
   \node[above right,gray,yshift=3.5mm,font=\tiny,text width=.9\paperwidth] at (current page.south west) {\DisableCol\citetitle{cousout2021principles}~\cite[p.~722]{cousout2021principles},\citetitle{DBLP:journals/annals/GiacobazziR22}~\cite[pp.~37]{DBLP:journals/annals/GiacobazziR22}};
\end{tikzpicture}
\end{uncoverenv}
\end{frame}


\subsubsection{Terminology}

\newsavebox\GraphHeaven
\begin{frame}[c]{Abstract \textcolor{gray}{Interpretation}}
\frametitle<1>{Abstract \textcolor{gray}{Interpretation}}%
\frametitle<2-|handout:1>{Concrete \textcolor{gray}{Interpretation}}%
\frametitle<11-|handout:1>{Abstract \textcolor{gray}{Interpretation}}%
% häufige Visualisierung
\begin{lrbox}{\GraphHeaven}
\pgfmathsetseed{42}%
\begin{tikzpicture}[line cap=round]
   \pgfonlayer{foreground}
   \draw[Kite-Kite,very thick] (0,3.5) node[below right,yshift=1mm] {{\onslide<3->{\(x(t)\)}}} |- (8,0) node[above left] {{\onslide<2->{\(t\)}}}; % time vs. x at tat time
   \endpgfonlayer
   \colorlet{@}{red}
   \onslide<4->{\only<5->{\colorlet{@}{gray}}\draw[very thick,@] (0,1) plot [smooth] coordinates {(0,1) (1,2) (2,1) (3,2) (4,1) (5,2) (6,1) (7,2) (8,1)}; % x(t)
   }
   \colorlet{@}{red}
   \onslide<5->{\only<6->{\colorlet{@}{gray}}\draw[very thick,@] (0,0) plot [smooth] coordinates {(0,0) (1,1) (2,2) (3,2) (4,2.5) (5,2.5) (6,.5) (7,.6) (8,.6)}; % x(t)
   }
   \colorlet{@}{red}
   \onslide<6->{\only<7->{\colorlet{@}{gray}}\draw[very thick,@] (0,1.5) plot [smooth] coordinates {(0,1.5) (1,2) (2,2.5) (3,2) (4,2) (5,2.5) (6,2.5) (7,2) (8,2.5)}; % x(t)
   }
   \onslide<8->{
      \foreach \i in {0,...,5} {
         \pgfmathsetmacro{\randA}{rnd*0.33}
         \pgfmathsetmacro{\randB}{rand*0.5}
         \pgfmathsetmacro{\randC}{rand*0.4}
         \draw[gray] (0,1.5+\randA) plot [smooth] coordinates {(0,1.5+\randA) (1,2-\randB) (2,2.5-\randA) (3,2-\randB) (4,2+\randA) (5,2.5) (6,2.5+\randA) (7,2-\randA) (8,2.5+\randB)} node[inner sep=0pt] (a-\i) {};
         \draw[gray] (0,0+\randA) plot [smooth] coordinates {(0,0+\randA) (1,1-\randB) (2,2-\randB) (3,2+\randC) (4,2.5-\randA) (5,2.5-\randB) (6,.5+\randC) (7,.6+\randB) (8,.6+\randC)} node[inner sep=0pt] (b-\i) {};
         \draw[gray] (0,1+\randB) plot [smooth] coordinates {(0,1-\randC) (1,2-\randB) (2,1+\randB) (3,2-\randA) (4,1+\randA) (5,2-\randB) (6,1) (7,2-\randC) (8,1+\randA)} node[inner sep=0pt] (c-\i) {};
      }
   }
   % fit to all nodes to get the bounding box
   \node[fit=(a-0) (a-1) (a-2) (a-3) (a-4) (a-5) (b-0) (b-1) (b-2) (b-3) (b-4) (b-5) (c-0) (c-1) (c-2) (c-3) (c-4) (c-5),inner sep=0pt] (big-ghost) {~};
   \onslide<9->{
      \draw[decorate,thick,decoration={brace,amplitude=5pt,raise=2pt},gray] (big-ghost.north east) -- (big-ghost.south east) node[midway,right=7pt,gray,align=left] (@doc) {Collecting Semantics\textsuperscript{\cite[91]{cousout2021principles}}};
   }
   \onslide<10->{
      \node[below right,xshift=0mm,yshift=5mm,font=\small,text width=6cm,opacity=.5] at (@doc.south west) {\begin{itemize}
         \item Maybe impossible to compute statically
         \item \ldots~or very expensive (\faCaretRight~\textit{dynamic})
         \item[\faCaretRight] Abstract Interpretation to the rescue
      \end{itemize}};
   }
   \pgfonlayer{background}
   \pgfinterruptboundingbox
   \only<11|handout:0>{
   \fill[red,opacity=.175,even odd rule] plot [smooth] coordinates {(0,0) (1,0.4) (2,0.5) (3,1) (4,.8) (5,1) (6,.1) (7,0.2) (8.03,.2) (8.03,3) (7,2.8) (6,3) (5,2.95) (4,2.85) (3,2.75) (2,2.65) (1,2.5) (0,2) } -- cycle; 
   }
   \only<12->{
   \fill[red,opacity=.175,even odd rule] plot [smooth] coordinates {(0,0) (1,0.4) (2,0.5) (3,1) (4,.8) (5,1) (6,.1) (7,0.2) (8.03,.2) (8.03,3) (7,2.8) (6,3) (5,2.95) (4,2.85) (3,2.75) (2,2.65) (1,2.5) (0,2) } -- cycle (6,1.85) circle[radius=4mm]; 
   }
   \onslide<11->{%
      \draw[Circle-,red,rounded corners=4pt] (7.75,2.9) -- ++(.35,.5) -- ++(.5,0) node[right,align=left] {(Trace) Abstraction\textsuperscript{\cite[92]{cousout2021principles}}\\[-2pt]\footnotesize\color{gray}just one of many};
   }
   \endpgfinterruptboundingbox
   \onslide<13->{
      \node (@b1) at (6,1.85) {\small\faBug};
      \node (@b2) at (3,.35) {\small\faBug};
      \node (@b3) at (7,2.5) {\small\faBug};
   }
   \onslide<14->{
      \node[above left=-1mm,green] at(@b2.south east) {\scriptsize\faCheck};
   }
   \onslide<15->{
      \node[above left=-1mm,green] at(@b1.south east) {\scriptsize\faCheck};
   }
   \onslide<16->{
      \node[above left=-1mm,yshift=1pt,orange] at(@b3.south east) {\scriptsize\faQuestion};
   }
   \endpgfonlayer
   \path[use as bounding box] (0,0) rectangle (8,3.5);
\end{tikzpicture}
\end{lrbox}
\begin{tikzpicture}[overlay,remember picture]
   \node[xshift=2.5mm] at(current page.center) {\usebox\GraphHeaven};
   \node[above right,gray,yshift=3.5mm,font=\tiny,text width=.9\paperwidth] at (current page.south west) {\DisableCol See \citetitle{cousout2012casual}~\cite{cousout2012casual}};
\end{tikzpicture}
\end{frame}
\newsavebox\PowersetZHasse
\newsavebox\TestBox
% TOOD: measure and only box if larger?
\begin{lrbox}{\PowersetZHasse}
\def\S#1{\savebox\TestBox{\footnotesize\absexpr{\Set{#1}}}\ifdim\ht\TestBox>5mm\makebox[5mm][c]{\usebox\TestBox}\else\usebox\TestBox\fi}\color{gray}%
\begin{tikzpicture}
   \matrix (A) [matrix of nodes, row sep=1mm, column sep=-2mm]
   {
       & & & \kern-4mm\S{-4, 0, 1, 9}\kern-4mm & & & \\
       & & \S{-4,0,1} & \ldots & \S{0,1,9} & & \\
      & \S{-4,0} & \ldots & \S{0,1} & \ldots & \S{1,9} & \\
      \S{-4} & \ldots & \S{0} & \ldots & \S{1} & \ldots & \S{9} \\
      & & & \absexpr{\emptyset} & & & \\
   };
   \scope[line cap=round]
   \draw (A-1-4) -- (A-2-3) -- (A-3-2) -- (A-4-1) (A-4-1.south) -- (A-5-4);
   \draw (A-1-4) -- (A-2-5) -- (A-3-6) -- (A-4-7) (A-4-7.south) -- (A-5-4);
   \draw (A-3-2) -- (A-4-3) -- (A-3-4) (A-3-4) -- (A-4-5) -- (A-3-6);
   \draw (A-2-3) -- (A-3-4) -- (A-2-5);
   \draw (A-4-3) -- (A-5-4) -- (A-4-5);
   \draw[densely dotted] (A-5-4) -- ++(-1,0.05)  (A-5-4) -- ++(1,0.05);
   \foreach[count=\y] \i in {4,3,2,1} {
      \draw[densely dotted] (A-\y-\i.north west) -- ++(-.4,0.14);
      \node[left=3.5mm] at(A-\y-\i.west) {\footnotesize\ldots};
      \pgfmathsetmacro\other{int(8-\i)}
      \draw[densely dotted] (A-\y-\other.north east) -- ++(.4,0.14);
      \node[right=3.5mm] at(A-\y-\other.east) {\footnotesize\ldots};
   }
   \node[above=3.5mm] (pz) at(A-1-4.north) {\absexpr{\Z}};
   \draw[densely dotted] (pz) -- ++(-1.25,-0.1) (pz) -- ++(1.25,-0.1);
   \draw[-Kite] ([yshift=1cm,xshift=-3mm]current bounding box.south west) -- ([yshift=-5mm]current bounding box.north west) node[midway,left,font=\scriptsize] {\rotatebox{90}{\absexpr{\partof \asdef\eq \subseteq}}};
   \endscope
\end{tikzpicture}
\end{lrbox}
\begin{frame}{Terminology}
   \begin{itemize}
      \item \textbf{Property}\onslide<2->{ --- Set of states/traces that satisfy that property}\\
            \onslide<3->{\textcolor{gray}{Even integers: \absexpr{\text{P} = \Set{ z \in \Z \Given \exists k \in \Z : z = 2k} = \Set{0, 2, 4, 6, \ldots} \subseteq \P(\tikzmarknode{universe}{\Z})}}}
            \medskip

            \onslide<5->{\centerline{\absexpr{\tikzmarknode{ff}{\emptyset} \subseteq \tikzmarknode{p1}{\text{P}_1} \subseteq \tikzmarknode{p2}{\text{P}_2} \subseteq \tikzmarknode{tt}{\Universe}}}}
            \vspace*{4.9em}

      \item<10-> \textbf{Partial Order} \onslide<11->{--- A \tikzmarknode{reflexive}{reflexive}, \tikzmarknode{transitive}{transitive}, \tikzmarknode{antisymmetric}{antisymmetric} relation on a set}\\
            \textcolor{gray}{\onslide<15->{\absexpr{(\Z, \leq)}}\onslide<16->{,\quad\absexpr{(\P(\Z), \subseteq)},\quad\ldots}}
            % domains special kinds of partial orders
   \end{itemize}
   
   \begin{tikzpicture}[overlay,remember picture,line cap=round]
      \onslide<4->{
         \draw[Kite-,gray] ([yshift=-2pt]universe.south) to[out=310,in=180] ++(.4,-.25) node[right] {\small universe (\absexpr{\Universe})};
      }
      \onslide<6->{\draw[Kite-,gray] ([yshift=-2pt]ff.south) to[out=230,in=0] ++(-.4,-.25) node[left] {\small strongest};}
      \onslide<7->{\draw[Kite-,gray] ([yshift=-2pt,xshift=-2pt]p1.south) to[out=260,in=0] ++(-.4,-.55) node[left] {\small stronger};}
      \onslide<8->{\draw[Kite-,gray] ([yshift=-2pt,xshift=-2pt]p2.south) to[out=280,in=180] ++(.4,-.55) node[right] {\small weaker};}
      \onslide<9->{\draw[Kite-,gray] ([yshift=-2pt]tt.south) to[out=310,in=180] ++(.4,-.25) node[right] {\small weakest};}
      
      \onslide<12->{\draw[Kite-,gray] ([yshift=2pt]reflexive.north) to[out=130,in=0] ++(-.4,.215) node[left] {\small \absexpr{\forall x \in X : x \partof x}};}
      \onslide<13->{\draw[Kite-,gray] ([yshift=2pt]transitive.north) -- ++(0,.3) node[above=-1pt] {\small \absexpr{\forall x, y, z \in X : x \partof y \land y \partof z \implies x \partof z}};}
      \onslide<14->{\draw[Kite-,gray] ([yshift=2pt]antisymmetric.north) to[out=50,in=180] ++(.4,.215) node[right] {\kern-1pt\small\absexpr{\forall x, y \in X : x \partof y \land y \partof x \implies x = y}};}
      
      \node[above right,gray,yshift=3.5mm,font=\tiny,text width=.9\paperwidth] at (current page.south west) {\DisableCol\citetitle{cousout2021principles}~\cite[15]{cousout2021principles},\citetitle{DBLP:journals/ftpl/Mine17}~\cite[18]{DBLP:journals/ftpl/Mine17}};
      \onslide<17->{%
      \node[above left,yshift=3.5mm] at(current page.south east) {\scalebox{.65}{\usebox\PowersetZHasse}};
      }
   \end{tikzpicture}
\end{frame}

\def\S#1{{\footnotesize\absexpr{\Set{#1}}}}

\begin{frame}[fragile]{Chains and Lattices}
\begin{onlyenv}<1|handout:0>
\begin{tikzpicture}
   \matrix (A) [matrix of nodes, row sep=2.5mm, column sep=-2mm]
   {
       & &  & \kern-4mm\S{-4, 0, 1, 9}\kern-4mm & & & \\
       & & \S{-4,0,1} & \ldots & \S{0,1,9} & & \\
      & \S{-4,0} & \ldots & \S{0,1} & \ldots & \S{1,9} & \\
      \S{-4} & \ldots & \S{0} & \ldots & \S{1} & \ldots & \S{9} \\
      & & & \absexpr{\emptyset} & & & \\
   };
   \scope[line cap=round]
   \draw (A-1-4) -- (A-2-3) -- (A-3-2) -- (A-4-1) (A-4-1.south) -- (A-5-4);
   \draw (A-1-4) -- (A-2-5) -- (A-3-6) -- (A-4-7) (A-4-7.south) -- (A-5-4);
   \draw (A-3-2) -- (A-4-3) -- (A-3-4) (A-3-4) -- (A-4-5) -- (A-3-6);
   \draw (A-2-3) -- (A-3-4) -- (A-2-5);
   \draw (A-4-3) -- (A-5-4) -- (A-4-5);
   \draw[densely dotted] (A-5-4) -- ++(-1,0.05)  (A-5-4) -- ++(1,0.05);
   \foreach[count=\y] \i in {4,3,2,1} {
      \draw[densely dotted] (A-\y-\i.north west) -- ++(-.4,0.14);
      \node[left=3.5mm] at(A-\y-\i.west) {\footnotesize\ldots};
      \pgfmathsetmacro\other{int(8-\i)}
      \draw[densely dotted] (A-\y-\other.north east) -- ++(.4,0.14);
      \node[right=3.5mm] at(A-\y-\other.east) {\footnotesize\ldots};
   }
   \node[above=3.5mm] (pz) at(A-1-4.north) {\absexpr{\P(\Z)}};
   \draw[densely dotted] (pz) -- ++(-1.25,-0.1) (pz) -- ++(1.25,-0.1);
   \draw[-Kite] ([yshift=1cm,xshift=-3mm]current bounding box.south west) -- ([yshift=-5mm]current bounding box.north west) node[midway,left,font=\scriptsize] {\rotatebox{90}{\absexpr{\partof \asdef\eq \subseteq}}};
   \endscope
\end{tikzpicture}
\end{onlyenv}
\begin{onlyenv}<2-|handout:1>
\begin{tikzpicture}
   \matrix (A) [matrix of nodes, row sep=2.5mm, column sep=-2mm]
   {
       & &  & \I{-1}{\infty} & & & \\
       & & \I{-1}{1} & \ldots & \I{0}{9} & & \\
      & \I{-1}{0} & \ldots & \I{0}{1} & \ldots & \I{1}{9} & \\
      \I{-1}{-1} & \ldots & \I00 & \ldots & \I11 & \ldots & \I99 \\
      & & & \absexpr{\bot} & & & \\
   };
   \scope[line cap=round]
   \draw (A-2-3) -- (A-3-2) -- (A-4-1) -- (A-5-4);
   \draw (A-3-6) -- (A-4-7) -- (A-5-4);
   \draw (A-3-2) -- (A-4-3) -- (A-3-4) (A-3-4) -- (A-4-5) -- (A-3-6);
   \draw (A-2-3) -- (A-3-4);
   \draw (A-4-3) -- (A-5-4) -- (A-4-5);
   \draw[densely dotted] (A-2-5) -- (A-1-4) -- (A-2-3) (A-3-4) -- (A-2-5) -- (A-3-6);
   \draw[densely dotted] (A-5-4) -- ++(-1,0.05)  (A-5-4) -- ++(1,0.05);
   \foreach[count=\y] \i in {4,3,2,1} {
      \draw[densely dotted] (A-\y-\i.north west) -- ++(-.4,0.14);
      \node[left=3.5mm] at(A-\y-\i.west) {\footnotesize\ldots};
      \pgfmathsetmacro\other{int(8-\i)}
      \draw[densely dotted] (A-\y-\other.north east) -- ++(.4,0.14);
      \node[right=3.5mm] at(A-\y-\other.east) {\footnotesize\ldots};
   }
   \node[above=3.5mm] (pz) at(A-1-4.north) {\absexpr{\top}};
   \draw[densely dotted] (pz) -- ++(-1.25,-0.1) (pz) -- ++(1.25,-0.1);
   \draw[-Kite] ([yshift=1cm,xshift=-3mm]current bounding box.south west) -- ([yshift=-5mm]current bounding box.north west) node[midway,left,font=\scriptsize] {\rotatebox{90}{\absexpr{\partof \asdef\eq \dot\subseteq}}};
   \endscope
   \pgfonlayer{background}
   \scope[opacity=.175,transparency group]
   \onslide<6->{%
      \draw[red,line width=3mm,rounded corners=2mm,line cap=round] (pz.center) -- ++(1.25,-.42) coordinate (@edge) -- (A-1-4.center) -- (A-2-3.center) -- (A-3-2.center) -- (A-4-1.center) -- (A-5-4.center);
      \foreach \i in {pz,A-1-4,A-2-3,A-3-2,A-4-1,A-5-4} {
         \fill[red,rounded corners=5pt,line cap=round] (\i.south west) rectangle (\i.north east);
      }
   }
   \endscope
   \onslide<4->{
      \draw[Kite-,red,rounded corners=4pt] ([xshift=1.5mm,yshift=-1mm]A-3-4.north) -- ++(.25,.325) -- ++(3,0) node[below right,yshift=.7\baselineskip,align=left] {Least upper bound\\[-2pt]\footnotesize\color{gray}of \IntCC00 and \IntCC11\\[-3pt]\footnotesize\color{gray}lub, join, \absexpr{\lub}};
   }
   \onslide<5->{
      \draw[Kite-,red,rounded corners=4pt] ([xshift=1mm,yshift=1mm]A-4-3.south) -- ++(-.4,-.85) -- ++(-.25,0) node[below left,yshift=.7\baselineskip,align=right] {Greatest lower bound\\[-2pt]\footnotesize\color{gray}of \IntCC{-1}{0} and \IntCC01\\[-3pt]\footnotesize\color{gray}glb, meet, \absexpr{\glb}};
   }
   \onslide<6->{
      \draw[Circle-,red,rounded corners=4pt] ([xshift=-1mm]@edge) -- ++(.35,.5) -- ++(.5,0) node[below right,yshift=.7\baselineskip,align=left] {Chain\\[-2pt]\footnotesize\color{gray}a totally ordered subset\\[-4pt]{\footnotesize\color{gray}\onslide<7->{e.g., \absexpr{\IntCC{0}{0} \partof \IntCC{0}{9} \partof \IntCC{-10}{200}}}}};
   }
   \endpgfonlayer
   \pgfinterruptboundingbox
   \onslide<3->{%
      \draw[Kite-,gray] (A-5-4.south) to[out=-60,in=180] ++(.5,-.25) node[right,font=\scriptsize] {bottom, empty interval};
      \draw[Kite-,gray] (pz.north) to[out=120,in=0] ++(-.5,.25) node[left,font=\scriptsize] {top, \absexpr{\IntCC{-\infty}{\infty}}};
   }
   \endpgfinterruptboundingbox
\end{tikzpicture}
\begin{tikzpicture}[overlay,remember picture]
\onslide<8->{%
   \node[above left,yshift=4.345mm,text width=5.5cm] at(current page.south east) {\textbf{Complete Lattice}\\\absexpr{(X, \partof, \lub, \glb, \bot, \top)}\vspace{-1.5mm}\footnotesize
   {\begin{itemize} 
      \itemsep-1pt
      \item<9-> \absexpr{(X, \partof)} is a partial order
      \item<10-> \absexpr{\forall A \subseteq X : \lub A} and \absexpr{\glb A} exist
      \item<11-> \absexpr{\bot}/\absexpr{\top} as smallest/largest element
   \end{itemize}}
   };
}
\end{tikzpicture}
\end{onlyenv}
% birkhoff1940lattice
\begin{tikzpicture}[overlay,remember picture]
   \node[above right,gray,yshift=3.5mm,font=\tiny,text width=.9\paperwidth] at (current page.south west) {\citetitle{birkhoff1967lattice}~\cite{birkhoff1967lattice}, see also sublattices~\cite[25]{DBLP:journals/ftpl/Mine17}};
\end{tikzpicture}
\end{frame}

\subsubsection{Abstract Domains}
\begin{frame}{\insertsubsubsection~\qquad\textcolor{gray}{Numerical}}
\vspace*{3.5em}
\tikzset{@/.style={red}}\only<5->{\tikzset{@/.style={gray}}}%
\begin{tikzpicture}[line cap=round]
   \draw[-Kite] (0,-1) -- (0,1) node[below left] {\scriptsize y};
   \draw[-Kite] (-1,0) -- (1,0) node[below left] {\scriptsize x};
\scope[@]
\onslide<2->{%
   \fill (.1,.1) coordinate (@) circle[radius=1.5pt];
   \foreach \x/\y in {.2/.3,-.1/.2,-.3/-.2,.2/-.5,.5/.4,-.8/.7,-.9/-.4,.4/.4,.35/-.4,-.4/.3} {
      \fill (\x,\y) circle[radius=1.5pt];
      \draw[-{Kite[scale=.4]}] (@) -- (\x,\y) coordinate (@);
   }
}
\onslide<3->{%
   \fill (-.4,-.55) circle[radius=1.5pt];
   \fill (.5,-.8) circle[radius=1.5pt];
   \draw[-{Kite[scale=.4]}] (-.4,-.55) -- (.5,-.8);
}
\endscope
\onslide<4->{%
   \node[below=1mm] at(current bounding box.south) {\scriptsize Collecting Semantics};
}
\end{tikzpicture}\hfill\onslide<5->{%
\tikzset{@/.style={red}}\only<8->{\tikzset{@/.style={gray}}}%
\begin{tikzpicture}[line cap=round]
   \draw[-Kite] (0,-1) -- (0,1) node[below left] {\scriptsize y};
   \draw[-Kite] (-1,0) -- (1,0) node[below left] {\scriptsize x};
\pgfonlayer{background}
\scope[@]
\onslide<6->{%
   \fill[@,opacity=.4] (-.4,-1) rectangle (.7,1);
   \draw[gray,thin] (-.4,-1) -- ++(0,2);
   \draw[gray,thin] (.7,-1) -- ++(0,2);
}
\endscope
\endpgfonlayer
\onslide<7->{%
   \node[below=1mm] at(current bounding box.south) {\clap{\scriptsize Intervals \absexpr{x \in \IntCC ab}}};
}
\end{tikzpicture}}\hfill\onslide<9->{%
\tikzset{@/.style={red}}\only<11->{\tikzset{@/.style={gray}}}%
\begin{tikzpicture}[line cap=round]
   \draw[-Kite] (0,-1) -- (0,1) node[below left] {\scriptsize y};
   \draw[-Kite] (-1,0) -- (1,0) node[below left] {\scriptsize x};
\scope[@]
\onslide<10->{%
   \foreach \x in {-1,-.7,...,1} {
      \foreach \y in {-.9,-.6,...,1} {
         \fill (\x,\y) circle[radius=1.5pt];
      }
   }
}
\endscope
\onslide<10->{%
   \node[below=1mm] at(current bounding box.south) {\clap{\scriptsize Simple Congruences}};
}
\end{tikzpicture}}\hfill\onslide<11->{%
\tikzset{@/.style={red}}\only<13->{\tikzset{@/.style={gray}}}%
\begin{tikzpicture}[line cap=round]
   \draw[-Kite] (0,-1) -- (0,1) node[below left] {\scriptsize y};
   \draw[-Kite] (-1,0) -- (1,0) node[below left] {\scriptsize x};
\scope[@]
\onslide<12->{%
   \fill[opacity=.4,@] (-.5,-.3) |- ++(.8,.6) -- ++(0,-.4) -- ++(-.2,-.2) -- cycle;
   \draw[gray,thin] (-.5,-.3) |- ++(.8,.6) -- ++(0,-.4) -- ++(-.2,-.2) -- cycle;
   \draw[thick,@] (.6,.2) -- ++(-.8,-.8);
}
\endscope
\onslide<12->{%
   \node[below=1mm] at(current bounding box.south) {\clap{\scriptsize Pentagons}};
}
\end{tikzpicture}}
\medskip
\begin{multicols}{4}
\begin{itemize}
   \item<14-> Octagons
   \item<15-> Ellipses
   \item<16-> Exponentials
   \item<17-> Signs
\end{itemize}
\end{multicols}
\begin{tikzpicture}[overlay,remember picture]
   \node[above right,gray,yshift=3.5mm,font=\tiny,text width=.9\paperwidth] at (current page.south west) {\citetitle{cousout2021principles}~\cite{cousout2021principles}, \citetitle{DBLP:conf/sac/LogozzoF08}~\cite[25]{DBLP:conf/sac/LogozzoF08}};
\end{tikzpicture}
% non relational: pointer, shapes, cost, ...
\end{frame}

\subsubsection{Sign Analysis}
\begin{frame}[fragile]{\insertsubsubsection\hfill\textcolor{gray}{Simple Sign Domain}}
\begin{itemize}
   \item<3-> We still have no program semantics, but we can try\ldots
\begin{minted}[escapeinside=||,lineskip=1pt]{java}
int a = 0; |\tikzmark{n@a1}|
int b = 12; |\tikzmark{n@b1}|
int c = a + b; |\tikzmark{n@c1}|
int d = c - b; |\tikzmark{n@d1}|
\end{minted}
\begin{tikzpicture}[overlay,remember picture]
   \onslide<4->{%
      \coordinate (n@a1) at (pic cs:n@a1);
      \node[right=2cm] (n@a1) at (n@a1) {\AbstractInfo{a = 0}};
   }%
   \onslide<5->{%
      \coordinate (n@b1) at (pic cs:n@b1);
      \node[right] at (n@b1-|n@a1.west) {\AbstractInfo{b \geq 0}};
   }%
   \onslide<6->{%
      \coordinate (n@c1) at (pic cs:n@c1);
      \node[right] at (n@c1-|n@a1.west) {\AbstractInfo{c \geq 0\quad (={} 0~{}+{}\geq 0)}};
   }%
   \onslide<7->{%
      \coordinate (n@d1) at (pic cs:n@d1);
      \node[right] (n@d1) at (n@d1-|n@a1.west) {\AbstractInfo{d = \top\kern-1.5pt\quad (\geq{} 0~{}-{}\geq 0)}};
   }
\end{tikzpicture}
   \item<8-> But how to handle control flow? Loops? Recursion?
\end{itemize}

\begin{tikzpicture}[overlay,remember picture]
   \onslide<2->{\node[above left=.5mm,yshift=3.5mm] at(current page.south east) {\usebox\SimpleSignLattice};}
\end{tikzpicture}
\end{frame}


\subsubsection{Fixpoints}
\begin{frame}{\insertsubsubsection}
   \begin{itemize}
      \itemsep10pt
      \item<2-> For operators \(f: X \to X\) a \textbf{fixpoint} is a \(x \in X\) such that \(f(x) = x\)
      \item<3-> If we iterate \(f\) starting from some \(x_0 \in X\):\medskip
      \begin{itemize}[--]
         \itemsep15pt
         \item<4-> \parbox{3.25cm}{\smash{\tikz[baseline=-.85mm,gray]{\coordinate (@) at (0,0);\fill (@) circle[radius=2pt] node[below] {\tiny\(x_0\)};\foreach \i in {0,...,3} {
            \draw[-Kite] (@) -- ++(.5,0) coordinate[xshift=1pt] (@);
            \fill (@) circle[radius=2pt] node[below] {\tiny\(f_\i\)};
         }; \draw[Kite-] ([xshift=-1pt]@) to[out=120,in=60,looseness=10] ([xshift=2pt]@);}}} \onslide<5->{reach a fixpoint, \(f^p = f(f^p)\)}%\hfill\textcolor{gray}{\footnotesize\absexpr{f : \N \to \N, f(x) = x}}
         \item<6-> \parbox{3.25cm}{\smash{\tikz[baseline=-.85mm,gray]{\coordinate (@) at (0,0);\fill (@) circle[radius=2pt] node[below] {\tiny\(x_0\)};\foreach \i in {0,...,3} {
            \draw[-Kite] (@) -- ++(.5,0) coordinate[xshift=1pt] (@) coordinate (@\i);
            \fill (@) circle[radius=2pt] node[below] {\tiny\(f_\i\)};
         }; \draw[-Kite] (@) to[out=70,in=60] (@1);}}} \onslide<7->{reach a cycle, \(f^{p + \ell} = f^p,~~\ell > 0\)}% \hfill\textcolor{gray}{\footnotesize\absexpr{f : \N \to \N, f(x) = x + 1}}
         \item<8-> \parbox{3.25cm}{\smash{\tikz[baseline=-.85mm,gray]{\coordinate (@) at (0,0);\fill (@) circle[radius=2pt] node[below] {\tiny\(x_0\)};\foreach \i in {0,...,3} {
            \draw[-Kite] (@) -- ++(.5,0) coordinate[xshift=1pt] (@) coordinate (@\i);
            \fill (@) circle[radius=2pt] node[below] {\tiny\(f_\i\)};
         }; \draw[densely dotted,semithick] (@) -- ++(0.42,0)}}} \onslide<9->{iterate forever, \(\forall p \neq q \in \mathbb{N} : f^{p} \neq f^q\)}\hfill\textcolor{gray}{\footnotesize\absexpr{f : \N \to \N, f(x) = x + 1}}
      \end{itemize}
      \item<10-> If our function is monotonic, we can always find a fixpoint\textsuperscript{\cite{tarski1955lattice}} \tikzmarknode{tarski}{\strut}
      \item<12-> Analyzing, e.g. loops, we \enquote{go up} the lattice until we reach a least fixpoint
      \item<13-> Alternatively, we have to use \textit{widening} to reach a fixpoint
   \end{itemize}
\begin{tikzpicture}[overlay,remember picture]
   \node[above right,gray,yshift=3.5mm,font=\tiny,text width=.9\paperwidth] at (current page.south west) {\DisableCol\citetitle{tarski1955lattice}~\cite{tarski1955lattice}, \citetitle{kleene1952introduction}~\cite{kleene1952introduction}, \citetitle{cousout2021principles}~\cite[165]{cousout2021principles}};
   \onslide<11->{%
      \draw[-Kite,line cap=round,gray] ([yshift=1.5pt]pic cs:tarski) to[out=0,in=180] ++(.5,.25) node[right,font=\tiny,align=left] {for complete, nonempty lattices\\Tarski's Theorem};
   }
\end{tikzpicture}
\end{frame}

% \subsubsection{Interval Analysis, I}

\begin{lrbox}{\SimpleSignLattice}
\scriptsize
\begin{tikzpicture}[line cap=round,x=6.5mm,y=6.5mm,gray]
   \node (top) at (0,0) {\absexpr{\top}};
   \node (pos) at (-1,-1) {\absexpr{\geq 0}};
   \node (neg) at (1,-1) {\absexpr{\leq 0}};
   \node (zero) at (0,-2) {\absexpr{0}};
   \node (bot) at (0,-3) {\absexpr{\bot}};
   \draw (top) -- (pos) -- (zero) -- (neg) -- (top) (zero) -- (bot);
\end{tikzpicture}
\end{lrbox}
\newsavebox\IntervalLattice

\def\mark{\color{red}\let\I\MarkedI}
\def\MarkedI#1#2{\footnotesize\absexpr{\mathbf{\IntCC{#1}{#2}}}}
% \begin{frame}[fragile]{\insertsubsubsection}
% \begin{lrbox}{\IntervalLattice}
% \scriptsize
% \begin{tikzpicture}[line cap=round,x=6.5mm,y=6.5mm,gray]
%    \matrix (A) [matrix of nodes, row sep=2.5mm, column sep=-2mm]
%    {
%        & &  & \I{-1}{\infty} & & & \\
%        & & \I{-1}{1} & \ldots & \only<12-|handout:3>{\mark}\I{0}{2} & & \\
%       & \I{-1}{0} & \ldots & \only<9-11|handout:2>{\mark}\I{0}{1} & \ldots & \I{1}{2} & \\
%       \I{-1}{-1} & \ldots & \only<4-8|handout:1>{\mark}\I00 & \ldots & \only<7-8|handout:1>{\mark}\I11 & \ldots & \I22 \\
%       & & & \absexpr{\bot} & & & \\
%    };
%    \scope[line cap=round]
%    \draw (A-2-3) -- (A-3-2) -- (A-4-1) -- (A-5-4);
%    \draw (A-3-6) -- (A-4-7) -- (A-5-4);
%    \draw (A-3-2) -- (A-4-3) -- (A-3-4) (A-3-4) -- (A-4-5) -- (A-3-6);
%    \draw (A-2-3) -- (A-3-4) (A-3-4) -- (A-2-5) -- (A-3-6);
%    \draw (A-4-3) -- (A-5-4) -- (A-4-5);
%    \draw[densely dotted] (A-2-5) -- (A-1-4) -- (A-2-3);
%    \draw[densely dotted] (A-5-4) -- ++(-1,0.05)  (A-5-4) -- ++(1,0.05);
%    \foreach[count=\y] \i in {4,3,2,1} {
%       \draw[densely dotted] (A-\y-\i.north west) -- ++(-.4,0.14);
%       \node[left=3.5mm] at(A-\y-\i.west) {\footnotesize\ldots};
%       \pgfmathsetmacro\other{int(8-\i)}
%       \draw[densely dotted] (A-\y-\other.north east) -- ++(.4,0.14);
%       \node[right=3.5mm] at(A-\y-\other.east) {\footnotesize\ldots};
%    }
%    \node[above=3.5mm] (pz) at(A-1-4.north) {\absexpr{\top}};
%    \draw[densely dotted] (pz) -- ++(-1.25,-0.1) (pz) -- ++(1.25,-0.1);
%    \endscope
% \end{tikzpicture}
% \end{lrbox}\vspace*{-2em}%
% \begin{minted}[escapeinside=||,lineskip=1pt]{java}
% int x = 0; |\tikzmark{int@i1}\medskip|
% while(x < 2) {|\tikzmark{int@i2}\medskip|
%    x = x + 1; |\tikzmark{int@i3}|
% }|\tikzmark{int@i4}|
% \end{minted}
% \begin{tikzpicture}[overlay,remember picture]
%    \onslide<3->{%
%       \coordinate[yshift=2pt] (int@i1) at (pic cs:int@i1);
%       \node[right=2cm] (int@i1) at (int@i1) {\AbstractInfo{x_0 \in \IntCC00 }};
%    }%
%    \onslide<5->{%
%       \coordinate[yshift=2pt] (int@i2) at (pic cs:int@i2);
%       \node[above right,yshift=-1.5pt] (int@i2a) at (int@i2-|int@i1.west) {\AbstractInfo{\only<8-9,12|handout:2,3>{\boldmath}\text{\textcolor{gray}{\scriptsize[pre]}~}x_1 \in \only<-8|handout:1>{\IntCC00}\only<9-11|handout:2>{\IntCC01\quad (\IntCC00 \cup \IntCC11)}\only<12-|handout:3>{\IntCC02\quad (\IntCC01 \cup \IntCC12)}}};
%    }%
%    \onslide<6->{%
%       \node[below right,yshift=1.5pt] (int@i3a) at (int@i2-|int@i1.west) {\AbstractInfo{\only<10|handout:0>{\boldmath}\text{\textcolor{gray}{\scriptsize[in]}~}x_2 \in \only<-9|handout:1>{\IntCC00\quad (\IntCC00 \cap \IntOC{-\infty}1)}\only<10-|handout:2>{\IntCC01\quad (\IntCC01 \cap \IntOC{-\infty}1)}}};
%    }%
%    \onslide<7->{%
%       \coordinate[yshift=2pt] (int@i3) at (pic cs:int@i3);
%       \node[right] (int@i3) at (int@i3-|int@i1.west) {\AbstractInfo{\only<11|handout:0>{\boldmath}\text{}x_3 \in \only<-10|handout:1>{\IntCC11\quad (\IntCC00 \oplus \IntCC11)}\only<11-|handout:2>{\IntCC12\quad (\IntCC01 \oplus \IntCC11)}}};
%    }
%    \onslide<13-|handout:3>{%
%       \coordinate[yshift=2pt] (int@i4) at (pic cs:int@i4);
%       \node[right] (int@i4) at (int@i4-|int@i1.west) {\AbstractInfo{\text{\textcolor{gray}{\scriptsize[post]}~}x_4 \in \IntCC22 \quad (\IntCC02 \cap \IntCO2{\infty})}};
%    }
%    \onslide<14-|handout:3>{
%       \node[right=4.5cm] (int@i1@sign) at (int@i1.east) {\AbstractInfo{x_0 = 0}};
%       \node[right] at (int@i2a-|int@i1@sign.west) {\AbstractInfo{\text{\textcolor{gray}{\scriptsize[pre]}~}x_1 \geq 0}};
%       \node[right] at (int@i3a.east-|int@i1@sign.west) {\AbstractInfo{\text{\textcolor{gray}{\scriptsize[in]}~}x_2 \geq 0}};
%       \node[right] at (int@i3-|int@i1@sign.west) {\AbstractInfo{x_3 \geq 0}};
%       \node[right] at (int@i4-|int@i1@sign.west) {\AbstractInfo{\text{\textcolor{gray}{\scriptsize[post]}~}x_4 \geq 0}};
%       \node[above] at(int@i1@sign.north) {\footnotesize Signs};
%       \node[above] at(int@i1.north) {\footnotesize Intervals};
%    }
% \end{tikzpicture}

% \begin{tikzpicture}[overlay,remember picture]
%    \onslide<2->{\node[above left=.5mm,yshift=3.5mm] at(current page.south east) {\scalebox{.85}{\usebox\IntervalLattice}};}
%    \onslide<15->{
%       \node[above right=.5mm,yshift=3.5mm] at (current page.south west) {\scalebox{.85}{\usebox\SimpleSignLattice}};
%    }
% \end{tikzpicture}
% \end{frame}

\subsection{Soundness and Completeness}
\begin{frame}{Rice's Theorem}
   \begin{itemize}
      \itemsep12pt
      \item<2-> We want to prove properties of programs (e.g., no overflow, shapes,~\ldots)
      \item<3-> However, thanks to \citeauthor{rice1953classes}~\cite{rice1953classes} we know:\smallskip\\
      \begin{quote}<4->
         Rice's theorem states that all nontrivial semantic properties of programs are undecidable.~\cite[100]{cousout2021principles}
      \end{quote}
      \item<5-> We can not solve the halting problem
      \item<6-> We have to approximate the reality
   \end{itemize}
% \begin{tikzpicture}[overlay,remember picture]
%    \onslide<5->{\node[above left=.5mm,yshift=3.5mm] at(current page.south east) {\scalebox{.5}{\usebox\pinguA}};}
% \end{tikzpicture}
\end{frame}

\def\Box#1{\parbox{2cm}{\centering#1}}
\def\mark{\color{green}\bfseries}
\begin{frame}{The Confusion Matrix}
   \centering
   \begin{tabular}{c@{\hskip9pt}lcc}
      & & \multicolumn{2}{c}{\onslide<2->{\tikzmarknode{prediction}{\textbf{Prediction}}}} \\
      & & \onslide<3->{\textbf{Pos.}} & \onslide<4->{\textbf{Neg.}} \\[2mm]
      \multirow{2}{*}{\onslide<5->{\tikzmarknode{actual}{\rotatebox[origin=c]{90}{\textbf{Actual}\kern9pt}}}} 
      & \onslide<6->{\textbf{Pos.}} & \onslide<8->{\mark \Box{(TP) True Positive}} & \onslide<9->{\Box{(FN) False Negative}} \\[6mm]
      & \onslide<7->{\textbf{Neg.}} & \onslide<10->{\Box{(FP) False Positive}} & \onslide<11->{\mark\Box{(TN) True Negative}}
   \end{tabular}
   \begin{tikzpicture}[overlay,remember picture,gray,line cap=round]
      \onslide<2->{\draw[Kite-] ([yshift=2pt]prediction.north) to[out=80,in=190] ++(1,.5) node[right,font=\scriptsize] {E.g., do we claim there is an error?};}
      \onslide<5->{
         \draw[Kite-] ([xshift=-1.5pt]actual.west) to[out=180,in=80] ++(-.4,-1.2) node[below,font=\scriptsize] {E.g., is there really an error?};
      }
   \end{tikzpicture}
   \vspace*{2.45em}
   \begin{itemize}
      \item<12-> \parbox{1.8cm}{\strut\textbf{Precision:}} \(\text{\mark TP} / (\text{\mark TP} + \text{FP})\) \quad \textcolor{gray}{(\enquote{how many false alarms})}
      \item<13-> \parbox{1.8cm}{\strut\textbf{Recall:}} \(\text{\mark TP} / (\text{\mark TP} + \text{FN})\) \quad \textcolor{gray}{(\enquote{how many errors did we find})}
   \end{itemize}
\end{frame}

\begin{frame}{Soundness and Completeness}
   \onslide<2->{\textbf{Soundness}}\vspace*{-2mm}
   \begin{itemize} %TODO maybe adapt definitions
      \item<3-> All properties we derive are true (but we may miss some)
      \item<4-> If we report bugs for violated properties, we produce no false negative
   \end{itemize}
   \bigskip

   \onslide<5->{\textbf{Completeness}}\vspace*{-2mm}
   \begin{itemize} %TODO maybe adapt definitions
      \item<6-> We are able to infer all interesting properties in the program
      \item<7-> If we report bugs for violated properties, we produce no false positive
   \end{itemize}
   \begin{center}
      \onslide<8->{Abstract interpretation soundly over-approximates the program semantics}
   \end{center}
   % \note[itemize]{
   %    \item Sound: Vlt. sagen wir es gibt wo ein problem, wo es keins gibt, soundiness: restrict to some features
   %    \item Complete: Vlt. sagen wir es gibt kein problem, wo es eins gibt
   % }
\end{frame}


