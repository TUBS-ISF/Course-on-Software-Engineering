\subsection{Common Misunderstandings}
\begin{frame}{\insertsubsection}
	\begin{fancycolumns}
		\begin{example}{Misunderstanding \#1}
			\mycite{I've heard array access is slow.\\Now I try to avoid them.}
		\end{example}
		
		\begin{example}{Misunderstanding \#2}
			\mycite{I've heard loops are slow.\\Now I try to avoid them.}
		\end{example}
		
		\begin{example}{Misunderstanding \#3}
			\mycite{I've heard method calls are slow.\\Now I try to avoid them.}
		\end{example}
		\nextcolumn
		\begin{example}{Misunderstanding \#4}
			\mycite{I've heard objects are slow.\\Now I try to avoid them.}
		\end{example}
		
		\begin{example}{Misunderstanding \#5}
			\mycite{I've heard garbage collection is slow.\\Now I try to avoid it.}
		\end{example}
		
		\begin{example}{Misunderstanding \#6}
			\mycite{I'm new to programming and think that every instruction takes equally long.}
		\end{example}
	\end{fancycolumns}
\end{frame}

\subsection{Arrays, Loops, and Method Calls}
\begin{frame}{No Array Access?}
	\begin{fancycolumns}
		\begin{example}{Misunderstanding \#1}
			\mycite{I've heard array access is slow.\\Now I try to avoid them.}
		\end{example}
		\begin{definition}{{Tasks for Array Access a[b]}}
			\begin{itemize}
				\item evaluate the expression b (could be arbitrarily complex)
				\item compute offset = b * size of each field
				\item compute memory location = position of a + offset
				\item access memory location
				\item only for objects: use value as memory location
			\end{itemize}
		\end{definition}
		\nextcolumn
		\begin{example}{Example}
			a[b.n()] = a[b.n()-1] * a[b.n()-1];
		\end{example}
		\begin{example}{Simplified Example}
			Assumption 1: method n has no side-effects\\Assumption 2: method n cannot be overridden (e.g., a private method)\\~\\int n = b.n();\\a[n] = a[n-1] * a[n-1];
		\end{example}
		\begin{note}{Note}
			arrays are very common, compilers and hardware have plenty of optimizations
		\end{note}
	\end{fancycolumns}
\end{frame}

\begin{frame}{No Loops?}
	\begin{fancycolumns}
		\begin{example}{Misunderstanding \#2}
			\mycite{I've heard loops are slow.\\Now I try to avoid them.}
		\end{example}
		\begin{definition}{{Tasks for Loops}}
			\begin{itemize}
				\item create and initialize loop variable
				\item check loop condition
				\item \emph{run loop body}
				\item increment loop variable
				\item repeat from check loop condition
			\end{itemize}
		\end{definition}
		\nextcolumn
		\begin{example}{Example Loop}
			for (int i = 0; i++; i \textless\  3) \{ a[i] = i; \}
		\end{example}
		\begin{example}{Loop Unrolling}
			a[0] = 0; a[1] = 1; a[2] = 2;
		\end{example}
		\begin{note}{Do Not Avoid Loops!}
			\textbf{smells}: duplicated code, long method\\\textbf{compiler optimization}: loop unrolling (if number of runs is statically known and small enough)
		\end{note}
	\end{fancycolumns}
\end{frame}

\begin{frame}{No Method Calls?}
	\begin{fancycolumns}
		\begin{example}{Misunderstanding \#3}
			\mycite{I've heard method calls are slow.\\Now I try to avoid them.}
		\end{example}
		\begin{definition}{Tasks for Each Method Call}
			\begin{itemize}
				\item pass parameters and return address
				\item save registers
				\item run method body, pass return value
				\item restore registers, return to caller
			\end{itemize}
		\end{definition}
		\begin{note}{Do Not Avoid Method Calls!}
			\textbf{smells}: duplicated code, long method\\\textbf{refactoring}: extract method\\\textbf{compiler optimization}: method inlining
		\end{note}
		\nextcolumn
		\begin{exampletight}{Memory Layout: The Method Stack}
			\begin{tikzpicture}[every node/.style = {outer sep = 0pt, minimum width = 2cm, text depth = 0pt, draw}]
				\node[minimum height=1cm] (A) {};
				
				\node[below = 0pt of A] (B) {dynamic};
				\node[below = 0pt of B] (C) {static};
				\node[below = 0pt of C, minimum height=1cm] (D) {};
				\node[below = 0pt of D] (E) {return address};
				\node[below = 0pt of E] (F) {1$^\text{st}$ parameter};
				\node[below = 0pt of F] (G) {2$^\text{nd}$ parameter};
				\node[below = 0pt of G] (H) {$\ldots$};
				\node[below = 0pt of H] (I) {n$^\text{th}$ parameter};
				
				\node[left = -10pt of I.south west, align = right, font = \small, draw = none] {high\\addresses};
				
				\node[left = -10pt of A.north west, align = right, font = \small, draw = none] {low\\addresses};
				
				\path[-latex, gray]
				([xshift = 5pt, yshift = 5pt]A.north west) edge node[right, font=\small, draw = none] {next stack frame} ([xshift = 5pt, yshift = 15pt] A.north west)
				
				([xshift = 5pt, yshift = -5pt]I.south west) edge node[right, font=\small, draw = none] {previous stack frame} ([xshift = 5pt, yshift = -15pt] I.south west)
				;
				
				\path[line width = 2pt]
				
				([xshift = 5pt, yshift = -2pt]A.north east) edge[blue] node[align=left, right, font = \small, draw = none] {stack for evaluation\\of expressions} ([xshift = 5pt, yshift = 2pt]A.south east)
				
				([xshift = 5pt, yshift = -2pt]B.north east) edge[green] node[align=left, right, font = \small, draw = none] {local variables} ([xshift = 5pt, yshift = 2pt]C.south east)	
				
				([xshift = 5pt, yshift = -2pt]D.north east) edge[red] node[align=left, right, font = \small, draw = none] {saved registers} ([xshift = 5pt, yshift = 2pt]D.south east)	
				;
			\end{tikzpicture}
		\end{exampletight}
	\end{fancycolumns}
\end{frame}

\subsection{Objects and Garbage Collection}
\begin{frame}{No Objects?}
	\begin{fancycolumns}
		\begin{example}{Misunderstanding \#4}
			\mycite{I've heard objects are slow.\\Now I try to avoid them.}
		\end{example}
		\begin{definition}{Memory Layout for Objects}
			\begin{itemize}
				\item if life time not bound to a method, cannot be stored on method stack
				\item stored in free position on the heap
				\item pointer and derefence required
				\item need to store class information
				\item pointer to virtual method table for dynamic dispatch
				\item all fields, even inherited fields
			\end{itemize}
		\end{definition}
		\nextcolumn
		\begin{note}{Do Not Avoid Objects!}
			unless performance or memory consumption is a problem, then identify which objects (a) consume most memory and (b) have the shortest life time
		\end{note}
		\begin{example}{Thomas' Experience with LinkedList in Java}
			LinkedList can be problematic, as there is a list object for every entry in the list and list manipulations lead to new list objects\\~\\solution: use arrays or ArrayList instead\\~\\large speed-ups in FeatureIDE: \href{https://github.com/FeatureIDE/FeatureIDE/blob/b7732df29da1652c74b4342cd087bba436919815/plugins/de.ovgu.featureide.fm.core/src/org/prop4j/Node.java\#L46}{github.com}\\~\\also reported by others: \href{https://stackoverflow.com/questions/322715/when-to-use-linkedlist-over-arraylist-in-java}{stackoverflow.com}
		\end{example}
		% TODO search for link to commit removing linked lists in prop4j
	\end{fancycolumns}
\end{frame}

\begin{frame}[b]{No Garbage Collection?}
	\vspace{-10mm}
	\begin{fancycolumns}[b]
		\begin{example}{Misunderstanding \#5}
			\mycite{I've heard garbage collection is slow.\\Now I try to avoid it.}
		\end{example}
		\begin{definition}{Garbage Collection}
			\begin{itemize}
				\item find objects not referenced anymore
				\item free memory space assigned by them
				\item algorithms: reference counting, mark-and-sweep, copying collection
				\item causes random delays
			\end{itemize}
		\end{definition}
		\begin{note}{Reasons for Automatic Memory Management}
			\begin{itemize}
				\item simplified programming, programs
				\item fewer memory leaks
				\item improved safety, security, reliability
			\end{itemize}
		\end{note}
		\nextcolumn
		\hfill
		\pic[width=.6\linewidth]{emotions/fridge}\pause
		\begin{example}{Avoiding Garbage Collection}
			\begin{itemize}
				\item switch to a language with manual memory allocation (e.g., C/C++)
				\item use ownership and borrowing as in Rust
				\item deactivate garbage collection completely (only for programs with short runtime and low memory consumption)
				\item web services: do garbage collection when service is idle
			\end{itemize}
		\end{example}
	\end{fancycolumns}
\end{frame}
%For language implementation, this means that
%• the set of reachable objects has to be known
%• objects/record variables must be suitable for garbage collection
%• garbage collection must be implemented
%The first two aspects concern the compiler.

\subsection{Simplicity and Performance}
\begin{frame}{Speed = Instructions per Minute?}
	\begin{fancycolumns}
		\begin{example}{Misunderstanding \#6}
			\mycite{I'm new to programming and think that every instruction takes equally long.}
		\end{example}
		
		\begin{example}{Misunderstanding \#7}
			\mycite{I'm new to computers / smartphones and think that every instruction takes equally long.}
		\end{example}
		\nextcolumn
		\begin{note}{Hint}
			Use all language constructs and let compilers do their job.
			
			If performance is not sufficient, inspect bottleneck.
			
			Only reduce code quality in favor of performance where necessary.
		\end{note}
	\end{fancycolumns}
\end{frame}

\begin{frame}<2>{Simplicity over Performance}
	\slideSimplicityOverPerformance
\end{frame}
